\section{Focused inverse method}

The primary issue in the forward direction is to enumerate the propositions for
which we need to derive inference rules. As the calculus of derived rules has
only neutral sequents as premises and conclusions, we need only generate rules
for propositions that occur in neutral sequents; we call them \emph{frontier
  propositions}. To find the frontier propositions in a goal sequent, we
abstractly replay the focusing and active phases to identify the phase
transitions. Each transition from an active to a focal phase produces a frontier
proposition.

\[
  \decminus{f(p)} = \emptyset \qquad
  \decplus{f(p)} = \decplusminus{a(p)} = \{
  \decplusminus{\declinear{p}} \} \qquad \text{if $p$ right-based}
\]
\[
  \decplus{f(p)} = \emptyset \qquad
  \decminus{f(p)} = \decminusplus{a(p)} = \{
  \decminusplus{\declinear{p}} \} \qquad \text{if $p$ left-biased}
\]
\[
  \decminus{f(A \otimes B)} = \decminus{a(A \otimes B)} \qquad
  \decplus{f(A \otimes B)} = \decplus{f(A)}, \decplus{f(B)}
\]
\[
  \decminus{a(A\otimes B)} = \decminus{a(A)}, \decminus{a(B)} \qquad
  \decplus{a(A\otimes B)} = \decplus{f(A \otimes B)}, \decminuslinear{(A \otimes B)}
\]
\[
  \decminus{f(A \limp B)} = \decplus{f(A)}, \decminus{f(B)} \qquad
  \decplus{f(A \limp B)} = \decplus{a(A \limp B)}
\]
\[
  \decminus{a(A \limp B)} = \decminus{f(A \limp B)}, \decminuslinear{(A \limp
    B)}
  \qquad
  \decplus{a(A\limp B)} = \decminus{a(A)}, \decplus{a(B)}
\]

\begin{example}
  Put example here...
\end{example}

\begin{definition}[Frontier]
  Given a goal $\Gamma;\Delta \Longrightarrow Q$, its frontier contains

  \begin{enumerate}
  \item All top-level propositions in $\decunrestr{\decminus{\Gamma}},
    \decminuslinear{\Delta}$, and $\decpluslinear{Q}$;
  \item For any $A \in \Gamma, \Delta$, the collection $\decminus{f(A)}$;
  \item the collection $\decplus{f(Q)}$.
  \end{enumerate}
\end{definition}

In the preparatory phase for the inverse method, we calculate the frontier
propositions of the goal sequent. There is no need to generate initial sequents
separately, as the executions of (TODO: thesis says negative. why only
negative??) atoms in the frontier directly give us the necessary initial
sequents.

We that we do not consider negative left-biased atoms (or positive right-biased
atoms) when constructing derived rules from the formulas in the
frontier. Suppose we picked a negative left-biased atom $p$ from the frontier
and tried to build the associated derived rule. This amounts to finding a
derivation for $\relj{\flfrel{p}}{\Sigma}{s}$, which can only happen with an
application of the $conj^-$ rule:

\[
  \begin{prooftree}
    \fneuseq{\Gamma}{\Delta, p}{Q} \quad
    \[
      p \text{ left-biased}
      \justifies
      \flfrelj{p}{\fneuseq{\Gamma}{\Delta, p}{Q}}{\fneuseq{\Gamma}{\Delta}{Q}}
      \using{conj^-}
    \]
    \justifies
    \fneuseq{\Gamma}{\Delta, P}{Q}
    \using{\focminusrule}
  \end{prooftree}
\]

This leads to the following derived rule

\[
  \begin{prooftree}
    \fneuseq{\Gamma}{\Delta, p}{Q}
    \justifies
    \fneuseq{\Gamma}{\Delta, p}{Q}
  \end{prooftree}
\]

which is obviously redundant. The same happens if we tried to build a rule for a
positive right-biased atom. Nevertheless, the $conj^+$ and $conj^-$ rules are
essential to achieve completeness, as also shown by the example below.

\begin{example}
  Example... where $d$ is right-biased and all other atoms are left-biased, and
  $\Gamma \equiv R_1, R_2$.

  \[
    \begin{prooftree}
      \[
        \[ q \text{ left-biased} \justifies \rfocseq{\Gamma}{q}{q}\]
        \[
          \[
            \[
              \[
                \[
                  \[
                    \[
                      \[ d \text{ right-biased} \justifies \lfocseq{\Gamma}{\cdot}{d}{d} \]
                      \justifies
                      \btriseq{\Gamma}{d}{\cdot}{\cdot ; d}
                    \]
                    \justifies
                    \rfocseq{\Gamma}{d}{d}
                  \]
                  \[
                    \[
                      \[ d \text{ right-biased} \justifies \lfocseq{\Gamma}{\cdot}{d}{d} \]
                      \justifies
                      \btriseq{\Gamma}{d}{\cdot}{\cdot ; d}
                    \]
                    \justifies
                    \rfocseq{\Gamma}{d}{d}
                  \]
                  \justifies
                  \rfocseq{\Gamma}{d,d}{d \otimes d}
                \]
                \[
                  \[
                    \[
                      \[
                        \[ n \text{ left-biased} \justifies \rfocseq{\Gamma}{n}{n}\]
                        \[ n \text{ left-biased} \justifies \rfocseq{\Gamma}{n}{n}\]
                        \justifies
                        \rfocseq{\Gamma}{n,n}{n \otimes n}
                      \]
                      \[
                        d \text{ right-biased}
                        \justifies
                        \lfocseq{\Gamma}{\cdot}{d}{d}
                      \]
                      \justifies
                      \lfocseq{\Gamma}{n,n}{n \otimes n \limp d}{d}
                    \]
                    \justifies
                    \btriseq{\Gamma}{n,n}{\cdot}{\cdot ; d}
                    \using{\copyrule}
                  \]
                  \justifies
                  \rfocseq{\Gamma}{n,n}{d}
                \]
                \justifies
                \rfocseq{\Gamma}{n,d,d,n}{d \otimes d \otimes d}
              \]
              \justifies
              \btriseq{\Gamma}{n, d, d, n}{\cdot}{\cdot ; d \otimes d \otimes d}
              \using{\rfoc}
            \]
            \justifies
            \btriseq{\Gamma}{n}{d \otimes d \otimes n}{\cdot ; d \otimes d \otimes d}
            \using{\otimes L \times 3, \lact \times 3}
          \]
          \justifies
          \lfocseq{\Gamma}{n}{d \otimes d \otimes n}{d \otimes d \otimes d}
          \using{\lblur}
        \]
        \justifies
        \lfocseq{\Gamma}{q,n}{q \limp d \otimes d \otimes n}{d \otimes d \otimes d}
        \using{\limp L}
      \]
      \justifies
      \btriseq{\Gamma}{q,n}{\cdot}{\cdot ; d \otimes d \otimes d}
      \using{\copyrule}
    \end{prooftree}
  \]
\end{example}

The frontier of the full goal sequent

\[
  \fneuseq{q \limp d\otimes d\otimes n, n \otimes n \limp d}{q,n}{d \otimes d
    \otimes d}
\]

is given by the top-level decorations

\[
  \decminusunrestr{(q \limp d \otimes d \otimes n)},
  \decminusunrestr{(n \otimes n \limp d)},
  \decminuslinear{q}, \decminuslinear{n},
  \decpluslinear{(d \otimes d \otimes d)}
\]

together with the additionally computed subformulas
$\decminuslinear{d}, \decminuslinear{n}, \decpluslinear{d}$. The frontier
formulas give rise to the initial derived rules as follows

\begin{itemize}
\item $\decminuslinear{q}$ is a negative left-biased atom, hence it is ignored;
  same for $\decminuslinear{n}$;
\item $\decminuslinear{d}$ is a negative right-biased atom, hence we get an
  initial sequent as follows:

  \[
    \begin{prooftree}
      \[
        d \; \text{right-based}
        \justifies
        \relj{\flfrel{d}}{\cdot}{\fneuseq{\cdot}{\cdot}{d}}
        \using{\linit}
      \]
      \justifies
      \fneuseq{\cdot}{d}{d}
      \using{\focminusrule}
    \end{prooftree}
  \]

\item $\decpluslinear{d}$ is a positive right-biased atom, hence it is ignored;

\item The negative unrestricted formula $q \limp d \otimes d \otimes d$ gives
  rise to a derived rule as follows

  {
    \scriptsize{
      \[
        \begin{prooftree}
          \fneuseq{\Gamma}{\Delta, d, d, n}{Q}
          \[
            \[
              \[
                \[
                  \justifies
                  \factrelj{
                    \btriseq{\cdot}{d,d,n}{\cdot}{\cdot}
                  }{
                    \fneuseq{\Gamma}{\Delta,d,d,n}{Q}
                  }{
                    \fneuseq{\Gamma}{\Delta}{Q}
                  }
                \]
                \justifies
                \factrelj{
                  \btriseq{\cdot}{\cdot}{d \otimes d \otimes n}{\cdot}
                }{
                  \fneuseq{\Gamma}{\Delta,d,d,n}{Q}
                }{
                  \fneuseq{\Gamma}{\Delta}{Q}
                }
              \]
              \justifies
              \flfrelj{d \otimes d \otimes n}{\fneuseq{\Gamma}{\Delta,d,d,n}{Q}}{\fneuseq{\Gamma}{\Delta}{Q}}
            \]
            \[
              q \text{ left-biased}
              \justifies
              \frfrelj{q}{\cdot}{\fneuseq{\cdot}{q}{\cdot}}
            \]
            \justifies
            \flfrelj{q \limp d \otimes d \otimes
              d}{\fneuseq{\Gamma}{\Delta,d,d,n}{Q}}{\fneuseq{\Gamma}{\Delta,q}{Q}}
          \]
          \justifies
          \fneuseq{\Gamma, R_3}{\Delta, q}{Q}
        \end{prooftree}
      \]
    }
  }

  that is

  \[
    \begin{prooftree}
      \fneuseq{\Gamma}{\Delta, d, d, n}{Q}
      \justifies
      \fneuseq{\Gamma, R_3}{\Delta, q}{Q}
    \end{prooftree}
  \]

\item The negative unrestricted formula $n \otimes n \limp d$ gives rise to a
  derived rule as follows

  \[
    \begin{prooftree}
      \[
        \[
          \[
            n \text{ left-biased}
            \justifies
            \frfrelj{n}{\cdot}{\fneuseq{\cdot}{n}{\cdot}}
          \]
          \[
            n \text{ left-biased}
            \justifies
            \frfrelj{n}{\cdot}{\fneuseq{\cdot}{n}{\cdot}}
          \]
          \justifies
          \frfrelj{n \otimes n}{\cdot}{\fneuseq{\cdot}{n,n}{\cdot}}
        \]
        \[
          d \text{ right-biased}
          \justifies
          \flfrelj{d}{\cdot}{\fneuseq{\cdot}{\cdot}{d}}
        \]
        \justifies
        \flfrelj{n \otimes n \limp d}{\cdot}{\fneuseq{\cdot}{n,n}{d}}
      \]
      \justifies
      \fneuseq{R_1}{n,n}{d}
    \end{prooftree}
  \]

  that is

  \[
    \begin{prooftree}
      \justifies
      \fneuseq{R_1}{n,n}{d}
    \end{prooftree}
  \]

\item The positive linear formula $d \otimes d \otimes d$ gives rise to a
  derived rule as follows

  \[
    \begin{prooftree}
      \[
        \[
          d \text{ right-biased}
          \justifies
          \frfrelj{d}{s_1}{\fneuseq{\Gamma_1}{\Delta_1}{\cdot}}
        \]
        \[
          d \text{ right-biased}
          \justifies
          \frfrelj{d}{s_2}{\fneuseq{\Gamma_2}{\Delta_2}{\cdot}}
        \]
        \justifies
        \frfrelj{d \otimes d
        }{
          s_1 \cdot s_2
        }{
          \fneuseq{\Gamma_1,\Gamma_2}{\Delta_1,\Delta_2}{\cdot}
        }
      \]
      \[
        d \text{ right-biased}
        \justifies
        \frfrelj{d}{s_3}{\fneuseq{\Gamma_3}{\Delta_3}{\cdot}}
      \]
      \justifies
      \frfrelj{d \otimes d \otimes d}{s_1 \cdot s_2 \cdot s_3
      }{
        \fneuseq{\Gamma_1, \Gamma_2, \Gamma_3}{\Delta_1, \Delta_2, \Delta_3}{\cdot}
      }
    \end{prooftree}
  \]

  where $s_i \equiv \fneuseq{\Gamma_i}{\Delta_i}{d}$. Therefore

  \[
    \begin{prooftree}
      \fneuseq{\Gamma_1}{\Delta_1}{d}
      \quad
      \fneuseq{\Gamma_2}{\Delta_2}{d}
      \quad
      \fneuseq{\Gamma_3}{\Delta_3}{d}
      \justifies
      \fneuseq{\Gamma_1, \Gamma_2, \Gamma_3}{\Delta_1, \Delta_2, \Delta_3}{d
        \otimes d \otimes d}
    \end{prooftree}
  \]

\end{itemize}

Then, the goal sequent can be derived as follows:

\[
  \begin{prooftree}
    \[
      \[
        \justifies
        \fneuseq{\cdot}{d}{d}
      \]
      \[
        \justifies
        \fneuseq{\cdot}{d}{d}
      \]
      \[
        \justifies
        \fneuseq{R_1}{n,n}{d}
      \]
      \justifies
      \fneuseq{R_1}{d,d,n,n}{d \otimes d \otimes d}
    \]
    \justifies
    \fneuseq{R_1, R_3}{q,n}{d \otimes d \otimes d}
  \end{prooftree}
\]


%%% Local Variables:
%%% mode: latex
%%% TeX-master: "../docs"
%%% End:
