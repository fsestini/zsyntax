\section{Focused inverse method}

The last crucial step in building an automated deduction procedure for \zss{} is
to think of how to use our forward derived rule calculus with the inverse method
described in the preceding sections. On a more technical level, the primary
issue in the implementation of the inverse method is to enumerate the formulas
for which we need to derive inference rules. As the calculus of derived rules
has only neutral sequents as premises and conclusions, we only need to generate
rules for formulas that occur in neutral sequents; we call those \emph{frontier
  formulas} as in \cite{chaudhuri-thesis}. To find the frontier formulas in a
goal sequent, we abstractly replay the focusing and active phases to identify
the phase transitions: each transition produces a frontier formula.

To generate the frontier, we use \emph{decorated formulas} to keep track of the
polarity as well as the linearity of the formula. A positive polarity marks
formulas that will occur on the right of the sequent symbol, whereas a negarive
polarity marks those that will occur on the left. Moreover, formulas marked with
a dot are linear, whereas those marked with an exclamation mark are part of the
unrestricted context.

\[
  \textsf{DecFormula} ::= \decplus{\declinear{A}} \, | \,
    \decminus{\declinear{A}} \, | \,
    \decunrestr{A}
\]

where $A$ is a formula. Notice that we do not distinguish the polarity of
unrestricted formulas, as they are all negative.

\begin{definition}
  The auxiliary fuctions \textsf{fr}, \textsf{foc}, and \textsf{act} are
  inductively defined as follows:

  \[
    \begin{array}{c}
      \decminus{\textsf{fr}(p)} = \{\decminus{\declinear{p}}\} \qquad
      \decplus{\textsf{fr}(p)} = \emptyset \qquad
      \textsf{foc}(p) = \emptyset \qquad
      \textsf{act}(p) = \{\decminus{\declinear{p}}\} \\

      \decplus{\textsf{fr}(A \otimes B)} = \textsf{foc}(A \otimes B) \qquad
      \textsf{foc}(A \otimes B) = \textsf{foc}(A) \cup \textsf{foc}(B) \\

      \textsf{act}(A \otimes B) = \textsf{act}(A) \cup \textsf{act}(B) \\

      \decminus{\textsf{fr}(A \rightarrow_{\reactlist{}}^S B)} =
      \textsf{foc}(A) \cup \textsf{act}(B) \\
      \decplus{\textsf{fr}(A \rightarrow_{\reactlist{}}^S B)} =
      \textsf{act}(A) \cup \{ \decplus{\declinear{B}} \} \cup \textsf{fr}(B) \\

      \textsf{foc}(A \rightarrow_{\reactlist{}}^S B) =
      \decplus{\declinear{(A \rightarrow_{\reactlist{}}^S B)}} \cup
      \decplus{\textsf{fr}(A \rightarrow_{\reactlist{}}^S B)} \\

      \textsf{act}(A \rightarrow_{\reactlist{}}^S B) =
      \decminus{\declinear{(A \rightarrow_{\reactlist{}}^S B)}} \cup
      \decminus{\textsf{fr}(A \rightarrow_{\reactlist{}}^S B)}
    \end{array}
  \]
\end{definition}

% \[
%   \decplus{f(p)} = \emptyset \qquad
%   \decminus{f(p)} = \decminusplus{a(p)} = \{
%   \decminusplus{\declinear{p}} \} \qquad \text{if $p$ left-biased}
% \]
% \[
%   \decminus{f(A \otimes B)} = \decminus{a(A \otimes B)} \qquad
%   \decplus{f(A \otimes B)} = \decplus{f(A)}, \decplus{f(B)}
% \]
% \[
%   \decminus{a(A\otimes B)} = \decminus{a(A)}, \decminus{a(B)} \qquad
%   \decplus{a(A\otimes B)} = \decplus{f(A \otimes B)}, \decminuslinear{(A \otimes B)}
% \]
% \[
%   \decminus{f(A \limp B)} = \decplus{f(A)}, \decminus{f(B)} \qquad
%   \decplus{f(A \limp B)} = \decplus{a(A \limp B)}
% \]
% \[
%   \decminus{a(A \limp B)} = \decminus{f(A \limp B)}, \decminuslinear{(A \limp
%     B)}
%   \qquad
%   \decplus{a(A\limp B)} = \decminus{a(A)}, \decplus{a(B)}
% \]

\begin{definition}[Frontier]
  Given a goal sequent $\Gamma;\Delta \Longrightarrow Q$, its frontier is the
  set union of the following:

  \begin{enumerate}
  \item All top-level formulas in $\decunrestr{\Gamma},
    \decminuslinear{\Delta}$, and $\decpluslinear{Q}$;
  \item For any $A \in \Gamma, \Delta$, the set $\decminus{\textsf{fr}(A)}$;
  \item The set $\decplus{\textsf{fr}(Q)}$.
  \end{enumerate}
\end{definition}

In the preparatory phase for the inverse method, we calculate the frontier
formulas of the goal sequent. There is no need to compute the initial sequents
separately, as the generation of the derived rules on the frontier formulas
already does that. In particular, we just take as initial sequents all rules
with zero premises among those generated from the frontier formulas.

The generation proceeds as follows: for each formula $A$ in the frontier

\begin{itemize}
\item If $A$ is a positive linear atom or a positive linear conjunction, we
  generate the corresponding $focus$ derived rule;
\item If $A$ is a positive linear implication, we generate the corresponding
  $\rightarrow R$ derived rule;
\item If $A$ is a negative linear implication, we generate the corresponding
  $\rightarrow L$ derived rule;
\item If $A$ is a negative unrestricted implication, we generate the
  corresponding $\copyrule$ derived rule.
\end{itemize}

\begin{example}
  Consider the goal sequent
  $\zsyseq{A_1, A_2}{q,n}{ [(\emptyset,\ctrlset{})]}{d \otimes d \otimes d}$,
  and $A_1, A_2$ are axioms defined as follows:

  \begin{align*}
    A_1 & \equiv q \rightarrow_{[]}^{\emptyset} d \otimes d \otimes n \\
    A_2 & \equiv n \otimes n \rightarrow_{[(\emptyset, \ctrlset{})]}^{\emptyset} d
  \end{align*}

  where $[(\emptyset, \ctrlset{})]$ is just a shorthand for
  $(\emptyset, \ctrlset{}):[]$, and $\ctrlset{}$ is defined such that for any
  $\Delta$, $\respects{\Delta}{\ctrlset{}}$ iff $q \notin \Delta$.  A derivation
  of such sequent in $\zss$ is as follows:

  {\scriptsize{\[
    \begin{prooftree}
      \[
        \[\justifies\zsyseq{A_1, A_2}{q}{\emptyctrl{}}{q}\] \,
        \[
          \[
            \[
              \[
                \[
                  \[\justifies \zsyseq{A_1, A_2}{n}{\emptyctrl{}}{n}\]
                  \quad
                  \[\justifies \zsyseq{A_1, A_2}{n}{\emptyctrl{}}{n}\]
                  \justifies
                  \zsyseq{A_1, A_2}{n,n}{\emptyctrl{}}{n \otimes n}
                \]
                \,
                \[
                  \[
                    \[\justifies \zsyseq{A_1, A_2}{d}{\emptyctrl{}}{d}\]
                    \quad
                    \[\justifies \zsyseq{A_1, A_2}{d}{\emptyctrl{}}{d}\]
                    \justifies
                    \zsyseq{A_1, A_2}{d, d}{\emptyctrl{}}{d \otimes d}
                  \]
                  \quad
                  \[\justifies\zsyseq{A_1, A_2}{d}{\emptyctrl{}}{d}\]
                  \justifies
                  \zsyseq{A_1, A_2}{d, d, d}{\emptyctrl{}}{d \otimes d \otimes d}
                \]
                \justifies
                \zsyseq{A_1, A_2}{n \otimes n \rightarrow_{[(\emptyset,\ctrlset{})]}^{\emptyset{}} d,
                  d, d, n,n}{[(\emptyset,\ctrlset{})]}{d \otimes d \otimes d}
              \]
              \justifies
              \zsyseq{A_1, A_2}{d, d, n,n}{[(\emptyset,\ctrlset{})]}{d \otimes d
                \otimes d}
              \using{\copyrule}
            \]
            \justifies
            \zsyseq{A_1, A_2}{d \otimes d, n,n}{[(\emptyset,\ctrlset{})]}{d \otimes d
              \otimes d}
            \using{\otimes L}
          \]
          \justifies
          \zsyseq{A_1, A_2}{d \otimes d \otimes n,n}{[(\emptyset,\ctrlset{})]}{d \otimes d
            \otimes d}
          \using{\otimes L}
        \]
        \justifies
        \zsyseq{A_1, A_2}{q \rightarrow_{\emptyctrl{}}^{\emptyset{}} d \otimes d
          \otimes n, q,n}{[(\emptyset,\ctrlset{})]}{d \otimes d \otimes d}
      \]
      \justifies
      \zsyseq{A_1, A_2}{q,n}{[(\emptyset,\ctrlset{})]}{d \otimes d \otimes d}
      \using{\copyrule}
    \end{prooftree}
  \]}}

Let us see now how to derive the same goal sequent in the forward derived rule
calculus. The frontier of the full goal sequent

\[
  \zfneuseq{q \rightarrow d\otimes d\otimes n, n \otimes n \rightarrow d}{q,n}{[(\emptyset,\ctrlset{})]}
  {d \otimes d \otimes d}
\]

is given by the top-level decorations

\[
  \decunrestr{(q \rightarrow d \otimes d \otimes n)},
  \decunrestr{(n \otimes n \rightarrow d)},
  \decminuslinear{q}, \decminuslinear{n},
  \decpluslinear{(d \otimes d \otimes d)}
\]

together with the additionally computed subformulas
$\decminuslinear{d}, \decminuslinear{n}$. The frontier
formulas give rise to the derived rules as follows

\begin{itemize}
\item $\decminuslinear{q}$ is a negative atom, hence it is ignored.
  Same for $\decminuslinear{n}$ and $\decminuslinear{d}$;

\item The unrestricted formula $q \rightarrow d \otimes d \otimes d$ gives
  rise to a derived rule as follows

  {
    \scriptsize{
      \[
        \begin{prooftree}
          \zfneuseq{\Gamma}{\Delta, d, d, n}{\reactlist{}}{Q}
          \quad
          \[
            \justifies
            \frfrelj{q}{\cdot}{\fneuseq{\cdot}{q}{\cdot}}
          \]
          \,
          \[
            \[
              \justifies
              \factrelj{
                \bkwseq{d,d,n}{\cdot}{\cdot}
              }{
                \zfneuseq{\Gamma}{\Delta,d,d,n}{\reactlist{}}{Q}
              }{
                \zfneuseq{\Gamma}{\Delta}{\reactlist{}}{Q}
              }
            \]
            \justifies
            \factrelj{
              \bkwseq{\cdot}{d \otimes d \otimes n}{\cdot}
            }{
              \zfneuseq{\Gamma}{\Delta,d,d,n}{\reactlist{}}{Q}
            }{
              \zfneuseq{\Gamma}{\Delta}{\reactlist{}}{Q}
            }
          \]
          \justifies
          \zfneuseq{\Gamma, A_1}{\Delta, q}{\reactlist{}}{Q}
        \end{prooftree}
      \]
    }
  }

  that is

  \[
    \begin{prooftree}
      \zfneuseq{\Gamma}{\Delta, d, d, n}{\reactlist{}}{Q}
      \justifies
      \zfneuseq{\Gamma, A_1}{\Delta, q}{\reactlist{}}{Q}
    \end{prooftree}
  \]

\item The negative unrestricted formula $n \otimes n \rightarrow d$ gives rise
  to a derived rule as follows

  {\scriptsize{\[
    \begin{prooftree}
      \[
        \[
          \justifies
          \frfrelj{n}{\cdot}{\fneuseq{\cdot}{n}{\cdot}}
        \]
        \[
          \justifies
          \frfrelj{n}{\cdot}{\fneuseq{\cdot}{n}{\cdot}}
        \]
        \justifies
        \frfrelj{n \otimes n}{\cdot}{\fneuseq{\cdot}{n,n}{\cdot}}
      \]
      \[
        \[
          \justifies
          \factrelj{
            \bkwseq{d}{\cdot}{\cdot}
          }{
            \zfneuseq{\Gamma}{\Delta,d}{\reactlist{}}{Q}
          }{
            \zfneuseq{\Gamma}{\Delta}{\reactlist{}}{Q}
          }
        \]
        \justifies
        \factrelj{
          \bkwseq{\cdot}{d}{\cdot}
        }{
          \zfneuseq{\Gamma}{\Delta,d}{\reactlist{}}{Q}
        }{
          \zfneuseq{\Gamma}{\Delta}{\reactlist{}}{Q}
        }
      \]
      \quad
      \[
        \resplist{\Delta}{[(\emptyset, \ctrlset{})]}
        \proofdotseparation=0ex
        \proofdotnumber=0
        \leadsto
        \zfneuseq{\Gamma}{\Delta, d}{\reactlist{}}{Q}
      \]
      \justifies
      \zfneuseq{\Gamma, A_2}{\Delta, n, n}{
        \baseandplus{\Delta}{[(\emptyset,\ctrlset{})]}{\reactlist{}}
      }{Q}
    \end{prooftree}
  \]}}

  that is

  \[
    \begin{prooftree}
      \zfneuseq{\Gamma}{\Delta, d}{\reactlist{}}{Q}
      \justifies
      \zfneuseq{\Gamma, A_2}{\Delta, n, n}{
        \baseandplus{\Delta}{[(\emptyset,\ctrlset{})]}{\reactlist{}}
      }{Q}
      \using{\respects{\Delta}{\ctrlset{}}}
    \end{prooftree}
  \]

\item The positive linear formula $d \otimes d \otimes d$ gives rise to a
  derived rule as follows

  \[
    \begin{prooftree}
      \[
        \[
          \[\justifies \frfrelj{d}{\cdot}{\fneuseq{\cdot}{d}{\cdot}} \]
          \[\justifies \frfrelj{d}{\cdot}{\fneuseq{\cdot}{d}{\cdot}} \]
          \justifies
          \frfrelj{d \otimes d}{\cdot}{\fneuseq{\cdot}{d,d}{\cdot}}
        \]
        \[\justifies \frfrelj{d}{\cdot}{\fneuseq{\cdot}{d}{\cdot}} \]
        \justifies
        \frfrelj{d \otimes d \otimes d}{\cdot}{\fneuseq{\cdot}{d,d,d}{\cdot}}
      \]
      \justifies
      \zfneuseq{\cdot}{d,d,d}{\emptyctrl{}}{d \otimes d \otimes d}
    \end{prooftree}
  \]

  therefore, simply

  \[
    \begin{prooftree}
      \justifies
      \zfneuseq{\cdot}{d,d,d}{\emptyctrl{}}{d \otimes d \otimes d}
    \end{prooftree}
  \]

\end{itemize}

Then, the goal sequent can be fully derived with just two rule applications as
follows:

\[
  \begin{prooftree}
    \[
      \[\justifies\zfneuseq{\cdot}{d,d,d}{\emptyctrl{}}{d \otimes d \otimes d}\]
      \justifies
      \zfneuseq{A_2}{d,d,n,n}{[(\emptyset, \ctrlset{})]}{d \otimes d \otimes d}
    \]
    \justifies
    \zfneuseq{A_1, A_2}{q, n}{[(\emptyset, \ctrlset{})]}{d \otimes d \otimes d}
  \end{prooftree}
\]

\end{example}

The example above shows the power of combining the forward search of the inverse
method with the use of big-step derived rules from the focused calculus.  The
good proof-theoretical properties of our original sequent calculus allow us to
derive many properties and do a lot of automatic work in the phase preliminary
to proof-search, from the structure of the goal sequent only. This has the
effect that many redundant and inessential steps are eliminated from the actual
proof-search phase, where they would otherwise badly affect its computational
complexity.

%%% Local Variables:
%%% mode: latex
%%% TeX-master: "../docs"
%%% End:
