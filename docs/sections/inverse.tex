\section{The Inverse Method}

\subsection{Multiplicative non-determinism}

... TODO. .... rationale for a forward calulus...

... TODO ... description of the inverse method...


.... blah blah we need a forward calculus .... blah...

\subsection{Forward sequent calculus}

Since weakening and contraction are admissible for the unrestricted
zone, we can treat $\Gamma$ as a set.

\begin{figure}[h]
  \begin{mdframed}
    \[
      \begin{prooftree}
        \justifies
        \cdot ; P \longrightarrow P
        \using{\init}
      \end{prooftree}
      \qquad \qquad
      \begin{prooftree}
        \Gamma; \Delta, A \longrightarrow C
        \justifies
        \Gamma \cup \{A\}; \Delta \longrightarrow C
        \using{\copyrule}
      \end{prooftree}
    \]

    \[
      \begin{prooftree}
        \Gamma; \Delta \longrightarrow A
        \qquad
        \Gamma'; \Delta' \longrightarrow B
        \justifies
        \Gamma\cup \Gamma'; \Delta, \Delta' \longrightarrow A \otimes B
        \using{\otimes R}
      \end{prooftree}
      \qquad \qquad
      \begin{prooftree}
        \Gamma; \Delta, A, B \longrightarrow C
        \justifies
        \Gamma; \Delta, A \otimes B \longrightarrow C
        \using{\otimes L}
      \end{prooftree}
    \]

    \[
      \begin{prooftree}
        \Gamma; \Delta \longrightarrow A
        \qquad
        \Gamma'; \Delta', B \longrightarrow C
        \justifies
        \Gamma \cup \Gamma'; \Delta, \Delta', A \limp B \longrightarrow C
        \using{\limp L}
      \end{prooftree}
      \qquad \qquad
      \begin{prooftree}
        \Gamma; \Delta, A \longrightarrow B
        \justifies
        \Gamma; \Delta \longrightarrow A \limp B
        \using{\limp R}
      \end{prooftree}
    \]
  \end{mdframed}
  \caption{Forward sequent calculus}
\end{figure}

\begin{theorem}[Soundness]
  If $\Gamma ; \Delta \longrightarrow C$, then
  $\Gamma ; \Delta \Longrightarrow C$.
\end{theorem}
\begin{proof}
  Just notice that every forward derivation is also a backward derivation, with
  the difference that in the latter unrestricted contexts are copied in all
  the premises. The details are a straightforward induction on the height of a
  forward derivation, plus weakening for the backward calculus.
\end{proof}

\begin{theorem}[Completeness]
  If $\Gamma ; \Delta \Longrightarrow C$, then
  $\Gamma' ; \Delta \longrightarrow C$ for some $\Gamma' \subseteq \Gamma$.
\end{theorem}
\begin{proof}
  Straightforward induction on the backward derivation.
\end{proof}

\begin{example}
  The following is an example derivation of

  $$\frwdseq{R_1, R_3}{\cdot}{q \otimes n \limp d \otimes d \otimes d}$$

  where $R_1 \equiv n \otimes n \limp d$ and
  $R_3 \equiv q \limp d \otimes d \otimes n$.

  \[
    \begin{prooftree}
      \[
        \[
          \[
            \frwdseq{\cdot}{q}{q}
            \[
              \[
                \[
                  \frwdseq{}{d}{d}
                  \quad
                  \frwdseq{}{d}{d}
                  \justifies
                  \frwdseq{}{d, d}{d \otimes d}
                \]
                \quad
                \[
                  \[
                    \frwdseq{}{d}{d}
                    \quad
                    \[
                      \frwdseq{}{n}{n}
                      \quad
                      \frwdseq{}{n}{n}
                      \justifies
                      \frwdseq{}{n, n}{n \otimes n}
                    \]
                    \justifies
                    \frwdseq{}{n, n, n \otimes n \limp d}{d}
                  \]
                  \justifies
                  \frwdseq{R_1}{n, n}{d}
                \]
                \justifies
                \frwdseq{R_1}{n, d, d, n}{d \otimes d \otimes d}
              \]
              \justifies
              \frwdseq{R_1}{n, d \otimes d \otimes n}{d \otimes d \otimes d}
            \]
            \justifies
            \frwdseq{R_1}{q, n, q \limp d \otimes d \otimes n}{d \otimes d \otimes d}
          \]
          \justifies
          \frwdseq{R_1, R_3}{q, n}{d \otimes d \otimes d}
        \]
        \justifies
        \frwdseq{R_1, R_3}{q \otimes n}{d \otimes d \otimes d}
      \]
      \justifies
      \frwdseq{R_1, R_3}{\cdot}{q \otimes n \limp d \otimes d \otimes d}
    \end{prooftree}
  \]
\end{example}

...

\subsection{Subformula property}

The key technical property that makes the inverse method possible is the
\emph{subformula property}. Logical calculi with the subformula property are
such that derivations only contain subformulas of the conclusion formula. This
property means, in our case, that we only need to consider sequents composed of
subformulas of the goal sequent when searching for a cut-free proof.

We show that such property holds for our calculi by formalizing the idea in
terms of a \emph{subformula relation} for propositions. To do so, we decorate
subformulas with certain marks:

\begin{itemize}
\item \emph{Polarity}, written as a superscript + or - symbol;
\item \emph{Availability}, written as a superscript ! (for ``unrestricted'') or
  . (``linear''). Subformulas of an unrestricted formula \emph{do not} inherit
  the decoration.
\end{itemize}

The availability signs determine whether the formula is allowed to occur in the
unrestricted context, and thus serve as a guide for the copy rule.

\begin{definition}
  A decorated sequent is of the form $\Gamma^-_! ; \Delta^-_. \Longrightarrow
  C^+_.$ .
\end{definition}

\begin{definition}[Decorated subformula relation]
  The decorated subformula relation $\leq$ between decorated propositions is the
  reflexive-transitive closure of the following cases:

  \begin{alignat*}{2}
    & A^{\pm}_{.} \leq (A \otimes B)_a^{\pm} \qquad & B^{\pm}_{.} \leq (A \otimes
    B)_a^{\pm} \\
    & A^{\mp}_{.} \leq (A \limp B)_a^{\pm} & B^{\pm}_{.} \leq (A \limp
    B)_a^{\pm}
  \end{alignat*}
  \[
    A^{\pm}_{.} \leq A^{\pm}_{!}
  \]
\end{definition}

We assume the standard pointwise extension of this relation to sets of decorated
propositions:

\[
  S \leq T \equiv \forall A \in S, \exists B \in T : A \leq B
\]

\begin{definition}
  A decorated sequent $s_1 \equiv \Gamma^-_! ; \Delta^-_. \Longrightarrow C^+_.$
  is a subsequent of the decorated sequent
  $\Gamma'^-_! ; \Delta'^-_. \Longrightarrow C'^+_.$, written $s_1 \leq s_2$ if

  \[
    \subsequent{\Gamma'^-_! \cup \Delta'^-_. \cup C'^+_.}
    {\Gamma^-_! \cup \Delta^-_. \cup C^+_.}
  \]
\end{definition}

\begin{theorem}[Subformula property]
  If $\Gamma'; \Delta' \Longrightarrow C'$ appears in a proof of
  $\Gamma; \Delta \Longrightarrow C$, then

  \[
    \subsequent{\Gamma'; \Delta' \Longrightarrow C'}
    {\Gamma; \Delta \Longrightarrow C}
  \]
\end{theorem}
\begin{proof}
  By straightforward inspection of the rules of the backward calculus. It
  sufficies to observe that, if the following is such a rule:

  \[
    \begin{prooftree}
      s_1 \quad s_2 \quad \dots \quad s_n
      \justifies
      s
    \end{prooftree}
  \]

  then $s_i \leq s$ for all $i \in \{1, \dots, n\}$. The thesis follows by
  transitivity of the subformula relation.
\end{proof}

The same reasoning holds for the forward calculus, where the subsequent relation
is extended to forward sequents in the obvious way.

\begin{corollary}
  If $\Gamma'; \Delta' \longrightarrow C'$ appears in a proof of
  $\Gamma; \Delta \longrightarrow C$, then

  \[
    \subsequent{
      \Gamma'; \Delta' \longrightarrow C'
    }{
      \Gamma; \Delta \longrightarrow C
    }
  \]
\end{corollary}
\begin{proof}
  Immediate.
\end{proof}



%%% Local Variables:
%%% mode: latex
%%% TeX-master: "../docs"
%%% End:
