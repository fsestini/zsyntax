\section{The Inverse Method}

Backwards proof search for linear logic suffers from a resource management
problem, which stems from the lack of structural rules that makes even the
propositional fragment undecidable. One such problem is given by multiplicative
connectives, that exibit rules characterized by \emph{multiplicative
  non-determinism}. Consider, for example, the $\otimes R$ rule

\[
  \begin{prooftree}
    \Delta_1 \Longrightarrow A
    \qquad
    \Delta_2 \Longrightarrow B
    \justifies
    \Delta_1, \Delta_2 \Longrightarrow A \otimes B
  \end{prooftree}
\]

Suppose the rule above is used for backwards proof search. Without weakening,
the search procedure must infer a division of the linear resources of the
conclusion in order to distribute it among the premises. To guarantee
completeness, a backwards proof search procedure must either support
backtracking, or implement resource management in other ways.

Suppose now that the inference rules are read in the forward direction, that is
by using them to derive new sequents from previously established ones.
Multiplicative non-determinism does not exist in a forward reading, as we simply
conjoin the resources of the premises to construct the conclusion.  Proof search
in the forward direction thus starts from some initially known facts (usually,
some instantiation of the axioms), and then repeatedly applies the rules of the
calculus to accumulate new sequents until one of them is found to subsume the
goal (that is, validity of such sequents entails validity of the goal).

Given that in our case the main source of computational complexity is
multiplicative non-determinism, forward proof search seems to be the ideal
candidate to implement an automated theorem prover.  However, one advantage of
backwards proof search that the forward direction lacks is its goal-orientation:
we start from a goal sequent and then break the problem into subgoals, knowing
that validity of the latter implies validity of the former. In other words, a
goal-oriented search like the backwards one only considers sequents that are
directly useful to prove the initial goal. In the forward direction, the
application of the rules to already known sequents may generate new sequents
that are not at all useful to derive, or even related to, the goal. That is,
naive forward search is not goal-directed and therefore very inefficient.

Fortunately, there are ways to make forward reasoning more goal-directed, so
that one can have some of the advantages of backward reasoning which enjoying
the good resourse management properties of forward search.  The particular
forward search strategy we use is the Inverse Method. The Inverse Method is a
very general technique that can be applied to any sequent calculus enjoying the
subformula property. The method works as follows: first, a goal sequent of which
validity is to be decided is fixed. Then, the axioms and inference rules of the
logic are \emph{specialized} to the subformulas of the goal sequent, such that
the principal formula in all inference rules is a subformula of the goal
sequent. The search then proceeds by starting from the axioms and applying these
rules to construct new sequents by matching the premises against previously
derived sequents. New sequents that are not simply instances of sequents derived
earlier are themselves then used in the inference rules to derive newer
sequents. Eventually, assuming the search strategy is complete, either the goal
sequent is derived, or the search space is saturated and the goal sequent is
found to be unprovable.

The fundamental idea behing the inverse method is therefore to specialize the
entire logical calculus to the particular goal sequent, so that the only
sequents that get generated during forward proof search are subsequents of the
goal. This is sufficient to guarantee completeness in calculi with the
subformula property, and it also provides an efficient decision procedure for
calculi such as multiplicative linear logic, where the set of subsequents of any
goal sequent (and thus the search space) is finite.

In our calculus \zss{}, the possibility of introducing axioms in the proof
\emph{ad libitum} represented by the presence of the unrestricted context and
the $\copyrule$ rule ruins the properties that make plain multiplicative
intuitionistic linear logic decidable. In particular, in any derivation there is
no limit to the number of uses of the biological axioms, and since each use may
generate a sequent that is structurally different from all the others, there is
no way to easily deduce a height bound for all the derivations of a goal sequent
from the sequent alone, as it is the case for multiplicative linear logic. It
follows that it is not true, in general, that a naive application of the inverse
method would terminate for every goal sequent. Another interpretation of the
problem is to be found in the fact that the unrestricted context in our sequents
basically corresponds to a form of linear logic ! (``of course'')
exponentiation. Decidability of MELL (multiplicative exponential linear logic)
was an open problem until recently, and has been proved decidable using ideas
that go back to Curry and Kleene by constructing a sequent calculus which
absorbes contraction for exponentiated formulas in the rules of the other
logical connectives \cite{mell-dec}. Of course, the adaptation of those ideas to
a forward focused calculus is outside the scope of this work.

Nevertheless, even in the presence of our unrestricted contexts, the search
space is in most cases finite, exception made for some particularly bad
sequents. Consider, as an example, the following goal sequent

\[
  \bkwseq{A \rightarrow B, B \rightarrow A}{A}{C}
\]

It is of course not provable, but a an attempt to find a backward proof (the
problem is identical in the forward direction) may lead to a loop

\[
  \begin{prooftree}
    \[
      \[
        \justifies
        \bkwseq{A \rightarrow B, B \rightarrow A}{A}{A}
      \]
      \quad
      \[
        \[
          \[
            \justifies
            \bkwseq{A \rightarrow B, B \rightarrow A}{B}{B}
          \]
          \quad
          \[
            \justifies
            \bkwseq{A \rightarrow B, B \rightarrow A}{A}{C}
          \]
          \justifies
          \bkwseq{A \rightarrow B, B \rightarrow A}{B \rightarrow A, B}{C}
        \]
        \justifies
        \bkwseq{A \rightarrow B, B \rightarrow A}{B}{C}
      \]
      \justifies
      \bkwseq{A \rightarrow B, B \rightarrow A}{A \rightarrow B, A}{C}
    \]
    \justifies
    \bkwseq{A \rightarrow B, B \rightarrow A}{A}{C}
  \end{prooftree}
\]

We conjecture that these bad cases are not as frequent as it might seem, so that
it should not be necessary, at least in its early stages, to complicate the
development of a theorem prover to avoid as many such cases as possible. For our
implementation, we just decided to fix an upper bound to the size of the search
space. If the derivation fails due to a violation of this upper bound, the error
can be reported to the user who may then decide to retry with a bigger upper
bound.

\subsection{Forward calculus}

\subsubsection{Focusing}

The key insight of \cite{chaudhuri-thesis} is to use focusing \emph{and} the
inverse method together, by developing a focused forward calculus from which to
extract an efficient proof search procedure.  We again follow Chaudhuri's work
as a model and build a forward calculus for \zss{}. Since the idea is to
implement focusing (and the derived rules) in the forward direction, we do not
develop a forward calculus and then apply focusing techniques to it, but instead
directly modify the backward focused calculus of derived rules defined in the
previous section and adapt it for forward reasoning. The result will be the
forward focused calculus of derived rules that we want.

\subsubsection{Subformula property}

The key technical property that makes the inverse method possible is the
\emph{subformula property}. Logical calculi with the subformula property are
such that derivations only contain subformulas of the conclusion formula. This
means, in our case, that we only need to consider sequents composed of
subformulas of the goal sequent when searching for a cut-free proof.

When we derived the rules of the sequent calculus, we paid particular attention
in putting them in a form that allowed us to have a calculus with subformula
property. We do not show directly that such property holds for our forward
calculus of derived rules, but we limit ourselves to observe that every
derivation in our calculus corresponds to a structurally identical derivation of
the calculus in \cite{chaudhuri-thesis}. It is then easy to convince ourselves
that our calculus has the subformula property.

\subsection{Forward derived rules}

In Section~\ref{sec:focusing} we added a light form of focusing to our backward
calculus that allows us to generate derived rules that speed up the proof search
process. At the beginning of this section we introduced the inverse method,
another technique for automated deduction that is particularly useful in logics
(as the one we are using) with multiplicative connectives since it perform proof
search in the forward direction to eliminate multiplicative non-determinism. As
the final objective is to use forward search and the inverse method in
conjunction with (lightweight) focusing for Zsyntax, we need to adapt the
backward calculus of derived rules of the previous section to the forward
direction. Again, we follow \cite{chaudhuri-thesis} as a model.

\begin{definition}[Forward pre-sequent]
  A forward presequent is a term of the form
  $\zfneuseq{\Gamma}{\Delta}{\zeta}{\gamma}$, where either

  \begin{enumerate}
  \item $\zeta = \gamma = \cdot$, or
  \item $\zeta = \reactlist{}$ and $\gamma = C$, for some reaction list
    $\reactlist{}$ and formula $C$.
  \end{enumerate}
\end{definition}

\begin{definition}[Forward relation]
  Forward relations are inductively defined as follows:

  \begin{enumerate}
  \item $\relj{\frfrel{A}}{\Sigma}{s}$, where $A$ is a formula, $\Sigma$ is a
    list of forward neutral sequents, and $s$ is a forward presequent.
  \item $\relj{\factrel{\zsyseq{\Delta}{\Omega}{\zeta}{\xi}} }{\Sigma}{s}$,
    where $\Delta$, $\Omega$, $\Sigma$ and $s$ are as before, and either
    \begin{enumerate}
    \item $\zeta = \gamma = \cdot$, or
    \item $\zeta = \reactlist{}$ and $\xi = C$, for some reaction list
      $\reactlist{}$ and formula $C$.
    \end{enumerate}
  \end{enumerate}
\end{definition}

Figure~\ref{fig:fwdderrulecalculus} show the calculus to derived forward
relations as judgements. the intuitive interpretation of a forward relation is
the same as the backwards case, with the only difference that now we consider
the forward direction, hence we go from a list of known premise sequents
$\Sigma$ to the conclusion $s$ that follows from them with a single pair of
phases composed (top-down) of an active phase starting from the premises
followed by a focused phase ending in the conclusion sequent.

\begin{figure}[h]
  \begin{mdframed}
    \[
      \begin{prooftree}
        \justifies
        \relj{\frfrel{p}}{\cdot}{\zfneuseqempty{\cdot}{p}}
        \using{\init}
      \end{prooftree}
    \]

    \[
      \begin{prooftree}
        \justifies
        \relj{\frfrel{R}}{\zfneuseq{\Gamma}{\Delta}{\emptyctrl{}}{R}}{
          \zfneuseqempty{\Gamma}{\Delta}}
        \using{\faplus}
      \end{prooftree}
    \]

    \[
      \begin{prooftree}
        \relj{\frfrel{A}}{\Sigma_1}{\zfneuseqempty{\Gamma_1}{\Delta_1}}
        \qquad
        \relj{\frfrel{B}}{\Sigma_2}{\zfneuseqempty{\Gamma_2}{\Delta_2}}
        \justifies
        \relj{\frfrel{A \otimes B}}{\Sigma_1 \cdot \Sigma_2}{
          \zfneuseqempty{\Gamma_1,
            \Gamma_2}{\Delta_1, \Delta_2}}
        \using{\otimes F}
      \end{prooftree}
    \]

    \[
      \begin{prooftree}
        \relj{\factrel{
            \zsyseq{\Delta}{\Omega, A, B}{\zeta}{\xi}}
        }{\Sigma}{s}
        \justifies
        \relj{\factrel{
            \zsyseq{\Delta}{\Omega, A \otimes B}{\zeta}{\xi}}
        }{\Sigma}{s}
        \using{\otimes A}
      \end{prooftree}
      \qquad
      \begin{prooftree}
        \relj{\factrel{\zsyseq{\Delta, P}{\Omega}{\zeta}{\xi}}}{\Sigma}{s}
        \justifies
        \relj{\factrel{\zsyseq{\Delta}{\Omega, P}{\zeta}{\xi}}}{\Sigma}{s}
        \using{\actrule}
      \end{prooftree}
    \]

    \[
      \begin{prooftree}
        \justifies
        \relj{
          \factrel{\zsyseq{\Delta}{\cdot}{\cdot}{\cdot}}
        }{
          \zfneuseq{\Gamma}{\Delta, \Delta'}{\reactlist{}}{C}
        }{
          \zfneuseq{\Gamma}{\Delta'}{\reactlist{}}{C}
        }
        \using{\matchrule}
      \end{prooftree}
    \]

    \[
      \begin{prooftree}
        \justifies
        \relj{
          \factrel{\zsyseq{\Delta}{\cdot}{\reactlist{}}{C}}
        }{
          \zfneuseq{\Gamma}{\Delta, \Delta'}{\reactlist{}}{C}
        }{
          \zfneuseqempty{\Gamma}{\Delta'}
        }
        \using{\matchprimerule}
      \end{prooftree}
    \]

  \end{mdframed}
  \caption{Forward focused relations calculus}
  \label{fig:fwdrulescalculus}
\end{figure}

As in the backward case, we use the forward relations to build a calculus of
derived rules that has neutral (forward) sequents as premises and
conclusions. The full calculus is shown in Figure~\ref{fig:fwdderrulecalculus}.

\begin{figure}[h]
  \begin{mdframed}
\[
  \begin{prooftree}
    s_1 \quad \dots \quad s_n \quad
    (\relj{\frfrel{Q}}{s_1 \dots s_n}{\fneuseq{\Gamma}{\Delta}{\cdot}})
    \justifies
    \zfneuseq{\Gamma}{\Delta}{\emptyctrl{}}{Q}
    \using{\focplusrule}
  \end{prooftree}
\]

\[
  \begin{prooftree}
    \[
      \relj{\frfrel{A}}{s_1, \dots, s_n}{\zfneuseqempty{\Gamma_1}{\Delta_1}}
      \proofdotseparation=1.2ex
      \proofdotnumber=0
      \leadsto
      \relj{\factrel{
            \zsyseq{\cdot}{B}{\cdot}{\cdot}}
        }{s}{\zfneuseq{\Gamma_2}{\Delta_2}{\reactlist{}'}{C}}
    \]
    \quad
    \[
      \resplist{\Delta_2}{\reactlist{}}
      \proofdotseparation=1.2ex
      \proofdotnumber=0
      \leadsto
      s_1 \quad \dots \quad s_n \quad s
    \]
    \justifies
    \zfneuseq{\Gamma_1, \Gamma_2}{\Delta_1, \Delta_2,
      A \rightarrow_{\reactlist{}}^S B}{
      \baseandplus{\Delta_2}{\reactlist{}}{\reactlist{}'}
    }{C}
    \using{\rightarrow L}
  \end{prooftree}
\]

\[
  \begin{prooftree}
    \[
      \relj{\frfrel{A}}{s_1, \dots, s_n}{\zfneuseqempty{\Gamma_1}{\Delta_1}}
      \proofdotseparation=1.2ex
      \proofdotnumber=0
      \leadsto
      \relj{\factrel{
            \zsyseq{\cdot}{B}{\cdot}{\cdot}}
        }{s}{\zfneuseq{\Gamma_2}{\Delta_2}{\reactlist{}'}{C}}
    \]
    \quad
    \[
      \resplist{\Delta_2}{\reactlist{}}
      \proofdotseparation=1.2ex
      \proofdotnumber=0
      \leadsto
      s_1 \quad \dots \quad s_n \quad s
    \]
    \justifies
    \zfneuseq{\Gamma_1, \Gamma_2, A \rightarrow_{\reactlist{}}^{\emptyset} B
    }{\Delta_1, \Delta_2,
    }{
      \baseandplus{\Delta_2}{\reactlist{}}{\reactlist{}'}
    }{C}
    \using{\copyrule}
  \end{prooftree}
\]

\[
  \begin{prooftree}
    s \qquad
    \relj{\factrel{\zsyseq{\cdot}{A}{\reactlist{}}{B}}}{s}{
      \zfneuseqempty{\Gamma}{\Delta}}
    \justifies
    \zfneuseq{\Gamma}{\Delta}{\emptyctrl{}}{A
      \rightarrow_{\reactlist{}}^{\elembases{\Delta}} B}
    \using{\rightarrow R}
  \end{prooftree}
\]
  \end{mdframed}
  \caption{Forward focused derived rule calculus}
  \label{fig:fwdderrulecalculus}
\end{figure}

\begin{definition}
  The sequent $\zfneuseq{\Gamma}{\Delta}{\reactlist{}}{Q}$ is sound if
  $\ztriseq{\Gamma}{\Delta}{\cdot}{\reactlist{}}{Q}$.
\end{definition}

\begin{lemma}\label{fsoundnesslemma}
  If all sequents in $\Sigma$ are sound, then

  \begin{enumerate}
  \item If $\relj{\frfrel{A}}{\Sigma}{\zfneuseqempty{\Gamma}{\Delta}}$,
    then, $\zfocseq{\Gamma}{\Delta}{A}$;
  \item If
    $\relj{\factrel{\zsyseq{\Delta}{\Omega}{\cdot}{\cdot}}}{\Sigma}
    {\zfneuseq{\Gamma}{\Delta'}{\reactlist{}}{C}}$, then
    $\ztriseq{\Gamma}{\Delta, \Delta'}{\Omega}{\reactlist{}}{C}$;
  \item If
    $\relj{\factrel{\zsyseq{\Delta}{\Omega}{\reactlist{}}{C}}}{\Sigma}
    {\zfneuseqempty{\Gamma}{\Delta'}}$, then
    $\ztriseq{\Gamma}{\Delta, \Delta'}{\Omega}{\reactlist{}}{C}$.
  \end{enumerate}
\end{lemma}
\begin{proof}
  All points proved by simultaneous induction on the height of the
  derivation. We distinguish cases on the last rule used in the derivation:

  \begin{itemize}
  \item Case $\init$. Immediate.

  \item Case $\faplus{}$.

    \[
      \begin{prooftree}
        \justifies
        \relj{\frfrel{R}}{\zfneuseq{\Gamma}{\Delta}{\emptyctrl{}}{R}}{
          \zfneuseqempty{\Gamma}{\Delta}}
        \using{\faplus}
      \end{prooftree}
    \]

    By hypothesis $\zfneuseq{\Gamma}{\Delta}{\emptyctrl{}}{R}$ is sound, therefore
    $\ztriseq{\Gamma}{\Delta}{\cdot}{\emptyctrl{}}{R}$. Then,

    \[
      \begin{prooftree}
        \ztriseq{\Gamma}{\Delta}{\cdot}{\emptyctrl{}}{R}
        \justifies
        \zfocseq{\Gamma}{\Delta}{R}
        \using{blur}
      \end{prooftree}
    \]

  \item Case $\otimes F$.

    \[
      \begin{prooftree}
        \relj{\frfrel{A}}{\Sigma_1}{\fneuseq{\Gamma_1}{\Delta_1}{\cdot}}
        \qquad
        \relj{\frfrel{B}}{\Sigma_2}{\fneuseq{\Gamma_2}{\Delta_2}{\cdot}}
        \justifies
        \relj{\frfrel{A \otimes B}}{\Sigma_1 \cdot \Sigma_2}{\fneuseq{\Gamma_1,
            \Gamma_2}{\Delta_1, \Delta_2}{\cdot}}
        \using{\otimes F}
      \end{prooftree}
    \]

    Then, by inductive hypothesis, weakening and $\otimes R$:

    \[
      \begin{prooftree}
        \[
          \zfocseq{\Gamma_1}{\Delta_1}{A}
          \justifies
          \zfocseq{\Gamma_1, \Gamma_2}{\Delta_1}{A}
        \]
        \qquad
        \[
          \zfocseq{\Gamma_2}{\Delta_2}{B}
          \justifies
          \zfocseq{\Gamma_1, \Gamma_2}{\Delta_2}{B}
        \]
        \justifies
        \zfocseq{\Gamma_1, \Gamma_2}{\Delta_1, \Delta_2}{A \otimes B}
      \end{prooftree}
    \]

  \item Case $\otimes A$ and act: we thesis follows immediately by inductive
    hypothesis and an application of the corresponding rule of the focused calculus.

  \item Case $\matchrule$ and $\matchprimerule$: the thesis follows immediately
    from the hypothesis that $\Sigma$ are sound.
  \end{itemize}
\end{proof}

\begin{theorem}[Soundness]
  If $\zfneuseq{\Gamma}{\Delta}{\reactlist{}}{Q}$, then it is sound.
\end{theorem}
\begin{proof}
  By induction on the height of the derivation of
  $\zfneuseq{\Gamma}{\Delta}{\reactlist{}}{Q}$.

  \begin{enumerate}
  \item Case $focus$.
    If $Q$ is an implication, the rule is just an identity. Otherwise, by
    Lemma~\ref{fsoundnesslemma}, $\zfocseq{\Gamma}{\Delta}{Q}$. The thesis
    follows by an application of the $focus$ rule.

  \item Case $\rightarrow L$.
    The thesis follows by Lemma~\ref{fsoundnesslemma}, weakening, and
    $\rightarrow L$ of the focused calculus.

  \item Case $\copyrule$ is the same as $\rightarrow L$, with the addition of a
    concluding $\copyrule$ rule of the focused calculus.

  \item Case $\rightarrow R$.
    The thesis follows easily by Lemma~\ref{fsoundnesslemma}, and an application
    of $\rightarrow R$ of the focused calculus.

  \end{enumerate}
\end{proof}

Completeness is established with respect to the backward calculus of derived
rules. All proofs go through smoothly as expected, since the forward calculus is
structurally identical to the backward one.

\begin{definition}[Stronger forms]
  A forward sequent $\zfneuseq{\Gamma}{\Delta}{\zeta}{\gamma}$ is said to be
  stronger than a backward sequent
  $\zbneuseq{\Gamma}{\Delta}{\zeta}{\gamma}$, written as a relation
  $\preceq$, if $\Gamma \subseteq \Gamma'$, where $\zeta = \cdot$ or
  $\reactlist{}$ for some $\reactlist{}$, and $\gamma = \cdot$ or $C$ for some $C$.
\end{definition}

The above relation is assumed to be extended point-wise to ordered sequences of
sequents.

\begin{lemma}\label{fdercompllemma}
  \begin{enumerate}
  \item If $\brfrelj{A}{s}{\Sigma}$ and there exists a derivable sequence of
    sequents $\Sigma' \preceq \Sigma$, then $\frfrelj{A}{\Sigma'}{s'}$ for some
    $s' \preceq s$;
  \item If $\bactrelj{\zsyseq{\Delta}{\Omega}{\zeta}{\xi}}{s}{\Sigma}$ and
    there exists a derivable sequence $\Sigma' \preceq \Sigma$, then
    $\factrelj{\zsyseq{\Delta}{\Omega}{\zeta}{\xi}}{\Sigma'}{s'}$ for some
    $s' \preceq s$.
  \end{enumerate}
\end{lemma}
\begin{proof}
  All points proved simultaneously by an easy induction on the height of the
  backward relation derivation. We sketch a few cases below.

  \begin{enumerate}
  \item Case $\faplus$. Then

    \[
      \begin{prooftree}
        \justifies
        \relj{\brfrel{R}}{\bkwseq{\Gamma}{\Delta}{\cdot}}{\zsyseq{\Gamma}{\Delta}{\emptyctrl{}}{R}}
        \using{FA^+}
      \end{prooftree}
    \]

    Suppose there are $\Sigma' \preceq \Sigma$.
    Then, it must be that $\Sigma' \equiv
    \zfneuseq{\Gamma'}{\Delta}{\emptyctrl{}}{R}$, with $\Gamma' \subseteq \Gamma$.
    But then

    \[
      \begin{prooftree}
        \justifies
        \relj{\frfrel{R}}{\zfneuseq{\Gamma'}{\Delta}{\emptyctrl{}}{R}}{
          \zfneuseqempty{\Gamma'}{\Delta}}
        \using{\faplus}
      \end{prooftree}
    \]

    with $s' \equiv \zfneuseqempty{\Gamma'}{\Delta}$, which clearly satisfies
    the thesis as wanted.

  \item Case $\otimes A$.
    Suppose $\Sigma' \preceq \Sigma$. Then, the thesis follows by inductive
    hypothesis and an application of $\otimes A$ in the forward derived rule
    calculus.

  \item The other cases are treated similarly.
  \end{enumerate}
\end{proof}

\begin{theorem}[Completeness]
  If $\zbneuseq{\Gamma}{\Delta}{\reactlist{}}{Q}$, then
  $\zfneuseq{\Gamma'}{\Delta}{\reactlist{}}{Q}$ for some
  $\Gamma' \subseteq \Gamma$.
\end{theorem}
\begin{proof}
  Easy induction on the derivation of
  $\zbneuseq{\Gamma}{\Delta}{\reactlist{}}{Q}$ and application of
  Lemma~\ref{fdercompllemma} on the premises.
\end{proof}

%%% Local Variables:
%%% mode: latex
%%% TeX-master: "../docs"
%%% End:
