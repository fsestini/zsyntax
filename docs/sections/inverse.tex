\section{The Inverse Method}

\subsection{Rationale}

... TODO. .... rationale for a forward calulus...

... TODO ... description of the inverse method...


.... blah blah we need a forward calculus .... blah...

... blah we don't use the plain forward calculus directly, blah..

% \subsection{Forward sequent calculus}

% Since weakening and contraction are admissible for the unrestricted
% zone, we can treat $\Gamma$ as a set.

% \begin{figure}[h]
%   \begin{mdframed}
%     \[
%       \begin{prooftree}
%         \justifies
%         \zfwdseq{\cdot}{P}{\emptyset}{P}
%         \using{\init}
%       \end{prooftree}
%       \qquad \qquad
%       \begin{prooftree}
%         \zfwdseq{\Gamma}{\Delta, A}{\ctrlset{}}{C}
%         \justifies
%         \zfwdseq{\Gamma \cup \{A\}}{\Delta}{\ctrlset{}}{C}
%         \using{\copyrule}
%       \end{prooftree}
%     \]

%     \[
%       \begin{prooftree}
%         \Gamma; \Delta \longrightarrow A
%         \qquad
%         \Gamma'; \Delta' \longrightarrow B
%         \justifies
%         \Gamma\cup \Gamma'; \Delta, \Delta' \longrightarrow A \otimes B
%         \using{\otimes R}
%       \end{prooftree}
%       \qquad \qquad
%       \begin{prooftree}
%         \Gamma; \Delta, A, B \longrightarrow C
%         \justifies
%         \Gamma; \Delta, A \otimes B \longrightarrow C
%         \using{\otimes L}
%       \end{prooftree}
%     \]

%     \[
%       \begin{prooftree}
%         \Gamma; \Delta \longrightarrow A
%         \qquad
%         \Gamma'; \Delta', B \longrightarrow C
%         \justifies
%         \Gamma \cup \Gamma'; \Delta, \Delta', A \limp B \longrightarrow C
%         \using{\limp L}
%       \end{prooftree}
%       \qquad \qquad
%       \begin{prooftree}
%         \Gamma; \Delta, A \longrightarrow B
%         \justifies
%         \Gamma; \Delta \longrightarrow A \limp B
%         \using{\limp R}
%       \end{prooftree}
%     \]
%   \end{mdframed}
%   \caption{Forward sequent calculus}
% \end{figure}

% \begin{theorem}[Soundness]
%   If $\Gamma ; \Delta \longrightarrow C$, then
%   $\Gamma ; \Delta \Longrightarrow C$.
% \end{theorem}
% \begin{proof}
%   Just notice that every forward derivation is also a backward derivation, with
%   the difference that in the latter unrestricted contexts are copied in all
%   the premises. The details are a straightforward induction on the height of a
%   forward derivation, plus weakening for the backward calculus.
% \end{proof}

% \begin{theorem}[Completeness]
%   If $\Gamma ; \Delta \Longrightarrow C$, then
%   $\Gamma' ; \Delta \longrightarrow C$ for some $\Gamma' \subseteq \Gamma$.
% \end{theorem}
% \begin{proof}
%   Straightforward induction on the backward derivation.
% \end{proof}

% \begin{example}
%   The following is an example derivation of

%   $$\frwdseq{R_1, R_3}{\cdot}{q \otimes n \limp d \otimes d \otimes d}$$

%   where $R_1 \equiv n \otimes n \limp d$ and
%   $R_3 \equiv q \limp d \otimes d \otimes n$.

%   \[
%     \begin{prooftree}
%       \[
%         \[
%           \[
%             \frwdseq{\cdot}{q}{q}
%             \[
%               \[
%                 \[
%                   \frwdseq{}{d}{d}
%                   \quad
%                   \frwdseq{}{d}{d}
%                   \justifies
%                   \frwdseq{}{d, d}{d \otimes d}
%                 \]
%                 \quad
%                 \[
%                   \[
%                     \frwdseq{}{d}{d}
%                     \quad
%                     \[
%                       \frwdseq{}{n}{n}
%                       \quad
%                       \frwdseq{}{n}{n}
%                       \justifies
%                       \frwdseq{}{n, n}{n \otimes n}
%                     \]
%                     \justifies
%                     \frwdseq{}{n, n, n \otimes n \limp d}{d}
%                   \]
%                   \justifies
%                   \frwdseq{R_1}{n, n}{d}
%                 \]
%                 \justifies
%                 \frwdseq{R_1}{n, d, d, n}{d \otimes d \otimes d}
%               \]
%               \justifies
%               \frwdseq{R_1}{n, d \otimes d \otimes n}{d \otimes d \otimes d}
%             \]
%             \justifies
%             \frwdseq{R_1}{q, n, q \limp d \otimes d \otimes n}{d \otimes d \otimes d}
%           \]
%           \justifies
%           \frwdseq{R_1, R_3}{q, n}{d \otimes d \otimes d}
%         \]
%         \justifies
%         \frwdseq{R_1, R_3}{q \otimes n}{d \otimes d \otimes d}
%       \]
%       \justifies
%       \frwdseq{R_1, R_3}{\cdot}{q \otimes n \limp d \otimes d \otimes d}
%     \end{prooftree}
%   \]
% \end{example}

% ...

\subsection{Forward calculus}

\subsubsection{Subformula property}

The key technical property that makes the inverse method possible is the
\emph{subformula property}. Logical calculi with the subformula property are
such that derivations only contain subformulas of the conclusion formula. This
means, in our case, that we only need to consider sequents composed of
subformulas of the goal sequent when searching for a cut-free proof. Since the
number of such sequents is necessarily finite, a proof search procedure, like the
one we are going to implement, that only considers subsequents of the goal
sequent is guaranteed to terminate.

We do not show directly that such property holds for our calculus, but we limit
ourselves to observe that every derivation in the Zsyntax forward calculus above
corresponds to a structurally identical derivation in intuitionistic linear
logic as in [cmu thesis]. It is then easy to convince ourselves that our
calculus has the subformula property.

\subsection{Forward derived rules}

In Section [TODO] we added a light form of focusing to our backward calculus
that allows us to generate derived rules that speed up the proof search process.
In this section we introduced the inverse method, another technique for
automated deduction that is particularly useful in logics (as the one we are
using) with multiplicative connectives since it perform proof search in the
forward direction to eliminate multiplicative non-determinism. As the final
objective is to use forward search and the inverse method in conjunction with
(lightweight) focusing for Zsyntax, we need to adapt the backward calculus of
derived rules of the previous section to the forward direction. Again, we follow
[cmu thesis] as a model.

\begin{definition}[Forward pre-sequent]
  A forward presequent is a term of the form
  $\zfneuseq{\Gamma}{\Delta}{\zeta}{\gamma}$, where either

  \begin{enumerate}
  \item $\zeta = \gamma = \cdot$, or
  \item $\zeta = \ctrlset{}$ and $\gamma = C$, for some control set $\ctrlset{}$
    and formula $C$.
  \end{enumerate}
\end{definition}

\begin{definition}[Forward relation]
  Forward relations are inductively defined as follows:

  \begin{enumerate}
  \item $\relj{\frfrel{A}}{\Sigma}{s}$, where $A$ is a formula, $\Sigma$ is a
    list of forward neutral sequents, and $s$ is a forward presequent.
  \item $\relj{\factrel{\zsyseq{\Delta}{\Omega}{\zeta}{\xi}} }{\Sigma}{s}$,
    where $\Delta$, $\Omega$, $\Sigma$ and $s$ are as before, and either
    \begin{enumerate}
    \item $\zeta = \gamma = \cdot$, or
    \item $\zeta = \ctrlset{}$ and $\xi = C$, for some control set $\ctrlset{}$
      and formula $C$.
    \end{enumerate}
  \end{enumerate}
\end{definition}

Figure~\ref{fig:fwdderrulecalculus} show the calculus to derived forward
relations as judgements. As now we consider the forward direction, the intuitive
interpretation of a forward relation is now to go from a list of known premise
sequents $\Sigma$ to the conclusion $s$ that follows from them according to how
the relation has been derived.

\begin{figure}[h]
  \begin{mdframed}
    \[
      \begin{prooftree}
        \justifies
        \relj{\frfrel{p}}{\cdot}{\zfneuseqempty{\cdot}{p}}
        \using{\init}
      \end{prooftree}
    \]

    \[
      \begin{prooftree}
        \justifies
        \relj{\frfrel{R}}{\zfneuseq{\Gamma}{\Delta}{\emptyset}{R}}{
          \zfneuseqempty{\Gamma}{\Delta}}
        \using{\faplus}
      \end{prooftree}
    \]

    \[
      \begin{prooftree}
        \relj{\frfrel{A}}{\Sigma_1}{\zfneuseqempty{\Gamma_1}{\Delta_1}}
        \qquad
        \relj{\frfrel{B}}{\Sigma_2}{\zfneuseqempty{\Gamma_2}{\Delta_2}}
        \justifies
        \relj{\frfrel{A \otimes B}}{\Sigma_1 \cdot \Sigma_2}{
          \zfneuseqempty{\Gamma_1,
            \Gamma_2}{\Delta_1, \Delta_2}}
        \using{\otimes F}
      \end{prooftree}
    \]

    \[
      \begin{prooftree}
        \relj{\factrel{
            \zsyseq{\Delta}{\Omega, A, B}{\zeta}{\xi}}
        }{\Sigma}{s}
        \justifies
        \relj{\factrel{
            \zsyseq{\Delta}{\Omega, A \otimes B}{\zeta}{\xi}}
        }{\Sigma}{s}
        \using{\otimes A}
      \end{prooftree}
      \qquad
      \begin{prooftree}
        \relj{\factrel{\zsyseq{\Delta, P}{\Omega}{\zeta}{\xi}}}{\Sigma}{s}
        \justifies
        \relj{\factrel{\zsyseq{\Delta}{\Omega, P}{\zeta}{\xi}}}{\Sigma}{s}
        \using{\actrule}
      \end{prooftree}
    \]

    \[
      \begin{prooftree}
        \justifies
        \relj{
          \factrel{\zsyseq{\Delta}{\cdot}{\cdot}{\cdot}}
        }{
          \zfneuseq{\Gamma}{\Delta, \Delta'}{\ctrlset{}}{C}
        }{
          \zfneuseq{\Gamma}{\Delta'}{\ctrlset{}}{C}
        }
        \using{\matchrule}
      \end{prooftree}
    \]

    \[
      \begin{prooftree}
        \justifies
        \relj{
          \factrel{\zsyseq{\Delta}{\cdot}{\ctrlset{}}{C}}
        }{
          \zfneuseq{\Gamma}{\Delta, \Delta'}{\ctrlset{}}{C}
        }{
          \zfneuseqempty{\Gamma}{\Delta'}
        }
        \using{\matchprimerule}
      \end{prooftree}
    \]

  \end{mdframed}
  \caption{Forward focused relations calculus}
  \label{fig:fwdrulescalculus}
\end{figure}

As in the backward case, we use the forward relations to build a calculus of
derived rules that has neutral (forward) sequents as premises and
conclusions. Rules are again generated from focusable formulas, implications and
axioms. The full calculus is shown in Figure~\ref{fig:fwdderrulecalculus}.

\begin{figure}[h]
  \begin{mdframed}
\[
  \begin{prooftree}
    s_1 \quad \dots \quad s_n \quad
    (\relj{\frfrel{Q}}{s_1 \dots s_n}{\fneuseq{\Gamma}{\Delta}{\cdot}})
    \justifies
    \zfneuseq{\Gamma}{\Delta}{\emptyset}{Q}
    \using{\focplusrule}
  \end{prooftree}
\]

\[
  \begin{prooftree}
    \[
      \relj{\frfrel{A}}{s_1, \dots, s_n}{\zfneuseqempty{\Gamma_1}{\Delta_1}}
      \proofdotseparation=1.2ex
      \proofdotnumber=0
      \leadsto
      \relj{\factrel{
            \zsyseq{\cdot}{B}{\cdot}{\cdot}}
        }{s}{\zfneuseq{\Gamma_2}{\Delta_2}{\ctrlset{}'}{C}}
    \]
    \quad
    \[
      \respects{\Delta_2}{\ctrlset{}}
      \proofdotseparation=1.2ex
      \proofdotnumber=0
      \leadsto
      s_1 \quad \dots \quad s_n \quad s
    \]
    \justifies
    \zfneuseq{\Gamma_1, \Gamma_2}{\Delta_1, \Delta_2,
      A \rightarrow_{\ctrlset{}}^S B}{\ctrlset{}\cup\ctrlset{}'}{C}
    \using{\rightarrow L}
  \end{prooftree}
\]

\[
  \begin{prooftree}
    \[
      \relj{\frfrel{A}}{s_1, \dots, s_n}{\zfneuseqempty{\Gamma_1}{\Delta_1}}
      \proofdotseparation=1.2ex
      \proofdotnumber=0
      \leadsto
      \relj{\factrel{
            \zsyseq{\cdot}{B}{\cdot}{\cdot}}
        }{s}{\zfneuseq{\Gamma_2}{\Delta_2}{\ctrlset{}'}{C}}
    \]
    \quad
    \[
      \respects{\Delta_2}{\ctrlset{}}
      \proofdotseparation=1.2ex
      \proofdotnumber=0
      \leadsto
      s_1 \quad \dots \quad s_n \quad s
    \]
    \justifies
    \zfneuseq{\Gamma_1, \Gamma_2, A \rightarrow_{\ctrlset{}}^{\emptyset} B
    }{\Delta_1, \Delta_2,
      }{\ctrlset{}\cup\ctrlset{}'}{C}
    \using{\copyrule}
  \end{prooftree}
\]

\[
  \begin{prooftree}
    s \qquad
    \relj{\factrel{\zsyseq{\cdot}{A}{\ctrlset{}}{B}}}{s}{
      \zfneuseqempty{\Gamma}{\Delta}}
    \justifies
    \zfneuseq{\Gamma}{\Delta}{\emptyset}{A
      \rightarrow_{\ctrlset{}}^{\elembases{\Delta}} B}
    \using{\rightarrow R}
  \end{prooftree}
\]
  \end{mdframed}
  \caption{Forward focused derived rule calculus}
  \label{fig:fwdderrulecalculus}
\end{figure}

\begin{definition}
  The sequent $\zfneuseq{\Gamma}{\Delta}{\ctrlset{}}{Q}$ is sound if
  $\ztriseq{\Gamma}{\Delta}{\cdot}{\ctrlset{}}{Q}$.
\end{definition}

\begin{lemma}\label{fsoundnesslemma}
  If all sequents in $\Sigma$ are sound, then

  \begin{enumerate}
  \item If $\relj{\frfrel{A}}{\Sigma}{\zfneuseqempty{\Gamma}{\Delta}}$,
    then, $\zfocseq{\Gamma}{\Delta}{A}$;
  \item If
    $\relj{\factrel{\zsyseq{\Delta}{\Omega}{\cdot}{\cdot}}}{\Sigma}
    {\zfneuseq{\Gamma}{\Delta'}{\ctrlset{}}{C}}$, then
    $\ztriseq{\Gamma}{\Delta, \Delta'}{\Omega}{\ctrlset{}}{C}$;
  \item If
    $\relj{\factrel{\zsyseq{\Delta}{\Omega}{\ctrlset{}}{C}}}{\Sigma}
    {\zfneuseqempty{\Gamma}{\Delta'}}$, then
    $\ztriseq{\Gamma}{\Delta, \Delta'}{\Omega}{\ctrlset{}}{C}$.
  \end{enumerate}
\end{lemma}
\begin{proof}
  All points proved by simultaneous induction on the height of the
  derivation. We distinguish cases on the last rule used in the derivation:

  \begin{itemize}
  \item Case $\init$. Immediate.

  \item Case $\faplus{}$.


    \[
      \begin{prooftree}
        \justifies
        \relj{\frfrel{R}}{\zfneuseq{\Gamma}{\Delta}{\emptyset}{R}}{
          \zfneuseqempty{\Gamma}{\Delta}}
        \using{\faplus}
      \end{prooftree}
    \]

    By hypothesis $\zfneuseq{\Gamma}{\Delta}{\emptyset}{R}$ is sound, therefore
    $\ztriseq{\Gamma}{\Delta}{\cdot}{\emptyset}{R}$. Then,

    \[
      \begin{prooftree}
        \ztriseq{\Gamma}{\Delta}{\cdot}{\emptyset}{R}
        \justifies
        \zfocseq{\Gamma}{\Delta}{R}
        \using{blur}
      \end{prooftree}
    \]

  \item Case $\otimes F$.

    \[
      \begin{prooftree}
        \relj{\frfrel{A}}{\Sigma_1}{\fneuseq{\Gamma_1}{\Delta_1}{\cdot}}
        \qquad
        \relj{\frfrel{B}}{\Sigma_2}{\fneuseq{\Gamma_2}{\Delta_2}{\cdot}}
        \justifies
        \relj{\frfrel{A \otimes B}}{\Sigma_1 \cdot \Sigma_2}{\fneuseq{\Gamma_1,
            \Gamma_2}{\Delta_1, \Delta_2}{\cdot}}
        \using{\otimes F}
      \end{prooftree}
    \]

    Then, by inductive hypothesis, weakening and $\otimes R$:

    \[
      \begin{prooftree}
        \[
          \zfocseq{\Gamma_1}{\Delta_1}{A}
          \justifies
          \zfocseq{\Gamma_1, \Gamma_2}{\Delta_1}{A}
        \]
        \qquad
        \[
          \zfocseq{\Gamma_2}{\Delta_2}{B}
          \justifies
          \zfocseq{\Gamma_1, \Gamma_2}{\Delta_2}{B}
        \]
        \justifies
        \zfocseq{\Gamma_1, \Gamma_2}{\Delta_1, \Delta_2}{A \otimes B}
      \end{prooftree}
    \]

  \item Case $\otimes A$ and act: we thesis follows immediately by inductive
    hypothesis and an application of the corresponding rule of the focused calculus.

  \item Case $\matchrule$ and $\matchprimerule$: the thesis follows immediately
    from the hypothesis that $\Sigma$ are sound.
  \end{itemize}
\end{proof}

\begin{theorem}[Soundness]
  If $\zfneuseq{\Gamma}{\Delta}{\ctrlset{}}{Q}$, then it is sound.
\end{theorem}
\begin{proof}
  By induction on the height of the derivation of
  $\zfneuseq{\Gamma}{\Delta}{\ctrlset{}}{Q}$.

  \begin{enumerate}
  \item Case $focus$.
    If $Q$ is an implication, the rule is just an identity. Otherwise, by
    Lemma~\ref{fsoundnesslemma}, $\zfocseq{\Gamma}{\Delta}{Q}$. The thesis
    follows by an application of the $focus$ rule.

  \item Case $\rightarrow L$.
    The thesis follows by Lemma~\ref{fsoundnesslemma}, weakening, and
    $\rightarrow L$ of the focused calculus.

  \item Case $\copyrule$ is the same as $\rightarrow L$, with the addition of a
    concluding $\copyrule$ rule of the focused calculus.

  \item Case $\rightarrow R$.
    The thesis follows easily by Lemma~\ref{fsoundnesslemma}, and an application
    of $\rightarrow R$ of the focused calculus.

  \end{enumerate}
\end{proof}

Completeness is established with respect to the backward calculus of derived
rules. All proofs go through smoothly as expected, since the forward calculus is
structurally identical to the backward one.

\begin{definition}[Stronger forms]
  A forward sequent $\zfneuseq{\Gamma}{\Delta}{\zeta}{\gamma}$ is said to be
  stronger than a backward sequent
  $\zbneuseq{\Gamma}{\Delta}{\zeta}{\gamma}$, written as a relation
  $\preceq$, if $\Gamma \subseteq \Gamma'$, where $\zeta = \cdot$ or
  $\ctrlset{}$ for some $\ctrlset{}$, and $\gamma = \cdot$ or $C$ for some $C$.
\end{definition}

The above relation is assumed to be extended point-wise to ordered sequences of
sequents.

\begin{lemma}\label{fdercompllemma}
  \begin{enumerate}
  \item If $\brfrelj{A}{s}{\Sigma}$ and there exists a derivable sequence of
    sequents $\Sigma' \preceq \Sigma$, then $\frfrelj{A}{\Sigma'}{s'}$ for some
    $s' \preceq s$;
  \item If $\bactrelj{\zsyseq{\Delta}{\Omega}{\zeta}{\xi}}{s}{\Sigma}$ and
    there exists a derivable sequence $\Sigma' \preceq \Sigma$, then
    $\factrelj{\zsyseq{\Delta}{\Omega}{\zeta}{\xi}}{\Sigma'}{s'}$ for some
    $s' \preceq s$.
  \end{enumerate}
\end{lemma}
\begin{proof}
  All points proved simultaneously by an easy induction on the height of the
  backward relation derivation. We sketch a few cases below.

  \begin{enumerate}
  \item Case $\faplus$. Then

    \[
      \begin{prooftree}
        \justifies
        \relj{\brfrel{R}}{\bkwseq{\Gamma}{\Delta}{\cdot}}{\zsyseq{\Gamma}{\Delta}{\emptyset}{R}}
        \using{FA^+}
      \end{prooftree}
    \]

    Suppose there are $\Sigma' \preceq \Sigma$.
    Then, it must be that $\Sigma' \equiv
    \zfneuseq{\Gamma'}{\Delta}{\emptyset}{R}$, with $\Gamma' \subseteq \Gamma$.
    But then

    \[
      \begin{prooftree}
        \justifies
        \relj{\frfrel{R}}{\zfneuseq{\Gamma'}{\Delta}{\emptyset}{R}}{
          \zfneuseqempty{\Gamma'}{\Delta}}
        \using{\faplus}
      \end{prooftree}
    \]

    with $s' \equiv \zfneuseqempty{\Gamma'}{\Delta}$, which clearly satisfies
    the thesis as wanted.

  \item Case $\otimes A$.
    Suppose $\Sigma' \preceq \Sigma$. Then, the thesis follows by inductive
    hypothesis and an application of $\otimes A$ in the forward derived rule
    calculus.

  \item The other cases are treated similarly.
  \end{enumerate}
\end{proof}

\begin{theorem}[Completeness]
  If $\zbneuseq{\Gamma}{\Delta}{\ctrlset{}}{Q}$, then
  $\zfneuseq{\Gamma'}{\Delta}{\ctrlset{}}{Q}$ for some
  $\Gamma' \subseteq \Gamma$.
\end{theorem}
\begin{proof}
  Easy induction on the derivation of $\zbneuseq{\Gamma}{\Delta}{\ctrlset{}}{Q}$ and
  application of Lemma~\ref{fdercompllemma} on the premises.
\end{proof}



%%% Local Variables:
%%% mode: latex
%%% TeX-master: "../docs"
%%% End:
