\section{The Inverse Method}

\subsection{Multiplicative non-determinism}

... TODO. .... rationale for a forward calulus...

... TODO ... description of the inverse method...


.... blah blah we need a forward calculus .... blah...

... blah we don't use the plain forward calculus directly, blah..

% \subsection{Forward sequent calculus}

% Since weakening and contraction are admissible for the unrestricted
% zone, we can treat $\Gamma$ as a set.

% \begin{figure}[h]
%   \begin{mdframed}
%     \[
%       \begin{prooftree}
%         \justifies
%         \zfwdseq{\cdot}{P}{\emptyset}{P}
%         \using{\init}
%       \end{prooftree}
%       \qquad \qquad
%       \begin{prooftree}
%         \zfwdseq{\Gamma}{\Delta, A}{\ctrlset{}}{C}
%         \justifies
%         \zfwdseq{\Gamma \cup \{A\}}{\Delta}{\ctrlset{}}{C}
%         \using{\copyrule}
%       \end{prooftree}
%     \]

%     \[
%       \begin{prooftree}
%         \Gamma; \Delta \longrightarrow A
%         \qquad
%         \Gamma'; \Delta' \longrightarrow B
%         \justifies
%         \Gamma\cup \Gamma'; \Delta, \Delta' \longrightarrow A \otimes B
%         \using{\otimes R}
%       \end{prooftree}
%       \qquad \qquad
%       \begin{prooftree}
%         \Gamma; \Delta, A, B \longrightarrow C
%         \justifies
%         \Gamma; \Delta, A \otimes B \longrightarrow C
%         \using{\otimes L}
%       \end{prooftree}
%     \]

%     \[
%       \begin{prooftree}
%         \Gamma; \Delta \longrightarrow A
%         \qquad
%         \Gamma'; \Delta', B \longrightarrow C
%         \justifies
%         \Gamma \cup \Gamma'; \Delta, \Delta', A \limp B \longrightarrow C
%         \using{\limp L}
%       \end{prooftree}
%       \qquad \qquad
%       \begin{prooftree}
%         \Gamma; \Delta, A \longrightarrow B
%         \justifies
%         \Gamma; \Delta \longrightarrow A \limp B
%         \using{\limp R}
%       \end{prooftree}
%     \]
%   \end{mdframed}
%   \caption{Forward sequent calculus}
% \end{figure}

% \begin{theorem}[Soundness]
%   If $\Gamma ; \Delta \longrightarrow C$, then
%   $\Gamma ; \Delta \Longrightarrow C$.
% \end{theorem}
% \begin{proof}
%   Just notice that every forward derivation is also a backward derivation, with
%   the difference that in the latter unrestricted contexts are copied in all
%   the premises. The details are a straightforward induction on the height of a
%   forward derivation, plus weakening for the backward calculus.
% \end{proof}

% \begin{theorem}[Completeness]
%   If $\Gamma ; \Delta \Longrightarrow C$, then
%   $\Gamma' ; \Delta \longrightarrow C$ for some $\Gamma' \subseteq \Gamma$.
% \end{theorem}
% \begin{proof}
%   Straightforward induction on the backward derivation.
% \end{proof}

% \begin{example}
%   The following is an example derivation of

%   $$\frwdseq{R_1, R_3}{\cdot}{q \otimes n \limp d \otimes d \otimes d}$$

%   where $R_1 \equiv n \otimes n \limp d$ and
%   $R_3 \equiv q \limp d \otimes d \otimes n$.

%   \[
%     \begin{prooftree}
%       \[
%         \[
%           \[
%             \frwdseq{\cdot}{q}{q}
%             \[
%               \[
%                 \[
%                   \frwdseq{}{d}{d}
%                   \quad
%                   \frwdseq{}{d}{d}
%                   \justifies
%                   \frwdseq{}{d, d}{d \otimes d}
%                 \]
%                 \quad
%                 \[
%                   \[
%                     \frwdseq{}{d}{d}
%                     \quad
%                     \[
%                       \frwdseq{}{n}{n}
%                       \quad
%                       \frwdseq{}{n}{n}
%                       \justifies
%                       \frwdseq{}{n, n}{n \otimes n}
%                     \]
%                     \justifies
%                     \frwdseq{}{n, n, n \otimes n \limp d}{d}
%                   \]
%                   \justifies
%                   \frwdseq{R_1}{n, n}{d}
%                 \]
%                 \justifies
%                 \frwdseq{R_1}{n, d, d, n}{d \otimes d \otimes d}
%               \]
%               \justifies
%               \frwdseq{R_1}{n, d \otimes d \otimes n}{d \otimes d \otimes d}
%             \]
%             \justifies
%             \frwdseq{R_1}{q, n, q \limp d \otimes d \otimes n}{d \otimes d \otimes d}
%           \]
%           \justifies
%           \frwdseq{R_1, R_3}{q, n}{d \otimes d \otimes d}
%         \]
%         \justifies
%         \frwdseq{R_1, R_3}{q \otimes n}{d \otimes d \otimes d}
%       \]
%       \justifies
%       \frwdseq{R_1, R_3}{\cdot}{q \otimes n \limp d \otimes d \otimes d}
%     \end{prooftree}
%   \]
% \end{example}

% ...

\subsection{Subformula property}

The key technical property that makes the inverse method possible is the
\emph{subformula property}. Logical calculi with the subformula property are
such that derivations only contain subformulas of the conclusion formula. This
means, in our case, that we only need to consider sequents composed of
subformulas of the goal sequent when searching for a cut-free proof. Since the
number of such sequents is necessarily finite, a proof search procedure, like the
one we are going to implement, that only considers subsequents of the goal
sequent is guaranteed to terminate.

We do not show directly that such property holds for our calculus, but we limit
ourselves to observe that every derivation in the Zsyntax forward calculus above
corresponds to a structurally identical derivation in intuitionistic linear
logic as in [cmu thesis]. It is then easy to convince ourselves that our
calculus has the subformula property.

%%% Local Variables:
%%% mode: latex
%%% TeX-master: "../docs"
%%% End:
