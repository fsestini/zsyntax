\section{Further work}

The following sections point out multiple directions towards which the work
presented here could be extended.

\subsection{Restoring the original Z-conditional operator}

In Section~\ref{sec:zsyntax} we explained how the original Z-conditional of
\cite{adding-logic} contains an implicit existential quantification: the
introduction of a $A \rightarrow B$ formula witnesses the fact that \emph{there
  exists some} transition from $A$ to $B$, with certain characteristics, whereas
the elimination rule allows one to use a conditional formula $A \rightarrow B$
provided the surrounding context respects \emph{all known instances} of the
biological transition represented by such formula.

The logic presented here takes a different approach, in the sense that the
conditional operator with implicit quantification is dropped in favor of one
that is equipped with explicit information about the deduction that was used to
establish it. As we explained in Section 2, it is possible to satisfactorily
reintroduce a connective with the same semantics of the one given in
\cite{adding-logic} by keeping the conditional introduced here and existentially
quantifying over elementary bases and reaction lists.

As we have seen, the quantifier-free fragment that we presented and that has
been implemented in the proving tool is already interesting and quite useful
alone. Nevertheless, the addition of quantifiers to it undoubtedly configures
itself as one of the most interesting extensions that could be done in the
future.

\subsection{Higher-order conditionals}

Even though from a proof-theoretic point of view there are no limitations on the
level of nesting of conditional formulas, the current implementation disallows
the user to specify higher-order conditionals in the goal sequents.  The reason
is that our theorem prover was designed with the idea that users should not be
required to specify, or even known anything about, reaction lists or any other
monotonicity control facilities when using the tool.

It follows that, without quantifiers, it is not clear how to use higher-order
conditionals in a way that is clear and intuitive to the user.  \footnote{Notice
  that we consider implication formulas in the linear context of a goal sequent
  as ``higher-order'' conditionals, as
  $\Gamma, A \rightarrow B \vdash C \simeq \Gamma \vdash (A \rightarrow B)
  \rightarrow C$.}  As an example, consider the following goal sequent issued by
a user: $\Gamma, A \rightarrow B \Longrightarrow \Delta$. In order to establish
the validity of this sequent, the prover has to decorate the conditional
$A \rightarrow B$ with an elementary base and a reaction list, since this is how
the underlying logic is defined. However, it is not clear how to do this
decoration in a way that corresponds with the intuitive meaning that the user
assigns to the specified conditional, which is to test if the sequent is
derivable for \emph{some} instance of $A \rightarrow B$.  Therefore, the goal
sequent above would have been best specified as
$\Gamma, \exists l e . (A \rightarrow_l^e B) \Longrightarrow \Delta$.

Consider now another example goal sequent,
$\Gamma, (A \rightarrow B) \rightarrow C \Longrightarrow \Delta$.
Again, the user probably wanted to represent the possibility to obtain a $C$
given the possibility to go from $A$ to $B$ in \emph{some} way, as it would be
if the conditional was given the original interpretation of \cite{adding-logic}.
And again, without quantifiers we are instead forced to assign a particular
elementary base and reaction list to $A \rightarrow B$, yielding a formula that
represents a transition from \emph{that} particular instance of $A \rightarrow
B$ to $C$, which is not quite was the user intended to give.
A more appropriate translation of this goal sequent would have been, again,
something like
$\Gamma, \exists e' l' . ((\exists e l . (A \rightarrow_l^e B)) \rightarrow_{l'}^{e'} C) \Longrightarrow \Delta$.

In conclusion, it is our opinion that to be able to use higher-order
conditionals in a way that reflects the usual
intuition of the user in the most accurate way, quantifiers are needed in the
logic and the corresponding implementation.
It should be noted, however, that a theorem prover with a restricted use of
conditionals like the one we implemented is still expressive enough for many
interesting cases. In particular, all examples of using Zsyntax to encode
biochemical pathways as deductions given in \cite{2010paper},
\cite{adding-logic}, \cite{melanoma} do not require higher-order conditional and
have been easily formalized and checked in our implementation.

\subsection{Remaining connectives}

The logic of Zsyntax was originally devised \cite{2010paper} with conjunction
and implication connectives, and then extended \cite{adding-logic} with a
conjunctive unit, an additive conjunction to express external choice, and an
additive disjunction operator to express internal choice.

The logic presented here relates to the fragment of Zsyntax with multiplicative
conjunction and implication only. These two connectives alone are already
expressive enough to encode many useful situations, let alone all example use
cases of Zsyntax syntax in the literature (\cite{2010paper},
\cite{melanoma}). Moreover, this fragment already includes the most interesting
and powerful connective of Zsyntax, namely the Z-conditional, so it is worth
studying by itself.

We therefore choose to implement this fragment as it is sufficiently expressive
to be useful in practice, sufficiently representative of the whole concept of
Zsyntax and controlled monotonicity to represent a meaningful treatise of how an
automated proof search procedure for Zsyntax could be implemented, and
sufficiently small to be manageable in the small time available.

Given that the main challenges regarding the implementation of Zsyntax are given
by the conditional operator, we speculate that the extension of the present work
to the remaining connectives, again following \cite{chaudhuri-thesis} as a
model, should be rather straightforward. This is because the additional
connectives are not different from their linear logic counterparts in any
fundamental way, so already existing literature on the subject can be leveraged
with little effort.

%%% Local Variables:
%%% mode: latex
%%% TeX-master: "../docs"
%%% End:
