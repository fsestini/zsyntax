\section{Further work}

The following sections point out multiple directions towards which the work
presented here could be extended. Our opinion is that most of them involve
technical rather than conceptual difficulties in their realization.

\subsection{Restoring the original Z-conditional operator}

In Section~\ref{sec:zsyntax} we explained how the original Z-conditional of
\cite{adding-logic} contains an implicit existential quantification: the
introduction of a $A \rightarrow B$ formula witnesses the fact that \emph{there
  exists some} transition from $A$ to $B$, with certain characteristics, whereas
the elimination rule allows one to use a conditional formula $A \rightarrow B$
provided the surrounding context respects \emph{all known instances} of the
biological transition represented by such formula. The logic presented
here takes a different approach, in the sense that the conditional operator with
implicit quantification is dropped in favor of one that is equipped with
explicit information about the deduction that was used to establish it.

As already explained, this alternative conditional is more powerful than the one
in \cite{adding-logic}, an obvious consequence of the fact that it carries more
information. To restore the original Z-conditional, and thus the possibility of
expressing formulas and proofs in the original language of Zsyntax, it is
sufficient to restore what our conditional dropped: the existential
quantification. That is, the original Z-conditional can be obtained from ours by
just existentially quantifying the data regarding elementary bases and reactions
lists. In this setting, our alternative conditional can be considered as a
predicate over terms of type ``elementary base'' and ``reaction list'', where
these terms can of course be arbitrary variables that can be bound by
quantifiers.

\[
  A \rightarrow B \quad \equiv \quad \exists e l . A \rightarrow_l^e B
\]

Then, the original introduction rule becomes a derived rule:

\[
  \begin{prooftree}
    \[
      \[
        \Gamma, A
        \leadsto
        B
      \]
      \qquad
      \Gamma, \Delta
      \justifies
      A \rightarrow_{\mathrm{L}}^{\elembases{\Gamma}} B, \Delta
    \]
    \justifies
    \exists s l . A \rightarrow_{l}^{s} B, \Delta
    \quad \equiv \quad
    A \rightarrow B, \Delta
  \end{prooftree}
\]

where $\mathrm{L}$ represents the concrete reaction list associated to the
deduction $\Gamma, A \vdash B$.
In order to understand how this new conditional, formulated as a predicate,
interacts with the other rules and relates to the original Z-conditional
operator, we must extend the definition of elementary base to account for the
presence of free variables. To start with

\[
  \elembases{\exists x . A} = \elembases{A}
\]

As free variables represent a kind of metalinguistic universal quantification, a
formula $A \rightarrow^e B$ represents a transition with an arbitrary elementary
base, so that any use of it must work out correctly for any possible
substitution of $e$ for a concrete elementary base.  As our logical system is
meant to reflect a world characterized by a dynamically growing knowledge, it
makes sense to restrict universal quantification to the domain that it known
\emph{at that current moment}. Under this interpretation, a natural definition
of elementary base for $A \rightarrow^e B$ if one that considers the elementary
bases for all formulas of that type that are known at the moment. This turns out
to correspond to the definition given in \cite{adding-logic}

\[
  \elembases{A \rightarrow^e B} =
  \{ \Delta \, | \, A \rightarrow^{\Delta} B \ \text{is known to hold} \}
\]

the correspondence is not perfect, given that \cite{adding-logic} considers when
$\Delta, A \models B$ is known to hold, rather than $A \rightarrow^{\Delta} B$.
Our definition does make sense, however: if $\Delta, A \models B$ holds, then
surely $A \rightarrow^{\Delta} B$ is valid.

Consider now how an elimination rule could be
derived, using our definition of the original Z-conditional as an existentially
quantified one:

\[
  \begin{prooftree}
    \Gamma, \exists l . A \rightarrow_l B, A
    \, \equiv \,
    \Gamma, A \rightarrow B, A
    \qquad
    %\resplist{\Gamma}{l}
    (\star)
    \quad
    \[
      \Gamma, A \rightarrow_l B, A
      \qquad
      \resplist{\Gamma}{l}
      \justifies
      \Gamma, B
      \using{\rightarrow \mathcal{E}}
    \]
    \justifies
    \Gamma, B
    \using{\exists \mathcal{E}}
  \end{prooftree}
\]

where $l$ is intended to be a fresh free variable that gets discharged as
part of the existential elimination rule.
To have an elimination rule that corresponds to the original one given in
\cite{adding-logic}, we have to define a side-condition, indicated
above as $(\star)$, that is logically equivalent (or sufficiently similar) to
the original one, and that entails $\resplist{\Gamma}{l}$.
The real connection between our interpretation and the original Z-conditional
thus crucially depends on the interpretation of $\resplist{\Gamma}{l}$ when $l$
is a free variable.

Continuing on the same line of reasoning as with elementary bases, to prove
something from $A \rightarrow_l B$ with $l$ free is to prove it for any
possible substitution of $l$ to non-variable
reaction lists $\mathrm{list}$, such that
$A \rightarrow_{\mathrm{list}} B$ is known to hold at the moment the
proof is done.
Then

\[
  (\star) \equiv
  \resplist{\Gamma}{\mathfrak{L}_{A \rightarrow B}} \equiv
  \text{for all known } A \rightarrow_{\mathrm{list}} B,
  \resplist{\Gamma}{\mathrm{list}}
\]

and similarly, extending the $\resplist{}{}$ relation to

\[
  \resplist{\Gamma}{l} \equiv
  \text{for all known } A \rightarrow_{\mathrm{list}} B,
  \resplist{\Gamma}{\mathrm{list}}
\]

when $A \rightarrow_l B$ and $l$ is a free variable. Notice that this is a
positive definition, in the sense that it considers what is currently known
about the system to give an interpretation of variables.
For this reason, it is different from the one given in \cite{adding-logic},
which is negative. In particular, the following is allowed:

\[
  \begin{prooftree}
    \Gamma, A \rightarrow B, A
    \justifies
    \Gamma, B
  \end{prooftree}
\]

only when $\Gamma$ is \emph{not known} to inhibit the reaction, that is,
when it is not true that $\Gamma, A \not \models B$ is known.




Under this treatment of existentials and free variables, it is
easy to see that our interpretation is very similar to the original Z-conditional
operator.

TODO: extend definition of elementary bases to ex. qu. formulas: (if e is a
variable, consider elem bases of all known transitions)

We use ``similar'' here because ...

- Positive notion, instead of negative notion as in [paper]
- Positive notion is more in line with the definition of elem bases: we consider
  as elem base the union of elem bases of all known transitions, so it makes
  sense to consider $l$ by substituting for all known react lists.

- It is not clear how the user should provide information on the contexts that
disallow a transition, regarding the treatment of free variables. The best
option is to make the machine compute it, by taking the union of the control
sets of all knwon deductions of a particular transition, and requiring that the
context is not in it. But this actually turns out to correspond to asking
provability for all known substitutions... So we could just stick with it...



The extension of the logic presented above with free variables and quantifiers
is mainly a tiresome rather than conceptually challenging work, as it should
only be an adaptation of [cmu thesis]. The only difficulty has to do with how to
compute the domain over which free variables range during a certain
proof. Consider a principal cut involving an existentially quantified
conditional that is first introduced and then eliminated:

\[
  \begin{prooftree}
    \[
      \Gamma \Longrightarrow A \rightarrow_{\mathrm{L}}^{\mathrm{E}} B
      \justifies
      \Gamma \Longrightarrow \exists e l . A \rightarrow_l^e B
    \]
    \qquad
    \[
      \Delta, A \rightarrow_l^e B \Longrightarrow C
      \justifies
      \Delta, \exists e l . A \rightarrow_l^e B \Longrightarrow C
    \]
    \justifies
    \Gamma, \Delta \Longrightarrow C
  \end{prooftree}
\]

where $\mathrm{E}, \mathrm{L}$ just stand for arbitrary non-variable elementary
bases and reaction lists. To be able to move the cut to the premises, the
premise on the right must have been derived with a domain of free variables that
\strong{includes} knowledge about the transition $A \rightarrow B$ that is proved in
the left premise. This requires us to carefully assign an explicit temporal
order interpretation to branches of a derivation, as the validity of parts of it
may crucially depend on facts that are established in others that are intended
to come before. This difficulties of course also arise when considering cuts of
a Z-conditional in the original sense of \cite{adding-logic}, since it stems
from the intrinsic existential quantification. The only difference is that in
our case the existential quantification is made explicit.

At this point, we could ask ourselves if it would have been better to just
consider the original Z-conditional with implicit quantification, and shape the
calculus around it. Firstly, notice that this would not have resulted in a
simpler logic. Getting to a satisfactory definition and implementation of the
Z-conditional operator \emph{is} an inevitably difficult task; this work just
exposes why it is so, it does not introduce any difficulty in itself.

Secondly, by hiding the quantification in the conditional, we would have removed
useful expressive power from the logic.  A transition $A \rightarrow B$ given as
an axiom is, for example, fundamentally different from the same transition
obtained after a long proof involving several intermediate reactions. The
ability to differentiate between such apperently identical biological types
yields a more expressive language, capable of proving facts that would have been
unprovable otherwise, as explained in Section~\ref{sec:zsyntax}.

The approach proposed here is thus based on the observation that it is better to
start with a more informative conditional operator (as the one presented here),
and then \emph{selectively} generalizing it only when really needed and with
very little cost, for example by quantifying the elementary base and reaction
list data on the conditional opearator to restore the semantics of the original
Z-conditional. Conversely, it is much more expensive to try to recover
information that we failed to express in the first place.

\subsection{Remaining connectives}

The logic of Zsyntax was originally devised [2010 paper] with conjunction and
implication connectives, and then extended \cite{adding-logic} with a
conjunctive unit, an additive conjunction to express external choice, and an
additive disjunction operator to express internal choice.

The logic presented here relates to the fragment of Zsyntax with multiplicative
conjunction and implication only. These two connectives alone are already
expressive enough to encode many useful situations, let alone all example use
cases of Zsyntax syntax in the literature ([2010 paper], [melanoma
paper]). Moreover, this fragment already includes the most interesting and
powerful connective of Zsyntax, namely the Z-conditional, so it is worth
studying by itself.

We therefore choose to implement this fragment as it is sufficiently expressive
to be useful in practice, sufficiently representative of the whole concept of
Zsyntax and controlled monotonicity to represent a meaningful treatise of how an
automated proof search procedure for Zsyntax could be implemented, and
sufficiently small to be manageable in the small time available.

Given that the main challenges regarding the implementation of Zsyntax are given
by the conditional operator, we speculate that the extension of the present work
to the remaining connectives, again following [cmu thesis] as a model, should be
rather straightforward. This is because the additional connectives are not
different from their linear logic counterparts in any fundamental way, so
already existing literature on the subject can be leveraged with little effort.

%%% Local Variables:
%%% mode: latex
%%% TeX-master: "../docs"
%%% End:
