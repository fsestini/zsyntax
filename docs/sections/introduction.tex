\section{Introduction}

\subsection{Zsyntax}

The main aim of \cite{adding-logic} is to suggest that logic can play a
significant role in the toolbox of molecular biology, particularly in the
systematic formalization of certain molecular mechanisms characterized by
state transitions, that can therefore sensibly be described as ``inferential''.
The article shows how biochemical pathways, i.e., transitions from a molecular
aggregate to another molecular aggregate, can be viewed as deductive processes.

As already pointed out by Girard \cite{ll-syntax-semantics}, state transition
systems, like the ones considered in biochemistry, can be effectively described
by linear logic.  In essence, by representing chemical equations as axioms, the
notion of linear consequence corresponds to the notion of accessible state from
an initial one. Moreover, the complete description of the intermediate state
transitions can easily be extracted from the deduction itself.
In \cite{adding-logic}, the authors follow Girard's suggestion and outline a
correspondence between linear deductions and (bio)chemical reactions, observing,
however, that a modification is needed in order to satifactorily represent
the kind of state transitions that are typical of biomolecular models.

In order to represent the transition from a molecular aggregate to another
molecular aggregate as a deductive process, a logical system called Zsyntax is
used. The formal language of Zsyntax is inductively defined from a set of atomic
formulas (defined in \cite{adding-logic} as the bonding language
$\bioformulas$), combined by some linear logic-inspired operators:

\begin{enumerate}
\item aggregative Z-conjunction $\otimes$ (which corresponds to multiplicative
  conjunction)
\item Z-conditional $\rightarrow$ (which corresponds to linear implication)
\item selective Z-conjunction $\wedge$ (which corresponds to additive
  conjunction)
\item Z-disjunction $\vee$ (which corresponds to additive disjunction)
\item a unit formula $\top$ (which corresponds to the multiplicative conjunction
  unit)
\end{enumerate}

Given the meaning of Zsyntax operators (thoroughly described in
\cite{adding-logic}), formulas in the resulting Zsyntax language correspond
to suitable types of molecular aggregates. A proof of $\Gamma \models \Delta$
then represents the biochemical pathway from the aggregate $\Gamma$ to the
aggregate $\Delta$, where $\models$ is defined in \cite{adding-logic} as the
analogous of a logical consequence relation. Proofs are built according to a
logical calculus formulated in natural deduction style.

The main difference between Zsyntax and linear logic lies in the
non-monotonicity of the logical consequence relation (and therefore of the
linear implication operator, here called Z-conditional). To clarify,
monotonicity means that validity of a formula $A \rightarrow B$ implies validity
of $A \otimes C \rightarrow B \otimes C$ for any $C$. This is at odds with the
fact that many important state transitions of interest in molecular biology are
context-sensitive: a reaction may not take place depending on the molecular
context in which it occurs.

The logical system described by the authors is therefore one that considers
context-sensitive state transitions: a transition may take place in every
molecular context that satisfies its control condition, which is a device used
to describe all contexts that instead inhibit such transition.
Control conditions are expressed in the formalism of Zsyntax as
\emph{control sets}, which are intended to be empirically determined (i.e., they
result from empirical knowledge obtained in the laboratory). The immediate
consequence is that their content typically changes over time, resulting in a
logical system that is \emph{open}: theorems may lose their status depending on
modification of the empirical knowledge.

\subsection{Automated deduction}

The objective of the work presented in this report is the development of an
automated theorem prover for (a suitable fragment of) the calculus of Zsyntax
presented in \cite{adding-logic}. The work required a study of the
proof-theoretic properties of the logic, a survey of the already existing body
of knowledge concerning automated deduction for similar logics (and in
particular linear logic), and the adaptation of procedures from the literature
to this specific case.

The logic that has been implemented does not correspond to the full calculus
presented in \cite{adding-logic}, but it is a fragment that has been selected to
be sufficiently expressive to be practically useful, sufficiently representative
of the cornerstone concepts of Zsyntax (namely, controlled monotonicity and
openness of the system), and enough compact to be manageable in the short time
available. In retrospective, this choice proved to be quite appropriate, as the
sole study and implementation of the aforementioned fragment already took
several weeks to be completed.

The development presented here followed a simple approach. A new logical
calculus for Zsyntax based on intuitionistic linear logic was defined,
specifically with proof search in mind. In particular, this new logic was
obtained by first studying the properties that differentiate Zsyntax from plain
linear logic, and then adding them to a linear logic sequent calculus in the
form of annotations.  This yielded a calculus with derivations that are
structurally identical (modulo annotations) to the ones of linear logic,
allowing us to (more or less easily) exploit already existing automated
deduction procedures. Annotations could then be used as a way to tune, or
restrict, the inference engine to account for the biological constraints.

\subsection{Main references and related work}

The main original controbution is to be found in the analysis of Zsyntax of
Section 2, and the definition of the logical foundation of the theorem prover in
the form of a sequent calculus given in Section 3.  The remaining sections deal
with the efficient implementation of an automated deduction procedure for the
calculus of Section 3, and are almost entirely based on the work of K. Chaudhuri
(\cite{chaudhuri-paper}, \cite{chaudhuri-thesis}). With his work, collected in
his PhD thesis, he shows how to combime two well-known automated reasoning
techniques for linear logic, namely focusing \cite{andreoli92} and the inverse
method \cite{inverse}, into a \emph{focused inverse method} for linear logic.

Chaudhuri's work is targeted to the full intuitionistic linear logic, so it was
possible to adapt most of the material in \cite{chaudhuri-thesis}
straightforwardly, since the logic described here and implemented in the proof
assistant can be seen as a small (and much simpler) fragment of propositional
intuitionistic linear logic. The original parts presented here concern the
addition of control set-based facilities to a standard backward sequent calculus
and the other calculi of \cite{chaudhuri-thesis}, and the modification of parts
of to correctly account for the order-sensitiveness of Zsyntax deductions.

As already mentioned, the whole work was an attempt to implement (a suitable
part of) the logic presented in \cite{adding-logic}.  Other articles by the same
authors provided useful reference, in particular \cite{non-mono} is where the
idea of adding annotations to sequents comes from, as well as the formalism used
for control sets.  Contrary to \cite{adding-logic}, and therefore the work
presented here, \cite{non-mono} does not consider implication operators, but
instead represents empirical knowledge as additional axioms to be added to the
formal system. It also presents a more powerful conjunction, capable of
expressing when two aggregates are in the same linear context but do not
interact with each other.

\subsection{Outline}

Section 2 gives some reasons why Zsyntax, as given in \cite{adding-logic}, is
not entirely satisfactory when it comes to machine implementation. It then
proceeds by defining an alternative formulation that is very much related but
subtly different, settting the foundation for the rest of the work.

Section 3 gives a cut-free sequent calculus translation of the logic of Section
2, and establishes soundness and completeness with respect to it. A proof of cut
admissibility is also given, which is required for the completeness proof to go
through.

Section 4 presents the idea of focusing as a way to improve proof search and
make it more computationally feasible by removing inessential
non-determinism. It then adapts the idea to our case, thus developing a focused
sequent calculus based on the standard one given in the previous section.

Section 5 presents the idea of forward reasoning as an alternative to the
more-frequently-used backward reasoning, and draws from the literature to show
how such alternative approach can be very effective to speed up inference for
linear logic and related formalisms, like the one used here.  It then again
takes \cite{chaudhuri-thesis} as a reference to represent the focused calculus
of the preceding section in the forward direction.

Section 6 explains the benefits of the forward focused calculus developed in the
previous sections, and gives some details on how to go from the definition of
the calculus to an implementation of it.

Section 7 gives some implementation details on the search procedure, based on
the forward focused calculus, and on the implemented theorem prover.

Section 8 presents a notion of derivation terms borrowed from
\cite{chaudhuri-thesis}, which can be useful as an alternative internal
representation of derivations as first class objects.

Section 9 concludes with some remarks on various ways in which the work
presented here could be extended in the future.

%%% Local Variables:
%%% mode: latex
%%% TeX-master: "../docs"
%%% End:
