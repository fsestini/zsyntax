\section{Zsyntax}

\subsection{Introduction}

... TODO ...

\subsection{Controlled monotonicity}

The property that separates the calculus of Zsyntax from plain linear logic is
its concept of controlled monotonicity. In linear logic, implication is
monotonic since $A \vdash B$ implies $A \otimes C \vdash B \otimes C$ for any
$C$. Zsyntax preserves the characteristics of linear logic as a logic of
resources, but has a non-monotonic logical consequence relation, or rather one
where such monotonicity is normally allowed apart from some specified cases. In
this sense it is similar to other formalisms such as default logic [TODO add
reference]. In the following we only repeat some of the definitions in [paper]
on which the following sections built upon. Details and examples on how Zsyntax
deals with monotonicity can be found in [paper].



\begin{definition}[Elementary bases]
  \begin{enumerate}
  \item An elementary base of a formula $A$ is a context
    $\Gamma \in \bioformulas^*$ defined inductively as follows:

    \begin{enumerate}
    \item If $A \in \bioformulas$, then $\Gamma = A$;
    \item If $A \equiv B \rightarrow C$, then $\Gamma$ is an elementary base of
      $A$ whenever $\Gamma, A \models B$ is known;
    \item If $A \equiv B \otimes C$, then $\Gamma = \Delta, \Delta'$ is an
      elementary base of $A$ whenever $\Delta$ and $\Delta'$ are, respectively,
      an elementary base for $A$ and $B$.
    \end{enumerate}

  \item Denote by $\elembases{A} \in \mathcal{P}(\bioformulas^*)$ the set of all
    elementary bases for the formula $A$;
  \item Given a Z-state $\Gamma = A_1, \dots, A_n$, let $\elembases{\Gamma}$ be
    the set of all Z-states $\Delta_1, \dots, \Delta_n$ such that $\Delta_i$ is
    in $\elembases{A}$ for all $i$.
  \end{enumerate}
\end{definition}

Notice that, given any context $\Gamma$, the set $\elembases{\Gamma}$ may change
over time as new theorems of the form $\Delta, A \models B$ are discovered.
Also notice that the definition of elementary base is not complete until we
define precisely what it means to \emph{know} that $\Gamma, A \models B$ (for
example, is this knowledge provided manually by the user or managed
automatically by the machine?). We leave this decision to the implementation
details.

[paper] then goes on to define control sets for conditionals $A \rightarrow B$
in terms of elementary bases, calling them \emph{elementary control sets}.  In
what follows, $\bioformulas^*$ denotes as usual the set of strings of formulas
of the bonding language (which are just Z-states formed by formulas of the
bonding language only.)

\begin{definition}
  The \emph{elementary control set}
  $\ctrlset{A \rightarrow B} \in \mathcal{P}(\bioformulas^*)$ associated with
  the conditional $A \rightarrow B$ can be defined as follows:

  \[
    \ctrlset{A \rightarrow B}^* = \{
    \Gamma \in \bioformulas^* \, | \, \exists \Delta \in \elembases{A \rightarrow
    B}
    \quad : \quad \Gamma, \Delta, A \not \models \Gamma, B \text{ is known}
    \}
  \]
\end{definition}

Again, this definition is incomplete unless we specify what it means to know
that $\Gamma, \Delta, A \not \models \Gamma, B$. Nevertheless, under this
definition, the controlled $\rightarrow$ elimination rule becomes the following:

\[
  \begin{prooftree}
    \Gamma, A \rightarrow B \, : \, \Gamma^* \cap \ctrlset{A\rightarrow B}^* =
    \emptyset
    \justifies
    \Gamma, B
  \end{prooftree}
\]

\subsection{Reasons for a different monotonicity control}

In the sections that follow, we will define an alternative provability relation
$\models$ for Zsyntax (which is just a different way to present the natural
deduction-style calculus as in [paper]), that is strictly related to the
one in [paper]. The purpose is to give a precise formalization of the
logic that we intend to implement, and at the same time establish the
similarities (and differences) with the original formulation.

Our provability relation , as well as the sequent calculus based on it, uses a
different kind of Z-conditional (linear implication) that the one described in
[paper], and a different (weaker) definition of control set.  The reason for the
first is logical, and lies in an asymmetry between introduction and elimination
rules of the conditional operator as presended in [paper]. The second has more
technical reasons, as it seemed necessary in order to get a satisfactory
cut-free sequent calculus.

\subsubsection{Conditional operator}

We now recall the definition of the Z-conditional operator in [paper], and
illustrate how the one used here is different. The introduction rule for the
Z-conditional is given in [paper] as follows:

\[
  \begin{prooftree}
    \Gamma, \Delta \qquad
    \[
      \Gamma, A
      \leadsto
      B
    \]
    \justifies
    A \rightarrow B, \Delta
  \end{prooftree}
\]

This rule introduces a particular \emph{instance} of $A \rightarrow B$.  Even
though the conclusion of the rule shows a generic forula $A \rightarrow B$, the
reason we have such a formula is because we know that a particular aggregate
$\Gamma$, with its particular elementary base, when paired with $A$ has a
transition to $B$. This transition (which is just a proof in the natural
deduction calculus) of course depends of the fact that the context in which this
transition is placed respects all control sets involved in it.

This introduction rule forgets all these details, as it unifies all proofs (or
reactions) of the form $\Gamma, A \vdash B$ under the type $A \rightarrow B$.
To be fair, information is not actually thrown away, as it is supposed to be
remembered as part of the control sets and elementary bases associated with the
type $A \rightarrow B$. This however does not solve the asymmetry between the
two rules. In fact, consider the elimination rule given in [paper]:

\[
  \begin{prooftree}
    \Gamma, A \rightarrow B \, : \, \elembases{\Gamma} \cap \ctrlset{A\rightarrow B}^* =
    \emptyset
    \justifies
    \Gamma, B
  \end{prooftree}
\]

From the definition of control set in [paper], we have that the deduction above
is allowed only if the context $\Gamma$ ``respects'' the control set
$\ctrlset{A\rightarrow B}^*$, in the sense that it is forbidden if $\Gamma$
is known to block \emph{some instance} of $A \rightarrow B$, that is, $\Gamma,
\Delta, A \not \models B$ is known for \emph{some} $\Delta$.

To see how this definition clashes with the introduction rule, consider the
following example. Suppose we know that $\Gamma, A \vdash B$ in any possible
context. In other words, the control set that we may want to associate with this
reaction is empty, as it is always allowed to happen. Then, intuitively, it
should be possible to derive the following:

\[
  \begin{prooftree}
    \[
      \Gamma, A, \Delta \qquad
      \[
        \Gamma, A
        \leadsto
        B
      \]
      \justifies
      A \rightarrow B, A, \Delta
      \using{\rightarrow\mathcal{I}}
    \]
    \justifies
    B, \Delta
    \using{\rightarrow\mathcal{E}}
  \end{prooftree}
\]

as we know that the aggregate of type $A \rightarrow B$ is really $\Gamma$ is
disguise, therefore we also know that $\Delta$ does not interact with its
reaction with $A$.

Suppose now that we know that $\Delta, \nabla, A \not \models B$ for some
$\nabla$. In Zsyntax, this is sufficient to block the validity of the deduction
above, since $\Delta$ would then violate the control set for $A \rightarrow B$
(which refers to \emph{all} known transitions of type $A \rightarrow B$, and so
also the one involving $\nabla$.) That is, a \emph{global property} of the
biological aggregates of type $A \rightarrow B$, referenced by the elimination
rule, prevents us to deduce a theorem that we certainly know to be
\emph{locally} possible, because we know from the introduction rule that the
particular instance of $A \rightarrow B$ that we are eliminating comes form
$\Gamma$, which is not really affected by the presence of $\Delta$.

If we translate the calculus into sequents as is, this asymmetry between local
instances and global properties, introduction and elimination rules, and thus
between right and left sequent rules, inevitably prevents an elegant cut-free
calculus to be found. To restore the symmetry, we decided to use a different
kind of implication, one that keeps explicit track of its control set and
elementary base. We write

\[
  A \limp_{\ctrlset{}}^{S} B
\]

to mean the type of a biological aggregate with elementary base $S$ that, if
paired with an element of type $A$, reacts to give an element of type $B$. In
other words, to know that $A \limp_{\ctrlset{}}^{S} B$ holds is to know that
$\Gamma, S, A \models \Gamma, B$, for any $\Gamma$ that respects the control set
$\ctrlset{}$.

Notice that, even though this new definitions allow us to restore some form of
symmetry between left and right rules for the conditional operator, the
resulting logic is still intrinsically non-commutative, as the success of a
derivation depends on the order in which inference rules are used. This is of
course a property that we want our logical system to have, if deductions are to
represent biological reactions. The price to be paid is in the resulting proof
theory, which gets more complicated and lacks other properties that more
standard logics have. As an example, despite what is said in [paper], the
deduction theorem fails to hold even in plain Zsyntax. In particular

\[
  \Gamma \vdash A \rightarrow B \not\implies \Gamma, A \vdash B
\]

As an example, consider $\Gamma \equiv A, A \rightarrow B, B \otimes B
\rightarrow C$, where $A \rightarrow B$ is only allowed to be used if $B$ is not
in the context. Then, $\Gamma \vdash B \rightarrow C$ is easily proved by first
eliminating $A \rightarrow B$ and then introducting $B$ to prove the conclusion.
However, there is no way to prove $\Gamma, B \vdash C$, as the presence of $B$
prevents $A \rightarrow B$ to be used.

The introduction of our modified conditional operator requires a new
definition of elementary base for a formula and for a (multi)set of formulas.

\begin{definition}[Elementary base]
  The elementary base $\elembases{A} \subseteq \bioformulas$ of a formula $A$ is
  inductively defined as follows:

  \begin{enumerate}
  \item $\elembases{A} = \{ A \}$, if $A \in \bioformulas$;
  \item $\elembases{A \otimes B} = \elembases{A} \cup \elembases{B}$;
  \item $\elembases{A \rightarrow_{\ctrlset{}}^S B} = S$.
  \end{enumerate}
\end{definition}

\subsubsection{Control sets}

As we said above, our weaker (with respect to the one given in [paper]) notion
of control sets has purely technical reasons. In particular, in order for the
proof of cut elimination to go through, we need the following fact about control
sets to hold:

\[
  \forall \Delta_1, \Delta_2, \ctrlset{},\quad
  \Delta_1 \notin \ctrlset{} \wedge \Delta_2 \notin \ctrlset{}
  \implies
  \Delta_1, \Delta_2 \notin \ctrlset{}
\]

where $\Delta_1, \Delta_2$ is the usual multiset union. This is of course not
true in general, at least for the notion of control set given in [paper].  We
speculate, however, that our contrained notion is still enough to cover many
interesting applications in molecular biology, so to justify the development of
an automated theorem prover based on it.

The constraint above basically forces the following concrete definition of
control set upon us:

\begin{definition}
  \begin{enumerate}
  \item A control set is a subset $\ctrlset{} \subseteq \bioformulas$;
  \item A (multi)set of formulas $\Delta$ \emph{respects} a control set
    $\ctrlset{}$, written $\respects{\Delta}{\ctrlset{}}$, if

    \[
      \forall A \in \ctrlset{}, A \notin \elembases{\Delta}
    \]

  \end{enumerate}
\end{definition}

\subsection{The calculus \eznd{}}

% As we have seen, the logical framework of Zsyntax is dynamic in the sense that
% knowledge about biochemical reactions changes over time as facts are
% discovered. In other terms, this means that the contents of elementary bases
% and control sets that we consider as part of our system change.

% To give a precise definition of proof (in natural deduction style) or derivation
% (in sequent calculus), however, we need a way to refer to a particular
% \emph{current state} of the system, that is, to the current value of the
% elementary bases and control sets. We thus give the concept of a \emph{world},
% which is just a structure that describes the state of the system in a given
% point in time.

% \begin{definition}[World]
%   A \emph{world assignment} is a map of type
%   $\mathcal{L} \times \mathcal{L} \to \mathcal{P}(\bioformulas^*)$.  A
%   \emph{world} $w$ is a pair of world assignments, respectively called the
%   \emph{control} map and the \emph{elementary} map.

%   \[
%     (w_{\text{ctrl}}, w_{\text{elem}}) \in W \equiv
%     (\mathcal{L} \times \mathcal{L} \to \mathcal{P}(\bioformulas^*))
%     \times (\mathcal{L} \times \mathcal{L} \to \mathcal{P}(\bioformulas^*))
%   \]
% \end{definition}

% Elementary bases and control sets are therefore from now on considered in
% reference to a particular world $w$: the only difference is that the control set
% of a formula $A \rightarrow B$ is given by $w_{\text{ctrl}}(A,B)$, whereas the set
% of elementary bases is given by $w_{\text{elem}}(A,B)$. Thus, the $\rightarrow$
% elimination rule can be rewritten as

% \[
%   \begin{prooftree}
%     \Gamma, A \rightarrow B \, : \,
%     \elembasesw{\Gamma}{w} \cap w_{\text{ctrl}}(A, B) = \emptyset
%     \justifies
%     \Gamma, B
%   \end{prooftree}
% \]

% where $\elembasesw{\Gamma}{w}$ is exactly as $\elembases{\Gamma}$ but using $w$
% for the elementary bases of the conditionals.

We now give an alternative formalization of the rules of Zsyntax, that uses the
modified conditional operator. As explained in the previous section, this yields
a strictly stronger calculus as far with respect to proofs involving
conditionals (if we forget our weaker notion of control set), as it is possible
to prove theorems that are not provable in plain \znd{}. The resulting lack of
soundness w.r.t. \znd{} is not a problem in our opinion, as we think that the
theorems that \znd{} misses should actually be provable.

% This calculus is
% nevertheless complete in a certain way: if a theorem holds in \znd{}, that it is
% possible to annotate all conditionals that appear in the proof in such a way
% that the resulting is also a proof in \eznd{}.

\eznd{} proofs involve a judgement which is basically the provability relation
of Zsyntax as in [paper], decorated with information about the control sets of
the biological reactions described by the relation.

\begin{definition}[Enhanced provability judgement]
  Enhanced provability judgements are judgements of the form

  \[
    \Gamma \models_{\ctrlset{}} \Delta
  \]

  where $\Gamma, \Delta$ are lists of formulas, and $\ctrlset{}$ is a control
  set.
\end{definition}

% The intuitive meaning of affirming a judgement of the form
% $\Delta \models_{\ctrlset{}}^w \nabla$ is to know that there exists a sequence
% of Zsyntax rule applications with which it is possible to reach the Z-state
% $\Gamma, \nabla$ from the Z-state $\Gamma, \Delta$, under the elementary bases
% and control sets given by the world $w$, provided that $\Gamma$ is compatible
% with the control set $\ctrlset{}$.

The intuitive meaning of affirming a judgement of the form
$\Delta \models_{\ctrlset{}} \nabla$ is to know that there exists a sequence of
Zsyntax rule applications with which it is possible to reach the Z-state
$\Gamma, \nabla$ from the Z-state $\Gamma, \Delta$, provided that $\Gamma$ is
compatible with the control set $\ctrlset{}$.

% \paragraph{World extension}

% A world $w$ codifies our knowledge given by the theorems (i.e., biological
% transitions) so far proved. Worlds are dynamic objects in the sense that they
% change as new reactions are discovered. In particular, we gain new information
% on the control set $\ctrlset{A \rightarrow B}$ whenever a new reaction of the
% form $\Gamma, A \models B$ is discovered. We therefore define a world extension
% operation as follows:

% \begin{definition}[World extension]
%   A derivation $\Gamma, A \models_{\ctrlset{}} B$ extends a world $w$ to produce
%   a world $w'$, written $w[\Gamma, A \models_{\ctrlset{}} B] = w'$, as follows:

%   \[
%     w[\Gamma, A \models_{\ctrlset{}} B]_{\text{ctrl}}(C,D) =
%     \begin{cases}
%       w_{\text{ctrl}}(A,B) \cup \ctrlset{}, & \text{if } A \rightarrow B \equiv
%       C \rightarrow D \\
%       w_{\text{ctrl}}(C,D), & \text{otherwise}
%     \end{cases}
%   \]
%   \[
%     w[\Gamma, A \models_{\ctrlset{}} B]_{\text{elem}}(C,D) =
%     \begin{cases}
%       w_{\text{elem}}(A,B) \cup w_{\text{elem}}(\Gamma), & \text{if } A \rightarrow B \equiv
%       C \rightarrow D \\
%       w_{\text{elem}}(C,D), & \text{otherwise}
%     \end{cases}
%   \]
% \end{definition}

% \paragraph{Technical detail about worlds}

% Proofs in the natural deduction calculus (and, analogously, derivations in the
% sequent calculus) consider a fixed world in all parts of the proof. In other
% terms, the \emph{current} world is of course allowed to (and will) change
% between proofs (this is how the system is dynamic), but it is \strong{not}
% allowed to change \strong{within} a proof. Even though it wouldn't be difficult
% to give a natural deduction calculus that accounts for world dynamics within the
% same proof, it becomes problematic when dealing with the corresponding sequent
% calculus derivations, and in particular in the proof of cut elimination. For
% this reason, we choose to consider only proofs with fixed worlds, and in
% particular we require that the world considered in the proof is aware of all
% transitions of the form $\Gamma, A \models B$ that may have been
% \emph{discovered} for the first time from the proof itself. This is actually a
% quite reasonable thing to do: given the dynamic nature of the calculus, every
% previously established theorem must be re-checked every time the current world
% is changed with new information. Therefore, it only saves time (and computation)
% to eagerly check a proof with a world as up-to-date as possible.

We now give a translation from the calculus \znd{} given in [paper] to our
enhanced calculus \eznd{}, which will be formed by inference rules having as
premises and conclusion judgements of the form
$\Gamma \models_{\ctrlset{}} \Delta$.

\paragraph{Empirical rules}

Empirical rules are not part of the calculus, but are treated as axioms as
hinted in Section [TODO] of [paper]. As an example, postulating that $A$
transitions to $B$ provided $\ctrlset{}$ is respected is the same as affirming
$\models A \rightarrow_{\ctrlset{}}^\emptyset B$.

\paragraph{$\rightarrow$ introduction}

% The rule in [paper] is as follows:

% \[
%   \begin{prooftree}
%     \Gamma, \Delta \qquad
%     \[
%       \Gamma, A
%       \leadsto
%       B
%     \]
%     \justifies
%     A \rightarrow B, \Delta
%   \end{prooftree}
% \]

% Thus, whenever it can be shown that $\Gamma, A \vdash_{\ctrlset{}} B$ for some
% $\ctrlset{}$, $\Gamma$ may be considered \emph{ipso facto} of type
% $A \rightarrow B$, that is
% $\Gamma, \Delta \vdash_{\emptyset} A \rightarrow B, \Delta$.
% Notice, however, that in this process we gain knowledge of the fact that
% $\Gamma, A \models B$, so the \emph{world} against which the side conditions are
% checked must be updated accordingly (or rather, must already be up to date, for
% what we said before). This leads to the following rule:

We have the following rule:

\[
  \begin{prooftree}
    \Gamma, A \models_{\ctrlset{}} B
    \justifies
    \Gamma, \Delta \models_{\emptyset}
    A \rightarrow_{\ctrlset{}}^{\elembases{\Gamma}} B, \Delta
  \end{prooftree}
\]

Notice how the information on elementary base and control set associated with
the conditional just introduced is preserved locally in the formula itself.
Also, the null control set in the conclusion is just a translation of the fact
that going from $\Gamma$ to $A \rightarrow B$ is just a matter of symbols, and
does not involve any real reaction.

% Here, the side condition $w[\Gamma,A\vdash_{\ctrlset{}}B] = w$ means that the
% proof may happen in an arbitrary world, provided it already contains the
% additional knowledge that the premise has just established. In other words, we
% require that the world against which we check the validity of the proof is a
% fixed point with respect to the extension operation represented by
% $\Gamma,A\models_{\ctrlset{}}B$.

\paragraph{$\rightarrow$ elimination}

The rule in [paper]

\[
  \begin{prooftree}
    \Gamma, A \rightarrow B, A \; : \;
    \elembases{\Gamma} \cap \ctrlset{A \rightarrow B}^* = \emptyset
    \justifies
    \Gamma, B
  \end{prooftree}
\]

Can be straighforwardly translated to our formalism as follows:

\[
  \begin{prooftree}
    \elembases{\Gamma} \notin \ctrlset{}
    \justifies
    \Gamma, A \rightarrow_{\ctrlset{}}^S B, A \vdash_{\ctrlset{}} \Gamma, B
  \end{prooftree}
\]

The $\ctrlset{}$ in the conclusion witnesses the fact that not only
$A\rightarrow B$ can be eliminated if $\Gamma$ respects the associated control
set, but also that any other context in which the whole transition is placed
must respect it.

\paragraph{$\otimes$ introduction and elimination}

The $\otimes$ introduction and elimination rules are straighforward:

\[
  \begin{prooftree}
    \justifies
    \Gamma, A, B \models_{\emptyset} \Gamma, A \otimes B
  \end{prooftree}
\]

\[
  \begin{prooftree}
    \justifies
    \Gamma, A \otimes B \models_{\emptyset} \Gamma, A, B
  \end{prooftree}
\]

\paragraph{Identity and composition}

Obviously, the trivial transition is allowed:

\[
  \begin{prooftree}
    \justifies
    \Gamma \models_{\emptyset} \Gamma
  \end{prooftree}
\]

Moreover, in order to have the $\models$ relation represent arbitrary sequences
of Zsyntax rule applications, we need a way to compose such sequences:

\[
  \begin{prooftree}
    \Gamma \models_{\ctrlset{1}} \Delta
    \qquad
    \Delta \models_{\ctrlset{2}} \nabla
    \justifies
    \Gamma \models_{\ctrlset{1} \cup \ctrlset{2}} \nabla
  \end{prooftree}
\]

The following easy to establish result holds:

\begin{proposition}
  The following rule is admissible:

  \[
    \begin{prooftree}
      \Gamma \models_{\ctrlset{}} \Delta
      \qquad
      \nabla \notin \ctrlset{}
      \justifies
      \Gamma, \nabla \models_{\ctrlset{}} \Delta, \nabla
    \end{prooftree}
  \]
\end{proposition}
\begin{proof}
  Straightforward induction on the derivations.
\end{proof}

The next section provides a sequent calculus that is proved sound and complete
with respect to \eznd{}.

% Notice how composition forces the two subsequences to happen under the same
% world $w$. We then have the following properties, which will be particularly
% useful when implementing the constraint-solving part of the automated deduction
% engine.

% \begin{proposition}
%   If $\Gamma \models_{\ctrlset{}}^w \Delta$, then any subderivation $\Gamma'
%   \models_{\ctrlset{}'}^{w'} \Delta'$ is such that $w = w'$.
% \end{proposition}
% \begin{proof}
%   Straightforward induction on the derivation.
% \end{proof}

% \begin{proposition}
%   If $\Gamma \models_{\ctrlset{}}^w \Delta$, then for every subderivation
%   $\Gamma', A \models_{\ctrlset{}'}^w B$ it holds that $w[\Gamma', A
%   \models_{\ctrlset{}'} B] = w$.
% \end{proposition}
% \begin{proof}
%   Follows directly from the proposition above.
% \end{proof}

%%% Local Variables:
%%% mode: latex
%%% TeX-master: "../docs"
%%% End:
