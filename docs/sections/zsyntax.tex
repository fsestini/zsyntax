\section{Zsyntax}

\subsection{Introduction}

... TODO ...

\subsection{Controlled monotonicity}

The calculus of Zsyntax revolves around the concept of controlled monotonicity:
in the particular context of linear logic, linear implication is monotonic since
$A \vdash B$ implies $A \otimes C \vdash B \otimes C$ for any $C$. Zsyntax
preserves the characteristics of linear logic as a logic of resources, but has a
non-monotonic logical consequence relation (and therefore, a linear implication
operator), or rather one where such monotonicity is allowed in all but some
specified cases. In this sense it is similar to other formalisms such as default
logic. In the following we only repeat some of the definitions in [paper] on
which the following sections built upon. Details and examples on how Zsyntax
deals with monotonicity can be found in [paper].

\begin{definition}[Elementary bases]
  \begin{enumerate}
  \item An elementary base of a formula $A$ is a context
    $\Gamma \in \bioformulas^*$ defined inductively as follows:

    \begin{enumerate}
    \item If $A \in \bioformulas$, then $\Gamma = A$;
    \item If $A \equiv B \rightarrow C$, then $\Gamma$ is an elementary base of
      $A$ whenever $\Gamma, A \models B$ is known;
    \item If $A \equiv B \otimes C$, then $\Gamma = \Delta, \Delta'$ is an
      elementary base of $A$ whenever $\Delta$ and $\Delta'$ are, respectively,
      an elementary base for $A$ and $B$.
    \end{enumerate}

  \item Denote by $\elembases{A} \in \mathcal{P}(\bioformulas^*)$ the set of all
    elementary bases for the formula $A$;
  \item Given a Z-state $\Gamma = A_1, \dots, A_n$, let $\elembases{\Gamma}$ be
    the set of all Z-states $\Delta_1, \dots, \Delta_n$ such that $\Delta_i$ is
    in $\elembases{A}$ for all $i$.
  \end{enumerate}
\end{definition}

Notice that, given any context $\Gamma$, the set $\elembases{\Gamma}$ may change
over time as new theorems of the form $\Delta, A \models B$ are discovered.
Also notice that the definition of elementary base is not complete until we
define precisely what it means to \emph{know} that $\Gamma, A \models B$ (for
example, is this knowledge provided manually by the user or managed
automatically by the machine?). We leave this decision to the implementation
details.

[paper] then goes on to define control sets for conditionals $A \rightarrow B$
in terms of elementary bases, calling them \emph{elementary control sets}.  In
what follows, $\bioformulas^*$ denotes as usual the set of strings of formulas
of the bonding language (which are just Z-states formed by formulas of the
bonding language only.)

\begin{definition}
  The \emph{elementary control set}
  $\ctrlset{A \rightarrow B} \in \mathcal{P}(\bioformulas^*)$ associated with
  the conditional $A \rightarrow B$ can be defined as follows:

  \[
    \ctrlset{A \rightarrow B}^* = \{
    \Gamma \in \bioformulas^* \, | \, \exists \Delta \in \elembases{A \rightarrow
    B}
    \quad : \quad \Gamma, \Delta, A \not \models \Gamma, B \text{ is known}
    \}
  \]
\end{definition}

Again, this definition is incomplete unless we specify what it means to know
that $\Gamma, \Delta, A \not \models \Gamma, B$. Nevertheless, under this
definition, the controlled $\rightarrow$ elimination rule becomes the following:

\[
  \begin{prooftree}
    \Gamma, A \rightarrow B \, : \, \Gamma^* \cap \ctrlset{A\rightarrow B}^* =
    \emptyset
    \justifies
    \Gamma, B
  \end{prooftree}
\]

In the following section we will give a slightly modified (but nevertheless
still precise and extremely faithful to the original) provability relation for
the Zsyntax natural deduction-style calculus in [paper], that not only describes
the biological transition but also takes into account the control sets involved
in it.

\subsection{An alternative natural deduction calculus}

As we have seen, the logical framework of Zsyntax is dynamic in the sense that
knowledge about biochemical reactions changes over time as facts are
discovered. In other terms, this means that the contents of elementary bases
and control sets that we consider as part of our system change.

To give a precise definition of proof (in natural deduction style) or derivation
(in sequent calculus), however, we need a way to refer to a particular
\emph{current state} of the system, that is, to the current value of the
elementary bases and control sets. We thus give the concept of a \emph{world},
which is just a structure that describes the state of the system in a given
point in time.

\begin{definition}[World]
  A \emph{world assignment} is a map of type
  $\mathcal{L} \times \mathcal{L} \to \mathcal{P}(\bioformulas^*)$.  A
  \emph{world} $w$ is a pair of world assignments, respectively called the
  \emph{control} map and the \emph{elementary} map.

  \[
    (w_{\text{ctrl}}, w_{\text{elem}}) \in W \equiv
    (\mathcal{L} \times \mathcal{L} \to \mathcal{P}(\bioformulas^*))
    \times (\mathcal{L} \times \mathcal{L} \to \mathcal{P}(\bioformulas^*))
  \]
\end{definition}

Elementary bases and control sets are therefore from now on considered in
reference to a particular world $w$: the only difference is that the control set
of a formula $A \rightarrow B$ is given by $w_{\text{ctrl}}(A,B)$, whereas the set
of elementary bases is given by $w_{\text{elem}}(A,B)$. Thus, the $\rightarrow$
elimination rule can be rewritten as

\[
  \begin{prooftree}
    \Gamma, A \rightarrow B \, : \,
    \elembasesw{\Gamma}{w} \cap w_{\text{ctrl}}(A, B) = \emptyset
    \justifies
    \Gamma, B
  \end{prooftree}
\]

where $\elembasesw{\Gamma}{w}$ is exactly as $\elembases{\Gamma}$ but using $w$
for the elementary bases of the conditionals.

\begin{definition}[Enhanced provability judgement]
  Enhanced provability judgements are judgements of the form

  \[
    \Gamma \models_{\ctrlset{}}^w \Delta
  \]

  where $\Gamma, \Delta$ are lists of formulas, $\ctrlset{}$ is a control set
  and $w \in W$ is a world.
\end{definition}

The intuitive meaning of affirming a judgement of the form
$\Delta \models_{\ctrlset{}}^w \nabla$ is to know that there exists a sequence
of Zsyntax rule applications with which it is possible to reach the Z-state
$\Gamma, \nabla$ from the Z-state $\Gamma, \Delta$, under the elementary bases
and control sets given by the world $w$, provided that $\Gamma$ is compatible
with the control set $\ctrlset{}$.

\paragraph{World extension}

A world $w$ codifies our knowledge given by the theorems (i.e., biological
transitions) so far proved. Worlds are dynamic objects in the sense that they
change as new reactions are discovered. In particular, we gain new information
on the control set $\ctrlset{A \rightarrow B}$ whenever a new reaction of the
form $\Gamma, A \models B$ is discovered. We therefore define a world extension
operation as follows:

\begin{definition}[World extension]
  A derivation $\Gamma, A \models_{\ctrlset{}} B$ extends a world $w$ to produce
  a world $w'$, written $w[\Gamma, A \models_{\ctrlset{}} B] = w'$, as follows:

  \[
    w[\Gamma, A \models_{\ctrlset{}} B]_{\text{ctrl}}(C,D) =
    \begin{cases}
      w_{\text{ctrl}}(A,B) \cup \ctrlset{}, & \text{if } A \rightarrow B \equiv
      C \rightarrow D \\
      w_{\text{ctrl}}(C,D), & \text{otherwise}
    \end{cases}
  \]
  \[
    w[\Gamma, A \models_{\ctrlset{}} B]_{\text{elem}}(C,D) =
    \begin{cases}
      w_{\text{elem}}(A,B) \cup w_{\text{elem}}(\Gamma), & \text{if } A \rightarrow B \equiv
      C \rightarrow D \\
      w_{\text{elem}}(C,D), & \text{otherwise}
    \end{cases}
  \]
\end{definition}

\paragraph{Technical detail about worlds}

Proofs in the natural deduction calculus (and, analogously, derivations in the
sequent calculus) consider a fixed world in all parts of the proof. In other
terms, the \emph{current} world is of course allowed to (and will) change
between proofs (this is how the system is dynamic), but it is \strong{not}
allowed to change \strong{within} a proof. Even though it wouldn't be difficult
to give a natural deduction calculus that accounts for world dynamics within the
same proof, it becomes problematic when dealing with the corresponding sequent
calculus derivations, and in particular in the proof of cut elimination. For
this reason, we choose to consider only proofs with fixed worlds, and in
particular we require that the world considered in the proof is aware of all
transitions of the form $\Gamma, A \models B$ that may have been
\emph{discovered} for the first time from the proof itself. This is actually a
quite reasonable thing to do: given the dynamic nature of the calculus, every
previously established theorem must be re-checked every time the current world
is changed with new information. Therefore, it only saves time (and computation)
to eagerly check a proof with a world as up-to-date as possible.

We now give a translation from the calculus given in [paper] to our enhanced
calculus, which will be formed by inference rules having as premises and
conclusion judgements of the form $\Gamma \models_{\ctrlset{}}^w \Delta$.

\paragraph{Empirical rules}

Empirical rules are not part of the calculus, but are treated as axioms as
hinted in Section [TODO] of [paper].

\paragraph{$\rightarrow$ introduction}

The rule in [paper] is as follows:

\[
  \begin{prooftree}
    \Gamma, \Delta \qquad
    \[
      \Gamma, A
      \leadsto
      B
    \]
    \justifies
    A \rightarrow B, \Delta
  \end{prooftree}
\]

Thus, whenever it can be shown that $\Gamma, A \vdash_{\ctrlset{}} B$ for some
$\ctrlset{}$, $\Gamma$ may be considered \emph{ipso facto} of type
$A \rightarrow B$, that is
$\Gamma, \Delta \vdash_{\emptyset} A \rightarrow B, \Delta$.
Notice, however, that in this process we gain knowledge of the fact that
$\Gamma, A \models B$, so the \emph{world} against which the side conditions are
checked must be updated accordingly (or rather, must already be up to date, for
what we said before). This leads to the following rule:

\[
  \begin{prooftree}
    \Gamma, A \vdash_{\ctrlset{}}^{w} B
    \justifies
    \Gamma, \Delta \vdash_{\emptyset}^{w} A
    \rightarrow B, \Delta
    \using{w[\Gamma,A\vdash_{\ctrlset{}}B] = w}
  \end{prooftree}
\]

Here, the side condition $w[\Gamma,A\vdash_{\ctrlset{}}B] = w$ means that the
proof may happen in an arbitrary world, provided it already contains the
additional knowledge that the premise has just established. In other words, we
require that the world against which we check the validity of the proof is a
fixed point with respect to the extension operation represented by
$\Gamma,A\models_{\ctrlset{}}B$.

\paragraph{$\rightarrow$ elimination}

The rule in [paper]

\[
  \begin{prooftree}
    \Gamma, A \rightarrow B, A \; : \;
    \elembases{\Gamma} \cap \ctrlset{A \rightarrow B}^* = \emptyset
    \justifies
    \Gamma, B
  \end{prooftree}
\]

Can be straighforwardly translated to our formalism as follows:

\[
  \begin{prooftree}
    \elembasesw{\Gamma}{w} \cap w_{\text{ctrl}}(A, B) = \emptyset
    \justifies
    \Gamma, A \rightarrow B, A \vdash_{w_{\text{ctrl}}(A,B)}^w \Gamma, B
  \end{prooftree}
\]

\paragraph{$\otimes$ introduction and elimination}

The $\otimes$ introduction and elimination rules are straighforward:

\[
  \begin{prooftree}
    \justifies
    \Gamma, A, B \models_{\emptyset}^w \Gamma, A \otimes B
  \end{prooftree}
\]

\[
  \begin{prooftree}
    \justifies
    \Gamma, A \otimes B \models_{\emptyset}^w \Gamma, A, B
  \end{prooftree}
\]

\paragraph{Identity and composition}

Obviously, the trivial transition is allowed in any world:

\[
  \begin{prooftree}
    \justifies
    \Gamma \models_{\emptyset}^w \Gamma
  \end{prooftree}
\]

Moreover, in order to have the $\models$ relation represent arbitrary sequences
of Zsyntax rule applications, we need a way to compose such sequences:

\[
  \begin{prooftree}
    \Gamma \models_{\ctrlset{1}}^w \Delta
    \qquad
    \Delta \models_{\ctrlset{2}}^w \nabla
    \justifies
    \Gamma \models_{\ctrlset{1} \cup \ctrlset{2}}^w \nabla
  \end{prooftree}
\]

Notice how composition forces the two subsequences to happen under the same
world $w$. We then have the following properties, which will be particularly
useful when implementing the constraint-solving part of the automated deduction
engine.

\begin{proposition}
  If $\Gamma \models_{\ctrlset{}}^w \Delta$, then any subderivation $\Gamma'
  \models_{\ctrlset{}'}^{w'} \Delta'$ is such that $w = w'$.
\end{proposition}
\begin{proof}
  Straightforward induction on the derivation.
\end{proof}

\begin{proposition}
  If $\Gamma \models_{\ctrlset{}}^w \Delta$, then for every subderivation
  $\Gamma', A \models_{\ctrlset{}'}^w B$ it holds that $w[\Gamma', A
  \models_{\ctrlset{}'} B] = w$.
\end{proposition}
\begin{proof}
  Follows directly from the proposition above.
\end{proof}

%%% Local Variables:
%%% mode: latex
%%% TeX-master: "../docs"
%%% End:
