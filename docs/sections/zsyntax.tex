\section{Zsyntax}

\subsection{Introduction}

... TODO ...

In the sections that follow we will give a slightly modified (but nevertheless
still precise and extremely faithful to the original) provability relation for
the Zsyntax natural deduction-style calculus in [paper], that not only describes
the biological transition but also takes into account the control sets involved
in it.

\subsection{Controlled monotonicity}

\begin{definition}[Elementary base]
  \begin{enumerate}
  \item An elementary base of a formula $A$ is defined inductively as follows:

    \begin{enumerate}
    \item ...
    \end{enumerate}

  \item Denote by $A^*$ the set of all elementary bases for the formula $A$;
  \item Given a Z-state $\Gamma = A_1, \dots, A_n$, let $\Gamma^*$ be the set of
    all Z-states $\Delta_1, \dots, \Delta_n$ such that $\Delta_i$ is in $A^*$
    for all $i$.
  \end{enumerate}
\end{definition}

Notice that, given any context $\Gamma$, the set $\Gamma^*$ may change over time
as new theorems of the form $\Gamma, A \models B$ are discovered, as this
influences the contents of $(A \rightarrow B)^*$.

[paper] then goes on to define control sets for conditionals $A \rightarrow B$
in terms of elementary bases, calling them \emph{elementary control sets}.

\begin{definition}
  The \emph{elementary control set} associated with the conditional
  $A \rightarrow B$ can be defined as follows:

  \[
    \ctrlset{A \rightarrow B}^* = \{
    \Gamma \in \mathcal{L}_{\odot}^* \, | \, \exists \Delta \in (A \rightarrow
    B)^*
    : \Gamma, \Delta, A \not \models \Gamma, B \text{ is known}
    \}
  \]
\end{definition}

Under this definition, the controller $\rightarrow$ elimination rule becomes the
following:

\[
  \begin{prooftree}
    \Gamma, A \rightarrow B \, : \, \Gamma^* \cap \ctrlset{A\rightarrow B}^* =
    \emptyset
    \justifies
    \Gamma, B
  \end{prooftree}
\]

\subsection{An alternative natural deduction calculus}

\begin{definition}[World]
  A \emph{world assignment} is a map of type
  $\mathcal{L} \times \mathcal{L} \to \mathcal{P}(\mathcal{L}_\odot)$.  A
  \emph{world} $w$ is a pair of world assignments, respectively called the
  \emph{control} map and the \emph{elementary} map.

  \[
    (w_{\text{ctrl}}, w_{\text{elem}}) \in W \equiv
    (\mathcal{L} \times \mathcal{L} \to \mathcal{P}(\mathcal{L}_\odot))
    \times (\mathcal{L} \times \mathcal{L} \to \mathcal{P}(\mathcal{L}_\odot))
  \]
\end{definition}

\begin{definition}[Enhanced provability judgement]
  Enhanced provability judgements are judgements of the form

  \[
    \Gamma \models_{\ctrlset{}}^w \Delta
  \]

  where $\Gamma, \Delta$ are lists of formulas, $\ctrlset{}$ is a control set
  and $w \in W$ is a world.
\end{definition}

The intuitive meaning of a judgement $\Delta \models_{\ctrlset{}}^w \nabla$ is
that there exists a sequence of Zsyntax rule applications with which it is
possible to reach the Z-state $\Gamma, \nabla$ from the Z-state
$\Gamma, \Delta$, under the elementary bases and control sets given by the world
$w$, provided that $\Gamma$ is compatible with the control set $\ctrlset{}$.

\paragraph{World extension}

A world $w$ codifies our knowledge given by the theorems (i.e., biological
transitions) so far proved. Worlds are dynamic objects in the sense that they
change as new reactions are discovered. In particular, we gain new information
on the control set $\ctrlset{A \rightarrow B}$ whenever a new reaction of the
form $\Gamma, A \models B$ is discovered. We therefore define a world extension
operation as follows:

\begin{definition}[World extension]
  A derivation $\Gamma, A \models_{\ctrlset{}} B$ extends a world $w$ to produce
  a world $w'$, written $w[\Gamma, A \models_{\ctrlset{}} B] = w'$, as follows:

  \[
    w[\Gamma, A \models_{\ctrlset{}} B]_{\text{ctrl}}(C,D) =
    \begin{cases}
      w_{\text{ctrl}}(A,B) \cup \ctrlset{}, & \text{if } A \rightarrow B \equiv
      C \rightarrow D \\
      w_{\text{ctrl}}(C,D), & \text{otherwise}
    \end{cases}
  \]
  \[
    w[\Gamma, A \models_{\ctrlset{}} B]_{\text{elem}}(C,D) =
    \begin{cases}
      w_{\text{elem}}(A,B) \cup w_{\text{elem}}(\Gamma), & \text{if } A \rightarrow B \equiv
      C \rightarrow D \\
      w_{\text{elem}}(C,D), & \text{otherwise}
    \end{cases}
  \]
\end{definition}

\paragraph{Empirical rules}

Empirical rules are not part of the calculus, but are treated as axioms as
hinted in Section [TODO] of [paper].

\paragraph{$\rightarrow$ introduction}

The rule in [paper] is as follows:

\[
  \begin{prooftree}
    \Gamma, \Delta \qquad
    \[
      \Gamma, A
      \leadsto
      B
    \]
    \justifies
    A \rightarrow B, \Delta
  \end{prooftree}
\]

Thus, whenever it can be shown that $\Gamma, A \vdash_{\ctrlset{}} B$ for some
$\ctrlset{}$, $\Gamma$ may be considered \emph{ipso facto} of type
$A \rightarrow B$, that is
$\Gamma, \Delta \vdash_{\emptyset} A \rightarrow B, \Delta$.
Notice, however, that in this process we gain knowledge of the fact that
$\Gamma, A \models B$, so the \emph{world} against which the side conditions are
checked must be updated accordingly. This leads to the following rule:

\[
  \begin{prooftree}
    \Gamma, A \vdash_{\ctrlset{}}^{w[\Gamma,A\vdash_{\ctrlset{}}B]} B
    \justifies
    \Gamma, \Delta \vdash_{\emptyset}^{w[\Gamma,A\vdash_{\ctrlset{}}B]n} A
    \rightarrow B, \Delta
  \end{prooftree}
\]

Here, $w[\Gamma,A\vdash_{\ctrlset{}}B]$ means that the proof may happen in an
arbitrary world, provided in contains the knowledge that the premise has
just established.

\paragraph{$\rightarrow$ elimination}

The rule in [paper]

\[
  \begin{prooftree}
    \Gamma, A \rightarrow B, A \; : \; \biocore{\Gamma} \cap \ctrlset{A
      \rightarrow B} = \emptyset
    \justifies
    \Gamma, B
  \end{prooftree}
\]

Can be straighforwardly translated to our formalism as follows:

\[
  \begin{prooftree}
    \textsf{resp}_w(\Gamma, A, B)
    \justifies
    \Gamma, A \rightarrow B, A \vdash_{w_{\text{ctrl}}(A,B)}^w \Gamma, B
  \end{prooftree}
\]

\paragraph{$\otimes$ introduction and elimination}

The $\otimes$ introduction and elimination rules are straighforward:

\[
  \begin{prooftree}
    \justifies
    \Gamma, A, B \models_{\emptyset}^w \Gamma, A \otimes B
  \end{prooftree}
\]

\[
  \begin{prooftree}
    \justifies
    \Gamma, A \otimes B \models_{\emptyset}^w \Gamma, A, B
  \end{prooftree}
\]

\paragraph{Identity and composition}

Obviously, the trivial transition is allowed in any world:

\[
  \begin{prooftree}
    \justifies
    \Gamma \models_{\emptyset}^w \Gamma
  \end{prooftree}
\]

Moreover, in order to have the $\models$ relation represent arbitrary sequences
of Zsyntax rule applications, we need a way to compose such sequences:

\[
  \begin{prooftree}
    \Gamma \models_{\ctrlset{1}}^w \Delta
    \qquad
    \Delta \models_{\ctrlset{2}}^w \nabla
    \justifies
    \Gamma \models_{\ctrlset{1} \cup \ctrlset{2}}^w \nabla
  \end{prooftree}
\]

Notice how composition forces the two subsequences to happen under the same
world $w$. We then have the following properties, which will be particularly
useful when implementing the constraint-solving part of the automated deduction
engine.

\begin{proposition}
  If $\Gamma \models_{\ctrlset{}}^w \Delta$, then any subderivation $\Gamma'
  \models_{\ctrlset{}'}^{w'} \Delta'$ is such that $w = w'$.
\end{proposition}
\begin{proof}
  Straightforward induction on the derivation.
\end{proof}

\begin{proposition}
  If $\Gamma \models_{\ctrlset{}}^w \Delta$, then for every subderivation
  $\Gamma', A \models_{\ctrlset{}'}^w B$ it holds that $w[\Gamma', A
  \models_{\ctrlset{}'} B] = w$.
\end{proposition}
\begin{proof}
  Follows directly from the proposition above.
\end{proof}

\subsubsection{Implementation details}

To slightly reduce the complexity of the automatic deduction process, we make
the following assumption:

\[
  \Delta \models \nabla \wedge \Gamma, \Delta \not \models \Gamma, \nabla
  \implies \forall \Gamma'
  \supseteq \Gamma, \; \Gamma', \Delta \not \models \Gamma', \nabla
\]

In words, we assume that if a context $\Gamma$ is such that it inhibits a
reaction, then so do all contexts $\Gamma'$ that extend it.

\subsubsection{Implementation details}

The two propositions above allow us to implement the proof search phase as
follows:

\begin{enumerate}
\item A structurally acceptable proof $\pi$ is found;
\item The current world $w$ is extended with knowledge about all derivations
  of the form $\Gamma, A \models B$ in $\pi$, to yield an updated world $w'$;
\item The proof in finally checked against the updated world $w'$.
\end{enumerate}

Notice that different derivations of the form $\Gamma, A \models B$ in the same
proof may influence each other in how the world is extended, for example when an
instance of one is part of the subderivation of another. This mutual dependence
can be solved by considering a per-proof monotone operator defining a one-step
world extension, and repeatedly applying such operator until a fixed point is
reached.

\begin{definition}
  World assignments form a join-semilattice

  \[
    \mathcal{A} = \langle \mathcal{L} \times \mathcal{L} \to
    \mathcal{P}(\mathcal{L}_\odot), \leq, \sqcup \rangle
  \]

  where

  \[
    w_1 \leq w_2 \equiv \forall F_1, F_2, w_1(F_1,F_2) \subseteq w_2(F_1,F_2)
  \]

  \[
    (w_1 \sqcup w_2)(F_1, F_2) = w_1(F_1,F_2) \cup w_2(F_1,F_2)
  \]
\end{definition}

Notice that, since we assume $\mathcal{L}_{\odot}$ to be finite, the set
$\mathcal{L} \times \mathcal{L} \to \mathcal{P}(\mathcal{L}_\odot)$ is finite
and so $\mathcal{A}$ is also a complete lattice. In particular, is has a top
element.

The order theoretic properties of world assigments extend to worlds in the
obvious way.

\begin{definition}
  Worlds form a join-semilattice
  $\mathcal{W} = \langle W, \leq, \sqcup \rangle$, where

  \[
    (w_{\text{ctrl}1}, w_{\text{elem}1}) \leq
    (w_{\text{ctrl}2}, w_{\text{elem}2}) \equiv
    w_{\text{ctrl}1} \leq w_{\text{ctrl}2} \wedge
    w_{\text{elem}1} \leq w_{\text{elem}2}
  \]
  \[
    (w_{\text{ctrl}1}, w_{\text{elem}1}) \sqcup
    (w_{\text{ctrl}2}, w_{\text{elem}2}) =
    (w_{\text{ctrl}1} \sqcup w_{\text{ctrl}2}, w_{\text{elem}1} \sqcup w_{\text{elem}2})
  \]
\end{definition}

Being $\mathcal{W}$ equipped with a top element $\top$, we have the rather
trivial property that every monotonic function over $W$ has a fixed point.

Given a derivation $\pi$ and a world $w$, the one-step extension
$\textsf{extend}_{\pi}(w) \in W$ is inductively defined as follows:

\begin{enumerate}
\item Case $\pi$ is ...
\end{enumerate}

\begin{proposition}
  The extension operator $\textsf{extend}_{\pi} : W \rightarrow W$ is monotone.
\end{proposition}
\begin{proof}
  TODO...
\end{proof}

Thus, the world against which a proof $\pi$ is checked is
$w' = \bigsqcup \{ \textsf{extend}_{\pi}^n(w) \, | \, n \in N \}$.

%%% Local Variables:
%%% mode: latex
%%% TeX-master: "../docs"
%%% End:
