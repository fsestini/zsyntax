\section{Analysis of Zsyntax}\label{sec:zsyntax}

\subsection{Controlled monotonicity}

The property that separates Zsyntax from plain linear logic is its concept of
controlled monotonicity. In linear logic, implication is monotonic since
$A \vdash B$ implies $A \otimes C \vdash B \otimes C$ for any $C$. Zsyntax
preserves the characteristics of linear logic as a logic of resources, but has a
non-monotonic logical consequence relation, or rather one where such
monotonicity is normally allowed apart from some specified cases. In this sense
it is similar to other formalisms such as default logic \cite{default}. Below we
only repeat some of the definitions in \cite{adding-logic} on which the next
sections built upon. More details and examples on how Zsyntax deals with monotonicity
can be found in \cite{adding-logic}.

In what follows, $\bioformulas$ indicates the set of biological atoms, or the
\emph{bonding language} (\cite{adding-logic}), whereas $\bioformulas^*$ is the
usual free monoid generated by $\bioformulas$ (i.e., lists of elements of
$\bioformulas$). \footnote{Notice that the set of linear (as in linear logic)
  contexts of biological atoms is just the quotient set of $\bioformulas$ under
  an equivalence relation that ignores element order.}

\begin{definition}[Zsyntax elementary bases]
  \begin{enumerate}
  \item An elementary base of a formula $A$ is a context
    $\Gamma \in \bioformulas^*$ defined by structural induction on $A$ as
    follows:

    \begin{enumerate}
    \item If $A \in \bioformulas$, then $\Gamma = A$;
    \item If $A \equiv B \rightarrow C$, then $\Gamma$ is an elementary base of
      $A$ whenever $\Gamma, A \models B$ is known;
    \item If $A \equiv B \otimes C$, then $\Gamma = \Delta, \Delta'$ is an
      elementary base of $A$ whenever $\Delta$ and $\Delta'$ are, respectively,
      an elementary base for $A$ and $B$.
    \end{enumerate}

  \item Denote by $\elembases{A} \in \mathcal{P}(\bioformulas^*)$ the set of all
    elementary bases for the formula $A$. That is, we in general associate to a
    formula $A$ a set of elementary contexts.
  \item Given a Z-state $\Gamma = A_1, \dots, A_n$, let $\elembases{\Gamma}$ be
    the set of all Z-states $\Delta_1, \dots, \Delta_n$ such that $\Delta_i$ is
    in $\elembases{A}$ for all $i$.
  \end{enumerate}
\end{definition}

Notice that, given any context $\Gamma$, the set $\elembases{\Gamma}$ may change
over time as new theorems of the form $\Delta, A \models B$ are discovered.
Also notice that the definition of elementary base is not complete until we
define precisely what it means to \emph{know} that $\Gamma, A \models B$ (for
example, is this knowledge provided manually by the user or managed
automatically by the machine?). We leave this decision to the implementation
details.

\cite{adding-logic} then goes on to define control sets for conditionals
$A \rightarrow B$ in terms of elementary bases, calling them \emph{elementary
  control sets}.

\begin{definition}
  The \emph{elementary control set}
  $\ctrlset{A \rightarrow B} \in \mathcal{P}(\bioformulas^*)$ associated with
  the conditional $A \rightarrow B$ can be defined as follows:

  \[
    \ctrlset{A \rightarrow B}^* = \{
    \Gamma \in \bioformulas^* \;\; | \;\; \exists \Delta \in \elembases{A \rightarrow
    B}
    \;\; : \;\; \Gamma, \Delta, A \not \models \Gamma, B \text{ is known}
    \}
  \]
\end{definition}

Again, this definition is incomplete unless we specify what it means to know
that $\Gamma, \Delta, A \not \models \Gamma, B$. Nevertheless, under this
definition, the controlled $\rightarrow$ elimination rule becomes the following:

\[
  \begin{prooftree}
    \Gamma, A \rightarrow B \, : \, \Gamma^* \cap \ctrlset{A\rightarrow B}^* =
    \emptyset
    \justifies
    \Gamma, B
  \end{prooftree}
\]

\subsection{Reasons for a different conditional operator}

In \cite{adding-logic}, the authors describe Zsyntax, a logical calculus for
biological reactions which from now on will be referred to as \znd{} for
short. In the sections that follow, we will define an alternative natural
deduction calculus that is strictly related to, albeit in some way different to,
the one given in \cite{adding-logic}. The calculus is given in terms of an
annotated provability relation $\Gamma \models_\star \Delta$ meaning that there
exists a transition from $\Gamma$ to $\Delta$ using the rules of the calculus,
provided some constraints $\star$ are respected. We use the provability relation
instead of inference rules as taking the rules of \cite{adding-logic} and
annotating them would have been less elegant.  The purpose is to give a precise
formalization of the logic that we intend to translate to sequent calculus and
then implement, and at the same time clarify the similarities (and differences)
with \znd{} as originally defined in \cite{adding-logic}.

The fundamental difference between our reference calculus \eznd{} and \znd{}
lies in a different kind of Z-conditional (linear implication) operator. The
reason is logical, and lies in an somewhat excessive generality of the operator
that may lead to a perceived asymmetry between introduction and elimination
rules.  We now recall the definition of the Z-conditional operator of
\cite{adding-logic}, and illustrate how the one used here is different. The
introduction rule for the Z-conditional is given in \cite{adding-logic} as
follows:

\[
  \begin{prooftree}
    \Gamma, \Delta \qquad
    \[
      \Gamma, A
      \leadsto
      B
    \]
    \justifies
    A \rightarrow B, \Delta
  \end{prooftree}
\]

Intuitively, this rule introduces a particular \emph{instance} of
$A \rightarrow B$.  Even though the conclusion of the rule shows a generic
formula $A \rightarrow B$, the reason we have such a formula is because we know
that a particular aggregate $\Gamma$, with its particular elementary base, when
paired with $A$ has a transition to $B$. This transition (which is just a proof
in the natural deduction calculus) of course depends on the fact that the
context in which this transition is placed respects all control sets involved in
it.

This introduction rule forgets all these details, as it unifies all proofs (or
reactions) of the form $\Gamma, A \vdash B$ under the type $A \rightarrow B$.
To be fair, information is not actually thrown away, as it is supposed to be
remembered as part of the control sets and elementary bases associated with the
type $A \rightarrow B$. This however does not solve the asymmetry between the
two rules. In fact, consider the elimination rule given in \cite{adding-logic}:

\[
  \begin{prooftree}
    \Gamma, A \rightarrow B \, : \, \elembases{\Gamma} \cap \ctrlset{A\rightarrow B}^* =
    \emptyset
    \justifies
    \Gamma, B
  \end{prooftree}
\]

From the definition of control set in \cite{adding-logic}, we have that the
deduction above is allowed only if the context $\Gamma$ ``respects'' the control
set $\ctrlset{A\rightarrow B}^*$, in the sense that it is forbidden if $\Gamma$
is known to block \emph{some instance} of $A \rightarrow B$, that is,
$\Gamma, \Delta, A \not \models B$ is known for \emph{some} $\Delta$.

To see how this definition clashes with a possible intuition regarding the
introduction rule, consider the following example. Suppose we know that
$\Gamma, A \vdash B$ in any possible context. In other words, the control set
that we may want to associate with this reaction is empty, as the reaction is
always allowed to happen. Then, intuitively, it should be possible to derive the
following:

\[
  \begin{prooftree}
    \[
      \Gamma, A, \Delta \qquad
      \[
        \Gamma, A
        \leadsto
        B
      \]
      \justifies
      A \rightarrow B, A, \Delta
      \using{\rightarrow\mathcal{I}}
    \]
    \justifies
    B, \Delta
    \using{\rightarrow\mathcal{E}}
  \end{prooftree}
\]

as we know that the aggregate of type $A \rightarrow B$ is really $\Gamma$ is
disguise, therefore we also know that $\Delta$ does not interact with its
reaction with $A$.

Suppose now that we know that $\Delta, \nabla, A \not \models B$ for some
$\nabla$. In Zsyntax, this is sufficient to block the validity of the deduction
above, since $\Delta$ would then violate the control set for $A \rightarrow B$
(which refers to \emph{all} known transitions of type $A \rightarrow B$, and so
also the one involving $\nabla$.) That is, a \emph{global property} of the
biological aggregates of type $A \rightarrow B$, referenced by the elimination
rule, prevents us to deduce a theorem that we certainly know to be
\emph{locally} possible, because we know from the introduction rule that the
particular instance of $A \rightarrow B$ that we are eliminating comes form
$\Gamma$, which is not really affected by the presence of $\Delta$.

In essence, the Z-conditional described above is \emph{too general}, as it
implicitly contains an existential quantification that is hidden in its
introduction and elimination rules: to introduce an element of type
$A \rightarrow B$ is to know that \emph{there exists} an aggregate that, paired
with $A$, transitions to $B$. To prove something by elimination on an element of
type $A \rightarrow B$ is to prove it without knowledge on which aggregate was
used to establish $A \rightarrow B$, that is, to prove it for an arbitrary
$A \rightarrow B$ (this indeed corresponds to considering all of them, or at
least all of them according to our knowledge of the system so far, as in the
elimination rule in \cite{adding-logic}).

In other words, we do not have the possibility to refer to a \emph{specific}
instance of a conditional formula in our deductions. Our claim is that such
increase in the expressive power of the logic is something to be desired in a
formal language for molecular biology. In our opinion, a conditional operator
should be used that preserves as much information as possible, so that proofs
like the one above are allowed by the logic, and not only justified by one's
intuition. Thus, the conditional operator that is introduced in the next
sections is one that keeps explicit track of both the aggregate from which it
originated and the history of transitions that were used to establish its
validity. In addition, it will be shown how cut elimination goes through
smoothly, since all information is preserved and available.

\subsection{Non-commutativity of derivations}

The logic of Zsyntax, regardless of the conditional operator that is used, is
intrinsically non-commutative, in the sense that the success of a derivation
depends crucially on the order in which inference rules are applied. This is of
course a property that we want our logical system to have, if deductions are to
represent biological reactions. The price to be paid is in the resulting proof
theory, which gets more complicated and lacks properties that more standard
logics have. As an example, the deduction theorem fails to hold even in plain
Zsyntax. In particular

\[
  \Gamma \vdash A \rightarrow B \not\implies \Gamma, A \vdash B
\]

To see it concretely, consider $\Gamma \equiv A, A \rightarrow B, B \otimes B
\rightarrow C$, where $A \rightarrow B$ is only allowed to be used if $B$ is not
in the context. Then, $\Gamma \vdash B \rightarrow C$ is easily proved by first
eliminating $A \rightarrow B$ and then introducting $B$ to prove the conclusion.
However, there is no way to prove $\Gamma, B \vdash C$, as the presence of $B$
prevents $A \rightarrow B$ to be used.

\subsection{An alternative Z-conditional operator}

Our calculus \eznd{} will replace the conditional operator in \znd{} with the
following:

\[
  A \rightarrow_{\reactlist{}}^S B
\]

where $A,B$ are formulas, $S$ is a set of atomic formulas of the bonding
language $\bioformulas$, and $l$ is a list defined in a suitable way below. The
meaning of such annotations will become clear as we introduce the concepts.

The introduction of our modified conditional operator requires a new definition
of elementary base for a formula and for a (multi)set of formulas. Recall that
elementary bases in \cite{adding-logic} are sets of elementary contexts of
elements of the bonding language $\bioformulas$. Such definition is necessary,
as conditional operators of type $A \rightarrow B$ refer to \emph{all} known
reactions of such type. Since in our formalism there is a one-to-one
correspondence between an instance of a conditional operator and \emph{a single}
class of reacting aggregates with the same elementary base, we must redefine
elementary bases accordingly. For this reason, our elementary bases are just
linear contexts of formulas of the bonding language, rather than sets of
contexts.

\begin{definition}[Elementary base]
  The elementary base $\elembases{A} \in \bioformulas^*$ of a formula $A$ is
  inductively defined as follows:

  \begin{enumerate}
  \item $\elembases{A} = A$, if $A \in \bioformulas$;
  \item $\elembases{A \otimes B} = \elembases{A}, \elembases{B}$;
  \item $\elembases{A \rightarrow_{\reactlist{}}^S B} = S$.
  \end{enumerate}
\end{definition}

Control sets are defined as before, that is, as sets of linear contexts (lists)
of atomic formulas of the bonding language. We now define the type of lists that
will be used as part of conditional formulas.

\begin{definition}[Reaction list]
  \begin{enumerate}
  \item A \emph{reaction list} is a list of pairs where the first component is
    an elementary base, and the second component is a control set. Concatenation
    of reaction lists, written $\listplus{l_1}{l_2}$, is defined in the usual
    way;
  \item A context $\Delta$ is said to \emph{respect} a reaction list $l$,
    written $\resplist{\Delta}{l}$, if

    \[
      \forall (\nabla, \ctrlset{})\in l, \respects{\nabla, \Delta}{\ctrlset{}}
    \]

  \item The operation of adjoining an elementary base $\Delta$ to a reaction
    list $l$, written $\basepluslist{\Delta}{l}$, is recursively defined as
    follows:

    \begin{align*}
      \basepluslist{\Delta}{[]} &= [] \\
      \basepluslist{\Delta}{(\nabla;\ctrlset{}):l} &= (\Delta,\nabla;\ctrlset{}):\basepluslist{\Delta}{l}
    \end{align*}
  \end{enumerate}
\end{definition}

As the meaning of our conditional operator is tightly related to its
meta-linguistic counterpart, we posticipate giving its intuitive meaning until
the definition of the provability relation that we will consider, and the
related inference rules.

\subsection{The calculus \eznd{}}

We now give an alternative formalization of the rules of Zsyntax, that uses the
modified conditional operator. As it will be clear, our modification adds some
expressive power when conditionals are involved, since reactions (like the one
given in the previous section) that are not provable in \znd{} will be in
\eznd{}.  The resulting lack of soundness w.r.t. \znd{} is both obvious and not
problematic. Obvious since we have understood that the Z-conditional of \znd{}
and the one given here are of different strenghts, as asserting that a reaction
with particular characteristics is valid is a stronger statement than just
asserting that one exists. It follows that some reactions that are provable in
\eznd{} will not be provable in \znd{}. This leads to the reason why such lack
of soundness is not problematic, as more expressiveness is obviously better than
less.

Having said that, it is nevertheless reasonable to ask ourselves how to express
Zsyntax assertions in our language. We discuss this point in subsequent
sections, but it is already clear that if we were able to reintroduce the
original Z-conditional operator in addition to the new one, the resulting
calculus would turn out to be strictly stronger than, and complete with respect
to, \znd{}.

\eznd{} proofs involve a judgement which is basically the provability relation
of Zsyntax as in \cite{adding-logic}, decorated with information about the
series of biological reactions that were used to establish the proof. Such
\emph{history} is represented in terms of the previously defined reaction lists.

\begin{definition}[Enhanced provability judgement]
  Enhanced provability judgements are judgements of the form

  \[
    \Gamma \models_{\reactlist{}} \Delta
  \]

  where $\Gamma, \Delta$ are lists of formulas, and $\reactlist{}$ is a reaction
  list.
\end{definition}

Here, the reaction list $l$ gives a kind of history of reactions that brought
from $\Gamma$ to $\Delta$: roughly, to a pair $(\nabla, \ctrlset{}) \in l$ we
associate the information that $\Gamma$, during its transition to become
$\Delta$, reached a Z-state which elementary base was $\nabla$, and used a
transition with control set $\ctrlset{}$ to advance to the next intermediate
aggregate.  In other words, the reaction list $l$ keeps track of the
intermediate forms that $\Gamma$ assumed in its transition to $\Delta$, and the
control sets of the conditional formulas that allowed such intermediate
reactions. See the definition of the calculus for a better, more precise
intuition on what information reaction lists encode.

The intuitive meaning of affirming a judgement of the form
$\Delta \models_{\reactlist{}} \nabla$ is to know that there exists a sequence
of \eznd{} rule applications with which it is possible to reach the Z-state
$\Gamma, \nabla$ from the Z-state $\Gamma, \Delta$, for any $\Gamma$ that is
compatible to $l$, or \emph{respects} it. Intuitively, this means that $\Gamma$
does not interfere with any of the intermediate reactions that bring $\Delta$ to
$\nabla$.

The reaction list annotations on the conditional operator have the same meaning
as the ones used in the provability relation, as the former is just an object
language internalization of the latter, meta-linguistic concept.  Elementary
base annotations in a conditional formula, say $A \rightarrow^S B$, are instead
used to keep track of the real biological nature of the element of type
$A \rightarrow B$ that we are considering.
In particular, to know that $A \rightarrow^S B$ is true is to know that $S, A
\models B$. This interpretation justifies our definition of elementary base:
since an element of type $A \rightarrow^S B$ really represents an aggregate with
elementary base $S$, it makes perfect sense to consider $S$ as the elementary
base of $A \rightarrow^S B$.

We now take the calculus \znd{} given in \cite{adding-logic} as a model to define our
modified calculus \eznd{}, which will be formed by inference rules having as
premises and conclusion judgements of the form
$\Gamma \models_{\reactlist{}} \Delta$.

\paragraph{Empirical rules}

Empirical rules are not part of the calculus, but are treated as axioms as
hinted at the end of Section 5 of \cite{adding-logic}. In our system, to
postulate that $A$ reacts into $B$ provided that the control set $\ctrlset{}$ is
respected can be done by affirming that the following formula is true in the
empty context:

\[
  A \rightarrow_{(\emptyset, \ctrlset{}):[]}^\emptyset B
\]

This is coherent with our interpretation of the annotations of conditional
operators. An axiom is indeed by definition true in the empty context (hence the
empty elementary base in the formula), and represents an atomic transition
(hence the empty elementary base in the reaction list, as an atomic transition
has no intermediate steps) which can happen unless $\ctrlset{}$ is respected
(hence the control set $\ctrlset{}$ in the reaction list.)

As axioms can be used \emph{ad libitum} in a proof, we need a way to introduce
them at any stage:

\[
  \begin{prooftree}
    A \ \text{axiom}
    \justifies
    \Gamma \models_{[]} \Gamma, A
  \end{prooftree}
\]

\paragraph{$\rightarrow$ introduction}

We have the following rule:

\[
  \begin{prooftree}
    \Gamma, A \models_{\reactlist{}} B
    \justifies
    \Gamma, \Delta \models_{[]}
    A \rightarrow_{\reactlist{}}^{\elembases{\Gamma}} B, \Delta
  \end{prooftree}
\]

Notice how the information on the reaction that was proved to establish the
truth of the conditional just introduced is preserved locally in the formula
itself. Also, notice how the empty reaction list in the conclusion represent the
fact that going from $\Gamma$ to $A \rightarrow B$ does not involve any
biological reaction.

\paragraph{$\rightarrow$ elimination}

The rule in \cite{adding-logic}

\[
  \begin{prooftree}
    \Gamma, A \rightarrow B, A \; : \;
    \elembases{\Gamma} \cap \ctrlset{A \rightarrow B}^* = \emptyset
    \justifies
    \Gamma, B
  \end{prooftree}
\]

Can be translated to our formalism as follows:

\[
  \begin{prooftree}
    \resplist{\Gamma}{\reactlist{}}
    \justifies
    \Gamma, A \rightarrow_{\reactlist{}}^S B, A
    \models_{\basepluslist{\Gamma}{\reactlist{}}} \Gamma, B
  \end{prooftree}
\]

The intuition behind the reaction list in the conclusion is quite simple.  The
conditional witnesses the fact that we can transition from $A$ to $B$ with some
intermediate reactions described by $\reactlist{}$, in every context $\Gamma$
that respects them. Therefore, we can safely add $\Gamma$ to the initial
aggregate to get to $\Gamma, B$ using the conditional at hand.

The reaction list of the overall transition that results from this is composed
of the initial reaction list $\reactlist{}$ of the conditional that has been
used, with the addition of (the elementary base of) $\Gamma$ in all its
components. This is to represent the fact that the intermediate reactions of the
conclusion are just those of $\reactlist{}$ with the addition of a passive
context $\Gamma$ alongside.

\paragraph{$\otimes$ introduction and elimination}

The $\otimes$ introduction and elimination rules are straighforward:

\[
  \begin{prooftree}
    \justifies
    \Gamma, A, B \models_{[]} \Gamma, A \otimes B
  \end{prooftree}
\]

\[
  \begin{prooftree}
    \justifies
    \Gamma, A \otimes B \models_{[]} \Gamma, A, B
  \end{prooftree}
\]

\paragraph{Identity and composition}

Obviously, the trivial transition is allowed:

\[
  \begin{prooftree}
    \justifies
    \Gamma \models_{[]} \Gamma
  \end{prooftree}
\]

Moreover, in order to have the $\models$ relation represent arbitrary sequences
of Zsyntax rule applications, we need a way to compose such sequences:

\[
  \begin{prooftree}
    \Gamma \models_{\reactlist{1}} \Delta
    \qquad
    \Delta \models_{\reactlist{2}} \nabla
    \justifies
    \Gamma \models_{\listplus{\reactlist{1}}{\reactlist{2}}} \nabla
  \end{prooftree}
\]

We can formalize our intuition about reaction list annotations and controlled
monotonicity with the following rule:

\[
  \begin{prooftree}
    \Gamma \models_{\reactlist{}} \Delta
    \qquad
    \resplist{\nabla}{\reactlist{}}
    \justifies
    \Gamma, \nabla \models_{\basepluslist{\nabla}{\reactlist{}}} \Delta, \nabla
  \end{prooftree}
\]

which is obviously close, both in its form and in its justification, to the
$\rightarrow$ elimination rule. The admissibility of this rule is justified by a
simple examination of the rules of the calculus, so we skip a detailed
proof.

\subsection{Restoring the original Zsyntax conditional operator}

As already explained, this alternative conditional is not logically equivalent
to the one in \cite{adding-logic}. To restore the original Z-conditional, and
thus the possibility of expressing formulas and proofs in the original language
of Zsyntax, it is sufficient to restore what our conditional removed: the
existential quantification. That is, the original Z-conditional can be obtained
from ours by just existentially quantifying over the additional data regarding
elementary bases and reactions lists in our conditional operator. To do this, we
can consider our conditional as a predicate over terms of type ``elementary
base'' and ``reaction list'', where these terms can of course be arbitrary
variables that can be bound by quantifiers.

\[
  A \rightarrow B \quad \equiv \quad \exists e l . A \rightarrow_l^e B
\]

Then, the original introduction rule becomes a derived rule composed of
$\rightarrow \mathcal{I}$ followed by $\exists \mathcal{I}$:

\[
  \begin{prooftree}
    \[
      \[
        \Gamma, A
        \leadsto
        B
      \]
      \qquad
      \Gamma, \Delta
      \justifies
      A \rightarrow_{\mathrm{L}}^{\elembases{\Gamma}} B, \Delta
      \using{\rightarrow \mathcal{I}}
    \]
    \justifies
    \exists s l . A \rightarrow_{l}^{s} B, \Delta
    \quad \equiv \quad
    A \rightarrow B, \Delta
    \using{\exists \mathcal{I}}
  \end{prooftree}
\]

where $\mathrm{L}$ represents the concrete reaction list associated with the
deduction $\Gamma, A \vdash B$.
In order to understand how this new conditional, formulated as a predicate,
interacts with the other rules and relates to the original Z-conditional
operator, we must extend the definition of elementary base to account for the
presence of free variables. To start with, we should extend the definition of
elementary base to quantifiers as follows:

\[
  \elembases{\exists x . A} = \elembases{A}
\]

As free variables represent a kind of metalinguistic universal quantification, a
formula $A \rightarrow^e B$ represents a transition with an arbitrary elementary
base, so that any use of it must work out correctly for any possible
substitution of $e$ for a concrete elementary base.  As our logical system is
meant to reflect a world characterized by a dynamically growing knowledge, it
makes sense to make variables range over what is known \emph{at that current
  moment}. Under this interpretation, a natural definition of elementary base
for $A \rightarrow^e B$ if one that considers the elementary bases for all
formulas of that type that are known at the moment. This requires us to consider
elementary bases as sets of contexts rather than simple contexts, leading to a
definition that turns out to correspond to the one given in \cite{adding-logic}

\[
  \elembases{A \rightarrow^e B} =
  \{ \Delta \, | \, A \rightarrow^{\Delta} B \ \text{is known to be valid} \}
\]

The correspondence is not perfect, given that \cite{adding-logic} considers when
$\Delta, A \models B$ is known to hold, rather than $A \rightarrow^{\Delta} B$.
Our definition does make sense, however: if $\Delta, A \models B$ holds, then
surely $A \rightarrow^{\Delta} B$ is valid.  It is not very clear though how a
theorem proving software is supposed to discover new facts, like the validity of
some $A \rightarrow^{\Delta} B$ (should it be under user input? should it
discover them automatically? should it do both?).

Consider now how an elimination rule could be derived, using our definition of
the original Z-conditional as an existentially quantified one:

\[
  \begin{prooftree}
    \Gamma, \exists l . A \rightarrow_l B, A
    \, \equiv \,
    \Gamma, A \rightarrow B, A
    \qquad
    %\resplist{\Gamma}{l}
    (\star)
    \quad
    \[
      \Gamma, A \rightarrow_l B, A
      \qquad
      \resplist{\Gamma}{l}
      \justifies
      \Gamma, B
      \using{\rightarrow \mathcal{E}}
    \]
    \justifies
    \Gamma, B
    \using{\exists \mathcal{E}}
  \end{prooftree}
\]

where $l$ is intended to be a fresh free variable that gets discharged as
part of the existential elimination rule.
To have an elimination rule that corresponds to the original one given in
\cite{adding-logic}, we have to define a side-condition, indicated
above as $(\star)$, that is logically equivalent (or sufficiently similar) to
the original one, and that entails $\resplist{\Gamma}{l}$.
The real connection between our interpretation and the original Z-conditional
thus crucially depends on the interpretation of $\resplist{\Gamma}{l}$ when $l$
is a free variable.

Continuing on the same line of reasoning as with elementary bases, to prove
something from $A \rightarrow_l B$ with $l$ free is to prove it for any
possible substitution of $l$ to non-variable
reaction lists $L$, such that
$A \rightarrow_{L} B$ is known to hold at the moment the
proof is done.
Then

\[
  (\star) \equiv
  \resplist{\Gamma}{\mathfrak{L}_{A \rightarrow B}} \equiv
  \text{for all known } A \rightarrow_{L} B,
  \; \resplist{\Gamma}{L}
\]

and similarly, extending the $\resplist{}{}$ relation to

\[
  \resplist{\Gamma}{l} \equiv
  \text{for all known } A \rightarrow_{L} B,
  \resplist{\Gamma}{L}
\]

when $A \rightarrow_l B$ and $l$ is a free variable. Notice that this is a
positive definition, in the sense that it considers what is currently known
about the system to give an interpretation of variables.
For this reason, it is different from the one given in \cite{adding-logic},
which is negative. In particular, there the following is allowed:

\[
  \begin{prooftree}
    \Gamma, A \rightarrow B, A
    \justifies
    \Gamma, B
  \end{prooftree}
\]

only when $\Gamma$ is \emph{not known} to inhibit the reaction, that is, when it
is not true that $\Gamma, A \not \models B$ is known. We therefore can
identify, in general, two possible interpretations of formulas involving free
variables,

\begin{enumerate}
\item $\Gamma, A(x) \vdash \Delta$ holds if it holds for all substitutions $t$
  for $x$ such that $A(t)$ is currently known to hold;
\item $\Gamma, A(x) \vdash \Delta$ holds if we do not have reason to reject its
  validity, that is, we do not currently know of any substitution $t$ for $x$
  such that $\Gamma, A(t) \not \vdash \Delta$.
\end{enumerate}

The first, ``positive'' interpretation is more in line with the definition of
elementary bases given above: we consider a free variable to range over all
current knowledge; hence, we consider a formula $A(x)$ valid if valid for any
possible substitution. We prefer this interpretation as it is more conservative,
in the sense that it validates formulas as theorems on the basis of known facts
justifying them, and not on the lack of facts disproving them.

Implementation-wise, this positive approach of course works well only if the
theorem prover is very clever in discovering new things and adding them to the
set of known facts. Otherwise, the user may see some obviously valid sequents
being rejected, only because the software was not able to collect enough facts
to prove them valid. Similarly it is again not very clear, if one choses to
implement the second option, how the theorem prover is supposed to gain
knowledge about transitions that do \emph{not} hold.

% TODO: expand on this.
% Moreover, the two
% definitions would actually correspond in an automated theorem prover if we
% assumed that what the system considers to be known is all and exactly all that
% is really known.

Besides considerations about which of the two possible interpretations is more
appropriate, the extension of the logic presented in this report with free
variables and quantifiers is mainly a tiresome rather than conceptually
challenging work, and it should be possible to adapt the work of
\cite{chaudhuri-thesis} to our case. One significant difficulty has to
do with how to compute the domain over which free variables range \emph{during a
  certain proof}. Consider a principal cut involving an existentially quantified
conditional that is first introduced and then eliminated:

\[
  \begin{prooftree}
    \[
      \Gamma \Longrightarrow A \rightarrow_{\mathrm{L}}^{\mathrm{E}} B
      \justifies
      \Gamma \Longrightarrow \exists e l . A \rightarrow_l^e B
    \]
    \qquad
    \[
      \Delta, A \rightarrow_l^e B \Longrightarrow C
      \justifies
      \Delta, \exists e l . A \rightarrow_l^e B \Longrightarrow C
    \]
    \justifies
    \Gamma, \Delta \Longrightarrow C
  \end{prooftree}
\]

where $\mathrm{E}, \mathrm{L}$ just stand for arbitrary non-variable elementary
bases and reaction lists. To be able to move the cut to the premises, the
premise on the right must have been derived with its free variables ranging over
a domain that \strong{includes} knowledge about the transition $A \rightarrow B$
that is proved in the left premise. In other words, to make the cut elimination
theorem go through, we need some parts of a derivation to know about other parts
of the \emph{same} derivation. This requires us to carefully assign an explicit
temporal order to branches of a derivation, as the validity of parts of it may
crucially depend on facts that are established in others that are intended to
come before.

At this point, we could ask ourselves if it would have been better to just
consider the original Z-conditional with implicit quantification, and shape the
calculus around it without an additional, more informative operator. Firstly,
notice that this would not have resulted in a simpler logic. Getting to a
satisfactory definition and implementation of the Z-conditional operator
\emph{is} an inevitably difficult task; this work just exposes why it is so, it
does not introduce any difficulty in itself. The problem with cut elimination
described above, for example, would have arisen in the same way if we considered
cuts of Z-conditionals in the original sense of \cite{adding-logic}, since it
has to do with the interaction between quantifiers and the dynamic
interpretation of the domain over which variables range. The only difference is
that in our case, existential quantification and free variables are made
explicit.

Secondly, by hiding quantification in the informal semantic explanation of the
conditional operator, we would have removed useful expressive power from the
logic. A transition $A \rightarrow B$ given as an axiom is, for example,
fundamentally different from the same transition obtained after a long proof
involving several intermediate reactions. In the formalism presented here this
difference is evident, as the two formulas obviously would end up having very
different reaction lists. The ability to differentiate between such apperently
identical biological types yields a more expressive language, capable of proving
facts that would have been unprovable otherwise, as explained in
Section~\ref{sec:zsyntax}.

The approach proposed here is thus based on the general view, which in many ways
characterizes constructive mathematics (\cite{richman}, \cite{unif}), according
to which it is better to start with a more informative logic, and then introduce
ways to \emph{selectively} generalize it only when really needed and with very
little cost, for example by quantifying the elementary base and reaction list
data on the conditional opearator to restore the semantics of the original
Z-conditional. Conversely, it is much more expensive to try to recover
information that we failed to express in the first place.

As we just explained, a proper implementation of quantifiers and free variables
requires significant efforts both from logical/proof-theoretic and from the
implementation standpoints. The first is needed to reach a proper understanding
of what quantifiers and free variables should \emph{mean} in a formalism like
Zsyntax. The second has to do with technicalities like unification and the
extension of the proof theory presented here to account for free variables,
which, while being not very interesting from a philosophical point of view, are
nevertheless quite time-consuming. This is indeed one of the reasons why
quantifiers had to be left out from the work presented here and its associated
implementation.

%%% Local Variables:
%%% mode: latex
%%% TeX-master: "../docs"
%%% End:
