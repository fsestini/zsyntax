\section{Implementation details}

\subsection{The $\relsym$ type}

As mentioned before, inference rules are implemented as sort of ``currified''
functions that take a single premise sequent as input and return either the
conclusion sequent (if there are no more premises in the rule) or a new,
partially applied rule that waits for the next premises.

In particular, a rule may be

\begin{itemize}
\item Zero-ary, that is, just a conclusion sequent;
\item Unary, that is, a function from the type of sequents to the type of rules.
\end{itemize}

However, rules are really matched against input sequents, and these matches may
fail. So to model a rule as a function from the type of sequents to the type of
rules, we have to include a third case of ``failing rules'' as elements of the
type itself.  All these three scenarios, as well as the particular types
involved, can be abstracted away and modelled as elements of the following
recursive type, that represents ``currified'' relations betweeen input elements
of type $A$ and output elements of type $B$.

\[
  \relty{A}{B} = 1 + (B + (A \rightarrow \relty{A}{B}))
\]

\[
  \mu : \relty{A}{(\relty{A}{B})} \to \relty{A}{B}
\]
\begin{align*}
  \mu\ \inl \star & = \inl \star \\
  \mu\ \inr(\inl{} c) & = c \\
  \mu\ \inr(\inr{} g) & = \inr(\inr(\mu \circ g))
\end{align*}

%%% Local Variables:
%%% mode: latex
%%% TeX-master: "../docs"
%%% End:
