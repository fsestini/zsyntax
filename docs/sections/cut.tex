\section{Appendix: lengthy proofs}

\subsection{Theorem~\ref{backwardcutelim}}

The proof is inspired by the one given in [paper] for the full linear
logic. In particular, we consider four ``classes'' of cuts and show
admissibility for all of them. The theorem follows from the fact that these
four classes cover all possible uses of the cut rule.
We sometimes omit the side conditions about reaction lists and control sets
when trivially satisfied.

\begin{description}
\item[Identity cuts] One of the premises is obtained by an application of the
  identity axiom.

  \begin{enumerate}
  \item Left case.

    \[
      \begin{prooftree}
        \[ \justifies \zsyseq{\Gamma}{A}{[]}{A} \]
        \qquad
        \zsyseq{\Gamma}{\Delta_2, A}{\reactlist{2}}{B}
        \justifies
        \zsyseq{\Gamma}{A, \Delta_2}{\reactlist{2}}{B}
      \end{prooftree}
    \]

    Then, just take the second premise.

  \item Right case.

    \[
      \begin{prooftree}
        \zsyseq{\Gamma}{\Delta_1}{\reactlist{1}}{A}
        \qquad
        \zsyseq{\Gamma}{A}{[]}{A}
        \justifies
        \zsyseq{\Gamma}{\Delta_1}{\reactlist{1}}{A}
      \end{prooftree}
    \]

    Then, just take the first premise.
  \end{enumerate}
\item[Principal cuts] The cut formula is principal in both premises. We
  distinguish the two possible cases of non-atomic formulas.

  \begin{enumerate}
  \item Case $\otimes$.


    \[
      \begin{prooftree}
        \[
          \zsyseq{\Gamma}{\Delta_1}{[]}{A}
          \qquad
          \zsyseq{\Gamma}{\Delta_2}{[]}{B}
          \justifies
          \zsyseq{\Gamma}{\Delta_1, \Delta_2}{[]}{A \otimes B}
        \]
        \qquad
        \[
          \zsyseq{\Gamma}{\Delta_3, A, B}{\reactlist{}}{C}
          \justifies
          \zsyseq{\Gamma}{\Delta_3, A \otimes B}{\reactlist{}}{C}
        \]
        \justifies
        \zsyseq{\Gamma}{\Delta_1, \Delta_2}{\reactlist{}}{C}
      \end{prooftree}
    \]

    Then, by inductive hypothesis, we have

    \[
      \begin{prooftree}
        \zsyseq{\Gamma}{\Delta_2}{[]}{B}
        \qquad
        \[
          \zsyseq{\Gamma}{\Delta_1}{[]}{A}
          \qquad
          \zsyseq{\Gamma}{\Delta_3, A, B}{\ctrlset{}}{C}
          \justifies
          \zsyseq{\Gamma}{\Delta_1, \Delta_3, B}{\ctrlset{}}{C}
        \]
        \justifies
        \zsyseq{\Gamma}{\Delta_1, \Delta_2}{\ctrlset{}}{C}
      \end{prooftree}
    \]

  \item Case $\rightarrow_{\reactlist{}}^S$.

    \[
      \begin{prooftree}
        \[
          \zsyseq{\Gamma}{\Delta_1, A}{\reactlist{1}}{B}
          \justifies
          \zsyseq{\Gamma}{\Delta_1}{[]}{A \rightarrow_{\reactlist{1}}^{\elembases{\Delta_1}} B}
        \]
        \qquad
        \[
          \zsyseq{\Gamma}{\Delta_2}{[]}{A}
          \qquad
          \zsyseq{\Gamma}{\Delta_3, B}{\reactlist{2}}{C}
          \justifies
          \zsyseq{\Gamma}{\Delta_2, \Delta_3, A
            \rightarrow_{\reactlist{1}}^{\elembases{\Delta_1}}
            B}{\baseandplus{\Delta_3}{\reactlist{1}}{\reactlist{2}}}{C}
          \using{\resplist{\Delta_3}{\reactlist{1}}}
        \]
        \justifies
        \zsyseq{\Gamma}{\Delta_1, \Delta_2, \Delta_3}{
          \baseandplus{\Delta_3}{\reactlist{1}}{\reactlist{2}}
        }{C}
      \end{prooftree}
    \]

    Then, by inductive hypothesis, we have

    \[
      \begin{prooftree}
        \[
          \zsyseq{\Gamma}{\Delta_2}{[]}{A}
          \qquad
          \zsyseq{\Gamma}{\Delta_1, A}{\reactlist{1}}{B}
          \justifies
          \zsyseq{\Gamma}{\Delta_1, \Delta_2}{\reactlist{1}}{B}
        \]
        \quad
        \zsyseq{\Gamma}{\Delta_3, B}{\reactlist{2}}{C}
        \justifies
        \zsyseq{\Gamma}{\Delta_1, \Delta_2, \Delta_3}{
          \baseandplus{\Delta_3}{\reactlist{1}}{\reactlist{2}}
        }{C}
        \using{\resplist{\Delta_3}{\reactlist{1}}}
      \end{prooftree}
    \]

  \end{enumerate}
\item[Left-commutative cuts] The cut formula is a side formula in the left
  premise. The cut is then simply moved up to the premises, where the
  inductive hypothesis is used.

  \begin{enumerate}
  \item Case $\otimes L$.

    \[
      \begin{prooftree}
        \[
          \zsyseq{\Gamma}{\Delta_1, C, D}{\reactlist{1}}{A}
          \justifies
          \zsyseq{\Gamma}{\Delta_1, C \otimes D}{\reactlist{1}}{A}
          \using{\otimes L}
        \]
        \qquad
        \zsyseq{\Gamma}{\Delta_2, A}{\reactlist{2}}{B}
        \justifies
        \zsyseq{\Gamma}{\Delta_1, \Delta_2, C \otimes D}{
          \baseandplus{\Delta_2}{\reactlist{1}}{\reactlist{2}}
        }{B}
        \using{\resplist{\Delta_2}{\reactlist{1}}}
      \end{prooftree}
    \]

    becomes

    \[
      \begin{prooftree}
        \[
          \zsyseq{\Gamma}{\Delta_1, C, D}{\reactlist{1}}{A}
          \qquad
          \zsyseq{\Gamma}{\Delta_2, A}{\reactlist{2}}{B}
          \justifies
          \zsyseq{\Gamma}{\Delta_1, \Delta_2, C, D}{
            \baseandplus{\Delta_2}{\reactlist{1}}{\reactlist{2}}
          }{B}
          \using{\resplist{\Delta_2}{\reactlist{1}}}
        \]
        \justifies
        \zsyseq{\Gamma}{\Delta_1, \Delta_2, C \otimes D}{
          \baseandplus{\Delta_2}{\reactlist{1}}{\reactlist{2}}
        }{B}
        \using{\otimes L}
      \end{prooftree}
    \]

  \item Case $\copyrule$.

    \[
      \begin{prooftree}
        \[
          \zsyseq{\Gamma, C}{\Delta_1, C}{\reactlist{1}}{A}
          \justifies
          \zsyseq{\Gamma, C}{\Delta_1}{\reactlist{1}}{A}
        \]
        \qquad
        \zsyseq{\Gamma, C}{\Delta_2, A}{\reactlist{2}}{B}
        \justifies
        \zsyseq{\Gamma, C}{\Delta_1, \Delta_2}{
          \baseandplus{\Delta_2}{\reactlist{1}}{\reactlist{2}}
        }{B}
        \using{\resplist{\Delta_2}{\reactlist{1}}}
      \end{prooftree}
    \]

    becomes

    \[
      \begin{prooftree}
        \[
          \zsyseq{\Gamma, C}{\Delta_1, C}{\reactlist{1}}{A}
          \qquad
          \zsyseq{\Gamma, C}{\Delta_2, A}{\reactlist{2}}{B}
          \justifies
          \zsyseq{\Gamma, C}{\Delta_1, C, \Delta_2}{
            \baseandplus{\Delta_2}{\reactlist{1}}{\reactlist{2}}
          }{B}
          \using{\resplist{\Delta_2}{\reactlist{1}}}
        \]
        \justifies
        \zsyseq{\Gamma, C}{\Delta_1, \Delta_2}{
          \baseandplus{\Delta_2}{\reactlist{1}}{\reactlist{2}}
        }{B}
        \using{\copyrule}
      \end{prooftree}
    \]

  \item Case $\rightarrow L$.

    \[
      \begin{prooftree}
        \[
          \zsyseq{\Gamma}{\Delta_1}{[]}{C}
          \qquad
          \zsyseq{\Gamma}{\Delta_2, D}{\reactlist{1}}{A}
          \justifies
          \zsyseq{\Gamma}{\Delta_1, \Delta_2, C \rightarrow_{\reactlist{}}^S D
          }{
            \baseandplus{\Delta_2}{\reactlist{}}{\reactlist{1}}
          }{A}
          \using{\resplist{\Delta_2}{\reactlist{}}}
        \]
        \quad
        \zsyseq{\Gamma}{\Delta_3, A}{\reactlist{2}}{B}
        \justifies
        \zsyseq{\Gamma}{\Delta_1, \Delta_2, \Delta_3, C
          \rightarrow_{\reactlist{}}^S D}{
          \baseandplus{\Delta_3}{(\baseandplus{\Delta_2}{\reactlist{}}{\reactlist{1}})}{l_2}
        }{B}
        \using{
          \resplist{\Delta_3}{
            (\baseandplus{\Delta_2}{\reactlist{}}{\reactlist{1}})}
        }
      \end{prooftree}
    \]

    Since
    $\resplist{\Delta_3}{
      (\baseandplus{\Delta_2}{\reactlist{}}{\reactlist{1}})}$, we also have
    $\resplist{\Delta_3}{\reactlist{1}}$ and
    $\resplist{\Delta_3}{\basepluslist{\Delta_2}{\reactlist{}}}$.  But then, by
    predicate logic and the definitions regarding reaction lists,
    $\resplist{\Delta_3}{\basepluslist{\Delta_2}{\reactlist{}}}$ is equivalent to

    \[
      \forall (\nabla, \ctrlset{}) \in l,
      \respects{(\elembases{\Delta_3}, \elembases{\Delta_2})}{\ctrlset{}}
    \]

    that is, $\resplist{(\Delta_3,\Delta_2)}{\reactlist{}}$. Therefore,

    \[
      \begin{prooftree}
        \zsyseq{\Gamma}{\Delta_1}{[]}{C}
        \quad
        \[
          \zsyseq{\Gamma}{\Delta_2, D}{\reactlist{1}}{A}
          \qquad
          \zsyseq{\Gamma}{\Delta_3, A}{\reactlist{2}}{B}
          \justifies
          \zsyseq{\Gamma}{\Delta_2, \Delta_3, D}{
            \baseandplus{\Delta_3}{\reactlist{1}}{\reactlist{2}}
          }{A}
          \using{\resplist{\Delta_3}{ \reactlist{1}}}
        \]
        \justifies
        \zsyseq{\Gamma}{\Delta_1, \Delta_2, \Delta_3, C
          \rightarrow_{\reactlist{}}^S D}{
          \baseandplus{(\Delta_2,\Delta_3)}{l}{(\baseandplus{\Delta_3}{\reactlist{1}}{\reactlist{2}})}
        }{B}
        \using{
          \resplist{(\Delta_3,\Delta_2)}{\reactlist{}}
        }
      \end{prooftree}
    \]

    Notice that the equivalences below follow by set theory and predicate
    logic, and by associativity of list concatenation:

    \begin{align*}
      \baseandplus{(\Delta_2,\Delta_3)}{l}{(\baseandplus{\Delta_3}{\reactlist{1}}{\reactlist{2}})}
      & = \listplus{
        (\baseandplus{(\Delta_2,\Delta_3)}{l}{\basepluslist{\Delta_3}{\reactlist{1}}})
        }{
        \reactlist{2}
        } \\
      & = \listplus{
        (
        \basepluslist{\Delta_3}{(\baseandplus{\Delta_2}{\reactlist{}}{\reactlist{1}})}
        % \baseandplus{(\Delta_2,\Delta_3)}{l}{\basepluslist{\Delta_3}{\reactlist{1}}}
        )
        }{
        \reactlist{2}
        }
    \end{align*}

    Hence the thesis.

  \end{enumerate}

\item[Right-commutative cuts] The cut formula is a side formula in the right
  premise.

  \begin{enumerate}
  \item Case $\otimes L$.

    \[
      \begin{prooftree}
        \zsyseq{\Gamma}{\Delta_1}{\reactlist{1}}{A}
        \qquad
        \[
          \zsyseq{\Gamma}{\Delta_2, C, D, A}{\reactlist{2}}{B}
          \justifies
          \zsyseq{\Gamma}{\Delta_2, C \otimes D, A}{\reactlist{2}}{B}
        \]
        \justifies
        \zsyseq{\Gamma}{\Delta_1, \Delta_2, C \otimes D}{
          \baseandplus{\Delta_2}{\reactlist{1}}{\reactlist{2}}
        }{B}
        \using{\resplist{(\Delta_2, C \otimes D)}{\reactlist{1}}}
      \end{prooftree}
    \]

    Recalling the definition of elementary base, we have
    $\elembases{\Delta_2, C\otimes D} = \elembases{\Delta_2, C, D}$, hence
    $C\otimes D$ and $C,D$ are safely interchangeable in every context
    above. Therefore, we have


    \[
      \begin{prooftree}
        \[
          \zsyseq{\Gamma}{\Delta_1}{\reactlist{1}}{A}
          \qquad
          \zsyseq{\Gamma}{\Delta_2, C, D, A}{\reactlist{2}}{B}
          \justifies
          \zsyseq{\Gamma}{\Delta_1, \Delta_2, C, D}{
            \baseandplus{\Delta_2}{\reactlist{1}}{\reactlist{2}}
          }{B}
          \using{\resplist{(\Delta_2, C, D)}{\reactlist{1}}}
        \]
        \justifies
        \zsyseq{\Gamma}{\Delta_1, \Delta_2, C \otimes D}{
          \baseandplus{\Delta_2}{\reactlist{1}}{\reactlist{2}}
        }{B}
      \end{prooftree}
    \]

  \item Case $\copyrule$.


    \[
      \begin{prooftree}
        \zsyseq{\Gamma,C}{\Delta_1}{\reactlist{1}}{A}
        \qquad
        \[
          \zsyseq{\Gamma,C}{\Delta_2, C, A}{\reactlist{2}}{B}
          \justifies
          \zsyseq{\Gamma,C}{\Delta_2, A}{\reactlist{2}}{B}
          \using{\copyrule}
        \]
        \justifies
        \zsyseq{\Gamma,C}{\Delta_1, \Delta_2}{
          \baseandplus{\Delta_2}{\reactlist{1}}{\reactlist{2}}
        }{B}
        \using{\resplist{\Delta_2}{\reactlist{1}}}
      \end{prooftree}
    \]

    Then, by inductive hypothesis, we have

    \[
      \begin{prooftree}
        \[
          \zsyseq{\Gamma,C}{\Delta_1}{\reactlist{1}}{A}
          \qquad
          \zsyseq{\Gamma,C}{\Delta_2, C, A}{\reactlist{2}}{B}
          \justifies
          \zsyseq{\Gamma,C}{\Delta_2, C}{
            \baseandplus{(\Delta_2, C)}{\reactlist{1}}{\reactlist{2}}
          }{B}
          \using{\resplist{(\Delta_2, C)}{\reactlist{1}}}
        \]
        \justifies
        \zsyseq{\Gamma,C}{\Delta_1, \Delta_2}{
          \baseandplus{(\Delta_2, C)}{\reactlist{1}}{\reactlist{2}}
        }{B}
        \using{\copyrule}
      \end{prooftree}
    \]

    Recall that, by assumption, every formula $C$ in the unrestricted context
    is such that $\elembases{C} = \emptyset$. In particular, this means that
    $\Delta_2,C$ is equivalent to $\Delta_2$ in every context of the original
    derivation, since $\elembases{\Delta_2,C} = \elembases{\Delta_2}$. It
    follows that the derivation above is clearly valid.

  \item Case $\rightarrow L$.

    We can assume the cut formula to be active in the left premise, for
    otherwise we would treat the cut as a left-commutative case.  Then, the
    left premise must be the conclusion of either a $\rightarrow R$ or a
    $\otimes R$ rule. Both rules conclude with an empty list, so we can assume
    the list of the left premise to be empty. We distinguish two cases based
    on the position of the cut formula in the derivation of right premise.

    \begin{enumerate}
    \item Case the cut formula ends up in the left premise of the cut.
      \[
        \begin{prooftree}
          \zsyseq{\Gamma}{\Delta}{[]}{A}
          \qquad
          \[
            \zsyseq{\Gamma}{\Delta_1, A}{[]}{C}
            \qquad
            \zsyseq{\Gamma}{\Delta_2, D}{\reactlist{2}}{B}
            \justifies
            \zsyseq{\Gamma}{\Delta_1, \Delta_2, A, C \rightarrow_{\reactlist{}}^S
              D}{
              \baseandplus{\Delta_2}{\reactlist{}}{\reactlist{2}}
            }{B}
            \using{\resplist{\Delta_2}{\reactlist{}}}
          \]
          \justifies
          \zsyseq{\Gamma}{\Delta, \Delta_1, \Delta_2,
            C \rightarrow_{\reactlist{}}^S D
          }{
            \baseandplus{\Delta_2}{\reactlist{}}{\reactlist{2}}
          }{B}
        \end{prooftree}
      \]

      Then, by inductive hypothesis

      \[
        \begin{prooftree}
          \[
            \zsyseq{\Gamma}{\Delta}{[]}{A}
            \qquad
            \zsyseq{\Gamma}{\Delta_1, A}{[]}{C}
            \justifies
            \zsyseq{\Gamma}{\Delta, \Delta_1}{[]}{C}
          \]
          \quad
          \zsyseq{\Gamma}{\Delta_2, D}{\reactlist{2}}{B}
          \justifies
          \zsyseq{\Gamma}{\Delta, \Delta_1, \Delta_2,
            C \rightarrow_{\reactlist{}}^S D
          }{
            \baseandplus{\Delta_2}{\reactlist{}}{\reactlist{2}}
          }{B}
        \end{prooftree}
      \]

    \item Case the cut formula ends up in the right premise of the derivation
      of the right premise of the cut.

      \[
        \begin{prooftree}
          \zsyseq{\Gamma}{\Delta}{[]}{A}
          \qquad
          \[
            \zsyseq{\Gamma}{\Delta_1}{[]}{C}
            \qquad
            \zsyseq{\Gamma}{\Delta_2, A, D}{\reactlist{2}}{B}
            \justifies
            \zsyseq{\Gamma}{\Delta_1, \Delta_2, A, C \rightarrow_{\reactlist{}}^S
              D}{
              \baseandplus{(\Delta_2, A)}{\reactlist{}}{\reactlist{2}}
            }{B}
            \using{\resplist{(\Delta_2, A)}{\reactlist{}}}
          \]
          \justifies
          \zsyseq{\Gamma}{\Delta, \Delta_1, \Delta_2,
            C \rightarrow_{\reactlist{}}^S D
          }{
            \baseandplus{(\Delta_2, A)}{\reactlist{}}{\reactlist{2}}
          }{B}
        \end{prooftree}
      \]

      Then, by inductive hypothesis

      \[
        \begin{prooftree}
          \zsyseq{\Gamma}{\Delta_1}{[]}{C}
          \qquad
          \[
            \zsyseq{\Gamma}{\Delta}{[]}{A}
            \qquad
            \zsyseq{\Gamma}{\Delta_2, A, D}{\reactlist{2}}{B}
            \justifies
            \zsyseq{\Gamma}{\Delta, \Delta_2, D}{\reactlist{2}}{B}
          \]
          \justifies
          \zsyseq{\Gamma}{\Delta, \Delta_1, \Delta_2,
            C \rightarrow_{\reactlist{}}^S D
          }{
            \baseandplus{(\Delta_2, \Delta)}{\reactlist{}}{\reactlist{2}}
          }{B}
          \using{\resplist{(\Delta_2, \Delta)}{\reactlist{}}}
        \end{prooftree}
      \]

      Which gives the thesis, recalling that if
      $\zsyseq{\Gamma}{\Delta}{[]}{A}$, then
      $\elembases{\Delta} = \elembases{A}$ by Lemma.
    \end{enumerate}

  \item Case $\rightarrow R$. We can assume, as before, that the cut formula
    is active in the left premise. Then,

    \[
      \begin{prooftree}
        \zsyseq{\Gamma}{\Delta_1}{[]}{A}
        \qquad
        \[
          \zsyseq{\Gamma}{\Delta_2, A, B}{\reactlist{}}{C}
          \justifies
          \zsyseq{\Gamma}{\Delta_2, A}{[]}{
            B \rightarrow_{\reactlist{}}^{\elembases{\Delta_2,A}} C}
        \]
        \justifies
        \zsyseq{\Gamma}{\Delta_1, \Delta_2}{
          []}{B \rightarrow_{\reactlist{}}^{\elembases{\Delta_2,A}} C}
      \end{prooftree}
    \]

    Then, by inductive hypothesis

    \[
      \begin{prooftree}
        \[
          \zsyseq{\Gamma}{\Delta_1}{[]}{A}
          \qquad
          \zsyseq{\Gamma}{\Delta_2, A, B}{\reactlist{}}{C}
          \justifies
          \zsyseq{\Gamma}{\Delta_1, \Delta_2, B}{\reactlist{}}{C}
        \]
        \justifies
        \zsyseq{\Gamma}{\Delta_1, \Delta_2}{
          []}{B \rightarrow_{\reactlist{}}^{\elembases{\Delta_1,\Delta_2}} C}
      \end{prooftree}
    \]

    which is just the thesis, recalling that if
    $\zsyseq{\Gamma}{\Delta}{[]}{A}$, then
    $\elembases{\Delta} = \elembases{A}$ by Lemma.
  \end{enumerate}
\end{description}

% \begin{lemma}\label{otimeslemma}\mbox{}
%   \begin{enumerate}
%   \item If $\ztriseq{\Gamma}{\Delta_1}{\Omega}{\emptyctrl}{A}$ and
%     $\zfocseq{\Gamma}{\Delta_2}{B}$, then
%     $\ztriseq{\Gamma}{\Delta_1, \Delta_2}{\Omega}{\emptyctrl}{A \otimes B}$;
%   \item If $\zfocseq{\Gamma}{\Delta_1}{A}$
%     $\ztriseq{\Gamma}{\Delta_2}{\Omega}{\emptyctrl}{B}$, then
%     $\ztriseq{\Gamma}{\Delta_1, \Delta_2}{\Omega}{\emptyctrl}{A \otimes B}$.
%   \end{enumerate}
% \end{lemma}
% \begin{proof}
%   In both cases, in $\Omega$ is non-empty we do a left- or right- commuting case
%   with the inductive hypothesis. Therefore, we can just assume $\Omega$ to be
%   empty.

%   \begin{enumerate}
%   \item

%     By induction on the left premise. If the last rule is a focus, the thesis
%     follows by $\otimes R$ and focus. If it is $\rightarrow R$ then $A$ is an
%     implication, and the thesis follows from a blur on the left premise and a
%     $\otimes R$. If it is a $\rightarrow L$, the thesis follows by just
%     commuting with the right premise of the $\rightarrow L$ rule instance.

%   \item Same as the first case.

%   \end{enumerate}
% \end{proof}

\subsection{Theorem~\ref{focusedcutelim}}

% \begin{theorem}\mbox{}

%   \begin{enumerate}
%   \item If $\ztriseq{\Gamma}{\Delta}{\Omega}{\ctrlset{1}}{A}$, then
%     \begin{enumerate}
%     \item If $\ztriseq{\Gamma}{\Delta'}{\Omega', A}{2}{C}$ and
%       $\respects{\Omega', \Delta'}{\ctrlset{1}}$, then

%       $\ztriseq{\Gamma}{\Delta, \Delta'}{\Omega, \Omega'}{
%         \ctrlset{1} \cup \ctrlset{2}}{C}$.
%     \item If $\ztriseq{\Gamma}{\Delta', A}{\Omega'}{2}{C}$ and
%       $\respects{\Omega', \Delta'}{\ctrlset{1}}$, then

%       $\ztriseq{\Gamma}{\Delta, \Delta'}{\Omega, \Omega'}{
%         \ctrlset{1} \cup \ctrlset{2}}{C}$;
%     \end{enumerate}

%   \item If $\ztriseq{\Gamma}{\Delta}{\Omega}{\emptyset}{A}$ and
%     $\zfocseq{\Gamma}{\Delta', A}{B}$, then

%     $\ztriseq{\Gamma}{\Delta, \Delta'}{\Omega}{\emptyset}{B}$.

%   \item If $\zfocseq{\Gamma}{\Delta}{A}$, then
%     \begin{enumerate}
%     \item If $\ztriseq{\Gamma}{\Delta'}{\Omega', A}{\ctrlset{}}{C}$, then
%       $\ztriseq{\Gamma}{\Delta'}{\Omega'}{\ctrlset{}}{C}$.
%     \item If $\ztriseq{\Gamma}{\Delta', A}{\Omega'}{\ctrlset{}}{C}$, then
%       $\ztriseq{\Gamma}{\Delta'}{\Omega'}{\ctrlset{}}{C}$.
%     \end{enumerate}

%   \end{enumerate}
% \end{theorem}
% \begin{proof}

The proof proceeds by simultaneous double induction on all six cuts, either on
the height of the derivation or on the size of cut formula. Again, we use the
proof in [cmu paper] as a model.

\paragraph{Cut 1a}

\begin{enumerate}
\item Case $\otimes L$, lact1, lact2 on the left premise are treated as left
  commutative cases similarly to [proof of cut for backward calculus].

\item Left focus.

  \[
    \begin{prooftree}
      \[
        \zfocseq{\Gamma}{\Delta}{A}
        \quad
        A\ \text{focusable}
        \justifies
        \ztriseq{\Gamma}{\Delta}{\cdot}{\emptyctrl{}}{A}
      \]
      \quad
      \ztriseq{\Gamma}{\Delta'}{\Omega', A}{\reactlist{}}{C}
      \justifies
      \ztriseq{\Gamma}{\Delta, \Delta'}{\Omega'}{\reactlist{}}{C}
    \end{prooftree}
  \]

  then, the thesis follows by induction hypothesis and cut 3a.

\item $\otimes L$, right commutative.

  \[
    \begin{prooftree}
      \ztriseq{\Gamma}{\Delta}{\Omega}{\reactlist{}}{A}
      \quad
      \[
        \ztriseq{\Gamma}{\Delta'}{\Omega', C, D, A}{\reactlist{}'}{C}
        \justifies
        \ztriseq{\Gamma}{\Delta'}{\Omega', C \otimes D, A}{\reactlist{}'}{C}
      \]
      \justifies
      \ztriseq{\Gamma}{\Delta, \Delta'}{\Omega, \Omega', C \otimes D}{
        \baseandplus{(\Delta', \Omega', C \otimes D)}{\reactlist{}}{\reactlist{}'}
      }{C}
      \using{\resplist{(\Omega', \Delta', C \otimes D)}{\reactlist{}}}
    \end{prooftree}
  \]

  then, by inductive hypothesis and remembering that
  $\resplist{(\Omega', \Delta', C \otimes D)}{\reactlist{}}$ implies
  $\resplist{(\Omega', \Delta', C, D)}{\reactlist{}}$ and
  $\baseandplus{(\Delta', \Omega', C \otimes D)}{\reactlist{}}{\reactlist{}'}
  = \baseandplus{(\Delta', \Omega', C, D)}{\reactlist{}}{\reactlist{}'}$,
  we have

  \[
    \begin{prooftree}
      \[
        \ztriseq{\Gamma}{\Delta}{\Omega}{\reactlist{}}{A}
        \qquad
        \ztriseq{\Gamma}{\Delta'}{\Omega', C, D, A}{\reactlist{}'}{C}
        \justifies
        \ztriseq{\Gamma}{\Delta, \Delta'}{\Omega, \Omega', C, D}{
          \baseandplus{(\Delta', \Omega', C, D)}{\reactlist{}}{\reactlist{}'}
        }{C}
        \using{\resplist{(\Omega', \Delta', C, D)}{\reactlist{}}}
      \]
      \justifies
      \ztriseq{\Gamma}{\Delta, \Delta'}{\Omega, \Omega', C \otimes D}{
        \baseandplus{(\Delta', \Omega', C, D)}{\reactlist{}}{\reactlist{}'}
      }{C}
    \end{prooftree}
  \]

\item Case act with cut formula as side formula are all easy right commutative
  cases.

\item Case right premise is act on the cut formula. Then the thesis follows by
  cut 1b.

\end{enumerate}

\paragraph{Cut 1b}

If $\Omega$ in the second premise is non-empty, we treat the cut as an usual
right-commutative case, similarly to [cut proof backw calculus]. Otherwise, we
can assume it to be empty.

\begin{enumerate}
\item Case $\otimes L$ and act in the left premise are all straightforward
  left commutative cases, as before.

\item Left focus.

  \[
    \begin{prooftree}
      \[
        \zfocseq{\Gamma}{\Delta}{A}
        \qquad
        A\ \text{focusable}
        \justifies
        \ztriseq{\Gamma}{\Delta}{\cdot}{\emptyctrl{}}{A}
      \]
      \quad
      \ztriseq{\Gamma}{\Delta', A}{\Omega'}{\reactlist{}}{C}
      \justifies
      \ztriseq{\Gamma}{\Delta, \Delta'}{\Omega, \Omega'}{\reactlist{}}{C}
    \end{prooftree}
  \]

  then, the thesis follows by induction hypothesis and cut 3b.

\item Case $\rightarrow L$, left commutative.

  \[
    \begin{prooftree}
      \[
        \zfocseq{\Gamma}{\Delta_1}{C}
        \qquad
        \ztriseq{\Gamma}{\Delta_2}{D}{\reactlist{1}}{A}
        \justifies
        \ztriseq{\Gamma}{\Delta_1, \Delta_2, C \rightarrow_{\reactlist{}}^S D}
        {\cdot}{
          \baseandplus{\Delta_2}{\reactlist{}}{\reactlist{1}}
        }{A}
        \using{\resplist{\Delta_2}{\reactlist{}}}
      \]
      \quad
      \ztriseq{\Gamma}{\Delta_3, A}{\cdot}{\reactlist{2}}{E}
      \justifies
      \ztriseq{\Gamma}{\Delta_1, \Delta_2, \Delta_3,
        C \rightarrow_{\reactlist{}}^S D}{\cdot}{
        \baseandplus{\Delta_3}{(\baseandplus{\Delta_2}{\reactlist{}}{\reactlist{1}})}{\reactlist{2}}
      }{E}
      \using{\resplist{\Delta_3}{
          (\baseandplus{\Delta_2}{\reactlist{}}{\reactlist{1}})}}
    \end{prooftree}
  \]


  Since
  $\resplist{\Delta_3}{
    (\baseandplus{\Delta_2}{\reactlist{}}{\reactlist{1}})}$, we also have
  $\resplist{\Delta_3}{ \reactlist{1}}$ and
  $\resplist{\Delta_3}{\basepluslist{\Delta_2}{\reactlist{}}}$. Then,
  $\resplist{(\Delta_3,\Delta_2)}{\reactlist{}}$. Also,

  \[
    \baseandplus{(\Delta_2,\Delta_3)}{l}{(\baseandplus{\Delta_3}{\reactlist{1}}{\reactlist{2}})}
    =\listplus{
      (
      \basepluslist{\Delta_3}{(\baseandplus{\Delta_2}{\reactlist{}}{\reactlist{1}})}
      )
    }{
      \reactlist{2}
    }
  \]

  Therefore, by inductive hypothesis

  \[
    \begin{prooftree}
      \zfocseq{\Gamma}{\Delta_1}{C}
      \quad
      \[
        \ztriseq{\Gamma}{\Delta_2}{D}{\reactlist{1}}{A}
        \qquad
        \ztriseq{\Gamma}{\Delta_3, A}{\cdot}{\reactlist{2}}{E}
        \justifies
        \ztriseq{\Gamma}{\Delta_2, \Delta_3}{D}{
          \baseandplus{\Delta_3}{\reactlist{1}}{\reactlist{2}}
        }{E}
        \using{\resplist{\Delta_3}{\reactlist{1}}}
      \]
      \justifies
      \ztriseq{\Gamma}{\Delta_1, \Delta_2, \Delta_3,
        C \rightarrow_{\reactlist{}}^S D}{\cdot}{
        \baseandplus{(\Delta_2,\Delta_3)}{l}{(\baseandplus{\Delta_3}{\reactlist{1}}{\reactlist{2}})}
      }{E}
      \using{\resplist{(\Delta_2, \Delta_3)}{\reactlist{}}}
    \end{prooftree}
  \]

  which is just the thesis.

\item Case $\copyrule$, left commutative. Same as $\rightarrow L$.

\item Case principal cut with $\rightarrow$.

  \[
    \begin{prooftree}
      \[
        \ztriseq{\Gamma}{\Delta}{A}{\reactlist{}}{B}
        \justifies
        \ztriseq{\Gamma}{\Delta}{\cdot}{\emptyctrl{}}
        {A \rightarrow_{\reactlist{}}^{\elembases{\Delta}} B}
      \]
      \quad
      \[
        \zfocseq{\Gamma}{\Delta'}{A}
        \qquad\qquad
        \ztriseq{\Gamma}{\Delta''}{B}{\reactlist{}'}{C}
        \justifies
        \ztriseq{\Gamma}{\Delta', \Delta'',
          A \rightarrow_{\reactlist{}}^{\elembases{\Delta}} B}{\cdot}{
          \baseandplus{\Delta''}{\reactlist{}}{\reactlist{}'}
        }{C}
        \using{\resplist{\Delta''}{\reactlist{}}}
      \]
      \justifies
      \ztriseq{\Gamma}{\Delta, \Delta', \Delta''}{\cdot}{
        \baseandplus{\Delta''}{\reactlist{}}{\reactlist{}'}
      }{C}
    \end{prooftree}
  \]

  Then, by inductive hypothesis, cut 3a, and cut 1a we get

  \[
    \begin{prooftree}
      \[
        \zfocseq{\Gamma}{\Delta'}{A}
        \qquad
        \ztriseq{\Gamma}{\Delta}{A}{\reactlist{}}{B}
        \justifies
        \ztriseq{\Gamma}{\Delta, \Delta'}{\cdot}{\reactlist{}}{B}
      \]
      \quad
      \ztriseq{\Gamma}{\Delta''}{B}{\reactlist{}'}{C}
      \justifies
      \ztriseq{\Gamma}{\Delta, \Delta', \Delta''}{\cdot}{
        \baseandplus{\Delta''}{\reactlist{}}{\reactlist{}'}
      }{C}
      \using{\resplist{\Delta''}{\reactlist{}}}
    \end{prooftree}
  \]

\item Case $\rightarrow L$, right commutative. If $A$ is a side formula in the
  first premise of the cut, we treat it as a left commutative case or a focus
  case. Otherwise, $A$ must be a conditional and the conclusion of the left
  premise must be $\rightarrow R$.

  \begin{enumerate}
  \item Case the cut formula comes from the left.

    \[
      \begin{prooftree}
        \[
          \ztriseq{\Gamma}{\Delta}{A}{\reactlist{}}{B}
          \justifies
          \ztriseq{\Gamma}{\Delta}{\cdot}{\emptyctrl}{
            A \rightarrow_{\reactlist{}}^{\elembases{\Delta}} B}
        \]
        \quad
        \[
          \zfocseq{\Gamma}{\Delta', A \rightarrow_{\reactlist{}}^{\elembases{\Delta}} B}{C}
          \qquad\quad
          \ztriseq{\Gamma}{\Delta''}{D}{\reactlist{}''}{E}
          \justifies
          \ztriseq{\Gamma}{\Delta', \Delta'',
            C \rightarrow_{\reactlist{}'}^{S} D,
            A \rightarrow_{\reactlist{}}^{\elembases{\Delta}} B}{\cdot}{
            \baseandplus{\Delta''}{\reactlist{}'}{\reactlist{}''}
          }{E}
          \using{\resplist{\Delta''}{\reactlist{}'}}
        \]
        \justifies
        \ztriseq{\Gamma}{\Delta, \Delta', \Delta'',
          C \rightarrow_{\reactlist{}'}^{S} D}{\cdot}{
          \baseandplus{\Delta''}{\reactlist{}'}{\reactlist{}''}
        }{E}
      \end{prooftree}
    \]

    Then, by cut 4 and Lemma,

    \[
      \begin{prooftree}
        \[
          \ztriseq{\Gamma}{\Delta}{\cdot}{\emptyctrl}{
            A \rightarrow_{\reactlist{}}^{\elembases{\Delta}} B}
          \qquad
          \zfocseq{\Gamma}{\Delta', A \rightarrow_{\reactlist{}}^{\elembases{\Delta}} B}{C}
          \justifies
          \ztriseq{\Gamma}{\Delta, \Delta'}{\cdot}{\emptyctrl}{C}
        \]
        \quad
        \ztriseq{\Gamma}{\Delta''}{D}{\reactlist{}''}{E}
        \justifies
        \ztriseq{\Gamma}{\Delta, \Delta', \Delta'',
          C \rightarrow_{\reactlist{}'}^{S} D}{\cdot}{
          \baseandplus{\Delta''}{\reactlist{}'}{\reactlist{}''}
        }{E}
        \using{\resplist{\Delta''}{\reactlist{}'}}
      \end{prooftree}
    \]

  \item Case the cut formula comes from the right.

    \[
      \begin{prooftree}
        \[
          \ztriseq{\Gamma}{\Delta}{A}{\reactlist{}}{B}
          \justifies
          \ztriseq{\Gamma}{\Delta}{\cdot}{\emptyctrl{}}{
            A \rightarrow_{\reactlist{}}^{\elembases{\Delta}} B}
        \]
        \quad
        \[
          \zfocseq{\Gamma}{\Delta'}{C}
          \qquad\quad
          \ztriseq{\Gamma}{\Delta'',
            A \rightarrow_{\reactlist{}}^{\elembases{\Delta}} B}{D}{\reactlist{}''}{E}
          \justifies
          \ztriseq{\Gamma}{\Delta', \Delta'',
            C \rightarrow_{\reactlist{}'}^{S} D,
            A \rightarrow_{\reactlist{}}^{\elembases{\Delta}} B}{\cdot}{
            \baseandplus{(\Delta'', \Delta)}{\reactlist{}'}{\reactlist{}''}
          }{E}
          \using{\resplist{(\Delta'', \Delta)}{\reactlist{}'}}
        \]
        \justifies
        \ztriseq{\Gamma}{\Delta, \Delta', \Delta'',
          C \rightarrow_{\reactlist{}'}^{S} D}{\cdot}{
          \baseandplus{(\Delta'', \Delta)}{\reactlist{}'}{\reactlist{}''}
        }{E}
      \end{prooftree}
    \]

    Then, by inductive hypothesis

    \[
      \begin{prooftree}
        \zfocseq{\Gamma}{\Delta'}{C}
        \quad
        \[
          \ztriseq{\Gamma}{\Delta}{\cdot}{\emptyctrl{}}{
            A \rightarrow_{\reactlist{}}^{\elembases{\Delta}} B}
          \qquad
          \ztriseq{\Gamma}{\Delta'',
            A \rightarrow_{\reactlist{}}^{\elembases{\Delta}} B}{D}{\reactlist{}''}{E}
          \justifies
          \ztriseq{\Gamma}{\Delta, \Delta''}{D}{\reactlist{}''}{E}
        \]
        \justifies
        \ztriseq{\Gamma}{\Delta, \Delta', \Delta'',
          C \rightarrow_{\reactlist{}'}^{S} D}{\cdot}{
          \baseandplus{(\Delta'', \Delta)}{\reactlist{}'}{\reactlist{}''}
        }{E}
        \using{\resplist{(\Delta'', \Delta)}{\reactlist{}'}}
      \end{prooftree}
    \]
  \end{enumerate}

\item Case $\copyrule$, right commutative is the same as $\rightarrow L$, right
  commutative.

\item Case $\rightarrow R$. Again, if the cut formula is a side formula, we
  treat the case as left commutative or focus. If the cut formula is active,
  then the first premise must be an instance of $\rightarrow R$.

  \[
    \begin{prooftree}
      \[
        \ztriseq{\Gamma}{\Delta}{A}{\reactlist{}}{B}
        \justifies
        \ztriseq{\Gamma}{\Delta}{\cdot}{\emptyctrl{}}{
          A \rightarrow_{\reactlist{}}^{\elembases{\Delta}} B}
      \]
      \quad
      \[
        \ztriseq{\Gamma}{\Delta',
          A \rightarrow_{\reactlist{}}^{\elembases{\Delta}} B}{C}{\reactlist{}'}{D}
        \justifies
        \ztriseq{\Gamma}{\Delta',
          A \rightarrow_{\reactlist{}}^{\elembases{\Delta}} B}{\cdot}{\emptyctrl{}}{
          C \rightarrow_{\reactlist{}'}^{\elembases{\Delta, \Delta'}} D}
      \]
      \justifies
      \ztriseq{\Gamma}{\Delta, \Delta'}{\cdot}{\emptyctrl{}}{
        C \rightarrow_{\reactlist{}'}^{\elembases{\Delta, \Delta'}} D}
    \end{prooftree}
  \]

  Then, by inductive hypothesis

  \[
    \begin{prooftree}
      \[
        \ztriseq{\Gamma}{\Delta}{\cdot}{\emptyctrl{}}{
          A \rightarrow_{\reactlist{}}^{\elembases{\Delta}} B}
        \qquad
        \ztriseq{\Gamma}{\Delta',
          A \rightarrow_{\reactlist{}}^{\elembases{\Delta}} B}{C}{\reactlist{}'}{D}
        \justifies
        \ztriseq{\Gamma}{\Delta, \Delta'}{C}{\reactlist{}'}{D}
      \]
      \justifies
      \ztriseq{\Gamma}{\Delta, \Delta'}{\cdot}{\emptyctrl{}}{
        C \rightarrow_{\reactlist{}'}^{\elembases{\Delta, \Delta'}} D}
    \end{prooftree}
  \]

\end{enumerate}

\paragraph{Cut 2}

By induction on the right premise. If it is an identity, the thesis follows by
just picking the left premise. If it is a blur, the thesis follows by cut
1b. If it is an application of $\otimes R$, the thesis follows by inductive
hypothesis with the same type 2 cut and Lemma~\ref{otimeslemma}, depending on
whether the cut formula ends up in the left or right premise of $\otimes R$.

\paragraph{Cut 3}

Cuts of type 3a are all treated easily as right commutative cases, using the
inductive hypothesis and cut 3a and 3b. The only exception is when the cut
formula is principal in the right premise, hence $A \equiv C \otimes D$. But
then, we have

\[
  \begin{prooftree}
    \[
      \zfocseq{\Gamma}{\Delta_1}{C}
      \qquad
      \zfocseq{\Gamma}{\Delta_2}{D}
      \justifies
      \zfocseq{\Gamma}{\Delta_1, \Delta_2}{C \otimes D}
    \]
    \quad
    \[
      \ztriseq{\Gamma}{\Delta}{\Omega, C, D}{\reactlist{}}{E}
      \justifies
      \ztriseq{\Gamma}{\Delta}{\Omega, C \otimes D}{\reactlist{}}{E}
    \]
    \justifies
    \ztriseq{\Gamma}{\Delta_1, \Delta_2, \Delta}{\Omega}{\reactlist{}}{E}
  \end{prooftree}
\]

hence, by inductive hypothesis

\[
  \begin{prooftree}
    \zfocseq{\Gamma}{\Delta_1}{C}
    \quad
    \[
      \zfocseq{\Gamma}{\Delta_2}{D}
      \qquad
      \ztriseq{\Gamma}{\Delta}{\Omega, C, D}{\reactlist{}}{E}
      \justifies
      \ztriseq{\Gamma}{\Delta_2, \Delta}{\Omega, C}{\reactlist{}}{E}
    \]
    \justifies
    \ztriseq{\Gamma}{\Delta_1, \Delta_2, \Delta}{\Omega}{\reactlist{}}{E}
  \end{prooftree}
\]

Cuts of type 3b are only possible if the cut formula is a conditional. But
then, the left premise of the cut must have been derived with a blur, hence
the thesis follows by cut 1b.

\paragraph{Cut 4}

As usual, if $\Omega$ is non-empty we use the inductive hypothesis and commute
on the left premise. Otherwise, we assume $\Omega$ to be empty and proceed by
induction on the derivation of the right premise.

\begin{enumerate}
\item Case $\init$. Then, the conclusion is just the left premise.

\item Case blur.

  \[
    \begin{prooftree}
      \ztriseq{\Gamma}{\Delta}{\cdot}{[]}{A}
      \qquad
      \[
        \ztriseq{\Gamma}{\Delta', A}{\cdot}{[]}{R}
        \justifies
        \zfocseq{\Gamma}{\Delta', A}{R}
      \]
      \justifies
      \ztriseq{\Gamma}{\Delta, \Delta'}{\cdot}{[]}{R}
    \end{prooftree}
  \]

  Then, by cut 1b, we have

  \[
    \begin{prooftree}
      \ztriseq{\Gamma}{\Delta}{\cdot}{[]}{A}
      \qquad
      \ztriseq{\Gamma}{\Delta', A}{\cdot}{[]}{R}
      \justifies
      \ztriseq{\Gamma}{\Delta, \Delta'}{\cdot}{[]}{R}
    \end{prooftree}
  \]

\item Case $\otimes$.

  \[
    \begin{prooftree}
      \ztriseq{\Gamma}{\Delta}{\cdot}{[]}{A}
      \qquad
      \[
        \zfocseq{\Gamma}{\Delta', A}{B}
        \qquad
        \zfocseq{\Gamma}{\Delta''}{C}
        \justifies
        \zfocseq{\Gamma}{\Delta', \Delta'', A}{B \otimes C}
      \]
      \justifies
      \ztriseq{\Gamma}{\Delta, \Delta', \Delta''}{\cdot}{[]}{B \otimes C}
    \end{prooftree}
  \]

  Then, by inductive hypothesis and Lemma,

  \[
    \begin{prooftree}
      \[
        \ztriseq{\Gamma}{\Delta}{\cdot}{[]}{A}
        \qquad
        \zfocseq{\Gamma}{\Delta', A}{B}
        \justifies
        \ztriseq{\Gamma}{\Delta, \Delta'}{\cdot}{[]}{B}
      \]
      \zfocseq{\Gamma}{\Delta''}{C}
      \justifies
      \ztriseq{\Gamma}{\Delta, \Delta', \Delta''}{\cdot}{[]}{B \otimes C}
    \end{prooftree}
  \]
\end{enumerate}

% \end{proof}

% We define an auxiliary function on formulas, \textsf{exp}, as follows:

% \begin{definition}
%   The function $\textsf{exp} : \mathcal{L} \to \mathcal{L}^*$ is inductively
%   defined as follows:

% \begin{align*}
%   \textsf{exp}(p) & = p \\
%   \textsf{exp}(A \otimes B) & = \textsf{exp}(A), \textsf{exp}(B) \\
%   \textsf{exp}(A \rightarrow_{\ctrlset{}}^S B) & = A \rightarrow_{\ctrlset{}}^S B
% \end{align*}
% \end{definition}

% \begin{lemma}\label{explemma}
%   If $\ztriseq{\Gamma}{\Delta, \textsf{exp}(A)}{\Omega}{\ctrlset{}}{C}$,
%   then $\ztriseq{\Gamma}{\Delta}{\Omega, A}{\ctrlset{}}{C}$.
% \end{lemma}
% \begin{proof}
%   By induction on $A$. If $A$ is an atom or an implication, the thesis
%   follows by act. If $A \equiv C \otimes D$, then
%   $\textsf{exp}(A) = \textsf{exp}(C \otimes D) = \textsf{exp}(C),
%   \textsf{exp}(D)$, so by inductive hypothesis

%   \[
%     \begin{prooftree}
%       \[
%         \[
%           \ztriseq{\Gamma}{\Delta, \textsf{exp}(C), \textsf{exp}(D)}{\Omega}{\ctrlset{}}{E}
%           \justifies
%           \ztriseq{\Gamma}{\Delta, \textsf{exp}(C)}{\Omega, D}{\ctrlset{}}{E}
%         \]
%         \justifies
%         \ztriseq{\Gamma}{\Delta}{\Omega, C, D}{\ctrlset{}}{E}
%       \]
%       \justifies
%       \ztriseq{\Gamma}{\Delta}{\Omega, C \otimes D}{\ctrlset{}}{E}
%     \end{prooftree}
%   \]
% \end{proof}

% \begin{lemma}[Identity expansions]\label{idexp}
%   For any formula $A$,
%   \begin{enumerate}
%   \item $\ztriseq{\Gamma}{\cdot}{A}{\emptyset}{A}$;
%   \item $\zfocseq{\Gamma}{\textsf{exp}(A)}{A}$.
%   \end{enumerate}
% \end{lemma}
% \begin{proof}
%   Both proved simultaneously by induction on $A$. If $A$ is an atom, both follow
%   trivially. Otherwise,

%   \begin{enumerate}
%   \item If $A \equiv C \otimes D$, then by inductive hypothesis

%     \[
%       \begin{prooftree}
%         \zfocseq{\Gamma}{\textsf{exp}(A)}{A}
%         \qquad
%         \zfocseq{\Gamma}{\textsf{exp}(B)}{B}
%         \justifies
%         \zfocseq{\Gamma}{\textsf{exp}(A), \textsf{exp}(B)}{A \otimes B}
%         \using{\otimes R}
%       \end{prooftree}
%     \]

%     Then, by Lemma~\ref{explemma} and the derivation above we get the first
%     point.

%     \[
%       \begin{prooftree}
%         \[
%           \[
%             \zfocseq{\Gamma}{\textsf{exp}(A)}{A}
%             \qquad
%             \zfocseq{\Gamma}{\textsf{exp}(B)}{B}
%             \justifies
%             \zfocseq{\Gamma}{\textsf{exp}(A), \textsf{exp}(B)}{A \otimes B}
%             \using{\otimes R}
%           \]
%           \justifies
%           \ztriseq{\Gamma}{\textsf{exp}(A), \textsf{exp}(B)}{\cdot}{\emptyset}{A \otimes B}
%         \]
%         \justifies
%         \ztriseq{\Gamma}{\cdot}{A \otimes B}{\emptyset}{A \otimes B}
%       \end{prooftree}
%     \]

%   \item If $A \equiv C \rightarrow_{\ctrlset{}}^S D$, then by inductive
%     hypothesis and Lemma~\ref{explemma} we get

%     \[
%       \begin{prooftree}
%         \[
%           \[
%             \[
%               \zfocseq{\Gamma}{\textsf{exp}(C)}{C}
%               \qquad
%               \ztriseq{\Gamma}{\cdot}{D}{\emptyset}{D}
%               \justifies
%               \ztriseq{\Gamma}{C \rightarrow_{\ctrlset{}}^S D,
%                 \textsf{exp}(C)}{}{\ctrlset{}}{D}
%             \]
%             \justifies
%             \ztriseq{\Gamma}{C \rightarrow_{\ctrlset{}}^S D}{C}{\ctrlset{}}{D}
%           \]
%           \justifies
%           \ztriseq{\Gamma}{C \rightarrow_{\ctrlset{}}^S D}{\cdot}{\emptyset}{
%             C \rightarrow_{\ctrlset{}}^S D}
%         \]
%         \justifies
%         \ztriseq{\Gamma}{\cdot}{C \rightarrow_{\ctrlset{}}^S D}{\emptyset}{
%           C \rightarrow_{\ctrlset{}}^S D}
%       \end{prooftree}
%     \]

%     The second point follows by blur on the derivation above.
%   \end{enumerate}
% \end{proof}

% \begin{lemma}\label{idhelplemma}
%   The following are derivable:
%   \begin{enumerate}
%   \item $\ztriseq{\Gamma}{P}{\cdot}{\emptyset}{P}$;
%   \item $\ztriseq{\Gamma}{\cdot}{A, B}{\emptyset}{A \otimes B}$;
%   \item $\ztriseq{\Gamma}{\cdot}{A \rightarrow_{\ctrlset{}}^S B,
%       A}{\ctrlset{}}{B}$;
%   \item $\ztriseq{\Gamma, A}{\cdot}{\cdot}{\emptyset}{A}$.
%   \end{enumerate}
% \end{lemma}
% \begin{proof}
%   All points proved easily with the same technique as the identity expansion
%   lemma.
% \end{proof}

% \begin{lemma}\label{activeinversion}\mbox{}
%   \begin{enumerate}
%   \item If $\ztriseq{\Gamma}{\Delta}{\Omega, A \otimes B}{\ctrlset{}}{C}$, then
%     $\ztriseq{\Gamma}{\Delta}{\Omega, A, B}{\ctrlset{}}{C}$;
%   \item If $\ztriseq{\Gamma}{\Delta}{\Omega, \Omega'}{\ctrlset{}}{C}$, then
%     $\ztriseq{\Gamma}{\Delta, \textsf{exp}(\Omega')}{\Omega}{\ctrlset{}}{C}$
%   \end{enumerate}
% \end{lemma}
% \begin{proof}
%   Point 1 is proved by cut with
%   $\ztriseq{\Gamma}{\cdot}{A, B}{\emptyset}{A \otimes B}$. Point 2 is proved by
%   repeated application of point 1 on $\otimes$ formulas until they become either
%   an atom or an implication, then by cut with
%   $\ztriseq{\Gamma}{P}{\cdot}{\emptyset}{P}$.
% \end{proof}

% \begin{theorem}[Completeness]
%   If $\zsyseq{\Gamma}{\Omega}{\ctrlset{}}{A}$, then
%   $\ztriseq{\Gamma}{\cdot}{\Omega}{\ctrlset{}}{A}$.
% \end{theorem}
% \begin{proof}
%   The proof proceeds as in [cmu thesis], by proving that all rules of the
%   backward calculus are admissible in the focused calculus.

%   \begin{enumerate}
%   \item Case $\init$: $\zsyseq{\Gamma}{A}{\emptyset}{A}$. The thesis follows by
%     Lemma~\ref{idexp}.
%   \item Case $\copyrule$. Then, by inductive hypothesis, Lemma~\ref{idhelplemma}
%     and cut admissibility, we have

%     \[
%       \begin{prooftree}
%         \ztriseq{\Gamma, A}{\cdot}{\cdot}{\emptyset}{A}
%         \qquad
%         \ztriseq{\Gamma, A}{\cdot}{\Omega, A}{\ctrlset{}}{C}
%         \justifies
%         \ztriseq{\Gamma, A}{\cdot}{\Omega}{\ctrlset{}}{C}
%       \end{prooftree}
%     \]

%   \item Case $\otimes L$. The thesis follows immediately by a single application
%     of the analogous $\otimes L$ rule in the focused calculus.
%   \item Case $\otimes R$. Then, by inductive hypothesis, Lemma~\ref{idhelplemma}
%     and cut admissibility, we have

%     \[
%       \begin{prooftree}
%         \ztriseq{\Gamma}{\cdot}{\Omega_2}{\emptyset}{B}
%         \quad
%         \[
%           \ztriseq{\Gamma}{\cdot}{\Omega_1}{\emptyset}{A}
%           \qquad
%           \ztriseq{\Gamma}{\cdot}{A, B}{\emptyset}{A \otimes B}
%           \justifies
%           \ztriseq{\Gamma}{\cdot}{\Omega_1, B}{\emptyset}{A \otimes B}
%         \]
%         \justifies
%         \ztriseq{\Gamma}{\cdot}{\Omega_1, \Omega_2}{\emptyset}{A \otimes B}
%       \end{prooftree}
%     \]

%   \item Case $\rightarrow L$. Then, by inductive hypothesis, Lemma~\ref{idhelplemma}
%     and cut admissibility, we have

%     \[
%       \begin{prooftree}
%         \ztriseq{\Gamma}{\cdot}{\Omega_1}{\emptyset}{A}
%         \quad
%         \[
%           \ztriseq{\Gamma}{\cdot}{
%             A, A \rightarrow_{\ctrlset{}}^S B
%           }{\ctrlset{}}{B}
%           \qquad
%           \ztriseq{\Gamma}{\cdot}{\Omega_2, B}{\ctrlset{}'}{C}
%           \justifies
%           \ztriseq{\Gamma}{\cdot}{
%             \Omega_2, A, A \rightarrow_{\ctrlset{}}^S B}{\ctrlset{} \cup \ctrlset{}'}{C}
%           \using{\respects{\Omega}{\ctrlset{}}}
%         \]
%         \justifies
%         \ztriseq{\Gamma}{\cdot}{\Omega_1, \Omega_2,
%         A \rightarrow_{\ctrlset{}}^S B}{\ctrlset{} \cup \ctrlset{}'}{C}
%       \end{prooftree}
%     \]

%   \item Case $\rightarrow R$. By inductive hypothesis and
%     Lemma~\ref{activeinversion}, we have

%     \[
%       \begin{prooftree}
%         \ztriseq{\Gamma}{\cdot}{\Omega, A}{\ctrlset{}}{B}
%         \justifies
%         \ztriseq{\Gamma}{\textsf{exp}(\Omega)}{A}{\ctrlset{}}{B}
%       \end{prooftree}
%     \]

%     It is an easy induction to see that
%     $\elembases{\Omega} = \elembases{\textsf{exp}(\Omega)}$. Therefore,

%     \[
%       \begin{prooftree}
%         \[
%           \ztriseq{\Gamma}{\cdot}{\Omega, A}{\ctrlset{}}{B}
%           \justifies
%           \ztriseq{\Gamma}{\textsf{exp}(\Omega)}{A}{\ctrlset{}}{B}
%         \]
%         \justifies
%         \ztriseq{\Gamma}{\textsf{exp}(\Omega)}{\cdot}{\emptyset}{
%           A \rightarrow_{\ctrlset{}}^{\elembases{\Omega}} B}
%       \end{prooftree}
%     \]

%     Then, by repeated application of act and $\otimes L$, we get the conclusion
%     $\ztriseq{\Gamma}{\cdot}{\Omega}{\emptyset}{
%           A \rightarrow_{\ctrlset{}}^{\elembases{\Omega}} B}$.

%   \end{enumerate}
% \end{proof}

%%% Local Variables:
%%% mode: latex
%%% TeX-master: "../docs"
%%% End:
