\section{Focused derivations}

Focused derivations originated in the context of the proof theory of linear
logic. The idea was introduced by Andreoli \cite{andreoli92} to reduce the
computational complexity of proof search in the sequent calculus.
In a focused calculus, non-atomic formulas are classified as asynchronous or
synchronous according to their behaviour during bottom-up proof search.
Asynchronous formulas are such that they can be decomposed on the right of the
sequent symbol without affecting provability, that is, they have an invertible
right (introduction) rule. An example is the $\limp$ connective of linear logic.

Synchronous formulas, on the other hand, are such that their decomposition on
the right can make a provable sequent evolve into a non-provable sequent, i.e.,
they have a non-invertible right rule. Decomposition for such connectives
therefore depends on ``synchronizing'' with other parts of the context.  An
example is given by the $\otimes$ connectives, which, given its multiplicative
nature, has a right rule that when used backwards involves a non-deterministic
choice on how to partition the context between the two premises.

This classification is sufficient for classical linear logic, which enjoys an
elegant formulation as a one-sided sequent calculus.  Intuitionistic calculi
are, in contrast, characterized by an asymmetry between the left and right part
of the sequent, to is makes more sense to talk about right- (respectively,
left-) synchronous connectives, characterized by a non-invertible right- (left-)
rule, and right- (left-) asynchronous connectives, characterized by an
invertible right- (left-) rule.  It turns our that, under this classification,
right-synchronous connectives are left-asynchronous, and vice versa.

It can be shown that proof search for (intuitionistic) linear logic can be
structured according to two phases without losing completeness. The active phase
acts on asynchronous formulas, and consists of applying left (right) rules to
left- (right-) asynchronous connectives until all connectives are synchronous.
In the focused phase that follows, some synchronous formula is selected and
becomes the ``focus'' of this phase. The phase consists of applying the
associated non-invertible rule to the focused formula and to any synchronous
subformula that is generated during the phase.
The power of focusing as a tool for automated deduction then is clear, and lies
in its reduction of the search space (once a formula is selected for focusing,
all rules are applied only to it or its subformulas).

Our logic lacks some of the properties of plain intuitionistic linear logic, and
this prevents us to adapt focusing from [thesis] to our case in a
straightforward way.
Contrary to ILL, we do not have a good symmetric subdivision of the connectives
into synchronous and asynchronous ones. To start with, non-invertibility of the
$\rightarrow R$ rule prevents us to treat $\rightarrow$ as a right asynchronous
connective. Moreover, the logic is strongly order-dependent, so rule
applications in a Zsyntax proof/derivation do not in general commute with each
other. Since focusing groups derivations under equivalence classes, many
derivations that are equivalent under focusing for plain linear logic are very
different for Zsyntax. It is therefore necessary to weaken the amount of
focusing in order to keep the calculus complete.
In order not to lose completeness, we are forced to keep the rules for
implications as they are. In other words, we lose $\rightarrow$ as both a
left-synchronous and a right-asynchronous connective.

In the weakened focused calculus that results, we can still identify the two
usual phases (active and focused), but the actions that are allowed during these
phases are restricted. As $\otimes$ retains all its properties from linear
logic, we can still focus on conjunctions on the right of the sequent symbol,
and freely eliminate conjunctions on the left. Hence, during the active phase,
all possible rules are applied to $\otimes$ propositions on the left. When only
atoms or implications remain, we may start a \emph{focused phase} if there is a
$\otimes$ formula on the right.

Our backward focusing calculus consists of the following new kinds of sequents:

\begin{table}[h]
  \centering
  \begin{tabular}{|l|l|}
    \hline
    \textbf{symbol} & \textbf{meaning} \\
    \hline
    $\zfocseq{\Gamma}{\nabla}{A}$ & focal sequent with $A$ under focus \\
    $\ztriseq{\Gamma}{\Delta}{\Omega}{\reactlist{}}{C}$ & active sequent \\
    \hline
  \end{tabular}
\end{table}

The additional context areas in the active sequents are meant to separate
formulas of different kinds. The rationale is that different formulas require
different approaches in the automatic proof search, therefore we need a way to
distinguish them in the sequent itself. Here, $\Delta$ is intended to be a
linear context of implicational or atomic formulas only, whereas $\Omega$ a
linear context that may contain all kinds of formulas.

For active sequents, the $\otimes$ formulas in $\Omega$ are decomposed until
they become either an implication or an atom, in which case they are transferred
to $\Delta$. Eventually, the active sequent is reduced to the form
$\ztriseq{\Gamma}{\Delta}{\cdot}{\reactlist{}}{C}$, which we call
\emph{neutral sequents} as in [cmu thesis]. From here we can either perform a
act on an implication (on the left or on the right), or launch a focused phase
selecting a focusable proposition on the conclusion.

\begin{definition}[Focusable proposition]
  A proposition is \emph{focusable} if it is a $\otimes$ formula or an atom.
\end{definition}

When we are in a neutral sequent, we may also copy a formula out of the
unrestricted context and immediately use a left rule on it. This is because
formulas in the unrestricted context do no interact with the linear context, so
it does not make any difference if they get copied right when we need them or in
another moment. The premises of the copy rule in the focused calculus resemble
those for $\rightarrow L$, since all formulas in the unrestricted context are
implications.

In what follows we use the convention, adapted from [cmu thesis], to have the
metavariable $P$ range over implications or atoms on the left of the sequent
symbol, $R$ over implications on the right of the sequent symbol, and $p$ over
atoms specifically.

\begin{figure}[h]
  \begin{mdframed}
    \[
      \begin{prooftree}
        \justifies
        \zfocseq{\Gamma}{p}{p}
        \using{\init}
      \end{prooftree}
      \qquad\qquad
      \begin{prooftree}
        \ztriseq{\Gamma}{\Delta, P}{\Omega}{\reactlist{}}{C}
        \justifies
        \ztriseq{\Gamma}{\Delta}{\Omega, P}{\reactlist{}}{C}
        \using{\lact}
      \end{prooftree}
    \]

    \[
      \begin{prooftree}
        \zfocseq{\Gamma}{\Delta}{C} \qquad Q\; \text{focusable}
        \justifies
        \ztriseq{\Gamma}{\Delta}{\cdot}{\emptyctrl}{C}
        \using{focus}
      \end{prooftree}
      \qquad \qquad
      \begin{prooftree}
        \ztriseq{\Gamma}{\Delta}{\cdot}{\emptyctrl}{R}
        \justifies
        \zfocseq{\Gamma}{\Delta}{R}
        \using{blur}
      \end{prooftree}
    \]

    \[
      \begin{prooftree}
        \zfocseq{\Gamma, A \rightarrow_{\reactlist{}}^\emptyset B}{\Delta_1}{A}
        \qquad
        \ztriseq{\Gamma, A \rightarrow_{\reactlist{}}^\emptyset B}{\Delta_2}{B}{\reactlist{}'}{C}
        \qquad
        \resplist{\Delta_2}{\reactlist{}}
        \justifies
        \ztriseq{\Gamma, A \rightarrow_{\reactlist{}}^\emptyset B}{
          \Delta_1, \Delta_2}{\cdot}{
          \baseandplus{\Delta_2}{\reactlist{}}{\reactlist{}'}
        }{C}
        \using{\copyrule}
      \end{prooftree}
    \]

    \[
      \begin{prooftree}
        \ztriseq{\Gamma}{\Delta}{\Omega, A, B}{\reactlist{}}{C}
        \justifies
        \ztriseq{\Gamma}{\Delta}{\Omega, A \otimes B}{\reactlist{}}{C}
        \using{\otimes L}
      \end{prooftree}
      \qquad \qquad
      \begin{prooftree}
        \zfocseq{\Gamma}{\Delta_1}{A}
        \qquad
        \zfocseq{\Gamma}{\Delta_2}{B}
        \justifies
        \zfocseq{\Gamma}{\Delta_1, \Delta_2}{A \otimes B}
        \using{\otimes R}
      \end{prooftree}
    \]

    \[
      \begin{prooftree}
        \zfocseq{\Gamma}{\Delta_1}{A}
        \qquad
        \ztriseq{\Gamma}{\Delta_2}{B}{\reactlist{}'}{C}
        \qquad
        \resplist{\Delta_2}{\reactlist{}}
        \justifies
        \ztriseq{\Gamma}{
          \Delta_1, \Delta_2,
          A \rightarrow_{\reactlist{}}^S B
        }{\cdot}{
          \baseandplus{\Delta_2}{\reactlist{}}{\reactlist{}'}
          }{C}
        \using{\rightarrow L}
      \end{prooftree}
    \]

    \[
      \begin{prooftree}
        \ztriseq{\Gamma}{\Delta}{A}{\reactlist{}}{B}
        \justifies
        \ztriseq{\Gamma}{\Delta}{\cdot}{\emptyset}{
          A \rightarrow_{\reactlist{}}^{\elembases{\Delta}} B}
        \using{\rightarrow R}
      \end{prooftree}
    \]
  \end{mdframed}
  \caption{Backward focused calculus}
  \label{fig:bkwdfocused}
\end{figure}


\begin{theorem}[Soundness]\mbox{}
  \begin{enumerate}
  \item If $\zfocseq{\Gamma}{\Delta}{A}$, then
    $\zsyseq{\Gamma}{\Delta}{\emptyset}{A}$;
  \item If $\ztriseq{\Gamma}{\Delta}{\Omega}{\reactlist{}}{C}$, then
    $\zsyseq{\Gamma}{\Delta, \Omega}{\reactlist{}}{C}$.
  \end{enumerate}
\end{theorem}
\begin{proof}
  The two assertions are proved simultaneously by straightforward induction on
  the height of the derivation.
\end{proof}

\begin{lemma}\label{impllemma}
  If $\ztriseq{\Gamma}{\Delta_1}{\Omega}{[]}{A}$,
  $\ztriseq{\Gamma}{\Delta_2}{B}{\reactlist{}'}{C}$, and
  $\resplist{\Delta_2}{\reactlist{}}$, then

  $\ztriseq{\Gamma}{\Delta_1, \Delta_2, A \rightarrow_{\reactlist{}}^S B}{\Omega}{
    \baseandplus{\Delta_2}{\reactlist{}}{\reactlist{}'}
  }{C}$.
\end{lemma}
\begin{proof}
  By induction on the left premise. If $\Omega$ is non-empty, we just use the
  inductive hypothesis and commute on the left. Otherwirse, we can assume
  $\Omega$ to be empty.

  \begin{enumerate}
  \item Case $\rightarrow$.

    \[
      \begin{prooftree}
        \zfocseq{\Gamma}{\Delta_1}{D}
        \qquad
        \ztriseq{\Gamma}{\Delta_1'}{E}{[]}{A}
        \justifies
        \ztriseq{\Gamma}{\Delta_1, \Delta_1',
          D \rightarrow_{[]}^S E}{\cdot}{[]}{A}
      \end{prooftree}
    \]

    Then, by inductive hypothesis

    \[
      \begin{prooftree}
        \zfocseq{\Gamma}{\Delta_1}{D}
        \,
        \[
          \ztriseq{\Gamma}{\Delta_1'}{E}{[]}{A}
          \qquad
          \ztriseq{\Gamma}{\Delta_2}{B}{\reactlist{}'}{C}
          \justifies
          \ztriseq{\Gamma}{\Delta_1', \Delta_2, A \rightarrow B}{E}{
            \baseandplus{\Delta_2}{\reactlist{}}{\reactlist{}'}}{C}
        \]
        \justifies
        \ztriseq{\Gamma}{\Delta_1, \Delta_1', \Delta_2,
          A \rightarrow B, D \rightarrow_{[]}^S E}{\cdot}{
          \baseandplus{\Delta_2}{\reactlist{}}{\reactlist{}'}
        }{C}
      \end{prooftree}
    \]

  \item Case focus.

    \[
      \begin{prooftree}
        \zfocseq{\Gamma}{\Delta_1}{A}
        \justifies
        \ztriseq{\Gamma}{\Delta_1}{\cdot}{[]}{A}
      \end{prooftree}
    \]

    Then, the thesis follows by an application of $\rightarrow L$.
  \item Case $\rightarrow R$. Then $A$ is an implication, therefore

    \[
      \begin{prooftree}
        \[
          \ztriseq{\Gamma}{\Delta_1}{\cdot}{[]}{A}
          \justifies
          \zfocseq{\Gamma}{\Delta_1}{A}
          \using{blur}
        \]
        \quad
        \ztriseq{\Gamma}{\Delta_2}{B}{\reactlist{}'}{C}
        \justifies
        \ztriseq{\Gamma}{\Delta_1, \Delta_2, A \rightarrow B}{\cdot}{\reactlist{}'}{C}
      \end{prooftree}
    \]
  \end{enumerate}
\end{proof}

\begin{lemma}\label{otimeslemma}\mbox{}
  \begin{enumerate}
  \item If $\ztriseq{\Gamma}{\Delta_1}{\Omega}{\emptyctrl}{A}$ and
    $\zfocseq{\Gamma}{\Delta_2}{B}$, then
    $\ztriseq{\Gamma}{\Delta_1, \Delta_2}{\Omega}{\emptyctrl}{A \otimes B}$;
  \item If $\zfocseq{\Gamma}{\Delta_1}{A}$
    $\ztriseq{\Gamma}{\Delta_2}{\Omega}{\emptyctrl}{B}$, then
    $\ztriseq{\Gamma}{\Delta_1, \Delta_2}{\Omega}{\emptyctrl}{A \otimes B}$.
  \end{enumerate}
\end{lemma}
\begin{proof}
  In both cases, in $\Omega$ is non-empty we do a left- or right- commuting case
  with the inductive hypothesis. Therefore, we can just assume $\Omega$ to be
  empty.

  \begin{enumerate}
  \item

    By induction on the left premise. If the last rule is a focus, the thesis
    follows by $\otimes R$ and focus. If it is $\rightarrow R$ then $A$ is an
    implication, and the thesis follows from a blur on the left premise and a
    $\otimes R$. If it is a $\rightarrow L$, the thesis follows by just
    commuting with the right premise of the $\rightarrow L$ rule instance.

  \item Same as the first case.

  \end{enumerate}
\end{proof}

\begin{theorem}\label{focusedcutelim}
  \begin{enumerate}
  \item If $\ztriseq{\Gamma}{\Delta}{\Omega}{\reactlist{1}}{A}$, then
    \begin{enumerate}
    \item If $\ztriseq{\Gamma}{\Delta'}{\Omega', A}{\reactlist{2}}{C}$ and
      $\resplist{\Omega', \Delta'}{\reactlist{1}}$, then

      $\ztriseq{\Gamma}{\Delta, \Delta'}{\Omega, \Omega'}{
        \baseandplus{(\Delta',\Omega')}{\reactlist{1}}{\reactlist{2}}}{C}$.
    \item If $\ztriseq{\Gamma}{\Delta', A}{\Omega'}{2}{C}$ and
      $\resplist{\Omega', \Delta'}{\reactlist{1}}$, then

      $\ztriseq{\Gamma}{\Delta, \Delta'}{\Omega, \Omega'}{
        \baseandplus{(\Delta', \Omega')}{\reactlist{1}}{\reactlist{2}}}{C}$;
    \end{enumerate}

  \item If $\ztriseq{\Gamma}{\Delta}{\Omega}{\emptyctrl}{A}$ and
    $\zfocseq{\Gamma}{\Delta', A}{B}$, then

    $\ztriseq{\Gamma}{\Delta, \Delta'}{\Omega}{\emptyctrl}{B}$.

  \item If $\zfocseq{\Gamma}{\Delta}{A}$, then
    \begin{enumerate}
    \item If $\ztriseq{\Gamma}{\Delta'}{\Omega', A}{\reactlist{}}{C}$, then
      $\ztriseq{\Gamma}{\Delta'}{\Omega'}{\reactlist{}}{C}$.
    \item If $\ztriseq{\Gamma}{\Delta', A}{\Omega'}{\reactlist{}}{C}$, then
      $\ztriseq{\Gamma}{\Delta'}{\Omega'}{\reactlist{}}{C}$.
    \end{enumerate}
  \item If $\ztriseq{\Gamma}{\Delta}{\Omega}{[]}{A}$ and
    $\zfocseq{\Gamma}{\Delta', A}{C}$, then
    $\ztriseq{\Gamma}{\Delta, \Delta'}{\Omega}{[]}{C}$.
  \end{enumerate}
\end{theorem}
\begin{proof}
  See appendix.
\end{proof}


We define an auxiliary function on formulas, \textsf{exp}, as follows:

\begin{definition}
  The function $\textsf{exp} : \mathcal{L} \to \mathcal{L}^*$ is inductively
  defined as follows:

\begin{align*}
  \textsf{exp}(p) & = p \\
  \textsf{exp}(A \otimes B) & = \textsf{exp}(A), \textsf{exp}(B) \\
  \textsf{exp}(A \rightarrow_{\reactlist{}}^S B) & = A \rightarrow_{\reactlist{}}^S B
\end{align*}
\end{definition}

\begin{lemma}\label{explemma}
  If $\ztriseq{\Gamma}{\Delta, \textsf{exp}(A)}{\Omega}{\reactlist{}}{C}$,
  then $\ztriseq{\Gamma}{\Delta}{\Omega, A}{\reactlist{}}{C}$.
\end{lemma}
\begin{proof}
  By induction on $A$. If $A$ is an atom or an implication, the thesis
  follows by act. If $A \equiv C \otimes D$, then
  $\textsf{exp}(A) = \textsf{exp}(C \otimes D) = \textsf{exp}(C),
  \textsf{exp}(D)$, so by inductive hypothesis

  \[
    \begin{prooftree}
      \[
        \[
          \ztriseq{\Gamma}{\Delta, \textsf{exp}(C), \textsf{exp}(D)}{\Omega}{\reactlist{}}{E}
          \justifies
          \ztriseq{\Gamma}{\Delta, \textsf{exp}(C)}{\Omega, D}{\reactlist{}}{E}
        \]
        \justifies
        \ztriseq{\Gamma}{\Delta}{\Omega, C, D}{\reactlist{}}{E}
      \]
      \justifies
      \ztriseq{\Gamma}{\Delta}{\Omega, C \otimes D}{\reactlist{}}{E}
    \end{prooftree}
  \]
\end{proof}

\begin{lemma}[Identity expansions]\label{idexp}
  For any formula $A$,
  \begin{enumerate}
  \item $\ztriseq{\Gamma}{\cdot}{A}{\emptyctrl{}}{A}$;
  \item $\zfocseq{\Gamma}{\textsf{exp}(A)}{A}$.
  \end{enumerate}
\end{lemma}
\begin{proof}
  Both proved simultaneously by induction on $A$. If $A$ is an atom, both follow
  trivially. Otherwise,

  \begin{enumerate}
  \item If $A \equiv C \otimes D$, then by inductive hypothesis

    \[
      \begin{prooftree}
        \zfocseq{\Gamma}{\textsf{exp}(A)}{A}
        \qquad
        \zfocseq{\Gamma}{\textsf{exp}(B)}{B}
        \justifies
        \zfocseq{\Gamma}{\textsf{exp}(A), \textsf{exp}(B)}{A \otimes B}
        \using{\otimes R}
      \end{prooftree}
    \]

    Then, by Lemma~\ref{explemma} and the derivation above we get the first
    point.

    \[
      \begin{prooftree}
        \[
          \[
            \zfocseq{\Gamma}{\textsf{exp}(A)}{A}
            \qquad
            \zfocseq{\Gamma}{\textsf{exp}(B)}{B}
            \justifies
            \zfocseq{\Gamma}{\textsf{exp}(A), \textsf{exp}(B)}{A \otimes B}
            \using{\otimes R}
          \]
          \justifies
          \ztriseq{\Gamma}{\textsf{exp}(A), \textsf{exp}(B)}{\cdot}{\emptyctrl{}}{A \otimes B}
        \]
        \justifies
        \ztriseq{\Gamma}{\cdot}{A \otimes B}{\emptyctrl{}}{A \otimes B}
      \end{prooftree}
    \]

  \item If $A \equiv C \rightarrow_{\reactlist{}}^S D$, then by inductive
    hypothesis and Lemma~\ref{explemma} we get

    \[
      \begin{prooftree}
        \[
          \[
            \[
              \zfocseq{\Gamma}{\textsf{exp}(C)}{C}
              \qquad
              \ztriseq{\Gamma}{\cdot}{D}{\emptyctrl{}}{D}
              \justifies
              \ztriseq{\Gamma}{C \rightarrow_{\reactlist{}}^S D,
                \textsf{exp}(C)}{}{\reactlist{}}{D}
            \]
            \justifies
            \ztriseq{\Gamma}{C \rightarrow_{\reactlist{}}^S D}{C}{\reactlist{}}{D}
          \]
          \justifies
          \ztriseq{\Gamma}{C \rightarrow_{\reactlist{}}^S D}{\cdot}{\emptyctrl{}}{
            C \rightarrow_{\reactlist{}}^S D}
        \]
        \justifies
        \ztriseq{\Gamma}{\cdot}{C \rightarrow_{\reactlist{}}^S D}{\emptyctrl{}}{
          C \rightarrow_{\reactlist{}}^S D}
      \end{prooftree}
    \]

    The second point follows by blur on the derivation above.
  \end{enumerate}
\end{proof}

\begin{lemma}\label{idhelplemma}
  The following are derivable:
  \begin{enumerate}
  \item $\ztriseq{\Gamma}{P}{\cdot}{\emptyctrl{}}{P}$;
  \item $\ztriseq{\Gamma}{\cdot}{A, B}{\emptyctrl{}}{A \otimes B}$;
  \item $\ztriseq{\Gamma}{\cdot}{A \rightarrow_{\reactlist{}}^S B,
      A}{\reactlist{}}{B}$;
  \item $\ztriseq{\Gamma, A}{\cdot}{\cdot}{\emptyctrl{}}{A}$.
  \end{enumerate}
\end{lemma}
\begin{proof}
  All points proved easily with the same technique as the identity expansion
  lemma.
\end{proof}

\begin{lemma}\label{activeinversion}\mbox{}
  \begin{enumerate}
  \item If $\ztriseq{\Gamma}{\Delta}{\Omega, A \otimes B}{\reactlist{}}{C}$, then
    $\ztriseq{\Gamma}{\Delta}{\Omega, A, B}{\reactlist{}}{C}$;
  \item If $\ztriseq{\Gamma}{\Delta}{\Omega, \Omega'}{\reactlist{}}{C}$, then
    $\ztriseq{\Gamma}{\Delta, \textsf{exp}(\Omega')}{\Omega}{\reactlist{}}{C}$
  \end{enumerate}
\end{lemma}
\begin{proof}
  Point 1 is proved by cut with
  $\ztriseq{\Gamma}{\cdot}{A, B}{\emptyctrl{}}{A \otimes B}$. Point 2 is proved by
  repeated application of point 1 on $\otimes$ formulas until they become either
  an atom or an implication, then by cut with
  $\ztriseq{\Gamma}{P}{\cdot}{\emptyctrl{}}{P}$.
\end{proof}

\begin{theorem}[Completeness]
  If $\zsyseq{\Gamma}{\Omega}{\reactlist{}}{A}$, then
  $\ztriseq{\Gamma}{\cdot}{\Omega}{\reactlist{}}{A}$.
\end{theorem}
\begin{proof}
  The proof proceeds as in [cmu thesis], by proving that all rules of the
  backward calculus are admissible in the focused calculus.

  \begin{enumerate}
  \item Case $\init$: $\zsyseq{\Gamma}{A}{\emptyctrl{}}{A}$. The thesis follows by
    Lemma~\ref{idexp}.
  \item Case $\copyrule$. Then, by inductive hypothesis, Lemma~\ref{idhelplemma}
    and cut admissibility, we have

    \[
      \begin{prooftree}
        \ztriseq{\Gamma, A}{\cdot}{\cdot}{\emptyctrl{}}{A}
        \qquad
        \ztriseq{\Gamma, A}{\cdot}{\Omega, A}{\reactlist{}}{C}
        \justifies
        \ztriseq{\Gamma, A}{\cdot}{\Omega}{\reactlist{}}{C}
      \end{prooftree}
    \]

  \item Case $\otimes L$. The thesis follows immediately by a single application
    of the analogous $\otimes L$ rule in the focused calculus.
  \item Case $\otimes R$. Then, by inductive hypothesis, Lemma~\ref{idhelplemma}
    and cut admissibility, we have

    \[
      \begin{prooftree}
        \ztriseq{\Gamma}{\cdot}{\Omega_2}{\emptyctrl{}}{B}
        \quad
        \[
          \ztriseq{\Gamma}{\cdot}{\Omega_1}{\emptyctrl{}}{A}
          \qquad
          \ztriseq{\Gamma}{\cdot}{A, B}{\emptyctrl{}}{A \otimes B}
          \justifies
          \ztriseq{\Gamma}{\cdot}{\Omega_1, B}{\emptyctrl{}}{A \otimes B}
        \]
        \justifies
        \ztriseq{\Gamma}{\cdot}{\Omega_1, \Omega_2}{\emptyctrl{}}{A \otimes B}
      \end{prooftree}
    \]

  \item Case $\rightarrow L$. Then, by inductive hypothesis,
    Lemma~\ref{idhelplemma} and cut admissibility, we have

    \[
      \begin{prooftree}
        \ztriseq{\Gamma}{\cdot}{\Omega_1}{\emptyctrl{}}{A}
        \quad
        \[
          \ztriseq{\Gamma}{\cdot}{
            A, A \rightarrow_{\ctrlset{}}^S B
          }{\reactlist{}}{B}
          \qquad
          \ztriseq{\Gamma}{\cdot}{\Omega_2, B}{\reactlist{}'}{C}
          \justifies
          \ztriseq{\Gamma}{\cdot}{
            \Omega_2, A, A \rightarrow_{\reactlist{}}^S B}{
            \baseandplus{\Omega_2}{\reactlist{}}{\reactlist{}'}
          }{C}
          \using{\resplist{\Omega_2}{\reactlist{}}}
        \]
        \justifies
        \ztriseq{\Gamma}{\cdot}{\Omega_1, \Omega_2,
          A \rightarrow_{\ctrlset{}}^S B}{
          \baseandplus{\Omega_2}{\reactlist{}}{\reactlist{}'}
      }{C}
      \end{prooftree}
    \]

  \item Case $\rightarrow R$. By inductive hypothesis and
    Lemma~\ref{activeinversion}, we have

    \[
      \begin{prooftree}
        \ztriseq{\Gamma}{\cdot}{\Omega, A}{\reactlist{}}{B}
        \justifies
        \ztriseq{\Gamma}{\textsf{exp}(\Omega)}{A}{\reactlist{}}{B}
      \end{prooftree}
    \]

    It is an easy induction to see that
    $\elembases{\Omega} = \elembases{\textsf{exp}(\Omega)}$. Therefore,

    \[
      \begin{prooftree}
        \[
          \ztriseq{\Gamma}{\cdot}{\Omega, A}{\reactlist{}}{B}
          \justifies
          \ztriseq{\Gamma}{\textsf{exp}(\Omega)}{A}{\reactlist{}}{B}
        \]
        \justifies
        \ztriseq{\Gamma}{\textsf{exp}(\Omega)}{\cdot}{\emptyctrl{}}{
          A \rightarrow_{\reactlist{}}^{\elembases{\Omega}} B}
      \end{prooftree}
    \]

    Then, by repeated application of act and $\otimes L$, we get the conclusion
    $\ztriseq{\Gamma}{\cdot}{\Omega}{\emptyctrl{}}{
          A \rightarrow_{\reactlist{}}^{\elembases{\Omega}} B}$.

  \end{enumerate}
\end{proof}

\subsection{Backward derived rules}

One of the benefits of focusing is the ability to generate derived ``big step''
inference rules: the intermediate results of a focusing or active phase are not
important, since those steps are ``forced'' in some way. Each derived rule
starts (at the bottom) with a neutral sequent from which a synchronous
proposition is selected for focus, and the focusing steps are followed. Then the
active rules are applied, and eventually we obtain a collection of neutral
sequents as the leaves. These neutral sequents are then treated as the premises
of the derived rule that produces the neutral sequent with which we started.

We first construct the backward derived rules. Then we will move to their
forward version. The general design is that intermediate sequents in the eager
active and focusing phases are not be stored in any sequent database; instead,
all sequents constructed during search are neutral sequents at the phase
boundaries. This is achieved by first precomputing the derived rules that
correspond to the frontier literals of the goal sequent.

For any given proposition, we are interested in constructing a derived inference
for the proposition corresponding to a single pair of focusing and inverse
phases. The idea is to interpret a proposition itself as the derived rules that
it embodies. Every propostiion is viewed as a relation between the conclusion of
the rule and its premises at the leaves of the bipole. Both the conclusion and
the premises are neutral sequents, which we indicate as
$\bneuseq{\Gamma}{\Delta}{Q}$.

There are three classes of relational interpretations:

\begin{enumerate}
\item Right focal relations for the focus formula $A$, written $\brfrel{A}$;
\item Left focal relations for the focus formula $A$, written $\blfrel{A}$;
\item Active relations, written
  $\bactrel{\btriseq{\Gamma}{\Delta}{\Omega}{\xi}}$, where $\xi$ is either
  $\cdot$ or a proposition $C$.
\end{enumerate}

The full set of relations is given in Figure~\ref{fig:bkwdrelations}. Each
relation $R$ takes as input the conclusion sequent $s$, and produces a sequence
of premise sequents $\Sigma = s_1, \dots, s_n$; we write this as
$\relj{R}{s}{\Sigma}$. We use these relations to form a calculus of derived
rules which acts on neutral sequents only, as given in
Figure~\ref{fig:bkwdderivedcalculus}.

\begin{figure}[h]
  \begin{mdframed}
    \[
      \begin{prooftree}
        \justifies
        \relj{\brfrel{p}}{\bneuseq{\Gamma}{p}{\cdot}}{\cdot}
        \using{\init}
      \end{prooftree}
    \]

    \[
      \begin{prooftree}
        \relj{\brfrel{A}}{\bneuseq{\Gamma}{\Delta_1}{\cdot}}{\Sigma_1}
        \qquad
        \relj{\brfrel{B}}{\bneuseq{\Gamma}{\Delta_2}{\cdot}}{\Sigma_2}
        \justifies
        \relj{\brfrel{A \otimes B}}{\bneuseq{\Gamma}{\Delta_1,
            \Delta_2}{\cdot}}{\Sigma_1 \cdot \Sigma_2}
        \using{\otimes F}
      \end{prooftree}
    \]

    \[
      \begin{prooftree}
        \justifies
        \relj{\brfrel{R}}{\zbneuseq{\Gamma}{\Delta}{}{\cdot}}{
          \zbneuseq{\Gamma}{\Delta}{\emptyset}{R}}
        \using{\faplus}
      \end{prooftree}
    \]

    \[
      \begin{prooftree}
        \relj{\bactrel{\zsyseq{\Delta}{\Omega \cdot A \cdot B}{\zeta}{\xi}}}{s}{\Sigma}
        \justifies
        \relj{\bactrel{\zsyseq{\Delta}{\Omega \cdot A \otimes B}{\zeta}{\xi}}}{s}{\Sigma}
        \using{\otimes A}
      \end{prooftree}
    \]

    \[
      \begin{prooftree}
        \relj{
          \bactrel{
            \zsyseq{\Delta,P}{\Omega}{\zeta}{\xi}
          }
        }{s}{\Sigma}
        \justifies
        \relj{\bactrel{\zsyseq{\Delta}{\Omega, P}{\zeta}{\xi}}}{s}{\Sigma}
        \using{act}
      \end{prooftree}
    \]

    \[
      \begin{prooftree}
        \justifies
        \relj{
          \bactrel{\zsyseq
            {\Delta}{\cdot}{}{\cdot}}
        }{
          \zbneuseq{\Gamma}{\Delta'}{\reactlist{}}{Q}
        }{
          \zbneuseq{\Gamma}{\Delta, \Delta'}{\reactlist{}}{Q}
        }
        \using{match}
      \end{prooftree}
    \]

    \[
      \begin{prooftree}
        \justifies
        \relj{
          \bactrel{\zsyseq{\Delta}{\cdot}{\reactlist{}}{Q}}
        }{
          \bneuseq{\Gamma}{\Delta'}{\cdot}
        }{
          \zbneuseq{\Gamma}{\Delta, \Delta'}{\reactlist{}}{Q}
        }
        \using{match'}
      \end{prooftree}
    \]
  \end{mdframed}
  \caption{Backward relations for derived rules}
  \label{fig:bkwdrelations}
\end{figure}

\begin{figure}[h]
  \begin{mdframed}
    \[
      \begin{prooftree}
        (\relj{\brfrel{Q}}{\zbneuseq{\Gamma}{\Delta}{}{\cdot}}{\Sigma})
        \qquad \Sigma
        \justifies
        \zbneuseq{\Gamma}{\Delta}{\emptyctrl{}}{Q}
        \using{focus}
      \end{prooftree}
    \]

    \[
      \begin{prooftree}
        \[
        (\relj{\brfrel{A}}{\zbneuseq{\Gamma}{\Delta_1}{}{\cdot}}{\Sigma})
        \proofdotseparation=1.2ex
        \proofdotnumber=0
        \leadsto
        (\relj{\bactrel{\zsyseq{\cdot}{B}{}{\cdot}}}
        {\zbneuseq{\Gamma}{\Delta_2}{\reactlist{}'}{Q}}{\Sigma'})
        \]
        \qquad \Sigma, \Sigma'
        \qquad \resplist{\Delta_2}{\reactlist{}}
        \justifies
        \zbneuseq{\Gamma}{\Delta_1, \Delta_2,
          A \rightarrow_{\reactlist{}}^S B
        }{
          \baseandplus{\Delta_2}{\reactlist{}}{\reactlist{}'}
        }{Q}
        \using{\rightarrow L}
      \end{prooftree}
    \]

    \[
      \begin{prooftree}
        (\relj{\bactrel{\zsyseq{\cdot}{A}{\reactlist{}}{B}}}
        {\zbneuseq{\Gamma}{\Delta}{}{\cdot}}{\Sigma})
        \qquad \Sigma
        \justifies
        \zbneuseq{\Gamma}{\Delta}{\emptyctrl{}}{
          A \rightarrow_{\reactlist{}}^{\elembases{\Delta}} B}
        \using{\rightarrow R}
      \end{prooftree}
    \]

    \[
      \begin{prooftree}
        \[
          (\relj{\brfrel{A}}{\zbneuseq{\Gamma,
              A \rightarrow_{\reactlist{}}^{\emptyset} B
            }{\Delta_1}{}{\cdot}}{\Sigma})
        \proofdotseparation=1.2ex
        \proofdotnumber=0
        \leadsto
        (\relj{\bactrel{\zsyseq{\cdot}{B}{}{\cdot}}}
        {\zbneuseq{\Gamma,
            A \rightarrow_{\reactlist{}}^{\emptyset} B
          }{\Delta_2}{\reactlist{}'}{Q}}{\Sigma'})
        \]
        \qquad \Sigma, \Sigma'
        \qquad \resplist{\Delta_2}{\reactlist{}}
        \justifies
        \zbneuseq{\Gamma, A \rightarrow_{\reactlist{}}^{\emptyset} B}{\Delta_1, \Delta_2
        }{
          \baseandplus{\Delta_2}{\reactlist{}}{\reactlist{}'}
        }{Q}
        \using{\copyrule}
      \end{prooftree}
    \]

  \end{mdframed}
  \caption{Backward calculus of derived rules}
  \label{fig:bkwdderivedcalculus}
\end{figure}

\begin{definition}
  A sequent of the backward derived rule calculus
  $\zbneuseq{\Gamma}{\Delta}{\reactlist{}}{Q}$
  is sound if $\ztriseq{\Gamma}{\Delta}{\cdot}{\reactlist{}}{Q}$.
\end{definition}

\begin{lemma}\label{bkwdder-soundness-lemma}
  \begin{enumerate}
  \item If $\brfrelj{A}{\zbneuseq{\Gamma}{\Delta}{}{\cdot}}{\Sigma}$
    and $\Sigma$
    are sound, then

    $\zfocseq{\Gamma}{\Delta}{A}$.
  \item If
    $\bactrelj{\zsyseq{\Delta}{\Omega}{}{\cdot}}
    {\zbneuseq{\Gamma}{\Delta'}{\reactlist{}}{Q}}{\Sigma}$
    and $\Sigma$ are sound, then

    $\ztriseq{\Gamma}{\Delta, \Delta'}{\Omega}{\reactlist{}}{Q}$.
  \item If
    $\bactrelj{\zsyseq{\Delta}{\Omega}{\reactlist{}}{Q}}
    {\zbneuseq{\Gamma}{\Delta'}{}{\cdot}}{\Sigma}$
    and $\Sigma$ are sound, then

    $\ztriseq{\Gamma}{\Delta, \Delta'}{\Omega}{\reactlist{}}{Q}$.
  \end{enumerate}
\end{lemma}
\begin{proof}
  All points are proved simultaneously and follow immediately by mutual
  induction on the height of the derivations and use of the corresponding rules
  of the focused calculus.
\end{proof}

\begin{theorem}[Soundness]
  If $\zbneuseq{\Gamma}{\Delta}{\reactlist{}}{Q}$ then
  $\ztriseq{\Gamma}{\Delta}{\cdot}{\reactlist{}}{Q}$.
\end{theorem}
\begin{proof}
  Straightforward induction on the derivation of the backward derived rules
  calculus, and Lemma~\ref{bkwdder-soundness-lemma}.
\end{proof}

\begin{lemma}\label{completeness-lemma}
  \begin{enumerate}

  \item If $\zfocseq{\Gamma}{\Delta}{A}$, then for some $\Sigma$
    \begin{enumerate}
    \item $\relj{\brfrel{A}}{\zbneuseq{\Gamma}{\Delta}{}{\cdot}}{\Sigma}$, and
    \item $\Sigma$ are all derivable
    \end{enumerate}

  \item If $\ztriseq{\Gamma}{\Delta_1, \Delta_2}{\Omega}{\zeta \uplus \delta}{\xi \uplus
      \gamma}$ (where $x \uplus y$ means either $x$ or $y$ is
    empty), then for some $\Sigma$
    \begin{enumerate}
    \item
      $\bactrelj{\zsyseq{\Delta_1}{\Omega}{\zeta}{\xi}}
      {\zbneuseq{\Gamma}{\Delta_2}{\delta}{\gamma}}{\Sigma}$,
      and
    \item $\Sigma$ are all derivable.
    \end{enumerate}
  \end{enumerate}
\end{lemma}
\begin{proof}
  By simultaneous induction on the height of the derivation.
  \begin{enumerate}
  \item Case $\init$. Then, just apply the $\init$ relation.
  \item Case act. Straightforward use of the act rule for derived
    relations.
  \item Case $focus$.

    \[
      \begin{prooftree}
        \zfocseq{\Gamma}{\Delta_1, \Delta_2}{Q} \qquad Q\; \text{right-focusable}
        \justifies
        \ztriseq{\Gamma}{\Delta_1, \Delta_2}{\cdot}{\emptyctrl{}}{Q}
        \using{focus}
      \end{prooftree}
    \]

    Suppose $\gamma_1, \gamma_2$ are such that $\gamma_1 \uplus \gamma_2 =
    Q$, and $\zeta_1 \uplus \zeta_2 = \emptyctrl{}$. Then, the following:

    \[
      \relj{
        \bactrel{\zsyseq{\Delta_1}{\cdot}{\zeta_1}{\gamma_1}}
      }{
        \zbneuseq{\Delta_2}{\cdot}{\zeta_2}{\gamma_2}
      }{
        \zbneuseq{\Gamma}{\Delta_1, \Delta_2}{\emptyctrl{}}{Q}
      }
    \]

    is derivable with either $\matchrule$ or $\matchprimerule$. We need to show
    that $\zbneuseq{\Gamma}{\Delta_1, \Delta_2}{\emptyctrl{}}{Q}$ is derivable. By
    inductive hypothesis, we have
    $\relj{\brfrel{Q}}{\zbneuseq{\Gamma}{\Delta_1, \Delta_2}{}{\cdot}}{\Sigma}$ for
    some $\Sigma$ all derivable. But then,

    \[
      \begin{prooftree}
        \relj{\brfrel{Q}}{\zbneuseq{\Gamma}{\Delta_1, \Delta_2}{}{\cdot}}{\Sigma}
        \qquad \Sigma
        \justifies
        \zbneuseq{\Gamma}{\Delta_1, \Delta_2}{\emptyctrl{}}{Q}
        \using{focus}
      \end{prooftree}
    \]

  \item Case $blur$.

    \[
      \begin{prooftree}
        \ztriseq{\Gamma}{\Delta}{\cdot}{\emptyctrl{}}{R}
        \justifies
        \zfocseq{\Gamma}{\Delta}{R}
        \using{blur}
      \end{prooftree}
    \]

    By inductive hypothesis, we have
    $\relj{\bactrel{\zsyseq{\cdot}{\cdot}{\emptyctrl{}}{R}}}{\bneuseq{\Gamma}{\Delta}{\cdot}}{\Sigma}$,
    where all $\Sigma$ are derivable.
    But then is must have been derived with an instance of $\matchrule$, hence
    $\zbneuseq{\Gamma}{\Delta}{\emptyctrl{}}{R}$ is derivable.
    We then have

    \[
      \begin{prooftree}
        \justifies
        \relj{\brfrel{R}}{\bneuseq{\Gamma}{\Delta}{\cdot}}{
          \zbneuseq{\Gamma}{\Delta}{\emptyctrl{}}{R}
        }
        \using{FA^+}
      \end{prooftree}
    \]

  \item Case $\rightarrow L$.

    \[
      \begin{prooftree}
        \zfocseq{\Gamma}{\Delta_1}{A}
        \qquad
        \ztriseq{\Gamma}{\Delta_2}{B}{\reactlist{}'}{C}
        \qquad
        \resplist{\Delta_2}{\reactlist{}}
        \justifies
        \ztriseq{\Gamma}{
          \Delta_1, \Delta_2, A \rightarrow_{\reactlist{}}^S B
        }{\cdot}{\baseandplus{\Delta_2}{\reactlist{}}{\reactlist{}'}}{C}
        \using{\rightarrow L}
      \end{prooftree}
    \]

    Suppose $\Delta', \Delta'' = \Delta_1, \Delta_2,
    A\rightarrow_{\reactlist{}}^S B$, and $\gamma_1 \uplus \gamma_2 = C$,
    $\zeta_1 \uplus \zeta_2 = \baseandplus{\Delta_2}{\reactlist{}}{\reactlist{}'}$.
    We can apply $\matchrule$ to immediately derive


    \[
      \begin{prooftree}
        \justifies
        \relj{\bactrel{
            \zsyseq{
              \Delta'}{\cdot}{\zeta_1}{\gamma_1}}}
        {\zbneuseq{\Gamma}
          {\Delta''}{\zeta_2}{\gamma_2}}{\zbneuseq{
            \Gamma}{\Delta', \Delta''}{
            \baseandplus{\Delta_2}{\reactlist{}}{\reactlist{}'}
          }{C}}
        \using{\matchrule}
      \end{prooftree}
    \]

    We can see that
    $\zbneuseq{\Gamma} {\Delta', \Delta''}{
      \baseandplus{\Delta_2}{\reactlist{}}{\reactlist{}'}}{C}$ is
    derivable by inductive hypothesis on the premises and an application of
    $\rightarrow L$ of the derived rule calculus.

    \[
      \begin{prooftree}
        \[
        (\relj{\brfrel{A}}{\zbneuseq{\Gamma}{\Delta_1}{}{\cdot}}{\Sigma})
        \proofdotseparation=1.2ex
        \proofdotnumber=0
        \leadsto
        (\relj{\bactrel{\zsyseq{\cdot}{B}{}{\cdot}}}
        {\zbneuseq{\Gamma}{\Delta_2}{\reactlist{}'}{C}}{\Sigma'})
        \]
        \qquad \Sigma, \Sigma'
        \qquad \resplist{\Delta_2}{\reactlist{}}
        \justifies
        \zbneuseq{\Gamma}{\Delta_1, \Delta_2,
          A \rightarrow_{\reactlist{}}^S B
        }{
          \baseandplus{\Delta_2}{\reactlist{}}{\reactlist{}'}
        }{C}
        \using{\rightarrow L}
      \end{prooftree}
    \]

  \item The case $\copyrule$ is analogous to $\rightarrow L$.

  \item Case $\rightarrow R$.

    \[
      \begin{prooftree}
        \ztriseq{\Gamma}{\Delta_1, \Delta_2}{A}{\reactlist{}}{B}
        \justifies
        \ztriseq{\Gamma}{\Delta_1, \Delta_2}{\cdot}{\emptyctrl{}}{
          A \rightarrow_{\reactlist{}}^{\elembases{\Delta_1, \Delta_2}} B}
        \using{\rightarrow R}
      \end{prooftree}
    \]

    Again, suppose
    $\gamma_1 \uplus \gamma_2 = A \rightarrow_{\reactlist{}}^{\elembases{\Delta_1,
        \Delta_2}} B$, $\zeta_1 \uplus \zeta_2 = \emptyctrl{}$. We can immediately
    derive by either $\matchrule$ or $\matchprimerule$ the following

    \[
      \begin{prooftree}
        \justifies
        \relj{\bactrel{
            \zsyseq{
              \Delta_1}{\cdot}{\zeta_1}{\gamma_1}}}
        {\zbneuseq{\Gamma}
          {\Delta_2}{\zeta_2}{\gamma_2}}{\zbneuseq{
            \Gamma}{\Delta_1, \Delta_2}{
            \emptyctrl{}
          }{A \rightarrow_{\reactlist{}}^{\elembases{\Delta_1, \Delta_2}} B}}
        \using{\matchrule}
      \end{prooftree}
    \]

    To see that
    $\zbneuseq{ \Gamma}{\Delta_1, \Delta_2}{ \emptyctrl{}}{A
      \rightarrow_{\reactlist{}}^{\elembases{\Delta_1, \Delta_2}} B}$ is
    derivable, we use the inductive hypothesis on the premise of $\rightarrow R$
    and an application of the corresponding rule of the derived rule calculus.

    \[
      \begin{prooftree}
        (\relj{\bactrel{\zsyseq{\cdot}{A}{\reactlist{}}{B}}}
        {\zbneuseq{\Gamma}{\Delta_1, \Delta_2}{}{\cdot}}{\Sigma})
        \quad \Sigma
        \justifies
        \zbneuseq{\Gamma}{\Delta_1, \Delta_2}{\emptyctrl{}}{
          A \rightarrow_{\reactlist{}}^{\elembases{\Delta_1, \Delta_2}} B}
        \using{\rightarrow R}
      \end{prooftree}
    \]


  \item The remaining $\otimes L$ and $\otimes R$ cases are straightforward
    applications of the inductive hypothesis.
  \end{enumerate}
\end{proof}

\begin{theorem}[Completeness]
  If $\ztriseq{\Gamma}{\Delta}{\cdot}{\reactlist{}}{Q}$, then
  $\zbneuseq{\Gamma}{\Delta}{\reactlist{}}{Q}$.
\end{theorem}
\begin{proof}
  Straightforward induction on the derivation and application of
  Lemma~\ref{completeness-lemma}.
\end{proof}

%%% Local Variables:
%%% mode: latex
%%% TeX-master: "../docs"
%%% End:
