\section{Focused derivations}\label{sec:focusing}

Focused derivations originated in the context of the proof theory of linear
logic. The idea was introduced by Andreoli \cite{andreoli92} to reduce the
computational complexity of proof search in the sequent calculus.
In a focused calculus, non-atomic formulas are classified as asynchronous or
synchronous according to their behaviour during bottom-up proof search.
Asynchronous formulas are such that they can be decomposed on the right of the
sequent symbol without affecting provability, that is, they have an invertible
right (introduction) rule. An example is the $\limp$ connective of linear logic.

Synchronous formulas, on the other hand, are such that their decomposition on
the right can make a provable sequent evolve into a non-provable sequent, i.e.,
they have a non-invertible right rule. Decomposition for such connectives
therefore depends on ``synchronizing'' with other parts of the context.  An
example is given by the $\otimes$ connective, which, given its multiplicative
nature, has a right rule that involves a non-deterministic choice on how to
partition the context between the two premises when used backwards.

This classification is sufficient for classical linear logic, which enjoys an
elegant formulation as a one-sided sequent calculus.  Intuitionistic calculi
are, in contrast, characterized by an asymmetry between the left and right part
of the sequent, so the literature usually talks about right- (respectively,
left-) synchronous connectives, characterized by a non-invertible right- (left-)
rule, and right- (left-) asynchronous connectives, characterized by an
invertible right- (left-) rule.  It turns our that, under this classification,
right-synchronous connectives are left-asynchronous, and vice versa.

It can be shown that proof search for (intuitionistic) linear logic can be
structured according to two phases without losing completeness. The active phase
acts on asynchronous formulas, and consists of applying left (right) rules to
left- (right-) asynchronous connectives until all connectives in the sequent are
synchronous. In the focused phase that follows, some synchronous formula is
selected and becomes the ``focus'' of this phase. The phase consists of applying
the associated non-invertible rule to the focused formula and to any synchronous
subformula that is generated during the phase.  The power of focusing as a tool
for automated deduction thus lies in its elimination of non-determinism, that
inevitably reduces the search space considerably (once a formula is selected for
focusing, rules are only applied on it or its subformulas).

Our logic lacks some of the properties of plain intuitionistic linear logic,
preventing us to adapt focusing from \cite{chaudhuri-thesis} to our case in a
straightforward way. Contrary to ILL, we do not have a good symmetric
subdivision of the connectives into synchronous and asynchronous ones. To start
with, the $\rightarrow R$ rule is not invertible, so we cannot treat
$\rightarrow$ as a right asynchronous connective. Moreover, the logic is
strongly order-dependent, so rule applications in a Zsyntax proof do not in
general commute with each other. Since focusing imposes a precise order in the
application of inference rules, many derivations that are equivalent under
focusing in linear logic are very different in Zsyntax. It is therefore
necessary to tune the ``amount of focusing'' in order to keep the calculus
complete.

It should be quite clear that, to do so, we are forced to keep the rules for the
conditional connective as they are. In other words, we lose $\rightarrow$ as
both a left-synchronous and a right-asynchronous connective. We can still
identify the two phases (active and focused), but we will allow fewer actions to
be performed. As our $\otimes$ retains all the properties from its linear logic
counterpart, we can still focus on conjunctions on the right of the sequent
symbol, and freely eliminate conjunctions on the left. Hence, during the active
phase, all possible rules are applied to $\otimes$ propositions on the
left. When only atoms or implications remain, we may start a \emph{focused
  phase} if there is a $\otimes$ or atomic formula on the right.

Our backward focusing calculus consists of the following kinds of sequents:

\begin{table}[h]
  \centering
  \begin{tabular}{|l|l|}
    \hline
    \textbf{symbol} & \textbf{meaning} \\
    \hline
    $\zfocseq{\Gamma}{\nabla}{A}$ & focal sequent with $A$ under focus \\
    $\ztriseq{\Gamma}{\Delta}{\Omega}{\reactlist{}}{C}$ & active sequent \\
    \hline
  \end{tabular}
\end{table}

Since different formulas (from the point of view of focusing) require different
actions during focused proof search, we split the linear context into two
distinct area as a way to distinguish them in the sequent itself. Here, $\Delta$
is intended to be a linear context of implicational or atomic formulas only,
whereas $\Omega$ is a linear context that may contain all kinds of formulas.

When performing backward proof search on an active sequent, all $\otimes$
formulas in $\Omega$ are decomposed with a left rule until they become either an
implication or an atom, in which case they are transferred to
$\Delta$. Eventually, the active sequent is reduced to the form
$\ztriseq{\Gamma}{\Delta}{\cdot}{\reactlist{}}{C}$, which we call \emph{neutral
  sequents} as in \cite{chaudhuri-thesis}. From here we can either act on an
implication (on the left or on the right), or launch a focused phase selecting a
focusable proposition on the conclusion.

\begin{definition}[Focusable proposition]
  A proposition is \emph{focusable} if it is a $\otimes$ formula or an atom.
\end{definition}

When we are on a neutral sequent, we may also copy a formula out of the
unrestricted context and immediately use a left rule on it. This is because
formulas in the unrestricted context do no interact with the rest of the context
(they have empty elementary base), so our decision of introducing unrestricted
formulas right before using them does not compromise completeness. The premises
of the copy rule in the focused calculus resemble those for $\rightarrow L$,
since all formulas in the unrestricted context are implications.

In what follows we use the convention of having the metavariable $P$ range over
implications or atoms on the left of the sequent symbol, $R$ over implications
on the right of the sequent symbol, and $p$ over atoms.

\begin{figure}[h]
  \begin{mdframed}
    \[
      \begin{prooftree}
        \justifies
        \zfocseq{\Gamma}{p}{p}
        \using{\init}
      \end{prooftree}
      \qquad\qquad
      \begin{prooftree}
        \ztriseq{\Gamma}{\Delta, P}{\Omega}{\reactlist{}}{C}
        \justifies
        \ztriseq{\Gamma}{\Delta}{\Omega, P}{\reactlist{}}{C}
        \using{\lact}
      \end{prooftree}
    \]

    \[
      \begin{prooftree}
        \zfocseq{\Gamma}{\Delta}{C} \qquad Q\; \text{focusable}
        \justifies
        \ztriseq{\Gamma}{\Delta}{\cdot}{\emptyctrl}{C}
        \using{focus}
      \end{prooftree}
      \qquad \qquad
      \begin{prooftree}
        \ztriseq{\Gamma}{\Delta}{\cdot}{\emptyctrl}{R}
        \justifies
        \zfocseq{\Gamma}{\Delta}{R}
        \using{blur}
      \end{prooftree}
    \]

    \[
      \begin{prooftree}
        \zfocseq{\Gamma, A \rightarrow_{\reactlist{}}^\emptyset B}{\Delta_1}{A}
        \qquad
        \ztriseq{\Gamma, A \rightarrow_{\reactlist{}}^\emptyset B}{\Delta_2}{B}{\reactlist{}'}{C}
        \qquad
        \resplist{\Delta_2}{\reactlist{}}
        \justifies
        \ztriseq{\Gamma, A \rightarrow_{\reactlist{}}^\emptyset B}{
          \Delta_1, \Delta_2}{\cdot}{
          \baseandplus{\Delta_2}{\reactlist{}}{\reactlist{}'}
        }{C}
        \using{\copyrule}
      \end{prooftree}
    \]

    \[
      \begin{prooftree}
        \ztriseq{\Gamma}{\Delta}{\Omega, A, B}{\reactlist{}}{C}
        \justifies
        \ztriseq{\Gamma}{\Delta}{\Omega, A \otimes B}{\reactlist{}}{C}
        \using{\otimes L}
      \end{prooftree}
      \qquad \qquad
      \begin{prooftree}
        \zfocseq{\Gamma}{\Delta_1}{A}
        \qquad
        \zfocseq{\Gamma}{\Delta_2}{B}
        \justifies
        \zfocseq{\Gamma}{\Delta_1, \Delta_2}{A \otimes B}
        \using{\otimes R}
      \end{prooftree}
    \]

    \[
      \begin{prooftree}
        \zfocseq{\Gamma}{\Delta_1}{A}
        \qquad
        \ztriseq{\Gamma}{\Delta_2}{B}{\reactlist{}'}{C}
        \qquad
        \resplist{\Delta_2}{\reactlist{}}
        \justifies
        \ztriseq{\Gamma}{
          \Delta_1, \Delta_2,
          A \rightarrow_{\reactlist{}}^S B
        }{\cdot}{
          \baseandplus{\Delta_2}{\reactlist{}}{\reactlist{}'}
          }{C}
        \using{\rightarrow L}
      \end{prooftree}
    \]

    \[
      \begin{prooftree}
        \ztriseq{\Gamma}{\Delta}{A}{\reactlist{}}{B}
        \justifies
        \ztriseq{\Gamma}{\Delta}{\cdot}{\emptyset}{
          A \rightarrow_{\reactlist{}}^{\elembases{\Delta}} B}
        \using{\rightarrow R}
      \end{prooftree}
    \]
  \end{mdframed}
  \caption{Backward focused calculus}
  \label{fig:bkwdfocused}
\end{figure}


\begin{theorem}[Soundness]\mbox{}
  \begin{enumerate}
  \item If $\zfocseq{\Gamma}{\Delta}{A}$, then
    $\zsyseq{\Gamma}{\Delta}{\emptyset}{A}$;
  \item If $\ztriseq{\Gamma}{\Delta}{\Omega}{\reactlist{}}{C}$, then
    $\zsyseq{\Gamma}{\Delta, \Omega}{\reactlist{}}{C}$.
  \end{enumerate}
\end{theorem}
\begin{proof}
  The two assertions are proved simultaneously by straightforward induction on
  the height of the derivation.
\end{proof}

\subsection{Completeness}

Completeness is established with respect to the backward sequent calculus of the
previous section. The proof is again based on \cite{chaudhuri-thesis} and uses
cut admissibility of the focused calculus as a lemma, from which completeness
follows almost immediately.

The proofs of all lemmas and theorems in \cite{chaudhuri-thesis} that were
useful to our case have been redone in their entirety for our calculus, and
documented here. Apart from the convenience of having the main results and
theorems repeated here, the reason is that our focused calculus, despite being
heavily inspired and very similar to the one in \cite{chaudhuri-thesis}, exibits
some significant differences, namely a weaker form of focusing and monotonicity
control annotations. This required some proofs to be carried out is a slightly
different way than Chaudhuri's thesis, sometimes simplifying things, and some
other times complicating them a bit or requiring additional lemmas.

Nevertheless, the general idea is still very much the one of
\cite{chaudhuri-thesis}, so we avoid spelling out too many details that can be
found in the referenced work.

\begin{lemma}\label{impllemma}
  If $\ztriseq{\Gamma}{\Delta_1}{\Omega}{[]}{A}$,
  $\ztriseq{\Gamma}{\Delta_2}{B}{\reactlist{}'}{C}$, and
  $\resplist{\Delta_2}{\reactlist{}}$, then

  $\ztriseq{\Gamma}{\Delta_1, \Delta_2, A \rightarrow_{\reactlist{}}^S B}{\Omega}{
    \baseandplus{\Delta_2}{\reactlist{}}{\reactlist{}'}
  }{C}$.
\end{lemma}
\begin{proof}
  By induction on the left premise. If $\Omega$ is non-empty, we just use the
  inductive hypothesis and commute on the left. Otherwirse, we can assume
  $\Omega$ to be empty.

  \begin{enumerate}
  \item Case $\rightarrow$.

    \[
      \begin{prooftree}
        \zfocseq{\Gamma}{\Delta_1}{D}
        \qquad
        \ztriseq{\Gamma}{\Delta_1'}{E}{[]}{A}
        \justifies
        \ztriseq{\Gamma}{\Delta_1, \Delta_1',
          D \rightarrow_{[]}^S E}{\cdot}{[]}{A}
      \end{prooftree}
    \]

    Then, by inductive hypothesis

    \[
      \begin{prooftree}
        \zfocseq{\Gamma}{\Delta_1}{D}
        \,
        \[
          \ztriseq{\Gamma}{\Delta_1'}{E}{[]}{A}
          \qquad
          \ztriseq{\Gamma}{\Delta_2}{B}{\reactlist{}'}{C}
          \justifies
          \ztriseq{\Gamma}{\Delta_1', \Delta_2, A \rightarrow B}{E}{
            \baseandplus{\Delta_2}{\reactlist{}}{\reactlist{}'}}{C}
        \]
        \justifies
        \ztriseq{\Gamma}{\Delta_1, \Delta_1', \Delta_2,
          A \rightarrow B, D \rightarrow_{[]}^S E}{\cdot}{
          \baseandplus{\Delta_2}{\reactlist{}}{\reactlist{}'}
        }{C}
      \end{prooftree}
    \]

  \item Case focus.

    \[
      \begin{prooftree}
        \zfocseq{\Gamma}{\Delta_1}{A}
        \justifies
        \ztriseq{\Gamma}{\Delta_1}{\cdot}{[]}{A}
      \end{prooftree}
    \]

    Then, the thesis follows by an application of $\rightarrow L$.
  \item Case $\rightarrow R$. Then $A$ is an implication, therefore

    \[
      \begin{prooftree}
        \[
          \ztriseq{\Gamma}{\Delta_1}{\cdot}{[]}{A}
          \justifies
          \zfocseq{\Gamma}{\Delta_1}{A}
          \using{blur}
        \]
        \quad
        \ztriseq{\Gamma}{\Delta_2}{B}{\reactlist{}'}{C}
        \justifies
        \ztriseq{\Gamma}{\Delta_1, \Delta_2, A \rightarrow B}{\cdot}{\reactlist{}'}{C}
      \end{prooftree}
    \]
  \end{enumerate}
\end{proof}

\begin{lemma}\label{otimeslemma}\mbox{}
  \begin{enumerate}
  \item If $\ztriseq{\Gamma}{\Delta_1}{\Omega}{\emptyctrl}{A}$ and
    $\zfocseq{\Gamma}{\Delta_2}{B}$, then
    $\ztriseq{\Gamma}{\Delta_1, \Delta_2}{\Omega}{\emptyctrl}{A \otimes B}$;
  \item If $\zfocseq{\Gamma}{\Delta_1}{A}$
    $\ztriseq{\Gamma}{\Delta_2}{\Omega}{\emptyctrl}{B}$, then
    $\ztriseq{\Gamma}{\Delta_1, \Delta_2}{\Omega}{\emptyctrl}{A \otimes B}$.
  \end{enumerate}
\end{lemma}
\begin{proof}
  In both cases, in $\Omega$ is non-empty we do a left- or right- commuting case
  with the inductive hypothesis. Therefore, we can just assume $\Omega$ to be
  empty.

  \begin{enumerate}
  \item

    By induction on the left premise. If the last rule is a focus, the thesis
    follows by $\otimes R$ and focus. If it is $\rightarrow R$ then $A$ is an
    implication, and the thesis follows from a blur on the left premise and a
    $\otimes R$. If it is a $\rightarrow L$, the thesis follows by just
    commuting with the right premise of the $\rightarrow L$ rule instance.

  \item Same as the first case.

  \end{enumerate}
\end{proof}

\begin{theorem}\label{focusedcutelim}
  \begin{enumerate}
  \item If $\ztriseq{\Gamma}{\Delta}{\Omega}{\reactlist{1}}{A}$, then
    \begin{enumerate}
    \item If $\ztriseq{\Gamma}{\Delta'}{\Omega', A}{\reactlist{2}}{C}$ and
      $\resplist{\Omega', \Delta'}{\reactlist{1}}$, then

      $\ztriseq{\Gamma}{\Delta, \Delta'}{\Omega, \Omega'}{
        \baseandplus{(\Delta',\Omega')}{\reactlist{1}}{\reactlist{2}}}{C}$.
    \item If $\ztriseq{\Gamma}{\Delta', A}{\Omega'}{2}{C}$ and
      $\resplist{\Omega', \Delta'}{\reactlist{1}}$, then

      $\ztriseq{\Gamma}{\Delta, \Delta'}{\Omega, \Omega'}{
        \baseandplus{(\Delta', \Omega')}{\reactlist{1}}{\reactlist{2}}}{C}$;
    \end{enumerate}

  \item If $\ztriseq{\Gamma}{\Delta}{\Omega}{\emptyctrl}{A}$ and
    $\zfocseq{\Gamma}{\Delta', A}{B}$, then

    $\ztriseq{\Gamma}{\Delta, \Delta'}{\Omega}{\emptyctrl}{B}$.

  \item If $\zfocseq{\Gamma}{\Delta}{A}$, then
    \begin{enumerate}
    \item If $\ztriseq{\Gamma}{\Delta'}{\Omega', A}{\reactlist{}}{C}$, then
      $\ztriseq{\Gamma}{\Delta'}{\Omega'}{\reactlist{}}{C}$.
    \item If $\ztriseq{\Gamma}{\Delta', A}{\Omega'}{\reactlist{}}{C}$, then
      $\ztriseq{\Gamma}{\Delta'}{\Omega'}{\reactlist{}}{C}$.
    \end{enumerate}
  \item If $\ztriseq{\Gamma}{\Delta}{\Omega}{[]}{A}$ and
    $\zfocseq{\Gamma}{\Delta', A}{C}$, then
    $\ztriseq{\Gamma}{\Delta, \Delta'}{\Omega}{[]}{C}$.
  \end{enumerate}
\end{theorem}
\begin{proof}
  See appendix.
\end{proof}

As in \cite{adding-logic}, $\mathcal{L}$ represents the set of formulas of the
formal language. We define an auxiliary function on formulas, \textsf{exp}, as
follows.

\begin{definition}
  The function $\textsf{exp} : \mathcal{L} \to \mathcal{L}^*$ is inductively
  defined as follows:

\begin{align*}
  \textsf{exp}(p) & = p \\
  \textsf{exp}(A \otimes B) & = \textsf{exp}(A), \textsf{exp}(B) \\
  \textsf{exp}(A \rightarrow_{\reactlist{}}^S B) & = A \rightarrow_{\reactlist{}}^S B
\end{align*}
\end{definition}

\begin{lemma}\label{explemma}
  If $\ztriseq{\Gamma}{\Delta, \textsf{exp}(A)}{\Omega}{\reactlist{}}{C}$,
  then $\ztriseq{\Gamma}{\Delta}{\Omega, A}{\reactlist{}}{C}$.
\end{lemma}
\begin{proof}
  By induction on $A$. If $A$ is an atom or an implication, the thesis
  follows by act. If $A \equiv C \otimes D$, then
  $\textsf{exp}(A) = \textsf{exp}(C \otimes D) = \textsf{exp}(C),
  \textsf{exp}(D)$, so by inductive hypothesis

  \[
    \begin{prooftree}
      \[
        \[
          \ztriseq{\Gamma}{\Delta, \textsf{exp}(C), \textsf{exp}(D)}{\Omega}{\reactlist{}}{E}
          \justifies
          \ztriseq{\Gamma}{\Delta, \textsf{exp}(C)}{\Omega, D}{\reactlist{}}{E}
        \]
        \justifies
        \ztriseq{\Gamma}{\Delta}{\Omega, C, D}{\reactlist{}}{E}
      \]
      \justifies
      \ztriseq{\Gamma}{\Delta}{\Omega, C \otimes D}{\reactlist{}}{E}
    \end{prooftree}
  \]
\end{proof}

\begin{lemma}[Identity expansions]\label{idexp}
  For any formula $A$,
  \begin{enumerate}
  \item $\ztriseq{\Gamma}{\cdot}{A}{\emptyctrl{}}{A}$;
  \item $\zfocseq{\Gamma}{\textsf{exp}(A)}{A}$.
  \end{enumerate}
\end{lemma}
\begin{proof}
  Both proved simultaneously by induction on $A$. If $A$ is an atom, both follow
  trivially. Otherwise,

  \begin{enumerate}
  \item If $A \equiv C \otimes D$, then by inductive hypothesis

    \[
      \begin{prooftree}
        \zfocseq{\Gamma}{\textsf{exp}(A)}{A}
        \qquad
        \zfocseq{\Gamma}{\textsf{exp}(B)}{B}
        \justifies
        \zfocseq{\Gamma}{\textsf{exp}(A), \textsf{exp}(B)}{A \otimes B}
        \using{\otimes R}
      \end{prooftree}
    \]

    Then, by Lemma~\ref{explemma} and the derivation above we get the first
    point.

    \[
      \begin{prooftree}
        \[
          \[
            \zfocseq{\Gamma}{\textsf{exp}(A)}{A}
            \qquad
            \zfocseq{\Gamma}{\textsf{exp}(B)}{B}
            \justifies
            \zfocseq{\Gamma}{\textsf{exp}(A), \textsf{exp}(B)}{A \otimes B}
            \using{\otimes R}
          \]
          \justifies
          \ztriseq{\Gamma}{\textsf{exp}(A), \textsf{exp}(B)}{\cdot}{\emptyctrl{}}{A \otimes B}
        \]
        \justifies
        \ztriseq{\Gamma}{\cdot}{A \otimes B}{\emptyctrl{}}{A \otimes B}
      \end{prooftree}
    \]

  \item If $A \equiv C \rightarrow_{\reactlist{}}^S D$, then by inductive
    hypothesis and Lemma~\ref{explemma} we get

    \[
      \begin{prooftree}
        \[
          \[
            \[
              \zfocseq{\Gamma}{\textsf{exp}(C)}{C}
              \qquad
              \ztriseq{\Gamma}{\cdot}{D}{\emptyctrl{}}{D}
              \justifies
              \ztriseq{\Gamma}{C \rightarrow_{\reactlist{}}^S D,
                \textsf{exp}(C)}{}{\reactlist{}}{D}
            \]
            \justifies
            \ztriseq{\Gamma}{C \rightarrow_{\reactlist{}}^S D}{C}{\reactlist{}}{D}
          \]
          \justifies
          \ztriseq{\Gamma}{C \rightarrow_{\reactlist{}}^S D}{\cdot}{\emptyctrl{}}{
            C \rightarrow_{\reactlist{}}^S D}
        \]
        \justifies
        \ztriseq{\Gamma}{\cdot}{C \rightarrow_{\reactlist{}}^S D}{\emptyctrl{}}{
          C \rightarrow_{\reactlist{}}^S D}
      \end{prooftree}
    \]

    The second point follows by blur on the derivation above.
  \end{enumerate}
\end{proof}

\begin{lemma}\label{idhelplemma}
  The following are derivable:
  \begin{enumerate}
  \item $\ztriseq{\Gamma}{P}{\cdot}{\emptyctrl{}}{P}$;
  \item $\ztriseq{\Gamma}{\cdot}{A, B}{\emptyctrl{}}{A \otimes B}$;
  \item $\ztriseq{\Gamma}{\cdot}{A \rightarrow_{\reactlist{}}^S B,
      A}{\reactlist{}}{B}$;
  \item $\ztriseq{\Gamma, A}{\cdot}{\cdot}{\emptyctrl{}}{A}$.
  \end{enumerate}
\end{lemma}
\begin{proof}
  All points proved easily with the same technique as the identity expansion
  lemma.
\end{proof}

\begin{lemma}\label{activeinversion}\mbox{}
  \begin{enumerate}
  \item If $\ztriseq{\Gamma}{\Delta}{\Omega, A \otimes B}{\reactlist{}}{C}$, then
    $\ztriseq{\Gamma}{\Delta}{\Omega, A, B}{\reactlist{}}{C}$;
  \item If $\ztriseq{\Gamma}{\Delta}{\Omega, \Omega'}{\reactlist{}}{C}$, then
    $\ztriseq{\Gamma}{\Delta, \textsf{exp}(\Omega')}{\Omega}{\reactlist{}}{C}$
  \end{enumerate}
\end{lemma}
\begin{proof}
  Point 1 is proved by cut with
  $\ztriseq{\Gamma}{\cdot}{A, B}{\emptyctrl{}}{A \otimes B}$. Point 2 is proved by
  repeated application of point 1 on $\otimes$ formulas until they become either
  an atom or an implication, then by cut with
  $\ztriseq{\Gamma}{P}{\cdot}{\emptyctrl{}}{P}$.
\end{proof}

\begin{theorem}[Completeness]
  If $\zsyseq{\Gamma}{\Omega}{\reactlist{}}{A}$, then
  $\ztriseq{\Gamma}{\cdot}{\Omega}{\reactlist{}}{A}$.
\end{theorem}
\begin{proof}
  The proof proceeds as in \cite{chaudhuri-thesis}, by proving that all rules of
  the backward calculus are admissible in the focused calculus.

  \begin{enumerate}
  \item Case $\init$: $\zsyseq{\Gamma}{A}{\emptyctrl{}}{A}$. The thesis follows by
    Lemma~\ref{idexp}.
  \item Case $\copyrule$. Then, by inductive hypothesis, Lemma~\ref{idhelplemma}
    and cut admissibility, we have

    \[
      \begin{prooftree}
        \ztriseq{\Gamma, A}{\cdot}{\cdot}{\emptyctrl{}}{A}
        \qquad
        \ztriseq{\Gamma, A}{\cdot}{\Omega, A}{\reactlist{}}{C}
        \justifies
        \ztriseq{\Gamma, A}{\cdot}{\Omega}{\reactlist{}}{C}
      \end{prooftree}
    \]

  \item Case $\otimes L$. The thesis follows immediately by a single application
    of the analogous $\otimes L$ rule in the focused calculus.
  \item Case $\otimes R$. Then, by inductive hypothesis, Lemma~\ref{idhelplemma}
    and cut admissibility, we have

    \[
      \begin{prooftree}
        \ztriseq{\Gamma}{\cdot}{\Omega_2}{\emptyctrl{}}{B}
        \quad
        \[
          \ztriseq{\Gamma}{\cdot}{\Omega_1}{\emptyctrl{}}{A}
          \qquad
          \ztriseq{\Gamma}{\cdot}{A, B}{\emptyctrl{}}{A \otimes B}
          \justifies
          \ztriseq{\Gamma}{\cdot}{\Omega_1, B}{\emptyctrl{}}{A \otimes B}
        \]
        \justifies
        \ztriseq{\Gamma}{\cdot}{\Omega_1, \Omega_2}{\emptyctrl{}}{A \otimes B}
      \end{prooftree}
    \]

  \item Case $\rightarrow L$. Then, by inductive hypothesis,
    Lemma~\ref{idhelplemma} and cut admissibility, we have

    \[
      \begin{prooftree}
        \ztriseq{\Gamma}{\cdot}{\Omega_1}{\emptyctrl{}}{A}
        \quad
        \[
          \ztriseq{\Gamma}{\cdot}{
            A, A \rightarrow_{\ctrlset{}}^S B
          }{\reactlist{}}{B}
          \qquad
          \ztriseq{\Gamma}{\cdot}{\Omega_2, B}{\reactlist{}'}{C}
          \justifies
          \ztriseq{\Gamma}{\cdot}{
            \Omega_2, A, A \rightarrow_{\reactlist{}}^S B}{
            \baseandplus{\Omega_2}{\reactlist{}}{\reactlist{}'}
          }{C}
          \using{\resplist{\Omega_2}{\reactlist{}}}
        \]
        \justifies
        \ztriseq{\Gamma}{\cdot}{\Omega_1, \Omega_2,
          A \rightarrow_{\ctrlset{}}^S B}{
          \baseandplus{\Omega_2}{\reactlist{}}{\reactlist{}'}
      }{C}
      \end{prooftree}
    \]

  \item Case $\rightarrow R$. By inductive hypothesis and
    Lemma~\ref{activeinversion}, we have

    \[
      \begin{prooftree}
        \ztriseq{\Gamma}{\cdot}{\Omega, A}{\reactlist{}}{B}
        \justifies
        \ztriseq{\Gamma}{\textsf{exp}(\Omega)}{A}{\reactlist{}}{B}
      \end{prooftree}
    \]

    It is an easy induction to see that
    $\elembases{\Omega} = \elembases{\textsf{exp}(\Omega)}$. Therefore,

    \[
      \begin{prooftree}
        \[
          \ztriseq{\Gamma}{\cdot}{\Omega, A}{\reactlist{}}{B}
          \justifies
          \ztriseq{\Gamma}{\textsf{exp}(\Omega)}{A}{\reactlist{}}{B}
        \]
        \justifies
        \ztriseq{\Gamma}{\textsf{exp}(\Omega)}{\cdot}{\emptyctrl{}}{
          A \rightarrow_{\reactlist{}}^{\elembases{\Omega}} B}
      \end{prooftree}
    \]

    Then, by repeated application of act and $\otimes L$, we get the conclusion
    $\ztriseq{\Gamma}{\cdot}{\Omega}{\emptyctrl{}}{
          A \rightarrow_{\reactlist{}}^{\elembases{\Omega}} B}$.

  \end{enumerate}
\end{proof}

\subsection{Backward derived rules}

One of the benefits of focusing is the ability to generate derived ``big step''
inference rules: the intermediate results of a focusing or active phase are not
important, since those steps are ``forced'' in some way. Each derived rule
starts (at the bottom) with a neutral sequent from which a synchronous
proposition is selected for focus, and the focusing steps are followed. Then the
active rules are applied, and eventually we obtain a collection of neutral
sequents as the leaves. These neutral sequents are then treated as the premises
of the derived rule that produces the neutral sequent with which we started.

Since actions during the active and focused phase are completely predictable, we
can actually build derived \emph{big-step} rules in a systematic way. To this
aim, we now construct a calculus of backward derived rules, where the inference
rules correspond to big step derived rules of the backward focused calculus just
defined. The general design is that intermediate sequents produced in the active
and focusing phases need not be stored in any sequent database during proof
search; instead, all sequents constructed during search are neutral sequents at
the phase boundaries. When the next sections introduce the inverse method, it
will be shown how it is possible to precompute only the derived rules that are
needed for a given goal sequent.

For any given proposition, we are interested in constructing a derived rule for
that proposition corresponding to a single pair of focusing and active
phases. The idea is to interpret a proposition itself as the derived rules that
it embodies. Every proposition is viewed as a relation between the conclusion of
the rule and its premises at the leaves of the derived rule. Both the conclusion
and the premises are neutral sequents, which we indicate as
$\zbneuseq{\Gamma}{\Delta}{\reactlist{}}{Q}$.

There are two classes of relational interpretations:

\begin{enumerate}
\item Focal relations for the focused formula $A$, written $\brfrel{A}$;
\item Active relations, written
  $\bactrel{\ztriseq{\Gamma}{\Delta}{\Omega}{\zeta}{\xi}}$, where $\xi$ is
  either $\cdot$ or a proposition $C$, and $\zeta$ is either $\cdot$ or a
  reaction list $l$.
\end{enumerate}

The full set of relations is given in Figure~\ref{fig:bkwdrelations}. Each
relation $R$ takes as input the conclusion sequent $s$, and produces a sequence
of premise sequents $\Sigma = s_1, \dots, s_n$; we write this as
$\relj{R}{s}{\Sigma}$. The intuition is that when $\relj{R}{s}{\Sigma}$ holds,
then there exists a derivation from the premise sequents $\Sigma$ to the
conclusion $s$ in the backward focused calculus with a single pair of phases
consisting, bottom-up, of a focused phase starting from the conclusion sequent,
followed by an active phase that ends on the premises.

We use these relations to form a calculus of derived
rules acting on neutral sequents only (Figure~\ref{fig:bkwdderivedcalculus}).

\begin{figure}[h]
  \begin{mdframed}
    \[
      \begin{prooftree}
        \justifies
        \relj{\brfrel{p}}{\bneuseq{\Gamma}{p}{\cdot}}{\cdot}
        \using{\init}
      \end{prooftree}
    \]

    \[
      \begin{prooftree}
        \relj{\brfrel{A}}{\bneuseq{\Gamma}{\Delta_1}{\cdot}}{\Sigma_1}
        \qquad
        \relj{\brfrel{B}}{\bneuseq{\Gamma}{\Delta_2}{\cdot}}{\Sigma_2}
        \justifies
        \relj{\brfrel{A \otimes B}}{\bneuseq{\Gamma}{\Delta_1,
            \Delta_2}{\cdot}}{\Sigma_1 \cdot \Sigma_2}
        \using{\otimes F}
      \end{prooftree}
    \]

    \[
      \begin{prooftree}
        \justifies
        \relj{\brfrel{R}}{\zbneuseq{\Gamma}{\Delta}{}{\cdot}}{
          \zbneuseq{\Gamma}{\Delta}{\emptyset}{R}}
        \using{\faplus}
      \end{prooftree}
    \]

    \[
      \begin{prooftree}
        \relj{\bactrel{\zsyseq{\Delta}{\Omega \cdot A \cdot B}{\zeta}{\xi}}}{s}{\Sigma}
        \justifies
        \relj{\bactrel{\zsyseq{\Delta}{\Omega \cdot A \otimes B}{\zeta}{\xi}}}{s}{\Sigma}
        \using{\otimes A}
      \end{prooftree}
    \]

    \[
      \begin{prooftree}
        \relj{
          \bactrel{
            \zsyseq{\Delta,P}{\Omega}{\zeta}{\xi}
          }
        }{s}{\Sigma}
        \justifies
        \relj{\bactrel{\zsyseq{\Delta}{\Omega, P}{\zeta}{\xi}}}{s}{\Sigma}
        \using{act}
      \end{prooftree}
    \]

    \[
      \begin{prooftree}
        \justifies
        \relj{
          \bactrel{\zsyseq
            {\Delta}{\cdot}{}{\cdot}}
        }{
          \zbneuseq{\Gamma}{\Delta'}{\reactlist{}}{Q}
        }{
          \zbneuseq{\Gamma}{\Delta, \Delta'}{\reactlist{}}{Q}
        }
        \using{\matchrule}
      \end{prooftree}
    \]

    \[
      \begin{prooftree}
        \justifies
        \relj{
          \bactrel{\zsyseq{\Delta}{\cdot}{\reactlist{}}{Q}}
        }{
          \bneuseq{\Gamma}{\Delta'}{\cdot}
        }{
          \zbneuseq{\Gamma}{\Delta, \Delta'}{\reactlist{}}{Q}
        }
        \using{\matchprimerule}
      \end{prooftree}
    \]
  \end{mdframed}
  \caption{Backward relations for derived rules}
  \label{fig:bkwdrelations}
\end{figure}

\begin{figure}[h]
  \begin{mdframed}
    \[
      \begin{prooftree}
        (\relj{\brfrel{Q}}{\zbneuseq{\Gamma}{\Delta}{}{\cdot}}{\Sigma})
        \qquad \Sigma
        \justifies
        \zbneuseq{\Gamma}{\Delta}{\emptyctrl{}}{Q}
        \using{focus}
      \end{prooftree}
    \]

    \[
      \begin{prooftree}
        \[
        (\relj{\brfrel{A}}{\zbneuseq{\Gamma}{\Delta_1}{}{\cdot}}{\Sigma})
        \proofdotseparation=1.2ex
        \proofdotnumber=0
        \leadsto
        (\relj{\bactrel{\zsyseq{\cdot}{B}{}{\cdot}}}
        {\zbneuseq{\Gamma}{\Delta_2}{\reactlist{}'}{Q}}{\Sigma'})
        \]
        \qquad \Sigma, \Sigma'
        \qquad \resplist{\Delta_2}{\reactlist{}}
        \justifies
        \zbneuseq{\Gamma}{\Delta_1, \Delta_2,
          A \rightarrow_{\reactlist{}}^S B
        }{
          \baseandplus{\Delta_2}{\reactlist{}}{\reactlist{}'}
        }{Q}
        \using{\rightarrow L}
      \end{prooftree}
    \]

    \[
      \begin{prooftree}
        (\relj{\bactrel{\zsyseq{\cdot}{A}{\reactlist{}}{B}}}
        {\zbneuseq{\Gamma}{\Delta}{}{\cdot}}{\Sigma})
        \qquad \Sigma
        \justifies
        \zbneuseq{\Gamma}{\Delta}{\emptyctrl{}}{
          A \rightarrow_{\reactlist{}}^{\elembases{\Delta}} B}
        \using{\rightarrow R}
      \end{prooftree}
    \]

    \[
      \begin{prooftree}
        \[
          (\relj{\brfrel{A}}{\zbneuseq{\Gamma,
              A \rightarrow_{\reactlist{}}^{\emptyset} B
            }{\Delta_1}{}{\cdot}}{\Sigma})
        \proofdotseparation=1.2ex
        \proofdotnumber=0
        \leadsto
        (\relj{\bactrel{\zsyseq{\cdot}{B}{}{\cdot}}}
        {\zbneuseq{\Gamma,
            A \rightarrow_{\reactlist{}}^{\emptyset} B
          }{\Delta_2}{\reactlist{}'}{Q}}{\Sigma'})
        \]
        \qquad \Sigma, \Sigma'
        \qquad \resplist{\Delta_2}{\reactlist{}}
        \justifies
        \zbneuseq{\Gamma, A \rightarrow_{\reactlist{}}^{\emptyset} B}{\Delta_1, \Delta_2
        }{
          \baseandplus{\Delta_2}{\reactlist{}}{\reactlist{}'}
        }{Q}
        \using{\copyrule}
      \end{prooftree}
    \]

  \end{mdframed}
  \caption{Backward calculus of derived rules}
  \label{fig:bkwdderivedcalculus}
\end{figure}

\begin{definition}
  A sequent of the backward derived rule calculus
  $\zbneuseq{\Gamma}{\Delta}{\reactlist{}}{Q}$
  is sound if $\ztriseq{\Gamma}{\Delta}{\cdot}{\reactlist{}}{Q}$.
\end{definition}

\begin{lemma}\label{bkwdder-soundness-lemma}
  \begin{enumerate}
  \item If $\brfrelj{A}{\zbneuseq{\Gamma}{\Delta}{}{\cdot}}{\Sigma}$
    and $\Sigma$
    are sound, then

    $\zfocseq{\Gamma}{\Delta}{A}$.
  \item If
    $\bactrelj{\zsyseq{\Delta}{\Omega}{}{\cdot}}
    {\zbneuseq{\Gamma}{\Delta'}{\reactlist{}}{Q}}{\Sigma}$
    and $\Sigma$ are sound, then

    $\ztriseq{\Gamma}{\Delta, \Delta'}{\Omega}{\reactlist{}}{Q}$.
  \item If
    $\bactrelj{\zsyseq{\Delta}{\Omega}{\reactlist{}}{Q}}
    {\zbneuseq{\Gamma}{\Delta'}{}{\cdot}}{\Sigma}$
    and $\Sigma$ are sound, then

    $\ztriseq{\Gamma}{\Delta, \Delta'}{\Omega}{\reactlist{}}{Q}$.
  \end{enumerate}
\end{lemma}
\begin{proof}
  All points are proved simultaneously and follow immediately by mutual
  induction on the height of the derivations and use of the corresponding rules
  of the focused calculus.
\end{proof}

\begin{theorem}[Soundness]
  If $\zbneuseq{\Gamma}{\Delta}{\reactlist{}}{Q}$ then
  $\ztriseq{\Gamma}{\Delta}{\cdot}{\reactlist{}}{Q}$.
\end{theorem}
\begin{proof}
  Straightforward induction on the derivation of the backward derived rules
  calculus, and Lemma~\ref{bkwdder-soundness-lemma}.
\end{proof}

\begin{lemma}\label{completeness-lemma}
  \begin{enumerate}

  \item If $\zfocseq{\Gamma}{\Delta}{A}$, then for some $\Sigma$
    \begin{enumerate}
    \item $\relj{\brfrel{A}}{\zbneuseq{\Gamma}{\Delta}{}{\cdot}}{\Sigma}$, and
    \item $\Sigma$ are all derivable
    \end{enumerate}

  \item If $\ztriseq{\Gamma}{\Delta_1, \Delta_2}{\Omega}{\zeta \uplus \delta}{\xi \uplus
      \gamma}$ (where $x \uplus y$ means either $x$ or $y$ is
    empty), then for some $\Sigma$
    \begin{enumerate}
    \item
      $\bactrelj{\zsyseq{\Delta_1}{\Omega}{\zeta}{\xi}}
      {\zbneuseq{\Gamma}{\Delta_2}{\delta}{\gamma}}{\Sigma}$,
      and
    \item $\Sigma$ are all derivable.
    \end{enumerate}
  \end{enumerate}
\end{lemma}
\begin{proof}
  By simultaneous induction on the height of the derivation.
  \begin{enumerate}
  \item Case $\init$. Then, just apply the $\init$ relation.
  \item Case act. Straightforward use of the act rule for derived
    relations.
  \item Case $focus$.

    \[
      \begin{prooftree}
        \zfocseq{\Gamma}{\Delta_1, \Delta_2}{Q} \qquad Q\; \text{right-focusable}
        \justifies
        \ztriseq{\Gamma}{\Delta_1, \Delta_2}{\cdot}{\emptyctrl{}}{Q}
        \using{focus}
      \end{prooftree}
    \]

    Suppose $\gamma_1, \gamma_2$ are such that $\gamma_1 \uplus \gamma_2 =
    Q$, and $\zeta_1 \uplus \zeta_2 = \emptyctrl{}$. Then, the following:

    \[
      \relj{
        \bactrel{\zsyseq{\Delta_1}{\cdot}{\zeta_1}{\gamma_1}}
      }{
        \zbneuseq{\Delta_2}{\cdot}{\zeta_2}{\gamma_2}
      }{
        \zbneuseq{\Gamma}{\Delta_1, \Delta_2}{\emptyctrl{}}{Q}
      }
    \]

    is derivable with either $\matchrule$ or $\matchprimerule$. We need to show
    that $\zbneuseq{\Gamma}{\Delta_1, \Delta_2}{\emptyctrl{}}{Q}$ is derivable. By
    inductive hypothesis, we have
    $\relj{\brfrel{Q}}{\zbneuseq{\Gamma}{\Delta_1, \Delta_2}{}{\cdot}}{\Sigma}$ for
    some $\Sigma$ all derivable. But then,

    \[
      \begin{prooftree}
        \relj{\brfrel{Q}}{\zbneuseq{\Gamma}{\Delta_1, \Delta_2}{}{\cdot}}{\Sigma}
        \qquad \Sigma
        \justifies
        \zbneuseq{\Gamma}{\Delta_1, \Delta_2}{\emptyctrl{}}{Q}
        \using{focus}
      \end{prooftree}
    \]

  \item Case $blur$.

    \[
      \begin{prooftree}
        \ztriseq{\Gamma}{\Delta}{\cdot}{\emptyctrl{}}{R}
        \justifies
        \zfocseq{\Gamma}{\Delta}{R}
        \using{blur}
      \end{prooftree}
    \]

    By inductive hypothesis, we have
    $\relj{\bactrel{\zsyseq{\cdot}{\cdot}{\emptyctrl{}}{R}}}{\bneuseq{\Gamma}{\Delta}{\cdot}}{\Sigma}$,
    where all $\Sigma$ are derivable.
    But then is must have been derived with an instance of $\matchrule$, hence
    $\zbneuseq{\Gamma}{\Delta}{\emptyctrl{}}{R}$ is derivable.
    We then have

    \[
      \begin{prooftree}
        \justifies
        \relj{\brfrel{R}}{\bneuseq{\Gamma}{\Delta}{\cdot}}{
          \zbneuseq{\Gamma}{\Delta}{\emptyctrl{}}{R}
        }
        \using{FA^+}
      \end{prooftree}
    \]

  \item Case $\rightarrow L$.

    \[
      \begin{prooftree}
        \zfocseq{\Gamma}{\Delta_1}{A}
        \qquad
        \ztriseq{\Gamma}{\Delta_2}{B}{\reactlist{}'}{C}
        \qquad
        \resplist{\Delta_2}{\reactlist{}}
        \justifies
        \ztriseq{\Gamma}{
          \Delta_1, \Delta_2, A \rightarrow_{\reactlist{}}^S B
        }{\cdot}{\baseandplus{\Delta_2}{\reactlist{}}{\reactlist{}'}}{C}
        \using{\rightarrow L}
      \end{prooftree}
    \]

    Suppose $\Delta', \Delta'' = \Delta_1, \Delta_2,
    A\rightarrow_{\reactlist{}}^S B$, and $\gamma_1 \uplus \gamma_2 = C$,
    $\zeta_1 \uplus \zeta_2 = \baseandplus{\Delta_2}{\reactlist{}}{\reactlist{}'}$.
    We can apply $\matchrule$ to immediately derive


    \[
      \begin{prooftree}
        \justifies
        \relj{\bactrel{
            \zsyseq{
              \Delta'}{\cdot}{\zeta_1}{\gamma_1}}}
        {\zbneuseq{\Gamma}
          {\Delta''}{\zeta_2}{\gamma_2}}{\zbneuseq{
            \Gamma}{\Delta', \Delta''}{
            \baseandplus{\Delta_2}{\reactlist{}}{\reactlist{}'}
          }{C}}
        \using{\matchrule}
      \end{prooftree}
    \]

    We can see that
    $\zbneuseq{\Gamma} {\Delta', \Delta''}{
      \baseandplus{\Delta_2}{\reactlist{}}{\reactlist{}'}}{C}$ is
    derivable by inductive hypothesis on the premises and an application of
    $\rightarrow L$ of the derived rule calculus.

    \[
      \begin{prooftree}
        \[
        (\relj{\brfrel{A}}{\zbneuseq{\Gamma}{\Delta_1}{}{\cdot}}{\Sigma})
        \proofdotseparation=1.2ex
        \proofdotnumber=0
        \leadsto
        (\relj{\bactrel{\zsyseq{\cdot}{B}{}{\cdot}}}
        {\zbneuseq{\Gamma}{\Delta_2}{\reactlist{}'}{C}}{\Sigma'})
        \]
        \qquad \Sigma, \Sigma'
        \qquad \resplist{\Delta_2}{\reactlist{}}
        \justifies
        \zbneuseq{\Gamma}{\Delta_1, \Delta_2,
          A \rightarrow_{\reactlist{}}^S B
        }{
          \baseandplus{\Delta_2}{\reactlist{}}{\reactlist{}'}
        }{C}
        \using{\rightarrow L}
      \end{prooftree}
    \]

  \item The case $\copyrule$ is analogous to $\rightarrow L$.

  \item Case $\rightarrow R$.

    \[
      \begin{prooftree}
        \ztriseq{\Gamma}{\Delta_1, \Delta_2}{A}{\reactlist{}}{B}
        \justifies
        \ztriseq{\Gamma}{\Delta_1, \Delta_2}{\cdot}{\emptyctrl{}}{
          A \rightarrow_{\reactlist{}}^{\elembases{\Delta_1, \Delta_2}} B}
        \using{\rightarrow R}
      \end{prooftree}
    \]

    Again, suppose
    $\gamma_1 \uplus \gamma_2 = A \rightarrow_{\reactlist{}}^{\elembases{\Delta_1,
        \Delta_2}} B$, $\zeta_1 \uplus \zeta_2 = \emptyctrl{}$. We can immediately
    derive by either $\matchrule$ or $\matchprimerule$ the following

    \[
      \begin{prooftree}
        \justifies
        \relj{\bactrel{
            \zsyseq{
              \Delta_1}{\cdot}{\zeta_1}{\gamma_1}}}
        {\zbneuseq{\Gamma}
          {\Delta_2}{\zeta_2}{\gamma_2}}{\zbneuseq{
            \Gamma}{\Delta_1, \Delta_2}{
            \emptyctrl{}
          }{A \rightarrow_{\reactlist{}}^{\elembases{\Delta_1, \Delta_2}} B}}
        \using{\matchrule}
      \end{prooftree}
    \]

    To see that
    $\zbneuseq{ \Gamma}{\Delta_1, \Delta_2}{ \emptyctrl{}}{A
      \rightarrow_{\reactlist{}}^{\elembases{\Delta_1, \Delta_2}} B}$ is
    derivable, we use the inductive hypothesis on the premise of $\rightarrow R$
    and an application of the corresponding rule of the derived rule calculus.

    \[
      \begin{prooftree}
        (\relj{\bactrel{\zsyseq{\cdot}{A}{\reactlist{}}{B}}}
        {\zbneuseq{\Gamma}{\Delta_1, \Delta_2}{}{\cdot}}{\Sigma})
        \quad \Sigma
        \justifies
        \zbneuseq{\Gamma}{\Delta_1, \Delta_2}{\emptyctrl{}}{
          A \rightarrow_{\reactlist{}}^{\elembases{\Delta_1, \Delta_2}} B}
        \using{\rightarrow R}
      \end{prooftree}
    \]


  \item The remaining $\otimes L$ and $\otimes R$ cases are straightforward
    applications of the inductive hypothesis.
  \end{enumerate}
\end{proof}

\begin{theorem}[Completeness]
  If $\ztriseq{\Gamma}{\Delta}{\cdot}{\reactlist{}}{Q}$, then
  $\zbneuseq{\Gamma}{\Delta}{\reactlist{}}{Q}$.
\end{theorem}
\begin{proof}
  Straightforward induction on the derivation and application of
  Lemma~\ref{completeness-lemma}.
\end{proof}

\subsection{Remarks}

We have seen that focusing can be used as a powerful tool to make proof search
in linear logic more feasible. Even though many proof-theoretical properties
that make focusing in the context of linear logic possible do not hold in our
case, we were still able to adapt some of the inspiring ideas to our backward
sequent calculus and obtain some improvements. The next section will continue
following \cite{chaudhuri-thesis}, with the conversion of the focused calculus
we just defined to the forward direction.

%%% Local Variables:
%%% mode: latex
%%% TeX-master: "../docs"
%%% End:
