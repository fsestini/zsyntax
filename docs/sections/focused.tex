\section{Focused derivations}

A backward focused proof has two phases. In the active phase all possible rules
are applied in an arbitrary order to asynchronous propositions. When only
synchronous propositions remain, one proposition is selected and a \emph{focused
  phase} for that proposition begins.

As out backward linear sequent calculus is two-sided, we have left- and right-
synchronous and asynchronous connectives.

\begin{table}[h]
  \centering
  \begin{tabular}{|l|l|}
    \hline
    \textbf{symbol} & \textbf{meaning} \\
    \hline
    $P$ & left-synchronous ($\limp$) \\
    $Q$ & right-synchronous ($\otimes$) \\
    $L$ & left-asynchronous ($\otimes$) \\
    $R$ & right-asynchronous ($\limp$) \\
    \hline
  \end{tabular}
\end{table}

The above table does not include the atomic propositions. For reasons explained
in [thesis], here we are forced to treat them as synchronous. However, we can
differenciate the atoms by means of a \emph{focusing bias}, which indicates
whether the atomic proposition under focus must immediately be derived in an
initial sequent.

The backward focusing calculus consists of the following kinds of sequents:

\begin{table}[h]
  \centering
  \begin{tabular}{|l|l|}
    \hline
    \textbf{symbol} & \textbf{meaning} \\
    \hline
    $\Gamma; \Delta \gg A$ & right-focal sequent with $A$ under focus \\
    $\Gamma; \Delta; A \ll Q$ & left-focal sequent with $A$ under focus \\
    $\Gamma; \Delta; \Omega \Longrightarrow C; \cdot$ & right-active sequent \\
    $\Gamma; \Delta; \Omega \Longrightarrow \cdot; Q$ & left-active sequent \\
    \hline
  \end{tabular}
\end{table}

Here, $\Delta$ contains only left-synchronous propositions, i.e., it is of the
form $P_1, \dots, P_n$. $\Omega$ is an ordered context of propositions which may
be synchronous of asynchronous.

For active sequents the right active propositions are decomposed until they
become right-synchronous, i.e. the sequent is of the form $\Gamma; \Delta;
\Omega \Longrightarrow Q; \cdot$. The right hand side is then changed to $\cdot;
Q$. Similarly, the propositions in $\Omega$ are decomposed except when the
proposition is left-synchronous, in which case it is transferred to $\Delta$.

Eventually, the active sequent is reduced to the form $\Gamma; \Delta; \cdot
\Longrightarrow \cdot \cdot; Q$, which we call \emph{neutral sequents}.
A focusing phase is launched from such a neutral sequent by selecting a
\emph{focusable} proposition and giving it the corresponsing focus.

\begin{definition}[Focusable proposition]
  \begin{enumerate}
  \item A proposition is right-focusable if it is right-synchronous and not a
    right-biased atom;
  \item A proposition is left-focusable if it is left-sunchronous and not a
    left-biased atom.
  \end{enumerate}
\end{definition}

When we are in a neutral sequent, we may copy a proposition out of the
unrestricted context and immediately focus on it, regardless of whether it is
focusable or not. If this proposition is actually left-asynchronous, then we
will immediately remove focus on it and transition to an active phase.

There are two forms of the initial sequent, corresponding to the two focusing
biases. If the focal proposition becomes atomic, we terminate with one of the
two initial forms. If the focal proposition is asynchronous, we blur the focus.
If the focal proposition is atomic and of the wrong bias, then also we blur the
focus, but in this case we transition directly to the neutral sequent instead of
entering the active phase.

Decomposing focal propositions uses non-invertible rules for that proposition,
and focus is maintained to the operands of the top-level connective of the
proposition.


\[
  \begin{prooftree}
    p \; \text{left-biased}
    \justifies
    \Gamma; p \gg p
    \using{rinit}
  \end{prooftree}
  \qquad \qquad
  \begin{prooftree}
    p \; \text{right-biased}
    \justifies
    \Gamma; \cdot; p \ll p
    \using{linit}
  \end{prooftree}
\]

\[
  \begin{prooftree}
    \Gamma; \Delta; \Omega \Longrightarrow \cdot; Q
    \justifies
    \Gamma; \Delta; \Omega \Longrightarrow Q; \cdot
    \using{ract}
  \end{prooftree}
  \qquad \qquad
  \begin{prooftree}
    \Gamma; \Delta, P; \Omega \cdot \Omega' \Longrightarrow \gamma
    \justifies
    \Gamma; \Delta; \Omega \cdot P \cdot \Omega' \Longrightarrow \gamma
    \using{lact}
  \end{prooftree}
\]

\[
  \begin{prooftree}
    \Gamma; \Delta \gg Q \qquad Q\; \text{right-focusable}
    \justifies
    \Gamma; \Delta; \cdot \Longrightarrow \cdot ; Q
    \using{rfoc}
  \end{prooftree}
  \qquad \qquad
  \begin{prooftree}
    \Gamma; \Delta; P \ll Q \qquad P\; \text{left-focusable}
    \justifies
    \Gamma; \Delta, P; \cdot \Longrightarrow \cdot; Q
    \using{lfoc}
  \end{prooftree}
\]

\[
  \begin{prooftree}
    \Gamma; \Delta; \cdot \Longrightarrow R; \cdot
    \justifies
    \Gamma; \Delta \gg R
    \using{rblur}
  \end{prooftree}
  \qquad \qquad
  \begin{prooftree}
    \Gamma; \Delta; L \Longrightarrow \cdot; Q
    \justifies
    \Gamma; \Delta; L \ll Q
    \using{lblur}
  \end{prooftree}
\]

\[
  \begin{prooftree}
    \Gamma; \Delta; \cdot \Longrightarrow \cdot ; p \qquad p \; \text{right-biased}
    \justifies
    \Gamma; \Delta \gg p
    \using{rblur^*}
  \end{prooftree}
  \qquad \qquad
  \begin{prooftree}
    \Gamma; \Delta, p; \cdot \Longrightarrow \cdot; Q \qquad p \; \text{left-biased}
    \justifies
    \Gamma; \Delta; p \ll Q
    \using{lblur^*}
  \end{prooftree}
\]

\[
  \begin{prooftree}
    \Gamma, A; \Delta ; A \ll Q
    \justifies
    \Gamma, A; \Delta; \cdot \Longrightarrow \cdot; Q
    \using{copy}
  \end{prooftree}
\]

\[
  \begin{prooftree}
    \Gamma; \Delta; \Omega \cdot A \cdot B \cdot \Omega' \Longrightarrow \gamma
    \justifies
    \Gamma; \Delta; \Omega \cdot A \otimes B \cdot \Omega' \Longrightarrow
    \gamma
    \using{\otimes L}
  \end{prooftree}
  \qquad \qquad
  \begin{prooftree}
    \Gamma; \Delta_1 \gg A \qquad \Gamma; \Delta_2 \gg B
    \justifies
    \Gamma; \Delta_1, \Delta_2 \gg A \otimes B
    \using{\otimes R}
  \end{prooftree}
\]

\[
  \begin{prooftree}
    \Gamma; \Delta_1; B \ll Q \qquad \Gamma; \Delta_2 \gg A
    \justifies
    \Gamma; \Delta_1, \Delta_2 ; A \limp B \ll Q
    \using{\limp L}
  \end{prooftree}
  \qquad \qquad
  \begin{prooftree}
    \Gamma; \Delta; \Omega \cdot A \Longrightarrow B; \cdot
    \justifies
    \Gamma; \Delta; \Omega \Longrightarrow A \limp B; \cdot
    \using{\limp R}
  \end{prooftree}
\]

\begin{theorem}[Soundness]
  \begin{enumerate}
  \item If $\rfocseq{\Gamma}{\Delta}{A}$, then $\Gamma; \Delta \Longrightarrow
    A$;
  \item If $\lfocseq{\Gamma}{\Delta}{A}{Q}$, then $\Gamma ; \Delta, A
    \Longrightarrow Q$;
  \item If $\btriseq{\Gamma}{\Delta}{\Omega}{C ; \cdot}$, then $\Gamma; \Delta,
    \Omega \Longrightarrow C$;
  \item If $\btriseq{\Gamma}{\Delta}{\Omega}{\cdot ; C}$, then $\Gamma; \Delta,
    \Omega \Longrightarrow C$.
  \end{enumerate}
\end{theorem}
\begin{proof}
  The four assertions are proved simultaneously by straightforward induction on
  the height of the derivation.
\end{proof}

\begin{theorem}[Completeness]
  If $\Gamma ; \Delta \Longrightarrow C$, then
  $\Gamma; \cdot ; \Omega \Longrightarrow C ; \cdot$, where $\Omega$ is some
  listing of $\Delta$.
\end{theorem}
\begin{proof}
  This proof is much more involved than the one for soundness, and can be found
  in [thesis] for the full intuitionistic linear logic. It can be easily adapted
  to our small fragment since it does not rely on the presence of all
  connectives.
\end{proof}

\subsection{Backward derived rules}

The primary benefit of focusing is the ability to generate derived ``big step''
inference rules: the intermediate results of a focusing or active phase are not
important, since those steps are ``forced'' in some way. Each derived rule
starts (at the bottom) with a neutral sequent from which a synchronous
proposition is selected for focus, and the focusing steps are followed. Then the
active rules are applied, and eventually we obtain a collection of neutral
sequents as the leaves. These neutral sequents are then treated as the premises
of the derived rule that produces the neutral sequent with which we started.

We first construct the backward derived rules. Then we will move to their
forward version. The general design is that intermediate sequents in the eager
active and focusing phases are not be stored in any sequent database; instead,
all sequents constructed during search are neutral sequents at the phase
boundaries. This is achieved by first precomputing the derived rules that
correspond to the frontier literals of the goal sequent.

For any given proposition, we are interested in constructing a derived inference
for the proposition corresponding to a single pair of focusing and inverse
phases. The idea is to interpret a proposition itself as the derived rules that
it embodies. Every propostiion is viewed as a relation between the conclusion of
the rule and its premises at the leaves of the bipole. Both the conclusion and
the premises are neutral sequents, which we indicate as
$\bneuseq{\Gamma}{\Delta}{Q}$.

There are three classes of relational interpretations:

\begin{enumerate}
\item Right focal relations for the focus formula $A$, written $\brfrel{A}$;
\item Left focal relations for the focus formula $A$, written $\blfrel{A}$;
\item Active relations, written
  $\bactrel{\btriseq{\Gamma}{\Delta}{\Omega}{\xi}}$, where $\xi$ is either
  $\cdot$ or a proposition $C$.
\end{enumerate}

The full set of relations is given in Figure~\ref{fig:bkwdrelations}. Each
relation $R$ takes as input the conclusion sequent $s$, and produces a sequence
of premise sequents $\Sigma = s_1, \dots, s_n$; we write this as
$\relj{R}{s}{\Sigma}$. We use these relations to form a calculus of derived
rules which acts on neutral sequents only, as given in
Figure~\ref{fig:bkwdderivedcalculus}.

\begin{figure}[h]
  \begin{mdframed}
    \[
      \begin{prooftree}
        p \; \text{right-biased}
        \justifies
        \relj{\blfrel{p}}{\bneuseq{\Gamma}{\cdot}{p}}{\cdot}
        \using{\linit}
      \end{prooftree}\qquad\qquad
      \begin{prooftree}
        p \; \text{left-biased}
        \justifies
        \relj{\brfrel{p}}{\bneuseq{\Gamma}{p}{\cdot}}{\cdot}
        \using{\rinit}
      \end{prooftree}
    \]

    \[
      \begin{prooftree}
        \relj{\brfrel{A}}{\bneuseq{\Gamma}{\Delta_1}{\cdot}}{\Sigma_1}
        \qquad
        \relj{\brfrel{B}}{\bneuseq{\Gamma}{\Delta_2}{\cdot}}{\Sigma_2}
        \justifies
        \relj{\brfrel{A \otimes B}}{\bneuseq{\Gamma}{\Delta_1,
            \Delta_2}{\cdot}}{\Sigma_1 \cdot \Sigma_2}
        \using{\otimes F}
      \end{prooftree}
    \]

    \[
      \begin{prooftree}
        \relj{\bactrel{\btriseq{\cdot}{\cdot}{\cdot}{R}}}{s}{\Sigma}
        \justifies
        \relj{\brfrel{R}}{s}{\Sigma}
        \using{\faplus}
      \end{prooftree}
      \qquad
      \begin{prooftree}
        p \; \text{right-biased}
        \justifies
        \relj{\brfrel{p}}{\bneuseq{\Gamma}{\Delta}{\cdot}}{\bneuseq{\Gamma}{\Delta}{p}}
        \using{conj^+}
      \end{prooftree}
    \]

    \[
      \begin{prooftree}
        \relj{\blfrel{B}}{\bneuseq{\Gamma}{\Delta_1}{Q}}{\Sigma_1} \qquad
        \relj{\brfrel{A}}{\bneuseq{\Gamma}{\Delta_2}{\cdot}}{\Sigma_2}
        \justifies
        \relj{\blfrel{A \limp B}}{\bneuseq{\Gamma}{\Delta_1,\Delta_2}{Q}}{\Sigma_1
          \cdot \Sigma_2}
        \using{\limp F}
      \end{prooftree}
    \]

    \[
      \begin{prooftree}
        \relj{\bactrel{\btriseq{\cdot}{\cdot}{L}{\cdot}}}{s}{\Sigma}
        \justifies
        \relj{\blfrel{L}}{s}{\Sigma}
        \using{FA^-}
      \end{prooftree}
      \qquad
      \begin{prooftree}
        p \; \text{left-biased}
        \justifies
        \relj{\blfrel{p}}{\bneuseq{\Gamma}{\Delta}{Q}}{\bneuseq{\Gamma}{\Delta, p}{Q}}
        \using{conj^-}
      \end{prooftree}
    \]

    \[
      \begin{prooftree}
        \relj{\bactrel{\btriseq{\Gamma}{\Delta}{\Omega \cdot A \cdot B \cdot
              \Omega'}{\xi}}}{s}{\Sigma}
        \justifies
        \relj{\bactrel{\btriseq{\Gamma}{\Delta}{\Omega \cdot A \otimes B \cdot
              \Omega'}{\xi}}}{s}{\Sigma}
        \using{\otimes A}
      \end{prooftree}
    \]
    \[
      \begin{prooftree}
        \relj{\bactrel{\btriseq{\Gamma}{\Delta}{\Omega \cdot A \cdot
              \Omega'}{B}}}{s}{\Sigma}
        \justifies
        \relj{\bactrel{\btriseq{\Gamma}{\Delta}{\Omega \cdot
              \Omega'}{A \limp B}}}{s}{\Sigma}
        \using{\limp A}
      \end{prooftree}
    \]

    \[
      \begin{prooftree}
        \relj{
          \bactrel{
            \btriseq{\Gamma}{\Delta,P}{\Omega \cdot \Omega'}{\xi}
          }
        }{s}{\Sigma}
        \justifies
        \relj{\bactrel{\btriseq{\Gamma}{\Delta}{\Omega \cdot P \cdot
              \Omega'}{\xi}}}{s}{\Sigma}
        \using{bact}
      \end{prooftree}
    \]

    \[
      \begin{prooftree}
        \justifies
        \relj{
          \bactrel{\btriseq{\Gamma}{\Delta}{\cdot}{\cdot}}
        }{
          \bneuseq{\Gamma'}{\Delta'}{Q}
        }{
          \bneuseq{\Gamma, \Gamma'}{\Delta, \Delta'}{Q}
        }
        \using{match}
      \end{prooftree}
    \]

    \[
      \begin{prooftree}
        \justifies
        \relj{
          \bactrel{\btriseq{\Gamma}{\Delta}{\cdot}{Q}}
        }{
          \bneuseq{\Gamma'}{\Delta'}{\cdot}
        }{
          \bneuseq{\Gamma, \Gamma'}{\Delta, \Delta'}{Q}
        }
        \using{match'}
      \end{prooftree}
    \]
  \end{mdframed}
  \caption{Backward relations for derived rules}
  \label{fig:bkwdrelations}
\end{figure}

\begin{figure}[h]
  \begin{mdframed}
    \[
      \begin{prooftree}
        (\relj{\brfrel{Q}}{\bneuseq{\Gamma}{\Delta}{\cdot}}{s_1 \cdot s_2 \dots s_n})
        \quad s_1 \quad s_2 \quad \dots \quad s_n
        \justifies
        \bneuseq{\Gamma}{\Delta}{Q}
        \using{right-focus}
      \end{prooftree}
    \]

    \[
      \begin{prooftree}
        (\relj{\blfrel{Q}}{\bneuseq{\Gamma}{\Delta}{Q}}{s_1 \cdot s_2 \dots s_n})
        \quad s_1 \quad s_2 \quad \dots \quad s_n
        \justifies
        \bneuseq{\Gamma}{\Delta, P}{Q}
        \using{left-focus}
      \end{prooftree}
    \]

    \[
      \begin{prooftree}
        (\relj{\blfrel{A}}{\bneuseq{\Gamma, A}{\Delta}{Q}}{s_1 \cdot s_2 \dots s_n})
        \quad s_1 \quad s_2 \quad \dots \quad s_n
        \justifies
        \bneuseq{\Gamma, A}{\Delta}{Q}
        \using{copy-focus}
      \end{prooftree}
    \]
  \end{mdframed}
  \caption{Backward calculus of derived rules}
  \label{fig:bkwdderivedcalculus}
\end{figure}

The following are some facts on the shape of the derivations for which we omit
the easy proof.

\begin{fact}\label{bkwdderfact}
  \begin{enumerate}
  \item If $\bactrelj{\btriseq{\Gamma}{\Delta}{\Omega}{\cdot}}{s}{\Sigma}$, then
    $s \equiv \bneuseq{\Gamma'}{\Delta'}{Q}$ for some $\Gamma', \Delta', Q$;
  \item If $\bactrelj{\btriseq{\Gamma}{\Delta}{\Omega}{C}}{s}{\Sigma}$, then
    $s \equiv \bneuseq{\Gamma'}{\Delta'}{\cdot}$ for some $\Gamma', \Delta'$.
  \end{enumerate}
\end{fact}

\begin{definition}
  A sequent of the backward derived rule calculus $\bneuseq{\Gamma}{\Delta}{Q}$
  is sound if $\btriseq{\Gamma}{\Delta}{\cdot}{\cdot ; Q}$.
\end{definition}

\begin{lemma}\label{bkwdder-soundness-lemma}
  \begin{enumerate}
  \item If $\brfrelj{A}{\bneuseq{\Gamma}{\Delta}{\cdot}}{\Sigma}$ and $\Sigma$
    are sound, then $\rfocseq{\Gamma}{\Delta}{A}$.
  \item If $\blfrelj{A}{\bneuseq{\Gamma}{\Delta}{Q}}{\Sigma}$ and $\Sigma$
    are sound, then $\lfocseq{\Gamma}{\Delta}{A}{Q}$.
  \item If
    $\bactrelj{\btriseq{\Gamma}{\Delta}{\Omega}{\cdot}}{\bneuseq{\Gamma'}{\Delta'}{Q}}{\Sigma}$
    and $\Sigma$ are sound, then $\btriseq{\Gamma, \Gamma'}{\Delta,
      \Delta'}{\Omega}{\cdot ; Q}$.
  \item If
    $\bactrelj{\btriseq{\Gamma}{\Delta}{\Omega}{\xi}}{\bneuseq{\Gamma'}{\Delta'}{\cdot}}{\Sigma}$
    and $\Sigma$ are sound, then
    $\btriseq{\Gamma, \Gamma'}{\Delta, \Delta'}{\Omega}{R ; \cdot}$ or
    $\btriseq{\Gamma, \Gamma'}{\Delta, \Delta'}{\Omega}{\cdot ; Q}$, depending
    on whether $\xi = R$ or $\xi = Q$.
  \end{enumerate}
\end{lemma}
\begin{proof}
  All points proved simultaneously by induction on the height of the
  derivations. The cases are briefely sketched below.

  \begin{enumerate}
  \item Case the last rule was $\linit$. The thesis follows by immediate
    application of the corresponding $\linit$ rule of the backward focused
    calculus.

  \item The $\rinit$ case is dual.

  \item Case the last rule was $\otimes F$. The thesis follows immediately by
    inductive hypothesis and application of the $\otimes R$ rule.

  \item Case the last rule was $\faplus$.

    \[
      \begin{prooftree}
        \relj{\bactrel{\btriseq{\cdot}{\cdot}{\cdot}{R}}}{s}{\Sigma}
        \justifies
        \relj{\brfrel{R}}{s}{\Sigma}
        \using{\faplus}
      \end{prooftree}
    \]

    By Fact~\ref{bkwdderfact}, $s \equiv \bneuseq{\Gamma}{\Delta}{\cdot}$ for
    some $\Gamma, \Delta$. By inductive hypothesis,
    $\btriseq{\Gamma}{\Delta}{\cdot}{R ; \cdot}$. The thesis follows immediately
    from an application of the $rblur$ rule.

  \item Case the last rule is conj+. The thesis follows immediately by
    hypothesis of soundness of the resulting sequents, and an application of
    $\rblurstar$.

  \item Case the last rule is $\limp F$. The thesis follows immediately by
    inductive hypothesis and an application of $\limp L$.

  \item Case the last rule is $\faminus$.

    \[
      \begin{prooftree}
        \relj{\bactrel{\btriseq{\cdot}{\cdot}{L}{\cdot}}}{s}{\Sigma}
        \justifies
        \relj{\blfrel{L}}{s}{\Sigma}
        \using{\faminus}
      \end{prooftree}
    \]

    By Fact~\ref{bkwdderfact}, $s \equiv \bneuseq{\Gamma}{\Delta}{Q}$. The
    thesis follows immediately by inductive hypothesis and an application of $lblur$.

  \item Case the last rule is $conj-$. The thesis follows immediately from
    the hypothesis of soundness of the result sequents, and an application of
    $\lblurstar$.

  \item The cases $\otimes A, \limp A, \bact$ are a straightforward application
    of the inductive hypothesis.

  \item In the base cases $\matchrule, \matchprimerule$, the thesis follows
    immediately from the hypothesis of $\Sigma$ being sound.

  \end{enumerate}
\end{proof}

\begin{theorem}[Soundness]
  If $\bneuseq{\Gamma}{\Delta}{Q}$ then $\bkwseq{\Gamma}{\Delta}{\cdot ; Q}$.
\end{theorem}
\begin{proof}
  Straightforward induction on the derivation, and
  Lemma~\ref{bkwdder-soundness-lemma}.
\end{proof}

\begin{lemma}\label{completeness-lemma}
  \begin{enumerate}

  \item If $\rfocseq{\Gamma}{\Delta}{A}$, then for some $\Sigma$
    \begin{enumerate}
    \item $\relj{\brfrel{A}}{\bneuseq{\Gamma}{\Delta}{\cdot}}{\Sigma}$, and
    \item $\Sigma$ are all derivable
    \end{enumerate}

  \item If $\lfocseq{\Gamma}{\Delta}{A}{Q}$, then for some $\Sigma$
    \begin{enumerate}
    \item $\blfrelj{A}{\bneuseq{\Gamma}{\Delta}{Q}}{\Sigma}$, and
    \item $\Sigma$ are all derivable
    \end{enumerate}

  \item If $\btriseq{\Gamma_1, \Gamma_2}{\Delta_1, \Delta_2}{\Omega}{\xi \uplus
      \gamma}$ (where $\xi \uplus \gamma$ means either $\xi$ or $\gamma$ is
    empty), then for some $\Sigma$
    \begin{enumerate}
    \item
      $\bactrelj{\btriseq{\Gamma_1}{\Delta_1}{\Omega}{\xi}}{\bneuseq{\Gamma_2}{\Delta_2}{\gamma}}{\Sigma}$,
      and
    \item $\Sigma$ are all derivable.
    \end{enumerate}
  \end{enumerate}
\end{lemma}
\begin{proof}
  By simultaneous induction on the height of the derivation.
  \begin{enumerate}
  \item Case $\rinit$:

    \[
      \begin{prooftree}
        p \; \text{left-biased}
        \justifies
        \rfocseq{\Gamma}{p}{p}
        \using{\rinit}
      \end{prooftree}
    \]

    then, just apply the $\rinit$ relation.
  \item Case $\linit$ is similar to the case above.
  \item Case $\ract$ is trivial, since the relations corresponding to the
    premise and conclusion sequents are identical.
  \item Case $\lact$. Straightforward use of the $\bact$ rule for derived
    relations.
  \item Case $\rfoc$.

    \[
      \begin{prooftree}
        \Gamma; \Delta \gg Q \qquad Q\; \text{right-focusable}
        \justifies
        \Gamma; \Delta; \cdot \Longrightarrow \cdot ; Q
        \using{rfoc}
      \end{prooftree}
    \]

    Suppose $\gamma_1, \gamma_2$ are such that $\gamma_1 \uplus \gamma_2 =
    Q$. Then, the following:

    \[
      \relj{
        \bactrel{\btriseq{\Gamma}{\Delta}{\cdot}{\gamma_1}}
      }{
        \bneuseq{\cdot}{\cdot}{\gamma_2}
      }{
        \bneuseq{\Gamma}{\Delta}{Q}
      }
    \]

    is derivable with either $\matchrule$ or $\matchprimerule$. We need to show
    that $\bneuseq{\Gamma}{\Delta}{Q}$ is derivable. By inductive hypothesis, we
    have $\relj{\brfrel{Q}}{\bneuseq{\Gamma}{\Delta}{\cdot}}{\Sigma}$ for some
    $\Sigma$ all derivable. But then,

    \[
      \begin{prooftree}
        \relj{\brfrel{Q}}{\bneuseq{\Gamma}{\Delta}{\cdot}}{\Sigma}
        \qquad \Sigma
        \justifies
        \bneuseq{\Gamma}{\Delta}{Q}
        \using{\rightfocusrule}
      \end{prooftree}
    \]

  \item Case $\lfoc$ is just analogous to $\rfoc$, with an application of
    $\leftfocusrule$.
  \item Case $\rblur$.

    \[
      \begin{prooftree}
        \Gamma; \Delta; \cdot \Longrightarrow R; \cdot
        \justifies
        \Gamma; \Delta \gg R
        \using{\rblur}
      \end{prooftree}
    \]

    By inductive hypothesis, we have
    $\relj{\bactrel{\btriseq{\cdot}{\cdot}{\cdot}{R}}}{\bneuseq{\Gamma}{\Delta}{\cdot}}{\Sigma}$,
    where all $\Sigma$ are derivable. But
    then, we can apply the rule $\faplus$ to get the thesis

    \[
      \begin{prooftree}
        \relj{\bactrel{\btriseq{\cdot}{\cdot}{\cdot}{R}}}{\bneuseq{\Gamma}{\Delta}{\cdot}}{\Sigma}
        \justifies
        \relj{\brfrel{R}}{\bneuseq{\Gamma}{\Delta}{\cdot}}{\Sigma}
        \using{FA^+}
      \end{prooftree}
    \]

  \item Case $\lblur$ is dual to $\rblur$.
  \item Case $\rblurstar$.

    \[
      \begin{prooftree}
        \Gamma; \Delta; \cdot \Longrightarrow \cdot ; p \qquad p \; \text{right-biased}
        \justifies
        \Gamma; \Delta \gg p
        \using{rblur^*}
      \end{prooftree}
    \]

    By inductive hypothesis,
    $\relj{\bactrel{\btriseq{\Gamma}{\Delta}{\cdot}{p}}}{\bneuseq{\cdot}{\cdot}{\cdot}}{\Sigma}$,
    that must have been derived by $\matchprimerule$. Therefore, $\Sigma \equiv
    \bneuseq{\Gamma}{\Delta}{p}$, which is by hypothesis derivable.
    Then, we can apply $\faplusstar$ to get
    $\relj{\brfrel{p}}{\bneuseq{\Gamma}{\Delta}{\cdot}}{\bneuseq{\Gamma}{\Delta}{p}}$.

  \item Case $\lblurstar$ is dual to $\rblurstar$.
  \item Case $\copyrule$.

    \[
      \begin{prooftree}
        \Gamma, A; \Delta ; A \ll Q
        \justifies
        \Gamma, A; \Delta; \cdot \Longrightarrow \cdot; Q
        \using{\copyrule}
      \end{prooftree}
    \]

    We can apply $\matchrule$ to immediately derive

    \[
      \begin{prooftree}
        \justifies
        \relj{\bactrel{\btriseq{\Gamma,
              A}{\Delta}{\cdot}{\cdot}}}{\bneuseq{\cdot}{\cdot}{Q}}{\bneuseq{\Gamma,
            A}{\Delta}{Q}}
        \using{\matchrule}
      \end{prooftree}
    \]

    and see that $\bneuseq{\Gamma, A}{\Delta}{Q}$ is derivable by
    the inductive hypothesis
    $\relj{\blfrel{A}}{\bneuseq{\Gamma,A}{\Delta}{Q}}{\Sigma}$ and an application
    of $\copyfocusrule$.
  \item The remaining rules of the connectives are straightforward application
    of the inductive hypothesis.
  \end{enumerate}
\end{proof}

\begin{theorem}[Completeness]
  If $\btriseq{\Gamma}{\Delta}{\cdot}{\cdot ; Q}$, then
  $\bneuseq{\Gamma}{\Delta}{Q}$.
\end{theorem}
\begin{proof}
  Straightforward application of Lemma~\ref{completeness-lemma}. The last rule
  used to derive $\btriseq{\Gamma}{\Delta}{\cdot}{\cdot ; Q}$ is one of
  $\lfoc, \rfoc$ or $\copyrule$; correspondingly, by the Lemma, we have the
  derived rules to use as premises of $\rightfocusrule, \leftfocusrule$ and
  $\copyfocusrule$ which derive $\bneuseq{\Gamma}{\Delta}{Q}$.
\end{proof}

\subsection{Forward derived rules}

In the following rules, $\gamma \setminus \xi$ is defined as $\gamma$ if $\xi = \cdot$,
and as $\cdot$ if $\gamma = \xi$.

\begin{figure}[h]
  \begin{mdframed}
    \[
      \begin{prooftree}
        p \; \text{left-biased}
        \justifies
        \relj{\frfrel{p}}{\cdot}{\fneuseq{\cdot}{p}{\cdot}}
        \using{\linit}
      \end{prooftree}
      \qquad\qquad
      \begin{prooftree}
        p \; \text{right-based}
        \justifies
        \relj{\flfrel{p}}{\cdot}{\fneuseq{\cdot}{\cdot}{p}}
        \using{\linit}
      \end{prooftree}
    \]

    \[
      \begin{prooftree}
        \relj{\factrel{\btriseq{\cdot}{\cdot}{\cdot}{R}}}{\Sigma}{s}
        \justifies
        \relj{\frfrel{R}}{\Sigma}{s}
        \using{\faplus}
      \end{prooftree}
      \qquad \qquad
      \begin{prooftree}
        \relj{\factrel{\btriseq{\cdot}{\cdot}{L}{\cdot}}}{\Sigma}{s}
        \justifies
        \relj{\flfrel{L}}{\Sigma}{s}
        \using{\faminus}
      \end{prooftree}
    \]

    \[
      \begin{prooftree}
        p \text{ right-biased}
        \justifies
        \frfrelj{p}{\fneuseq{\Gamma}{\Delta}{p}}{\fneuseq{\Gamma}{\Delta}{\cdot}}
        \using{conj^+}
      \end{prooftree}
    \]

    \[
      \begin{prooftree}
        p \text{ left-biased}
        \justifies
        \flfrelj{p}{\fneuseq{\Gamma}{\Delta, p}{Q}}{\fneuseq{\Gamma}{\Delta}{Q}}
        \using{conj^-}
      \end{prooftree}
    \]

    \[
      \begin{prooftree}
        \relj{\flfrel{B}}{\Sigma_1}{\fneuseq{\Gamma_1}{\Delta_1}{\gamma}}
        \qquad
        \relj{\frfrel{A}}{\Sigma_2}{\fneuseq{\Gamma_2}{\Delta_2}{\cdot}}
        \justifies
        \relj{\flfrel{A \limp B}}{\Sigma_1 \cdot \Sigma_2}{\fneuseq{\Gamma_1,
            \Gamma_2}{\Delta_1, \Delta_2}{\gamma}}
        \using{\limp F}
      \end{prooftree}
    \]

    \[
      \begin{prooftree}
        \relj{\frfrel{A}}{\Sigma_1}{\fneuseq{\Gamma_1}{\Delta_1}{\cdot}}
        \qquad
        \relj{\frfrel{B}}{\Sigma_2}{\fneuseq{\Gamma_2}{\Delta_2}{\cdot}}
        \justifies
        \relj{\frfrel{A \otimes B}}{\Sigma_1 \cdot \Sigma_2}{\fneuseq{\Gamma_1,
            \Gamma_2}{\Delta_1, \Delta_2}{\cdot}}
        \using{\otimes F}
      \end{prooftree}
    \]

    \[
      \begin{prooftree}
        \relj{\factrel{\btriseq{\Gamma}{\Delta}{\Omega \cdot A}{B}}}{\Sigma}{s}
        \justifies
        \relj{\factrel{\btriseq{\Gamma}{\Delta}{\Omega}{A \limp B}}}{\Sigma}{s}
        \using{\limp A}
      \end{prooftree}
    \]

    \[
      \begin{prooftree}
        \relj{\factrel{
            \btriseq{\Gamma}{\Delta}{\Omega \cdot A \cdot B \cdot \Omega'}{\xi}}
        }{\Sigma}{s}
        \justifies
        \relj{\factrel{
            \btriseq{\Gamma}{\Delta}{\Omega \cdot A \otimes B \cdot \Omega'}{\xi}}
        }{\Sigma}{s}
        \using{\otimes A}
      \end{prooftree}
    \]

    \[
      \begin{prooftree}
        \relj{\factrel{\btriseq{\Gamma}{\Delta, P}{\Omega \cdot \Omega'}{\xi}}}{\Sigma}{s}
        \justifies
        \relj{\factrel{\btriseq{\Gamma}{\Delta}{\Omega \cdot P \cdot \Omega'}{\xi}}}{\Sigma}{s}
        \using{\actrule}
      \end{prooftree}
    \]

    \[
      \begin{prooftree}
        \xi \subseteq \gamma
        \justifies
        \relj{
          \factrel{\btriseq{\Gamma}{\Delta}{\cdot}{\xi}}
        }{
          \fneuseq{\Gamma, \Gamma'}{\Delta, \Delta'}{\gamma}
        }{
          \fneuseq{\Gamma'}{\Delta'}{\gamma \setminus \xi}
        }
        \using{\matchrule}
      \end{prooftree}
    \]
  \end{mdframed}
  \caption{Forward focused derived rule calculus}
\end{figure}

The derived rule for positive subformulas is:

\[
  \begin{prooftree}
    s_1 \quad \dots \quad s_n \quad
    (\relj{\frfrel{Q}}{s_1 \dots s_n}{\fneuseq{\Gamma}{\Delta}{\cdot}})
    \justifies
    \fneuseq{\Gamma}{\Delta}{Q}
    \using{\focplusrule}
  \end{prooftree}
\]

Similarly, for negative propositions, we have two rules:

\[
  \begin{prooftree}
    s_1 \quad \dots \quad s_n \quad
    (\relj{\flfrel{P}}{s_1 \dots s_n}{\fneuseq{\Gamma}{\Delta}{Q}})
    \justifies
    \fneuseq{\Gamma}{\Delta, P}{Q}
    \using{\focminusrule}
  \end{prooftree}
\]

\[
  \begin{prooftree}
    s_1 \quad \dots \quad s_n \quad
    (\relj{\flfrel{A}}{s_1 \dots s_n}{\fneuseq{\Gamma}{\Delta}{Q}})
    \justifies
    \fneuseq{\Gamma, A}{\Delta}{Q}
    \using{\foccopyrule}
  \end{prooftree}
\]

\begin{fact}\label{forwardfact}
  \begin{enumerate}
  \item If $\relj{\factrel{\btriseq{\Gamma}{\Delta}{\Omega}{\xi}}}{\Sigma}
    {\fneuseq{\Gamma'}{\Delta'}{\gamma}}$, then either
    \begin{enumerate}
    \item $\xi = \cdot$ and $\gamma = Q$ for some right-synchronous $Q$;
    \item $\xi = C$ and $\gamma = \cdot$ for some $C$.
    \end{enumerate}

  \item If $\relj{\flfrel{A}}{\Sigma}{\fneuseq{\Gamma}{\Delta}{\gamma}}$, then
    $\gamma = Q$ for some right-synchronous $Q$.
  \end{enumerate}
\end{fact}

\begin{definition}
  The sequent $\fneuseq{\Gamma}{\Delta}{Q}$ is sound if
  $\btriseq{\Gamma}{\Delta}{\cdot}{\cdot ; Q}$.
\end{definition}

\begin{lemma}\label{fsoundnesslemma}
  If $\Sigma$ are sound, then

  \begin{enumerate}
  \item If $\relj{\frfrel{A}}{\Sigma}{\fneuseq{\Gamma}{\Delta}{\cdot}}$,
    then, $\rfocseq{\Gamma}{\Delta}{A}$;
  \item If $\relj{\flfrel{A}}{\Sigma}{\fneuseq{\Gamma}{\Delta}{Q}}$,
    then $\lfocseq{\Gamma}{\Delta}{A}{Q}$;
  \item If
    $\relj{\factrel{\btriseq{\Gamma}{\Delta}{\Omega}{\cdot}}}{\Sigma}
    {\fneuseq{\Gamma'}{\Delta'}{Q}}$, then
    $\btriseq{\Gamma, \Gamma'}{\Delta, \Delta'}{\Omega}{\cdot ; Q}$;
  \item If
    $\relj{\factrel{\btriseq{\Gamma}{\Delta}{\Omega}{C}}}{\Sigma}
    {\fneuseq{\Gamma'}{\Delta'}{\cdot}}$, then
    $\btriseq{\Gamma, \Gamma'}{\Delta, \Delta'}{\Omega}{C ; \cdot}$ or
    $\btriseq{\Gamma, \Gamma'}{\Delta, \Delta'}{\Omega}{\cdot ; C}$, depending
    on whether $C$ is right-asynchronous or right-synchronous, respectively.
  \end{enumerate}
\end{lemma}
\begin{proof}
  All points proved by simultaneous induction on the height of the
  derivation. We distinguish cases on the last rule used in the derivation:

  \begin{itemize}
  \item Case $\linit$. Hence,
    $\relj{\frfrel{p}}{\cdot}{\fneuseq{\cdot}{p}{\cdot}}$ and $p$ is
    left-biased. Then,

    \[
      \begin{prooftree}
        p \; \text{left-biased}
        \justifies
        \rfocseq{\cdot}{p}{p}
      \end{prooftree}
    \]

    By soundness of the backward focused calculus, $\bkwseq{\cdot}{p}{p}$.

  \item Case $\rinit$ is similar;
  \item Case $\faplus{}$.

    \[
      \begin{prooftree}
        \relj{
          \factrel{\btriseq{\cdot}{\cdot}{\cdot}{R}}
        }{
          \Sigma
        }{
          s
        }
        \justifies
        \relj{\frfrel{R}}{\Sigma}{s}
        \using{\faplus}
      \end{prooftree}
    \]

    By Fact~\ref{forwardfact}, $s \equiv
    \fneuseq{\Gamma'}{\Delta'}{\cdot}$ for some $\Gamma', \Delta'$, and by
    inductive hypothesis $\btriseq{\Gamma'}{\Delta'}{\cdot}{R ; \cdot}$. Then,

    \[
      \begin{prooftree}
        \btriseq{\Gamma'}{\Delta'}{\cdot}{R ; \cdot}
        \justifies
        \rfocseq{\Gamma'}{\Delta'}{R}
        \using{\rblur}
      \end{prooftree}
    \]

  \item Case $\faminus$.

    \[
      \begin{prooftree}
        \relj{\factrel{\btriseq{\cdot}{\cdot}{L}{\cdot}}}{\Sigma}{s}
        \justifies
        \relj{\flfrel{L}}{\Sigma}{s}
        \using{\faminus}
      \end{prooftree}
    \]
    By Fact~\ref{forwardfact}, $s \equiv \fneuseq{\Gamma'}{\Delta'}{Q}$, then by
    inductive hypothesis $\btriseq{\Gamma'}{\Delta'}{L}{\cdot ; Q}$. Then,

    \[
      \begin{prooftree}
        \btriseq{\Gamma'}{\Delta'}{L}{\cdot ; Q}
        \justifies
        \lfocseq{\Gamma'}{\Delta'}{L}{Q}
        \using{\lblur}
      \end{prooftree}
    \]

  \item Case $\limp F$.

    \[
      \begin{prooftree}
        \relj{\flfrel{B}}{\Sigma_1}{\fneuseq{\Gamma_1}{\Delta_1}{\gamma}}
        \qquad
        \relj{\frfrel{A}}{\Sigma_2}{\fneuseq{\Gamma_2}{\Delta_2}{\cdot}}
        \justifies
        \relj{\flfrel{A \limp B}}{\Sigma_1 \cdot \Sigma_2}{\fneuseq{\Gamma_1,
            \Gamma_2}{\Delta_1, \Delta_2}{\gamma}}
        \using{\limp F}
      \end{prooftree}
    \]

    Then, by Fact~\ref{forwardfact} and inductive hypothesis

    \[
      \lfocseq{\Gamma_1}{\Delta_1}{B}{Q} \qquad \qquad
      \rfocseq{\Gamma_2}{\Delta_2}{A}
    \]

    for some right-synchronous $Q$. Then, by weakening and $\limp L$:

    \[
      \begin{prooftree}
        \[
          \lfocseq{\Gamma_1}{\Delta_1}{B}{Q}
          \justifies
          \lfocseq{\Gamma_1, \Gamma_2}{\Delta_1}{B}{Q}
        \]
        \qquad
        \[
          \rfocseq{\Gamma_2}{\Delta_2}{A}
          \justifies
          \rfocseq{\Gamma_1, \Gamma_2}{\Delta_2}{A}
        \]
        \justifies
        \lfocseq{\Gamma_1, \Gamma_2}{\Delta_1, \Delta_2}{A \limp B}{Q}
      \end{prooftree}
    \]

  \item Case $\otimes F$.

    \[
      \begin{prooftree}
        \relj{\frfrel{A}}{\Sigma_1}{\fneuseq{\Gamma_1}{\Delta_1}{\cdot}}
        \qquad
        \relj{\frfrel{B}}{\Sigma_2}{\fneuseq{\Gamma_2}{\Delta_2}{\cdot}}
        \justifies
        \relj{\frfrel{A \otimes B}}{\Sigma_1 \cdot \Sigma_2}{\fneuseq{\Gamma_1,
            \Gamma_2}{\Delta_1, \Delta_2}{\cdot}}
        \using{\otimes F}
      \end{prooftree}
    \]

    Then, by inductive hypothesis

    \[
      \rfocseq{\Gamma_1}{\Delta_1}{A} \qquad \qquad
      \rfocseq{\Gamma_2}{\Delta_2}{B}
    \]

    Then, by weakening and $\otimes R$:

    \[
      \begin{prooftree}
        \[
          \rfocseq{\Gamma_1}{\Delta_1}{A}
          \justifies
          \rfocseq{\Gamma_1, \Gamma_2}{\Delta_1}{A}
        \]
        \qquad
        \[
          \rfocseq{\Gamma_2}{\Delta_2}{B}
          \justifies
          \rfocseq{\Gamma_1, \Gamma_2}{\Delta_2}{B}
        \]
        \justifies
        \rfocseq{\Gamma_1, \Gamma_2}{\Delta_1, \Delta_2}{A \otimes B}
      \end{prooftree}
    \]

  \item Case $\limp A$.

    \[
      \begin{prooftree}
        \relj{\factrel{\btriseq{\Gamma}{\Delta}{\Omega \cdot A}{B}}}{\Sigma}{s}
        \justifies
        \relj{\factrel{\btriseq{\Gamma}{\Delta}{\Omega}{A \limp B}}}{\Sigma}{s}
        \using{\limp A}
      \end{prooftree}
    \]

    By Fact~\ref{forwardfact}, $s \equiv \fneuseq{\Gamma'}{\Delta'}{\cdot}$ for
    some $\Gamma', \Delta'$. By inductive hypothesis $\btriseq{\Gamma,
      \Gamma'}{\Delta, \Delta'}{\Omega \cdot A}{B ; \cdot}$ (if $B$ is
    right-synchronous, it sufficies to apply $\ract$). Then, by $\limp R$

    \[
      \begin{prooftree}
        \btriseq{\Gamma,
          \Gamma'}{\Delta, \Delta'}{\Omega \cdot A}{B ; \cdot}
        \justifies
        \btriseq{\Gamma,
          \Gamma'}{\Delta, \Delta'}{\Omega}{A \limp B ; \cdot}
      \end{prooftree}
    \]

  \item Case $\otimes A$ is analogous.
  \item Case $act$.

    \[
      \begin{prooftree}
        \relj{\factrel{\btriseq{\Gamma}{\Delta, P}{\Omega \cdot \Omega'}{\xi}}}{\Sigma}{s}
        \justifies
        \relj{\factrel{\btriseq{\Gamma}{\Delta}{\Omega \cdot P \cdot \Omega'}{\xi}}}{\Sigma}{s}
        \using{act}
      \end{prooftree}
    \]

    \begin{enumerate}
    \item Case $\xi = \cdot$. Then, by Fact~\ref{forwardfact},
      $s \equiv \fneuseq{\Gamma'}{\Delta'}{Q}$ for some $Q$. Then, by inductive
      hypothesis,
      $\btriseq{\Gamma, \Gamma'}{\Delta, \Delta', P}{\Omega \cdot \Omega'}{\cdot
        ; Q}$. The thesis follows by application of the $lact$ rule.

    \item Case $\xi = C$ for some $C$. Then, by Fact~\ref{forwardfact},
      $s \equiv \fneuseq{\Gamma'}{\Delta'}{\cdot}$. Then, by inductive
      hypothesis,
      $\btriseq{\Gamma, \Gamma'}{\Delta, \Delta', P}{\Omega \cdot
        \Omega'}{\gamma}$, where $\gamma$ is either $C ; \cdot$ and $\cdot ;
      C$. The thesis follows again by application of the $lact$ rule.
    \end{enumerate}

  \item Case $match$.
    \begin{enumerate}
    \item Case $\xi = \cdot$. Then

      \[
        \begin{prooftree}
          \justifies
          \relj{
            \factrel{\btriseq{\Gamma}{\Delta}{\cdot}{\cdot}}
          }{
            \fneuseq{\Gamma, \Gamma'}{\Delta, \Delta'}{Q}
          }{
            \fneuseq{\Gamma'}{\Delta'}{Q}
          }
          \using{match}
        \end{prooftree}
      \]

      Then we need to prove the derivability of $\btriseq{\Gamma,
        \Gamma'}{\Delta, \Delta'}{\cdot}{\cdot ; Q}$, but this comes from the
      hypothesis that $\fneuseq{\Gamma, \Gamma'}{\Delta, \Delta'}{Q}$ be sound.

    \item Case $\xi = C$. Then, $C \equiv Q$ and

      \[
        \begin{prooftree}
          \justifies
          \relj{
            \factrel{\btriseq{\Gamma}{\Delta}{\cdot}{Q}}
          }{
            \fneuseq{\Gamma, \Gamma'}{\Delta, \Delta'}{Q}
          }{
            \fneuseq{\Gamma'}{\Delta'}{\cdot}
          }
          \using{match}
        \end{prooftree}
      \]

      The rest is as above.
    \end{enumerate}
  \end{itemize}
\end{proof}

\begin{theorem}[Soundness]
  If $\fneuseq{\Gamma}{\Delta}{Q}$, then it is sound.
\end{theorem}
\begin{proof}
  By induction on the height of the derivation of
  $\fneuseq{\Gamma}{\Delta}{Q}$. We distinguish three cases

  \begin{enumerate}
  \item Case $focplus$

    \[
      \begin{prooftree}
        s_1 \quad \dots \quad s_n \quad
        (\relj{\frfrel{Q}}{s_1 \dots s_n}{\fneuseq{\Gamma}{\Delta}{\cdot}})
        \justifies
        \fneuseq{\Gamma}{\Delta}{Q}
        \using{foc plus}
      \end{prooftree}
    \]

    By Lemma~\ref{fsoundnesslemma}, $\rfocseq{\Gamma}{\Delta}{Q}$. The thesis
    follows by $\rfoc$.

  \item Case $focminus$.

    \[
      \begin{prooftree}
        s_1 \quad \dots \quad s_n \quad
        (\relj{\flfrel{P}}{s_1 \dots s_n}{\fneuseq{\Gamma}{\Delta}{Q}})
        \justifies
        \fneuseq{\Gamma}{\Delta, P}{Q}
        \using{foc minus}
      \end{prooftree}
    \]

    By Lemma~\ref{fsoundnesslemma}, $\lfocseq{\Gamma}{\Delta}{P}{Q}$. The thesis
    follows by $\lfoc$.

  \item Case $!focminus$.

    \[
      \begin{prooftree}
        s_1 \quad \dots \quad s_n \quad
        (\relj{\flfrel{A}}{s_1 \dots s_n}{\fneuseq{\Gamma}{\Delta}{Q}})
        \justifies
        \fneuseq{\Gamma, A}{\Delta}{Q}
        \using{! foc minus}
      \end{prooftree}
    \]

    By Lemma~\ref{fsoundnesslemma}, $\lfocseq{\Gamma}{\Delta}{A}{Q}$. The thesis
    follows by $copy$.
  \end{enumerate}
\end{proof}

Completeness is established with respect to the backward calculus of derived
rules.

\begin{definition}[Stronger forms]
  A forward sequent $\fneuseq{\Gamma}{\Delta}{G}$ is said to be stronger than a
  backward derived sequent $\bneuseq{\Gamma'}{\Delta'}{G'}$, written as a
  relation $\preceq$, if $\Gamma \subseteq \Gamma'$, $\Delta = \Delta'$, and
  $G = G'$.
\end{definition}

The above relation is assumed to be extended point-wise to ordered sequences of
sequents.

\begin{lemma}\label{fdercompllemma}
  \begin{enumerate}
  \item If $\brfrelj{A}{s}{\Sigma}$ and there exists a derivable sequence of
    sequents $\Sigma' \preceq \Sigma$ for which $\frfrelj{A}{\Sigma'}{s'}$, then
    $s' \preceq s$;
  \item If $\blfrelj{A}{s}{\Sigma}$ and there exists a derivable sequence
    $\Sigma' \preceq \Sigma$ for which $\flfrelj{A}{\Sigma'}{s'}$, then $s'
    \preceq s$;
  \item If $\bactrelj{\btriseq{\Gamma}{\Delta}{\Omega}{\xi}}{s}{\Sigma}$ and
    there exists a derivable sequence $\Sigma' \preceq \Sigma$ for which
    $\factrelj{\btriseq{\Gamma}{\Delta}{\Omega}{\xi}}{\Sigma'}{s'}$, then
    $s' \preceq s$.
  \end{enumerate}
\end{lemma}
\begin{proof}
  All points proved simultaneously by induction on the height of the backward
  relation derivation.

  \begin{enumerate}
  \item Case $\linit$.

    \[
      \begin{prooftree}
        p \; \text{right-biased}
        \justifies
        \relj{\blfrel{p}}{\bneuseq{\Gamma}{\cdot}{p}}{\cdot}
        \using{\linit}
      \end{prooftree}
    \]

    Suppose there exists a derivable sequence $\Sigma' \preceq \cdot$ such that
    $\relj{\flfrel{p}}{\Sigma'}{s'}$ is derivable. Then, it must have been an
    application of the corresponding forward $\linit$ rule:

    \[
      \begin{prooftree}
        p \; \text{right-based}
        \justifies
        \relj{\flfrel{p}}{\cdot}{\fneuseq{\cdot}{\cdot}{p}}
        \using{\linit}
      \end{prooftree}
    \]

    where indeed $\Sigma' \equiv \cdot$ and $s' \equiv
    \fneuseq{\cdot}{\cdot}{p}$. The thesis follows immediately.

  \item Case $\rinit$. That is, we have a derivation

    \[
      \begin{prooftree}
        p \; \text{left-biased}
        \justifies
        \relj{\brfrel{p}}{\bneuseq{\Gamma}{p}{\cdot}}{\cdot}
        \using{\rinit}
      \end{prooftree}
    \]

    and suppose there are $\Sigma' \preceq \cdot$ and a derivation of
    $\relj{\frfrel{p}}{\Sigma'}{s'}$. Then, it must be that $\Sigma' = \cdot$ and

    \[
      \begin{prooftree}
        p \; \text{left-biased}
        \justifies
        \relj{\frfrel{p}}{\cdot}{\fneuseq{\cdot}{p}{\cdot}}
        \using{\rinit}
      \end{prooftree}
    \]

    Then $s' \equiv \fneuseq{\cdot}{p}{\cdot}$, which clearly satisfies the
    thesis.

  \item Case $\faplus$. Then

    \[
      \begin{prooftree}
        \relj{\bactrel{\btriseq{\cdot}{\cdot}{\cdot}{R}}}{s}{\Sigma}
        \justifies
        \relj{\brfrel{R}}{s}{\Sigma}
        \using{FA^+}
      \end{prooftree}
    \]

    Suppose there are $\Sigma' \preceq \Sigma$ such that
    $\frfrelj{R}{\Sigma'}{s'}$. Then, it must have been derived with an
    application of the rule $\faplus$.

    \[
      \begin{prooftree}
        \relj{\factrel{\btriseq{\cdot}{\cdot}{\cdot}{R}}}{\Sigma'}{s'}
        \justifies
        \relj{\frfrel{R}}{\Sigma'}{s'}
        \using{\faplus}
      \end{prooftree}
    \]

    The thesis follows immediately from the inductive hypothesis.

  \item Case $conj+$. Then,

    \[
      \begin{prooftree}
        p \; \text{right-biased}
        \justifies
        \relj{\brfrel{p}}{\bneuseq{\Gamma}{\Delta}{\cdot}}{\bneuseq{\Gamma}{\Delta}{p}}
        \using{conj^+}
      \end{prooftree}
    \]

    where $\Sigma \equiv \bneuseq{\Gamma}{\Delta}{p}$ and
    $s \equiv \bneuseq{\Gamma}{\Delta}{\cdot}$. Suppose there is some $\Sigma'
    \preceq \Sigma$ and $s'$ such that $\frfrelj{p}{\Sigma'}{s'}$.
    Then, it must have been derived by an application of $conj+$, hence
    $\Sigma' \equiv \fneuseq{\Gamma'}{\Delta'}{p}$ and
    $s' \equiv \fneuseq{\Gamma'}{\Delta'}{\cdot}$.

    By hypothesis, $\Sigma' \preceq \Sigma$ hence $\Gamma' \subseteq \Gamma$ and
    $\Delta = \Delta'$. It follows that $s' \preceq s$.

      \item Case $\otimes F$:

    \[
      \begin{prooftree}
        \relj{\brfrel{A}}{\bneuseq{\Gamma}{\Delta_1}{\cdot}}{\Sigma_1}
        \qquad
        \relj{\brfrel{B}}{\bneuseq{\Gamma}{\Delta_2}{\cdot}}{\Sigma_2}
        \justifies
        \relj{\brfrel{A \otimes B}}{\bneuseq{\Gamma}{\Delta_1,
            \Delta_2}{\cdot}}{\Sigma_1 \cdot \Sigma_2}
        \using{\otimes F}
      \end{prooftree}
    \]

    By hypothesis, there is $(\Sigma_1' \cdot \Sigma_2') \preceq (\Sigma_1 \cdot
    \Sigma_2)$ such that

    \[
      \begin{prooftree}
        \relj{\frfrel{A}}{\Sigma_1'}{\fneuseq{\Gamma_1'}{\Delta_1'}{\cdot}}
        \qquad
        \relj{\frfrel{B}}{\Sigma_2'}{\fneuseq{\Gamma_2'}{\Delta_2'}{\cdot}}
        \justifies
        \relj{\frfrel{A \otimes B}}{\Sigma_1' \cdot \Sigma_2'}{\fneuseq{\Gamma_1',
            \Gamma_2'}{\Delta_1', \Delta_2'}{\cdot}}
        \using{\otimes F}
      \end{prooftree}
    \]

    For some $\Gamma_1', \Gamma_2', \Delta_1', \Delta_2'$. By inductive
    hypothesis, $\Gamma_1' \subseteq \Gamma$, $\Gamma_2' \subseteq \Gamma$,
    $\Delta_1 = \Delta_1'$, $\Delta_2 = \Delta_2'$. It follows that $\Gamma_1,
    \Gamma_2 \subseteq \Gamma$ and $\Delta_1, \Delta_2 = \Delta_1', \Delta_2'$.

  \item Case $\limp F$.

    \[
      \begin{prooftree}
        \relj{\blfrel{B}}{\bneuseq{\Gamma}{\Delta_1}{Q}}{\Sigma_1} \qquad
        \relj{\brfrel{A}}{\bneuseq{\Gamma}{\Delta_2}{\cdot}}{\Sigma_2}
        \justifies
        \relj{\blfrel{A \limp B}}{\bneuseq{\Gamma}{\Delta_1,\Delta_2}{Q}}{\Sigma_1
          \cdot \Sigma_2}
        \using{\limp F}
      \end{prooftree}
    \]

    Suppose $(\Sigma_1' \cdot \Sigma_2') \preceq (\Sigma_1 \cdot \Sigma_2)$ such
    that $\relj{\flfrel{A \limp B}}{\Sigma_1' \cdot \Sigma_2'}{s'}$. Then, it
    must have been derived by an application of the corresponding $\limp F$
    rule.

    \[
      \begin{prooftree}
        \relj{\flfrel{B}}{\Sigma_1'}{\fneuseq{\Gamma_1'}{\Delta_1'}{Q'}}
        \qquad
        \relj{\frfrel{A}}{\Sigma_2'}{\fneuseq{\Gamma_2'}{\Delta_2'}{\cdot}}
        \justifies
        \relj{\flfrel{A \limp B}}{\Sigma_1' \cdot \Sigma_2'}{\fneuseq{\Gamma_1'
            \cup \Gamma_2'}{\Delta_1', \Delta_2'}{Q'}}
        \using{\limp F}
      \end{prooftree}
    \]

    By inductive hypothesis, $\Gamma_1' \subseteq \Gamma, \Gamma_2' \subseteq
    \Gamma, \Delta_1' = \Delta_1, \Delta_2' = \Delta_2, Q = Q'$. Therefore,
    $\Gamma_1' \cup \Gamma_2 \subseteq \Gamma$ and
    $\Delta_1', \Delta_2' = \Delta_1, \Delta_2$. The thesis follows immediately.

  \item Case $\otimes A$.

    \[
      \begin{prooftree}
        \relj{\bactrel{\btriseq{\Gamma}{\Delta}{\Omega \cdot A \cdot B \cdot
              \Omega'}{\xi}}}{s}{\Sigma}
        \justifies
        \relj{\bactrel{\btriseq{\Gamma}{\Delta}{\Omega \cdot A \otimes B \cdot
              \Omega'}{\xi}}}{s}{\Sigma}
        \using{\otimes A}
      \end{prooftree}
    \]

    Suppose $\Sigma' \preceq \Sigma$ such that
    $\relj{\factrel{ \btriseq{\Gamma}{\Delta}{\Omega \cdot A \otimes B \cdot
          \Omega'}{\xi}} }{\Sigma'}{s'}$. Then, it must have been derived by an
    application of the corresponding $\otimes A$ rule:

    \[
      \begin{prooftree}
        \relj{\factrel{
            \btriseq{\Gamma}{\Delta}{\Omega \cdot A \cdot B \cdot \Omega'}{\xi}}
        }{\Sigma'}{s'}
        \justifies
        \relj{\factrel{
            \btriseq{\Gamma}{\Delta}{\Omega \cdot A \otimes B \cdot \Omega'}{\xi}}
        }{\Sigma'}{s'}
        \using{\otimes A}
      \end{prooftree}
    \]

    The thesis follows immediately from the inductive hypothesis.

  \item Cases $\limp A$ and $bact$ are treated similarly.

  \item Case $\faminus$.

    \[
      \begin{prooftree}
        \relj{\bactrel{\btriseq{\cdot}{\cdot}{L}{\cdot}}}{s}{\Sigma}
        \justifies
        \relj{\blfrel{L}}{s}{\Sigma}
        \using{FA^-}
      \end{prooftree}
    \]

    Suppose there exists $\Sigma' \preceq \Sigma$ such that
    $\relj{\flfrel{L}}{\Sigma'}{s'}$. Then, it must have been derived with a
    corresponding application of $\faminus$:

    \[
      \begin{prooftree}
        \relj{\factrel{\btriseq{\cdot}{\cdot}{L}{\cdot}}}{\Sigma'}{s'}
        \justifies
        \relj{\flfrel{L}}{\Sigma'}{s'}
        \using{\faminus}
      \end{prooftree}
    \]

    By inductive hypothesis, $s' \preceq s$.

  \item Case $conj-$.

    \[
      \begin{prooftree}
        p \; \text{left-biased}
        \justifies
        \relj{\blfrel{p}}{\bneuseq{\Gamma}{\Delta}{Q}}{\bneuseq{\Gamma}{\Delta, p}{Q}}
        \using{conj^-}
      \end{prooftree}
    \]

    Suppose $\Sigma' \preceq \Sigma$ for which
    $\flfrelj{p}{\Sigma'}{s'}$. Then, it must have been derived by an
    application of the corresponding $conj-$:

    \[
      \begin{prooftree}
        p~\text{left-biased}
        \justifies
        \flfrelj{p}{\fneuseq{\Gamma'}{\Delta', p}{Q'}}{\fneuseq{\Gamma'}{\Delta'}{Q'}}
      \end{prooftree}
    \]

    By hypothesis, $\Gamma' \subseteq \Gamma, \Delta = \Delta', Q = Q'$. Hence
    $s' \preceq s$.

  \item Case $\matchrule$.

    \[
      \begin{prooftree}
        \justifies
        \relj{
          \bactrel{\btriseq{\Gamma}{\Delta}{\cdot}{\cdot}}
        }{
          \bneuseq{\Gamma'}{\Delta'}{Q}
        }{
          \bneuseq{\Gamma, \Gamma'}{\Delta, \Delta'}{Q}
        }
        \using{\matchrule}
      \end{prooftree}
    \]

    Suppose $\Sigma' \preceq \Sigma$ such that
    $\relj{ \factrel{\btriseq{\Gamma}{\Delta}{\cdot}{\cdot}} }{ \Sigma' }{ s'
    }$. Then, it must have been an application of the $\matchrule$ rule such that:

    \[
      \begin{prooftree}
        \justifies
        \relj{
          \factrel{\btriseq{\Gamma}{\Delta}{\cdot}{\cdot}}
        }{
          \fneuseq{\Gamma, \Gamma''}{\Delta, \Delta''}{Q'}
        }{
          \fneuseq{\Gamma''}{\Delta''}{Q'}
        }
        \using{\matchrule}
      \end{prooftree}
    \]

    remembering that $\Gamma, \Gamma''$ is supposed to be a partitioning of the
    unrestricted context of the input sequent, hence
    $\Gamma \cap \Gamma'' = \empty$.  From $\Sigma' \preceq \Sigma$ it follows
    that $Q \equiv Q'$ and $\Delta, \Delta' = \Delta, \Delta''$, hence $\Delta'
    = \Delta''$.
    Moreover, we have $\Gamma \cup \Gamma'' \subseteq \Gamma \cup
    \Gamma'$, implying $\Gamma'' \subseteq \Gamma'$. Therefore, $s' \preceq s$.

  \item Case $\matchprimerule$.

    \[
      \begin{prooftree}
        \justifies
        \relj{
          \bactrel{\btriseq{\Gamma}{\Delta}{\cdot}{Q}}
        }{
          \bneuseq{\Gamma'}{\Delta'}{\cdot}
        }{
          \bneuseq{\Gamma, \Gamma'}{\Delta, \Delta'}{Q}
        }
        \using{\matchprimerule}
      \end{prooftree}
    \]

    Suppose $\Sigma' \preceq \Sigma$ such that
    $\relj{\factrel{\btriseq{\Gamma}{\Delta}{Q}{\cdot}} }{ \Sigma' }{ s'
    }$. Then, it must have been an application of the $\matchrule$ rule such that:

    \[
      \begin{prooftree}
        \justifies
        \relj{
          \factrel{\btriseq{\Gamma}{\Delta}{\cdot}{Q}}
        }{
          \fneuseq{\Gamma, \Gamma''}{\Delta, \Delta''}{Q}
        }{
          \fneuseq{\Gamma''}{\Delta''}{\cdot}
        }
        \using{\matchrule}
      \end{prooftree}
    \]

    The rest is as in the $\matchrule$ case.
  \end{enumerate}
\end{proof}

\begin{theorem}[Completeness]
  If $\bneuseq{\Gamma}{\Delta}{Q}$, then $\fneuseq{\Gamma'}{\Delta}{Q}$ for some
  $\Gamma' \subseteq \Gamma$.
\end{theorem}
\begin{proof}
  Induction on the derivation of $\bneuseq{\Gamma}{\Delta}{Q}$ and
  Lemma~\ref{fdercompllemma}.
\end{proof}


%%% Local Variables:
%%% mode: latex
%%% TeX-master: "../docs"
%%% End:
