\section{Focused derivations}

TODO what are focused derivations...

In our case, a focused derivation has two phases as usual, but the actions that
are taken are slightly different. In the active phase all possible rules are
applied to $\otimes$ propositions on the left of the sequent symbol. When only
atoms or implications remain, one formula is selected and a \emph{focused phase}
may begin if that formula is a $\otimes$ formula on the right.

Contrary to other focused calculi, we do not have a symmetric subdivision of the
connectives into synchronous and asynchronous ones. In particular, the unusual
non-invertibility of the $\rightarrow R$ rule prevents us to treat $\rightarrow$
as a right asynchronous connective. Moreover, additional care is required when
adapting the ideas of focused linear logic to our case. The reason is that
Zsyntax derivations contains rule applications that are not commutative in
general: since focusing is a way to group similar derivations under suitable
equivalence classes, many derivations that are equivalent for plain linear logic
are very different for Zsyntax. It is therefore necessary to weaken the amount
of focusing in order to keep the calculus complete.

Our backward focusing calculus consists of the following new kinds of sequents:

\begin{table}[h]
  \centering
  \begin{tabular}{|l|l|}
    \hline
    \textbf{symbol} & \textbf{meaning} \\
    \hline
    $\zfocseq{\Gamma}{\nabla}{A}$ & focal sequent with $A$ under focus \\
    $\ztriseq{\Gamma}{\Delta}{\Omega}{\ctrlset{}}{C}$ & active sequent \\
    \hline
  \end{tabular}
\end{table}

The additional context areas in the active sequents are meant to separate
formulas of different kinds. The rationale is that different formulas require
different approaches in the automatic proof search, therefore we need a way to
distinguish them in the sequent itself. Here, $\Delta$ is intended to be a
linear context of implicational or atomic formulas only, whereas $\Omega$ a
linear context that may contain all kinds of formulas.

For active sequents, the $\otimes$ formulas in $\Omega$ are decomposed until
they become either an implication or an atom, in which case they are transferred
to $\Delta$. Eventually, the active sequent is reduced to the form
$\ztriseq{\Gamma}{\Delta}{\cdot}{\ctrlset{}}{C}$, which we call
\emph{neutral sequents} as in [cmu thesis]. From here we can either perform a
act on an implication (on the left or on the right), or launch a focused phase
selecting a focusable proposition on the conclusion.

\begin{definition}[Focusable proposition]
  A proposition is \emph{focusable} if it is a $\otimes$ formula or an atom.
\end{definition}

When we are in a neutral sequent, we may also copy a formula out of the
unrestricted context and immediately use a left rule on it. This is because
formulas in the unrestricted context do no interact with the linear context, so
it does not make any difference if they get copied right when we need them or in
another moment. The premises of the copy rule in the focused calculus resemble
those for $\rightarrow L$, since all formulas in the unrestricted context are
implications.

In what follows we use the convention, adapted from [cmu thesis], to have the
metavariable $P$ range over implications or atoms on the left of the sequent
symbol, $R$ over implications on the right of the sequent symbol, and $p$ over
atoms specifically.

\begin{figure}[h]
  \begin{mdframed}
\[
  \begin{prooftree}
    \justifies
    \zfocseq{\Gamma}{p}{p}
    \using{\init}
  \end{prooftree}
  \qquad\qquad
  \begin{prooftree}
    \ztriseq{\Gamma}{\Delta, P}{\Omega}{\ctrlset{}}{C}
    \justifies
    \ztriseq{\Gamma}{\Delta}{\Omega, P}{\ctrlset{}}{C}
    \using{\lact}
  \end{prooftree}
\]

\[
  \begin{prooftree}
    \zfocseq{\Gamma}{\Delta}{C} \qquad Q\; \text{focusable}
    \justifies
    \ztriseq{\Gamma}{\Delta}{\cdot}{\emptyset}{C}
    \using{focus}
  \end{prooftree}
  \qquad \qquad
  \begin{prooftree}
    \ztriseq{\Gamma}{\Delta}{\cdot}{\emptyset}{R}
    \justifies
    \zfocseq{\Gamma}{\Delta}{R}
    \using{blur}
  \end{prooftree}
\]

\[
  \begin{prooftree}
    \zfocseq{\Gamma, A \rightarrow_{\ctrlset{}}^\emptyset B}{\Delta_1}{A}
    \qquad
    \ztriseq{\Gamma, A \rightarrow_{\ctrlset{}}^\emptyset B}{\Delta_2}{B}{\ctrlset{}'}{C}
    \justifies
    \ztriseq{\Gamma, A \rightarrow_{\ctrlset{}}^\emptyset B}{
      \Delta_1, \Delta_2}{\cdot}{\ctrlset{} \cup \ctrlset{}'}{C}
    \using{copy}
  \end{prooftree}
\]

\[
  \begin{prooftree}
    \ztriseq{\Gamma}{\Delta}{\Omega, A, B}{\ctrlset{}}{C}
    \justifies
    \ztriseq{\Gamma}{\Delta}{\Omega, A \otimes B}{\ctrlset{}}{C}
    \using{\otimes L}
  \end{prooftree}
  \qquad \qquad
  \begin{prooftree}
    \zfocseq{\Gamma}{\Delta_1}{A}
    \qquad
    \zfocseq{\Gamma}{\Delta_2}{B}
    \justifies
    \zfocseq{\Gamma}{\Delta_1, \Delta_2}{A \otimes B}
    \using{\otimes R}
  \end{prooftree}
\]


\[
  \begin{prooftree}
    \zfocseq{\Gamma}{\Delta_1}{A}
    \qquad
    \ztriseq{\Gamma}{\Delta_2}{B}{\ctrlset{}'}{C}
    \justifies
    \ztriseq{\Gamma}{
      \Delta_1, \Delta_2,
      A \rightarrow_{\ctrlset{}}^S B
    }{\cdot}{\ctrlset{} \cup \ctrlset{}'}{C}
    \using{\rightarrow L}
  \end{prooftree}
\]

\[
  \begin{prooftree}
    \ztriseq{\Gamma}{\Delta}{A}{\ctrlset{}}{B}
    \justifies
    \ztriseq{\Gamma}{\Delta}{\cdot}{\emptyset}{
      A \rightarrow_{\ctrlset{}}^{\elembases{\Delta}} B}
    \using{\rightarrow R}
  \end{prooftree}
\]
\end{mdframed}
  \caption{Backward focused calculus}
  \label{fig:bkwdfocused}
\end{figure}


\begin{theorem}[Soundness]\mbox{}
  \begin{enumerate}
  \item If $\zfocseq{\Gamma}{\Delta}{A}$, then
    $\zsyseq{\Gamma}{\Delta}{\emptyset}{A}$;
  \item If $\ztriseq{\Gamma}{\Delta}{\Omega}{\ctrlset{}}{C}$, then
    $\zsyseq{\Gamma}{\Delta, \Omega}{\ctrlset{}}{C}$.
  \end{enumerate}
\end{theorem}
\begin{proof}
  The two assertions are proved simultaneously by straightforward induction on
  the height of the derivation.
\end{proof}


\begin{lemma}\label{otimeslemma}\mbox{}
  \begin{enumerate}
  \item If $\ztriseq{\Gamma}{\Delta_1}{\Omega}{\emptyset}{A}$ and
    $\zfocseq{\Gamma}{\Delta_2}{B}$, then
    $\ztriseq{\Gamma}{\Delta_1, \Delta_2}{\Omega}{\emptyset}{A \otimes B}$;
  \item If $\zfocseq{\Gamma}{\Delta_1}{A}$
    $\ztriseq{\Gamma}{\Delta_2}{\Omega}{\emptyset}{B}$, then
    $\ztriseq{\Gamma}{\Delta_1, \Delta_2}{\Omega}{\emptyset}{A \otimes B}$.
  \end{enumerate}
\end{lemma}
\begin{proof}
  In both cases, in $\Omega$ is non-empty we do a left- or right- commuting case
  with the inductive hypothesis. Therefore, we can just assume $\Omega$ to be
  empty.

  \begin{enumerate}
  \item

    By induction on the left premise. If the last rule is a focus, the thesis
    follows by $\otimes R$ and focus. If it is $\rightarrow R$ then $A$ is an
    implication, and the thesis follows from a blur on the left premise and a
    $\otimes R$. If it is a $\rightarrow L$, the thesis follows by just
    commuting with the right premise of the $\rightarrow L$ rule instance.

  \item Same as the first case.

  \end{enumerate}
\end{proof}

\begin{theorem}\mbox{}

  \begin{enumerate}
  \item If $\ztriseq{\Gamma}{\Delta}{\Omega}{\ctrlset{1}}{A}$, then
    \begin{enumerate}
    \item If $\ztriseq{\Gamma}{\Delta'}{\Omega', A}{2}{C}$ and
      $\respects{\Omega', \Delta'}{\ctrlset{1}}$, then

      $\ztriseq{\Gamma}{\Delta, \Delta'}{\Omega, \Omega'}{
        \ctrlset{1} \cup \ctrlset{2}}{C}$.
    \item If $\ztriseq{\Gamma}{\Delta', A}{\Omega'}{2}{C}$ and
      $\respects{\Omega', \Delta'}{\ctrlset{1}}$, then

      $\ztriseq{\Gamma}{\Delta, \Delta'}{\Omega, \Omega'}{
        \ctrlset{1} \cup \ctrlset{2}}{C}$;
    \end{enumerate}

  \item If $\ztriseq{\Gamma}{\Delta}{\Omega}{\emptyset}{A}$ and
    $\zfocseq{\Gamma}{\Delta', A}{B}$, then

    $\ztriseq{\Gamma}{\Delta, \Delta'}{\Omega}{\emptyset}{B}$.

  \item If $\zfocseq{\Gamma}{\Delta}{A}$, then
    \begin{enumerate}
    \item If $\ztriseq{\Gamma}{\Delta'}{\Omega', A}{\ctrlset{}}{C}$, then
      $\ztriseq{\Gamma}{\Delta'}{\Omega'}{\ctrlset{}}{C}$.
    \item If $\ztriseq{\Gamma}{\Delta', A}{\Omega'}{\ctrlset{}}{C}$, then
      $\ztriseq{\Gamma}{\Delta'}{\Omega'}{\ctrlset{}}{C}$.
    \end{enumerate}

  \end{enumerate}
\end{theorem}
\begin{proof}
  See appendix.
\end{proof}


We define an auxiliary function on formulas, \textsf{exp}, as follows:

\begin{definition}
  The function $\textsf{exp} : \mathcal{L} \to \mathcal{L}^*$ is inductively
  defined as follows:

\begin{align*}
  \textsf{exp}(p) & = p \\
  \textsf{exp}(A \otimes B) & = \textsf{exp}(A), \textsf{exp}(B) \\
  \textsf{exp}(A \rightarrow_{\ctrlset{}}^S B) & = A \rightarrow_{\ctrlset{}}^S B
\end{align*}
\end{definition}

\begin{lemma}\label{explemma}
  If $\ztriseq{\Gamma}{\Delta, \textsf{exp}(A)}{\Omega}{\ctrlset{}}{C}$,
  then $\ztriseq{\Gamma}{\Delta}{\Omega, A}{\ctrlset{}}{C}$.
\end{lemma}
\begin{proof}
  By induction on $A$. If $A$ is an atom or an implication, the thesis
  follows by act. If $A \equiv C \otimes D$, then
  $\textsf{exp}(A) = \textsf{exp}(C \otimes D) = \textsf{exp}(C),
  \textsf{exp}(D)$, so by inductive hypothesis

  \[
    \begin{prooftree}
      \[
        \[
          \ztriseq{\Gamma}{\Delta, \textsf{exp}(C), \textsf{exp}(D)}{\Omega}{\ctrlset{}}{E}
          \justifies
          \ztriseq{\Gamma}{\Delta, \textsf{exp}(C)}{\Omega, D}{\ctrlset{}}{E}
        \]
        \justifies
        \ztriseq{\Gamma}{\Delta}{\Omega, C, D}{\ctrlset{}}{E}
      \]
      \justifies
      \ztriseq{\Gamma}{\Delta}{\Omega, C \otimes D}{\ctrlset{}}{E}
    \end{prooftree}
  \]
\end{proof}

\begin{lemma}[Identity expansions]\label{idexp}
  For any formula $A$,
  \begin{enumerate}
  \item $\ztriseq{\Gamma}{\cdot}{A}{\emptyset}{A}$;
  \item $\zfocseq{\Gamma}{\textsf{exp}(A)}{A}$.
  \end{enumerate}
\end{lemma}
\begin{proof}
  Both proved simultaneously by induction on $A$. If $A$ is an atom, both follow
  trivially. Otherwise,

  \begin{enumerate}
  \item If $A \equiv C \otimes D$, then by inductive hypothesis

    \[
      \begin{prooftree}
        \zfocseq{\Gamma}{\textsf{exp}(A)}{A}
        \qquad
        \zfocseq{\Gamma}{\textsf{exp}(B)}{B}
        \justifies
        \zfocseq{\Gamma}{\textsf{exp}(A), \textsf{exp}(B)}{A \otimes B}
        \using{\otimes R}
      \end{prooftree}
    \]

    Then, by Lemma~\ref{explemma} and the derivation above we get the first
    point.

    \[
      \begin{prooftree}
        \[
          \[
            \zfocseq{\Gamma}{\textsf{exp}(A)}{A}
            \qquad
            \zfocseq{\Gamma}{\textsf{exp}(B)}{B}
            \justifies
            \zfocseq{\Gamma}{\textsf{exp}(A), \textsf{exp}(B)}{A \otimes B}
            \using{\otimes R}
          \]
          \justifies
          \ztriseq{\Gamma}{\textsf{exp}(A), \textsf{exp}(B)}{\cdot}{\emptyset}{A \otimes B}
        \]
        \justifies
        \ztriseq{\Gamma}{\cdot}{A \otimes B}{\emptyset}{A \otimes B}
      \end{prooftree}
    \]

  \item If $A \equiv C \rightarrow_{\ctrlset{}}^S D$, then by inductive
    hypothesis and Lemma~\ref{explemma} we get

    \[
      \begin{prooftree}
        \[
          \[
            \[
              \zfocseq{\Gamma}{\textsf{exp}(C)}{C}
              \qquad
              \ztriseq{\Gamma}{\cdot}{D}{\emptyset}{D}
              \justifies
              \ztriseq{\Gamma}{C \rightarrow_{\ctrlset{}}^S D,
                \textsf{exp}(C)}{}{\ctrlset{}}{D}
            \]
            \justifies
            \ztriseq{\Gamma}{C \rightarrow_{\ctrlset{}}^S D}{C}{\ctrlset{}}{D}
          \]
          \justifies
          \ztriseq{\Gamma}{C \rightarrow_{\ctrlset{}}^S D}{\cdot}{\emptyset}{
            C \rightarrow_{\ctrlset{}}^S D}
        \]
        \justifies
        \ztriseq{\Gamma}{\cdot}{C \rightarrow_{\ctrlset{}}^S D}{\emptyset}{
          C \rightarrow_{\ctrlset{}}^S D}
      \end{prooftree}
    \]

    The second point follows by blur on the derivation above.
  \end{enumerate}
\end{proof}

\begin{lemma}\label{idhelplemma}
  The following are derivable:
  \begin{enumerate}
  \item $\ztriseq{\Gamma}{P}{\cdot}{\emptyset}{P}$;
  \item $\ztriseq{\Gamma}{\cdot}{A, B}{\emptyset}{A \otimes B}$;
  \item $\ztriseq{\Gamma}{\cdot}{A \rightarrow_{\ctrlset{}}^S B,
      A}{\ctrlset{}}{B}$;
  \item $\ztriseq{\Gamma, A}{\cdot}{\cdot}{\emptyset}{A}$.
  \end{enumerate}
\end{lemma}
\begin{proof}
  All points proved easily with the same technique as the identity expansion
  lemma.
\end{proof}

\begin{lemma}\label{activeinversion}\mbox{}
  \begin{enumerate}
  \item If $\ztriseq{\Gamma}{\Delta}{\Omega, A \otimes B}{\ctrlset{}}{C}$, then
    $\ztriseq{\Gamma}{\Delta}{\Omega, A, B}{\ctrlset{}}{C}$;
  \item If $\ztriseq{\Gamma}{\Delta}{\Omega, \Omega'}{\ctrlset{}}{C}$, then
    $\ztriseq{\Gamma}{\Delta, \textsf{exp}(\Omega')}{\Omega}{\ctrlset{}}{C}$
  \end{enumerate}
\end{lemma}
\begin{proof}
  Point 1 is proved by cut with
  $\ztriseq{\Gamma}{\cdot}{A, B}{\emptyset}{A \otimes B}$. Point 2 is proved by
  repeated application of point 1 on $\otimes$ formulas until they become either
  an atom or an implication, then by cut with
  $\ztriseq{\Gamma}{P}{\cdot}{\emptyset}{P}$.
\end{proof}

\begin{theorem}[Completeness]
  If $\zsyseq{\Gamma}{\Omega}{\ctrlset{}}{A}$, then
  $\ztriseq{\Gamma}{\cdot}{\Omega}{\ctrlset{}}{A}$.
\end{theorem}
\begin{proof}
  The proof proceeds as in [cmu thesis], by proving that all rules of the
  backward calculus are admissible in the focused calculus.

  \begin{enumerate}
  \item Case $\init$: $\zsyseq{\Gamma}{A}{\emptyset}{A}$. The thesis follows by
    Lemma~\ref{idexp}.
  \item Case $\copyrule$. Then, by inductive hypothesis, Lemma~\ref{idhelplemma}
    and cut admissibility, we have

    \[
      \begin{prooftree}
        \ztriseq{\Gamma, A}{\cdot}{\cdot}{\emptyset}{A}
        \qquad
        \ztriseq{\Gamma, A}{\cdot}{\Omega, A}{\ctrlset{}}{C}
        \justifies
        \ztriseq{\Gamma, A}{\cdot}{\Omega}{\ctrlset{}}{C}
      \end{prooftree}
    \]

  \item Case $\otimes L$. The thesis follows immediately by a single application
    of the analogous $\otimes L$ rule in the focused calculus.
  \item Case $\otimes R$. Then, by inductive hypothesis, Lemma~\ref{idhelplemma}
    and cut admissibility, we have

    \[
      \begin{prooftree}
        \ztriseq{\Gamma}{\cdot}{\Omega_2}{\emptyset}{B}
        \quad
        \[
          \ztriseq{\Gamma}{\cdot}{\Omega_1}{\emptyset}{A}
          \qquad
          \ztriseq{\Gamma}{\cdot}{A, B}{\emptyset}{A \otimes B}
          \justifies
          \ztriseq{\Gamma}{\cdot}{\Omega_1, B}{\emptyset}{A \otimes B}
        \]
        \justifies
        \ztriseq{\Gamma}{\cdot}{\Omega_1, \Omega_2}{\emptyset}{A \otimes B}
      \end{prooftree}
    \]

  \item Case $\rightarrow L$. Then, by inductive hypothesis, Lemma~\ref{idhelplemma}
    and cut admissibility, we have

    \[
      \begin{prooftree}
        \ztriseq{\Gamma}{\cdot}{\Omega_1}{\emptyset}{A}
        \quad
        \[
          \ztriseq{\Gamma}{\cdot}{
            A, A \rightarrow_{\ctrlset{}}^S B
          }{\ctrlset{}}{B}
          \qquad
          \ztriseq{\Gamma}{\cdot}{\Omega_2, B}{\ctrlset{}'}{C}
          \justifies
          \ztriseq{\Gamma}{\cdot}{
            \Omega_2, A, A \rightarrow_{\ctrlset{}}^S B}{\ctrlset{} \cup \ctrlset{}'}{C}
          \using{\respects{\Omega}{\ctrlset{}}}
        \]
        \justifies
        \ztriseq{\Gamma}{\cdot}{\Omega_1, \Omega_2,
        A \rightarrow_{\ctrlset{}}^S B}{\ctrlset{} \cup \ctrlset{}'}{C}
      \end{prooftree}
    \]

  \item Case $\rightarrow R$. By inductive hypothesis and
    Lemma~\ref{activeinversion}, we have

    \[
      \begin{prooftree}
        \ztriseq{\Gamma}{\cdot}{\Omega, A}{\ctrlset{}}{B}
        \justifies
        \ztriseq{\Gamma}{\textsf{exp}(\Omega)}{A}{\ctrlset{}}{B}
      \end{prooftree}
    \]

    It is an easy induction to see that
    $\elembases{\Omega} = \elembases{\textsf{exp}(\Omega)}$. Therefore,

    \[
      \begin{prooftree}
        \[
          \ztriseq{\Gamma}{\cdot}{\Omega, A}{\ctrlset{}}{B}
          \justifies
          \ztriseq{\Gamma}{\textsf{exp}(\Omega)}{A}{\ctrlset{}}{B}
        \]
        \justifies
        \ztriseq{\Gamma}{\textsf{exp}(\Omega)}{\cdot}{\emptyset}{
          A \rightarrow_{\ctrlset{}}^{\elembases{\Omega}} B}
      \end{prooftree}
    \]

    Then, by repeated application of act and $\otimes L$, we get the conclusion
    $\ztriseq{\Gamma}{\cdot}{\Omega}{\emptyset}{
          A \rightarrow_{\ctrlset{}}^{\elembases{\Omega}} B}$.

  \end{enumerate}
\end{proof}

\subsection{Backward derived rules}

The primary benefit of focusing is the ability to generate derived ``big step''
inference rules: the intermediate results of a focusing or active phase are not
important, since those steps are ``forced'' in some way. Each derived rule
starts (at the bottom) with a neutral sequent from which a synchronous
proposition is selected for focus, and the focusing steps are followed. Then the
active rules are applied, and eventually we obtain a collection of neutral
sequents as the leaves. These neutral sequents are then treated as the premises
of the derived rule that produces the neutral sequent with which we started.

We first construct the backward derived rules. Then we will move to their
forward version. The general design is that intermediate sequents in the eager
active and focusing phases are not be stored in any sequent database; instead,
all sequents constructed during search are neutral sequents at the phase
boundaries. This is achieved by first precomputing the derived rules that
correspond to the frontier literals of the goal sequent.

For any given proposition, we are interested in constructing a derived inference
for the proposition corresponding to a single pair of focusing and inverse
phases. The idea is to interpret a proposition itself as the derived rules that
it embodies. Every propostiion is viewed as a relation between the conclusion of
the rule and its premises at the leaves of the bipole. Both the conclusion and
the premises are neutral sequents, which we indicate as
$\bneuseq{\Gamma}{\Delta}{Q}$.

There are three classes of relational interpretations:

\begin{enumerate}
\item Right focal relations for the focus formula $A$, written $\brfrel{A}$;
\item Left focal relations for the focus formula $A$, written $\blfrel{A}$;
\item Active relations, written
  $\bactrel{\btriseq{\Gamma}{\Delta}{\Omega}{\xi}}$, where $\xi$ is either
  $\cdot$ or a proposition $C$.
\end{enumerate}

The full set of relations is given in Figure~\ref{fig:bkwdrelations}. Each
relation $R$ takes as input the conclusion sequent $s$, and produces a sequence
of premise sequents $\Sigma = s_1, \dots, s_n$; we write this as
$\relj{R}{s}{\Sigma}$. We use these relations to form a calculus of derived
rules which acts on neutral sequents only, as given in
Figure~\ref{fig:bkwdderivedcalculus}.

\begin{figure}[h]
  \begin{mdframed}
    \[
      \begin{prooftree}
        \justifies
        \relj{\brfrel{p}}{\bneuseq{\Gamma}{p}{\cdot}}{\cdot}
        \using{\init}
      \end{prooftree}
    \]

    \[
      \begin{prooftree}
        \relj{\brfrel{A}}{\bneuseq{\Gamma}{\Delta_1}{\cdot}}{\Sigma_1}
        \qquad
        \relj{\brfrel{B}}{\bneuseq{\Gamma}{\Delta_2}{\cdot}}{\Sigma_2}
        \justifies
        \relj{\brfrel{A \otimes B}}{\bneuseq{\Gamma}{\Delta_1,
            \Delta_2}{\cdot}}{\Sigma_1 \cdot \Sigma_2}
        \using{\otimes F}
      \end{prooftree}
    \]

    \[
      \begin{prooftree}
        \justifies
        \relj{\brfrel{R}}{\zbneuseq{\Gamma}{\Delta}{}{\cdot}}{
          \zbneuseq{\Gamma}{\Delta}{\emptyset}{R}}
        \using{\faplus}
      \end{prooftree}
    \]

    \[
      \begin{prooftree}
        \relj{\bactrel{\zsyseq{\Delta}{\Omega \cdot A \cdot B}{\zeta}{\xi}}}{s}{\Sigma}
        \justifies
        \relj{\bactrel{\zsyseq{\Delta}{\Omega \cdot A \otimes B}{\zeta}{\xi}}}{s}{\Sigma}
        \using{\otimes A}
      \end{prooftree}
    \]

    \[
      \begin{prooftree}
        \relj{
          \bactrel{
            \zsyseq{\Delta,P}{\Omega}{\zeta}{\xi}
          }
        }{s}{\Sigma}
        \justifies
        \relj{\bactrel{\zsyseq{\Delta}{\Omega, P}{\zeta}{\xi}}}{s}{\Sigma}
        \using{act}
      \end{prooftree}
    \]

    \[
      \begin{prooftree}
        \justifies
        \relj{
          \bactrel{\zsyseq
            {\Delta}{\cdot}{}{\cdot}}
        }{
          \zbneuseq{\Gamma}{\Delta'}{\ctrlset{}}{Q}
        }{
          \zbneuseq{\Gamma}{\Delta, \Delta'}{\ctrlset{}}{Q}
        }
        \using{match}
      \end{prooftree}
    \]

    \[
      \begin{prooftree}
        \justifies
        \relj{
          \bactrel{\zsyseq{\Delta}{\cdot}{\ctrlset{}}{Q}}
        }{
          \bneuseq{\Gamma}{\Delta'}{\cdot}
        }{
          \zbneuseq{\Gamma}{\Delta, \Delta'}{\ctrlset{}}{Q}
        }
        \using{match'}
      \end{prooftree}
    \]
  \end{mdframed}
  \caption{Backward relations for derived rules}
  \label{fig:bkwdrelations}
\end{figure}

\begin{figure}[h]
  \begin{mdframed}
    \[
      \begin{prooftree}
        (\relj{\brfrel{Q}}{\zbneuseq{\Gamma}{\Delta}{\emptyset}{\cdot}}{\Sigma})
        \quad \Sigma
        \justifies
        \zbneuseq{\Gamma}{\Delta}{\emptyset}{Q}
        \using{focus}
      \end{prooftree}
    \]

    \[
      \begin{prooftree}
        \[
        (\relj{\brfrel{A}}{\zbneuseq{\Gamma}{\Delta_1}{\emptyset}{\cdot}}{\Sigma})
        \proofdotseparation=1.2ex
        \proofdotnumber=0
        \leadsto
        (\relj{\bactrel{\zsyseq{\cdot}{B}{\ctrlset{}'}{\cdot}}}
        {\zbneuseq{\Gamma}{\Delta_2}{\ctrlset{}'}{Q}}{\Sigma'})
        \]
        \quad \Sigma, \Sigma'
        \justifies
        \zbneuseq{\Gamma}{\Delta_1, \Delta_2,
          A \rightarrow_{\ctrlset{}}^S B
        }{\ctrlset{} \cup \ctrlset{}'}{Q}
        \using{\rightarrow L}
      \end{prooftree}
    \]

    \[
      \begin{prooftree}
        (\relj{\bactrel{\zsyseq{\cdot}{A}{\ctrlset{}}{B}}}
        {\zbneuseq{\Gamma}{\Delta}{\ctrlset{}}{\cdot}}{\Sigma})
        \quad \Sigma
        \justifies
        \zbneuseq{\Gamma}{\Delta}{\emptyset}{
          A \rightarrow_{\ctrlset{}}^{\elembases{\Delta}} B}
        \using{\rightarrow R}
      \end{prooftree}
    \]

    \[
      \begin{prooftree}
        \[
          (\relj{\brfrel{A}}{\zbneuseq{\Gamma,
              A \rightarrow_{\ctrlset{}}^S B
            }{\Delta_1}{\emptyset}{\cdot}}{\Sigma})
        \proofdotseparation=1.2ex
        \proofdotnumber=0
        \leadsto
        (\relj{\bactrel{\zsyseq{\cdot}{B}{\ctrlset{}'}{\cdot}}}
        {\zbneuseq{\Gamma,
            A \rightarrow_{\ctrlset{}}^S B
          }{\Delta_2}{\ctrlset{}'}{Q}}{\Sigma'})
        \]
        \quad \Sigma, \Sigma'
        \justifies
        \zbneuseq{\Gamma, A \rightarrow_{\ctrlset{}}^S B}{\Delta_1, \Delta_2
        }{\ctrlset{} \cup \ctrlset{}'}{Q}
        \using{\copyrule}
      \end{prooftree}
    \]

  \end{mdframed}
  \caption{Backward calculus of derived rules}
  \label{fig:bkwdderivedcalculus}
\end{figure}

\begin{definition}
  A sequent of the backward derived rule calculus
  $\zbneuseq{\Gamma}{\Delta}{\ctrlset{}}{Q}$
  is sound if $\ztriseq{\Gamma}{\Delta}{\cdot}{\ctrlset{}}{Q}$.
\end{definition}

\begin{lemma}\label{bkwdder-soundness-lemma}
  \begin{enumerate}
  \item If $\brfrelj{A}{\zbneuseq{\Gamma}{\Delta}{\emptyset}{\cdot}}{\Sigma}$
    and $\Sigma$
    are sound, then

    $\zfocseq{\Gamma}{\Delta}{A}$.
  \item If
    $\bactrelj{\zsyseq{\Delta}{\Omega}{\ctrlset{}}{\cdot}}
    {\zbneuseq{\Gamma}{\Delta'}{\ctrlset{}}{Q}}{\Sigma}$
    and $\Sigma$ are sound, then

    $\ztriseq{\Gamma}{\Delta, \Delta'}{\Omega}{\ctrlset{}}{Q}$.
  \item If
    $\bactrelj{\zsyseq{\Delta}{\Omega}{\ctrlset{}}{Q}}
    {\zbneuseq{\Gamma}{\Delta'}{\ctrlset{}}{\cdot}}{\Sigma}$
    and $\Sigma$ are sound, then

    $\ztriseq{\Gamma}{\Delta, \Delta'}{\Omega}{\ctrlset{}}{Q}$.
  \end{enumerate}
\end{lemma}
\begin{proof}
  All points are proved simultaneously and follow immediately by mutual
  induction on the height of the derivations and use of the corresponding rules
  of the focused calculus.
\end{proof}

\begin{theorem}[Soundness]
  If $\zbneuseq{\Gamma}{\Delta}{\ctrlset{}}{Q}$ then
  $\ztriseq{\Gamma}{\Delta}{\cdot}{\ctrlset{}}{Q}$.
\end{theorem}
\begin{proof}
  Straightforward induction on the derivation of the backward derived rules
  calculus, and Lemma~\ref{bkwdder-soundness-lemma}.
\end{proof}

\begin{lemma}\label{completeness-lemma}
  \begin{enumerate}

  \item If $\zfocseq{\Gamma}{\Delta}{A}$, then for some $\Sigma$
    \begin{enumerate}
    \item $\relj{\brfrel{A}}{\zbneuseq{\Gamma}{\Delta}{}{\cdot}}{\Sigma}$, and
    \item $\Sigma$ are all derivable
    \end{enumerate}

  \item If $\ztriseq{\Gamma}{\Delta_1, \Delta_2}{\Omega}{\zeta \uplus \delta}{\xi \uplus
      \gamma}$ (where $x \uplus y$ means either $x$ or $y$ is
    empty), then for some $\Sigma$
    \begin{enumerate}
    \item
      $\bactrelj{\zsyseq{\Delta_1}{\Omega}{\zeta}{\xi}}
      {\zbneuseq{\Gamma}{\Delta_2}{\delta}{\gamma}}{\Sigma}$,
      and
    \item $\Sigma$ are all derivable.
    \end{enumerate}
  \end{enumerate}
\end{lemma}
\begin{proof}
  By simultaneous induction on the height of the derivation.
  \begin{enumerate}
  \item Case $\init$. Then, just apply the $\init$ relation.
  \item Case act. Straightforward use of the act rule for derived
    relations.
  \item Case $focus$.

    \[
      \begin{prooftree}
        \zfocseq{\Gamma}{\Delta_1, \Delta_2}{Q} \qquad Q\; \text{right-focusable}
        \justifies
        \ztriseq{\Gamma}{\Delta_1, \Delta_2}{\cdot}{\emptyset}{Q}
        \using{focus}
      \end{prooftree}
    \]

    Suppose $\gamma_1, \gamma_2$ are such that $\gamma_1 \uplus \gamma_2 =
    Q$, and $\zeta_1 \uplus \zeta_2 = \emptyset$. Then, the following:

    \[
      \relj{
        \bactrel{\zsyseq{\Delta_1}{\cdot}{\zeta_1}{\gamma_1}}
      }{
        \zbneuseq{\Delta_2}{\cdot}{\zeta_2}{\gamma_2}
      }{
        \zbneuseq{\Gamma}{\Delta_1, \Delta_2}{\emptyset}{Q}
      }
    \]

    is derivable with either $\matchrule$ or $\matchprimerule$. We need to show
    that $\zbneuseq{\Gamma}{\Delta_1, \Delta_2}{\emptyset}{Q}$ is derivable. By
    inductive hypothesis, we have
    $\relj{\brfrel{Q}}{\zbneuseq{\Gamma}{\Delta_1, \Delta_2}{}{\cdot}}{\Sigma}$ for
    some $\Sigma$ all derivable. But then,

    \[
      \begin{prooftree}
        \relj{\brfrel{Q}}{\zbneuseq{\Gamma}{\Delta_1, \Delta_2}{}{\cdot}}{\Sigma}
        \qquad \Sigma
        \justifies
        \zbneuseq{\Gamma}{\Delta_1, \Delta_2}{\emptyset}{Q}
        \using{focus}
      \end{prooftree}
    \]

  \item Case $blur$.

    \[
      \begin{prooftree}
        \ztriseq{\Gamma}{\Delta}{\cdot}{\emptyset}{R}
        \justifies
        \zfocseq{\Gamma}{\Delta}{R}
        \using{blur}
      \end{prooftree}
    \]

    By inductive hypothesis, we have
    $\relj{\bactrel{\zsyseq{\cdot}{\cdot}{\emptyset}{R}}}{\bneuseq{\Gamma}{\Delta}{\cdot}}{\Sigma}$,
    where all $\Sigma$ are derivable.
    But then is must have been derived with an instance of $\matchrule$, hence
    $\zbneuseq{\Gamma}{\Delta}{\emptyset}{R}$ is derivable.
    We then have

    \[
      \begin{prooftree}
        \justifies
        \relj{\brfrel{R}}{\bneuseq{\Gamma}{\Delta}{\cdot}}{
          \zbneuseq{\Gamma}{\Delta}{\emptyset}{R}
        }
        \using{FA^+}
      \end{prooftree}
    \]

  \item Case $\rightarrow L$.

    \[
      \begin{prooftree}
        \zfocseq{\Gamma}{\Delta_1}{A}
        \qquad
        \ztriseq{\Gamma}{\Delta_2}{B}{\ctrlset{}'}{C}
        \qquad
        \respects{\Delta_2}{\ctrlset{}}
        \justifies
        \ztriseq{\Gamma}{
          \Delta_1, \Delta_2, A \rightarrow_{\ctrlset{}}^S B
        }{\cdot}{\ctrlset{} \cup \ctrlset{}'}{C}
        \using{\rightarrow L}
      \end{prooftree}
    \]

    Suppose $\Delta', \Delta'' = \Delta_1, \Delta_2,
    A\rightarrow_{\ctrlset{}}^S B$, and $\gamma_1 \uplus \gamma_2 = C$,
    $\zeta_1 \uplus \zeta_2 = \ctrlset{}\cup\ctrlset{}'$.
    We can apply $\matchrule$ to immediately derive


    \[
      \begin{prooftree}
        \justifies
        \relj{\bactrel{
            \zsyseq{
              \Delta'}{\cdot}{\zeta_1}{\gamma_1}}}
        {\zbneuseq{\Gamma}
          {\Delta''}{\zeta_2}{\gamma_2}}{\zbneuseq{
            \Gamma}{\Delta', \Delta''}{
            \ctrlset{}\cup\ctrlset{}'
          }{C}}
        \using{\matchrule}
      \end{prooftree}
    \]

    We can see that
    $\zbneuseq{\Gamma} {\Delta', \Delta''}{\ctrlset{}\cup\ctrlset{}'}{C}$ is
    derivable by inductive hypothesis on the premises and an application of
    $\rightarrow L$ of the derived rule calculus.

    \[
      \begin{prooftree}
        \[
        (\relj{\brfrel{A}}{\zbneuseq{\Gamma}{\Delta_1}{}{\cdot}}{\Sigma})
        \proofdotseparation=1.2ex
        \proofdotnumber=0
        \leadsto
        (\relj{\bactrel{\zsyseq{\cdot}{B}{}{\cdot}}}
        {\zbneuseq{\Gamma}{\Delta_2}{\ctrlset{}'}{C}}{\Sigma'})
        \]
        \quad \Sigma, \Sigma'
        \justifies
        \zbneuseq{\Gamma}{\Delta_1, \Delta_2,
          A \rightarrow_{\ctrlset{}}^S B
        }{\ctrlset{} \cup \ctrlset{}'}{C}
        \using{\rightarrow L}
      \end{prooftree}
    \]

  \item The case $\copyrule$ is analogous to $\rightarrow L$.

  \item Case $\rightarrow R$.

    \[
      \begin{prooftree}
        \ztriseq{\Gamma}{\Delta_1, \Delta_2}{A}{\ctrlset{}}{B}
        \justifies
        \ztriseq{\Gamma}{\Delta_1, \Delta_2}{\cdot}{\emptyset}{
          A \rightarrow_{\ctrlset{}}^{\elembases{\Delta_1, \Delta_2}} B}
        \using{\rightarrow R}
      \end{prooftree}
    \]

    Again, suppose
    $\gamma_1 \uplus \gamma_2 = A \rightarrow_{\ctrlset{}}^{\elembases{\Delta_1,
        \Delta_2}} B$, $\zeta_1 \uplus \zeta_2 = \emptyset$. We can immediately
    derive by either $\matchrule$ or $\matchprimerule$ the following

    \[
      \begin{prooftree}
        \justifies
        \relj{\bactrel{
            \zsyseq{
              \Delta_1}{\cdot}{\zeta_1}{\gamma_1}}}
        {\zbneuseq{\Gamma}
          {\Delta_2}{\zeta_2}{\gamma_2}}{\zbneuseq{
            \Gamma}{\Delta_1, \Delta_2}{
            \emptyset
          }{A \rightarrow_{\ctrlset{}}^{\elembases{\Delta_1, \Delta_2}} B}}
        \using{\matchrule}
      \end{prooftree}
    \]

    To see that
    $\zbneuseq{ \Gamma}{\Delta_1, \Delta_2}{ \emptyset }{A
      \rightarrow_{\ctrlset{}}^{\elembases{\Delta_1, \Delta_2}} B}$ is
    derivable, we use the inductive hypothesis on the premise of $\rightarrow R$
    and an application of the corresponding rule of the derived rule calculus.

    \[
      \begin{prooftree}
        (\relj{\bactrel{\zsyseq{\cdot}{A}{\ctrlset{}}{B}}}
        {\zbneuseq{\Gamma}{\Delta_1, \Delta_2}{}{\cdot}}{\Sigma})
        \quad \Sigma
        \justifies
        \zbneuseq{\Gamma}{\Delta_1, \Delta_2}{\emptyset}{
          A \rightarrow_{\ctrlset{}}^{\elembases{\Delta_1, \Delta_2}} B}
        \using{\rightarrow R}
      \end{prooftree}
    \]


  \item The remaining $\otimes L$ and $\otimes R$ cases are straightforward
    applications of the inductive hypothesis.
  \end{enumerate}
\end{proof}

\begin{theorem}[Completeness]
  If $\ztriseq{\Gamma}{\Delta}{\cdot}{\ctrlset{}}{Q}$, then
  $\zbneuseq{\Gamma}{\Delta}{\ctrlset{}}{Q}$.
\end{theorem}
\begin{proof}
  Straightforward induction on the derivation and application of
  Lemma~\ref{completeness-lemma}.
\end{proof}

%%% Local Variables:
%%% mode: latex
%%% TeX-master: "../docs"
%%% End:
