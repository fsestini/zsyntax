\section{Backward sequent calculus}

The objective of this section is to obtain a sequent calculus for \eznd{}, that
is amenable to efficient proof search while being sound and complete with
respect to \eznd{} itself. Given the obvious similarities between a Zsyntax
proof and a proof in intuitionistic linear logic, it seems reasonable to aim for
a slightly modified version of the sequent calculus for intuitionistic linear
logic in our search for such a calculus. The rest of this section does exactly
this.

In what follows, we try to build the rules of our sequent calculus by starting
from what would be a direct translation of the rules of \eznd{}, and then
refining them by cut until we reach a form that allows us to easily prove the
subformula property and cut elimination. As premises and conclusions we have
annotated sequents of the form

\[
  \zsyseq{\Gamma}{\Delta}{l}{C}
\]

where $\Gamma$ is the \emph{unrestricted context}, namely one where contraction
and weakening are accepted, $\Delta$ is the usual linear context, and $l$ is a
reaction list. The distinction between unrestricted and linear contexts in the
same sequent goes back to \cite{GIRARD1993201}, and provides in our setting a
simple and clear way to represent the ``globally known'' sets of biological
axioms that can be used freely during proof search.

The intuitive explanation is that a valid sequent of this form
witnesses the fact that, under the set $\Gamma$ of biological axioms available
in $\Gamma$, there exists a sequence of Zsyntax transitions that bring the
aggregate $\Delta$ to the aggregate $C$. In particular, this reaction happens in
any context that respects the reaction list, that is, in any $\Delta'$ such that
$\resplist{\Delta'}{l}$. By soundness, the validity of the sequent above
expresses that we know that $\Delta \models_{l} C$ holds in \eznd{}.

Throughout the rest of this work, we assume that the set of axioms is
``harmless'', in the sense that $\elembases{\Gamma} = \emptyset$.  This is the
same as saying that all formulas in $\Gamma$ must be conditionals with an empty
elementary base (which corresponds to our idea of axioms in this system).

We now give the rules of the calculus. To do so, we assume the following
form of cut rule to be available, even though we will not really include in the
calculus. This is because it will be shown later to be admissible.

\[
  \begin{prooftree}
    \zsyseq{\Gamma}{\Delta_1}{\reactlist{1}}{A}
    \qquad
    \zsyseq{\Gamma}{\Delta_2, A}{\reactlist{2}}{B}
    \justifies
    \zsyseq{\Gamma}{\Delta_1, \Delta_2}{
      \baseandplus{\Delta_2}{\reactlist{1}}{\reactlist{2}}
    }{B}
    \using{\resplist{\Delta_2}{l_1}}
  \end{prooftree}
\]

Let us examine and justify the reaction list annotation in the
conclusion. According to our interpretation of sequents, the left premise says
that there exists a reaction from $\Delta_1$ to $A$ with a series of
intermediate reactions described by $\reactlist{1}$, whereas the right premise
says that there exists a reaction from $\Delta_2, A$ to $B$ with a series of
intermediate reactions described by $\reactlist{2}$.
Moreover, we know that $\Delta_2$ respects the reaction list $\reactlist{1}$, so
that it can be added to $\Delta_1$ without interfering with its reactions.
This entitles us to assert that there exists a reaction from $\Delta_1, \Delta_2$
to $B$, as it sufficies to make $\Delta_1$ react to $A$, and then finally make
$\Delta_2,A$ react to $B$.
Notice that the reaction list describing this overall reaction is not simply the
concatenation of the premise lists, as now the reaction of the left premise has
gained $\Delta_2$ alongside. The conclusion list in therefore the concatenation
of $\reactlist{1}$ with the addition of $\elembases{\Delta_2}$ (since $\Delta_2$
is present as a side aggregate of \emph{all} intermediate reactions of the left
premise), and $\reactlist{2}$.

\paragraph{Identity}

The trivial reaction is the one that does nothing:

\[
  \begin{prooftree}
    \justifies
    \zsyseq{\Gamma}{A}{[]}{A}
  \end{prooftree}
\]

It will be proved later that we can rescrict $A$ to the atomic case while
retaining completeness.

\paragraph{Biological axioms}

The calculus has a built-in way to account for (biological) axioms with the
unrestricted context. To make an axiom $A$ available to a derivation, it
sufficies to take it from the unrestricted context and add it to the linear
context. Formulas in the unrestricted context can of course be used ad libitum:

\[
  \begin{prooftree}
    \zsyseq{\Gamma, A}{\Delta, A}{\reactlist{}}{C}
    \justifies
    \zsyseq{\Gamma, A}{\Delta}{\reactlist{}}{C}
  \end{prooftree}
\]

\paragraph{$\otimes$ left}

A direct translation from Zsyntax yields a rule that is already in an acceptable
form for a left rule.

\[
  \begin{prooftree}
    \zsyseq{\Gamma}{\Delta, A, B}{l}{C}
    \justifies
    \zsyseq{\Gamma}{\Delta, A \otimes B}{l}{C}
  \end{prooftree}
\]

\paragraph{$\otimes$ right}

The $\otimes$ introduction rule of the original calculus is the exact inverse of
the elimination rule, so a direct translation from Zsyntax to sequent calculus
would be the following:

\[
  \begin{prooftree}
    \zsyseq{\Gamma}{\Delta, A \otimes B}{\reactlist{}}{C}
    \justifies
    \zsyseq{\Gamma}{\Delta, A, B}{\reactlist{}}{C}
  \end{prooftree}
\]

This, however, is not a good rule as it augments the complexity of the sequent,
and breaks the left-right symmetry. We first notice that the above rule and the
following axiom are equivalent

\[
  \begin{prooftree}
    \justifies
    \zsyseq{\Gamma}{A,B}{[]}{A \otimes B}
  \end{prooftree}
\]

The second comes from the first by identity. The first comes from the second by
cut:

\[
  \begin{prooftree}
    \[ \justifies \zsyseq{\Gamma}{A,B}{[]}{A \otimes B} \]
    \qquad
    \zsyseq{\Gamma}{\Delta, A \otimes B}{\reactlist{}}{C}
    \justifies
    \zsyseq{\Gamma}{\Delta, A, B}{\reactlist{}}{C}
  \end{prooftree}
\]

However, we have to do some additional cuts to the axiom above in order to
obtain something that allows us to eliminate the cut rule later.


\[
  \begin{prooftree}
    \zsyseq{\Gamma}{\Delta_2}{[]}{B} \qquad
    \[
      \zsyseq{\Gamma}{\Delta_1}{[]}{A} \qquad
      \[\justifies \zsyseq{\Gamma}{A,B}{[]}{A \otimes B} \]
      \justifies
      \zsyseq{\Gamma}{\Delta_1, B}{[]}{A \otimes B}
    \]
    \justifies
    \zsyseq{\Gamma}{\Delta_1, \Delta_2}{[]}{A \otimes B}
  \end{prooftree}
\]

thus yielding the following rule

\[
  \begin{prooftree}
    \zsyseq{\Gamma}{\Delta_1}{[]}{A} \qquad
    \zsyseq{\Gamma}{\Delta_2}{[]}{B}
    \justifies
    \zsyseq{\Gamma}{\Delta_1, \Delta_2}{[]}{A \otimes B}
  \end{prooftree}
\]

which is also a bit more general, since it does not only says that $A \otimes B$
is entailed by $A,B$, but also by any pair of aggregates that can be transformed
into $A,B$ in a straightforward (i.e., not involving control set-constrained
reactions) way.

\paragraph{$\rightarrow$ introduction}

A direct translation to our single-succedent sequent calculus would be the
following:

\[
  \begin{prooftree}
    \zsyseq{\Gamma}{\Delta_1, A}{\reactlist{1}}{B} \qquad
    \zsyseq{\Gamma}{\Delta_2, A \rightarrow^{\elembases{\Delta_1}}_{\reactlist{1}}
      B}{\reactlist{2}}{C}
    \justifies
    \zsyseq{\Gamma}{\Delta_1, \Delta_2}{\reactlist{2}}{C}
  \end{prooftree}
\]

However, this rule again has the drawback that it increases the complexity of
the sequent backwards. Worse so, it introduces a conditional that does not occur
in the conclusion, to that we would have to guess it out of thin air when doing
proof search backwards. A better rule can be obtained as a special case of the
one above:

\[
  \begin{prooftree}
    \zsyseq{\Gamma}{\Delta_1, A}{\reactlist{1}}{B}
    \qquad
    \zsyseq{\Gamma}{A \rightarrow^{\elembases{\Delta_1}}_{\reactlist{1}}
      B}{[]}{A \rightarrow^{\elembases{\Delta_1}}_{\reactlist{1}}
      B}
    \justifies
    \zsyseq{\Gamma}{\Delta_1}{[]}{A \rightarrow^{\elembases{\Delta_1}}_{\reactlist{1}}
      B}
  \end{prooftree}
\]

thus yielding the following:

\[
  \begin{prooftree}
    \zsyseq{\Gamma}{\Delta, A}{\reactlist{}}{B}
    \justifies
    \zsyseq{\Gamma}{\Delta}{[]}{A \rightarrow^{\elembases{\Delta}}_{\reactlist{}}
      B}
  \end{prooftree}
\]

The equivalence is witnessed by the fact that the first rule can be obtained
from the second by cut:

\[
  \begin{prooftree}
    \[
      \zsyseq{\Gamma}{\Delta_1, A}{\reactlist{1}}{B}
      \justifies
      \zsyseq{\Gamma}{\Delta}{[]}{A \rightarrow^{\elembases{\Delta_1}}_{\reactlist{1}}
        B}
    \]
    \qquad
    \zsyseq{\Gamma}{\Delta_2, A \rightarrow^{\elembases{\Delta_1}}_{\reactlist{1}}
      B}{\reactlist{2}}{C}
    \justifies
    \zsyseq{\Gamma}{\Delta_1, \Delta_2}{\reactlist{2}}{C}
  \end{prooftree}
\]

\paragraph{$\rightarrow$ elimination}

The direct translation would be the following:

\[
  \begin{prooftree}
    \zsyseq{\Gamma}{\Delta, B}{\reactlist{}'}{C}
    \justifies
    \zsyseq{\Gamma}{\Delta, A \rightarrow^{\elembases{S}}_{\reactlist{}}
      B, A}{\listplus{\basepluslist{\Delta}{\reactlist{}}}{\reactlist{}'}}{C}
    \using{\resplist{\Delta}{\reactlist{}}}
  \end{prooftree}
\]

If we can get from $\Delta,B$ to $C$ with a series of transitions described by
$\reactlist{}'$, then we can do so if we do not have $B$ right away, but we can
get one by eliminating some conditional $A \rightarrow B$, provided of course
that the side aggregate $\Delta$ respects all transitions that are involved in
the reaction. The overall reaction list must then include the ones of the
conditional (in this case $\reactlist{}$) extended with $\Delta$, as we
explained when introducing the cut rule.

We can get to a better left rule that has only the implication as principal
formula with cut:

\[
  \begin{prooftree}
    \zsyseq{\Gamma}{\Delta_1}{[]}{A}\qquad
    \[
      \zsyseq{\Gamma}{\Delta_2, B}{\reactlist{2}}{C}
      \justifies
      \zsyseq{\Gamma}{\Delta_2, A \rightarrow^{\elembases{S}}_{\reactlist{1}}
        B, A}{\listplus{\basepluslist{\Delta_2}{\reactlist{1}}}{\reactlist{2}}}{C}
      \using{\resplist{\Delta_2}{\reactlist{1}}}
    \]
    \justifies
    \zsyseq{\Gamma}{\Delta_1, \Delta_2, A
      \rightarrow^{\elembases{S}}_{\reactlist{1}} B}{
      \listplus{\basepluslist{\Delta_2}{\reactlist{1}}}{\reactlist{2}}
      }{C}
  \end{prooftree}
\]

thus yielding the following rule:

\[
  \begin{prooftree}
    \zsyseq{\Gamma}{\Delta_1}{[]}{A}\qquad\qquad
    \zsyseq{\Gamma}{\Delta_2, B}{\reactlist{2}}{C}
    \justifies
    \zsyseq{\Gamma}{\Delta_1, \Delta_2, A
      \rightarrow^{\elembases{S}}_{\reactlist{1}} B}{
      \listplus{\basepluslist{\Delta_2}{\reactlist{1}}}{\reactlist{2}}
      }{C}
    \using{
      \resplist{\Delta_2}{\reactlist{1}}}
  \end{prooftree}
\]

The resulting sequent calculus is given in its entirety in
Figure~\ref{bkwseqcalc}, where the formula in the $\init$ axiom is assumed to be
logically atomic.

\begin{figure}[ht]
  \begin{mdframed}

    \[
      \begin{prooftree}
        \justifies
        \zsyseq{\Gamma}{A}{[]}{A}
        \using{\init}
      \end{prooftree}
      \qquad \qquad
      \begin{prooftree}
        \zsyseq{\Gamma, A}{\Delta, A}{\reactlist{}}{C}
        \justifies
        \zsyseq{\Gamma, A}{\Delta}{\reactlist{}}{C}
        \using{\copyrule}
      \end{prooftree}
    \]

    \[
      \begin{prooftree}
        \zsyseq{\Gamma}{\Delta_1}{[]}{A} \qquad
        \zsyseq{\Gamma}{\Delta_2}{[]}{B}
        \justifies
        \zsyseq{\Gamma}{\Delta_1, \Delta_2}{[]}{A \otimes B}
        \using{\otimes R}
      \end{prooftree}
    \]

    \[
      \begin{prooftree}
        \zsyseq{\Gamma}{\Delta, A \otimes B}{\reactlist{}}{C}
        \justifies
        \zsyseq{\Gamma}{\Delta, A, B}{\reactlist{}}{C}
        \using{\otimes L}
      \end{prooftree}
      \qquad \qquad
      \begin{prooftree}
        \zsyseq{\Gamma}{\Delta, A}{\reactlist{}}{B}
        \justifies
        \zsyseq{\Gamma}{\Delta}{[]}{A \limp^{\elembases{\Delta}}_{\reactlist{}}
          B}
        \using{\rightarrow R}
      \end{prooftree}
    \]

    \[
      \begin{prooftree}
        \zsyseq{\Gamma}{\Delta_1}{[]}{A}\qquad
        \zsyseq{\Gamma}{\Delta_2, B}{\reactlist{2}}{C}
        \qquad \resplist{\Delta_2}{\reactlist{1}}
        \justifies
        \zsyseq{\Gamma}{\Delta_1, \Delta_2, A \rightarrow^{S}_{\reactlist{1}} B}{
          \listplus{\basepluslist{\Delta_2}{\reactlist{1}}}{\reactlist{2}}}{C}
        \using{\rightarrow L}
      \end{prooftree}
    \]

  \end{mdframed}
  \caption{\label{bkwseqcalc} Annotated backward sequent calculus \zss{}.}
\end{figure}

\begin{lemma}[Identity expansion]
  The calculus \zss{}' with identity axiom allowing arbitrary formulas is
  equivalent to the calculus \zss{} with identity axiom restricted to atomic
  formulas.
\end{lemma}
\begin{proof}
  We show by structural induction on the involved formula that if
  $\zsyseq{\Gamma}{A \otimes B}{[]}{A \otimes B}$, then the same sequent can be
  derived with only atoms at the $\init$ axiom instances. There are as many
  cases as there are connectives to consider.

  \begin{enumerate}
  \item Case $\zsyseq{\Gamma}{A \otimes B}{[]}{A \otimes B}$. Then,

    \[
      \begin{prooftree}
        \[
          \zsyseq{\Gamma}{A}{[]}{A}
          \qquad
          \zsyseq{\Gamma}{B}{[]}{B}
          \justifies
          \zsyseq{\Gamma}{A, B}{[]}{A \otimes B}
          \using{\otimes R}
        \]
        \justifies
        \zsyseq{\Gamma}{A \otimes B}{[]}{A \otimes B}
        \using{\otimes L}
      \end{prooftree}
    \]

  \item Case $\zsyseq{\Gamma}{A \rightarrow_{\reactlist{}}^S B}{[]}{A
      \rightarrow_{\reactlist{}}^S B}$. Then, the following holds

    \[
      \begin{prooftree}
        \[
          \zsyseq{\Gamma}{A}{[]}{A}\qquad
          \zsyseq{\Gamma}{B}{[]}{B}
          \justifies
          \zsyseq{\Gamma}{A \rightarrow_{\reactlist{}}^S B, A}{\reactlist{}}{B}
        \]
        \justifies
        \zsyseq{\Gamma}{A \rightarrow_{\reactlist{}}^S B}{[]}{A
          \rightarrow_{\reactlist{}}^S B}
      \end{prooftree}
    \]

    since $\elembases{A \rightarrow^S_{\reactlist{}} B} = S$ and
    $\listplus{\basepluslist{\emptyset}{l}}{[]} = l$.
  \end{enumerate}
\end{proof}

The next theorem establishes that the cut rule introduced above is redundant, as
one would expect.

\begin{theorem}\label{backwardcutelim}[Cut admissibility]
  The following cut rule

  \[
    \begin{prooftree}
      \zsyseq{\Gamma}{\Delta_1}{\reactlist{1}}{A}
      \qquad
      \zsyseq{\Gamma}{\Delta_2, A}{\reactlist{2}}{B}
      \justifies
      \zsyseq{\Gamma}{\Delta_1, \Delta_2}{
        \listplus{\basepluslist{\Delta_2}{\reactlist{1}}}{\reactlist{2}}}{B}
      \using{\resplist{\Delta_2}{\reactlist{1}}}
    \end{prooftree}
  \]

  is admissible.
\end{theorem}
\begin{proof}
  See Appendix.
\end{proof}

We now prove soundness and completeness of the sequent calculus with respect to
$\eznd{}$. On a side note, since the theorem prover is an implementation of the
sequent calculus just presented, these two results mean that any difference
between the prover's underlying logic and Zsyntax can be assessed by looking at
\eznd{} alone.

\begin{theorem}[Soundness]
  If $\zsyseq{\Gamma}{\Delta}{\reactlist{}}{A}$, then
  $\Delta \models_{\reactlist{}} A$.
\end{theorem}
\begin{proof}
  By easy induction on the sequent derivation.

  \begin{enumerate}
  \item Case $\init$. Straightforward.
  \item Case $\copyrule$.

    \[
      \begin{prooftree}
        \zsyseq{\Gamma, A}{\Delta, A}{\reactlist{}}{C}
        \justifies
        \zsyseq{\Gamma, A}{\Delta}{\reactlist{}}{C}
        \using{\copyrule}
      \end{prooftree}
    \]

    Then, by inductive hypothesis

    \[
      \begin{prooftree}
        \Delta \models_{[]} \Delta, A
        \qquad
        \Delta,A \models_{\reactlist{}} C
        \justifies
        \Delta \models_{\reactlist{}} C
      \end{prooftree}
    \]

  \item $\otimes R$.

    \[
      \begin{prooftree}
        \zsyseq{\Gamma}{\Delta_1}{[]}{A} \qquad
        \zsyseq{\Gamma}{\Delta_2}{[]}{B}
        \justifies
        \zsyseq{\Gamma}{\Delta_1, \Delta_2}{[]}{A \otimes B}
        \using{\otimes R}
      \end{prooftree}
    \]

    Then, by inductive hypothesis

    \[
      \begin{prooftree}
        \[
          \[
            \Delta_1 \models_{[]} A
            \justifies
            \Delta_1, \Delta_2 \models_{[]} A, \Delta_2
          \]
          \qquad
          \[
            \Delta_2 \models_{[]} B
            \justifies
            \Delta_2, A \models_{[]} A,B
          \]
          \justifies
          \Delta_1,\Delta_2 \models_{[]} A,B
        \]
        \,
        A,B \models A \otimes B
        \justifies
        \Delta_1,\Delta_2 \models_{[]} A \otimes B
      \end{prooftree}
    \]

  \item The $\otimes L$ and $\rightarrow R$ cases are a straightforward
    application of the inductive hypothesis.

  \item Case $\rightarrow L$.

    \[
      \begin{prooftree}
        \zsyseq{\Gamma}{\Delta_1}{[]}{A}\qquad
        \zsyseq{\Gamma}{\Delta_2, B}{\reactlist{2}}{C}
        \qquad \resplist{\Delta_2}{\reactlist{1}}
        \justifies
        \zsyseq{\Gamma}{\Delta_1, \Delta_2, A \rightarrow^{S}_{\reactlist{1}} B}{
          \listplus{\basepluslist{\Delta_2}{\reactlist{1}}}{\reactlist{2}}}{C}
        \using{\rightarrow L}
      \end{prooftree}
    \]

    \[
      \begin{prooftree}
        \[
          \Delta_1 \models_{[]} A
          \justifies
          \Delta_1, \Delta_2, A \rightarrow B \models_{[]} A, \Delta_2, A
          \rightarrow B
        \]
        \,
        \[
          \[
            \resplist{\Delta_2}{\reactlist{1}}
            \justifies
            \Delta_2, A \rightarrow B, A \models_{[]} \Delta_2, B
          \]
          \qquad
          \Delta_2, B \models_{\reactlist{2}} C
          \justifies
          \Delta_2, A \rightarrow B, A \models_{\reactlist{2}} C
        \]
        \justifies
        \Delta_1, \Delta_2, A \rightarrow B \models_{\reactlist{2}} C
      \end{prooftree}
    \]

  \end{enumerate}
\end{proof}

\begin{lemma}
  For all set of axioms $\Gamma$ and contexts $\Delta$, the sequent
  $\zsyseq{\Gamma}{\Delta}{[]}{\Delta^{\otimes}}$ is derivable.
\end{lemma}
\begin{proof}
  Immediate. Just decompose the right-hand side with $\otimes R$ until either an
  atom or an implication is reached. Then, apply the identity principle.
\end{proof}

\begin{lemma}
  If $\zsyseq{\Gamma}{\Delta}{\reactlist{}}{C}$, then
  $\zsyseq{\Gamma}{\Delta^{\otimes}}{\reactlist{}}{C}$.
\end{lemma}
\begin{proof}
  Immediate.
\end{proof}

\begin{theorem}[Completeness]
  If $\Delta \models_{\reactlist{}} \nabla$ using a set of axioms $\Gamma$, then
  $\zsyseq{\Gamma}{\Delta}{\reactlist{}}{\nabla^{\otimes}}$.
\end{theorem}
\begin{proof}
  By induction on the \eznd{} proof.

  \begin{enumerate}
  \item The reflexive case is immediate.
  \item

    \[
      \begin{prooftree}
        A \ \text{axiom}
        \justifies
        \Delta \models_{[]} \Delta, A
      \end{prooftree}
    \]

    Then $A$ must be in $\Gamma$, hence $\Gamma \equiv \Gamma', A$. Therefore

    \[
      \begin{prooftree}
        \[
          \zsyseq{\Gamma', A}{\Delta}{[]}{\Delta^{\otimes}}
          \qquad
          \zsyseq{\Gamma', A}{A}{[]}{A}
          \justifies
          \zsyseq{\Gamma', A}{\Delta, A}{[]}{\Delta^{\otimes} \otimes A}
          \using{\otimes R}
        \]
        \justifies
        \zsyseq{\Gamma', A}{\Delta}{[]}{\Delta^{\otimes} \otimes A}
        \using{\copyrule}
      \end{prooftree}
    \]

  \item

    \[
      \begin{prooftree}
        \Delta, A \models_{\reactlist{}} B
        \justifies
        \Delta, \nabla \models_{[]}
        A \rightarrow_{\reactlist{}}^{\elembases{\Delta}} B, \nabla
      \end{prooftree}
    \]

    Then, by inductive hypothesis

    \[
      \begin{prooftree}
        \[
          \zsyseq{\Gamma}{\Delta, A}{\reactlist{}}{B}
          \justifies
          \zsyseq{\Gamma}{\Delta}{[]}{
            A \rightarrow_{\reactlist{}}^{\elembases{\Delta}} B}
          \using{\rightarrow R}
        \]
        \qquad
        \zsyseq{\Gamma}{\nabla}{[]}{\nabla^{\otimes}}
        \justifies
        \zsyseq{\Gamma}{\Delta, \nabla}{[]}{
          A \rightarrow_{\reactlist{}}^{\elembases{\Delta}} B \otimes
          \nabla^{\otimes}}
        \using{\otimes R}
      \end{prooftree}
    \]

  \item

    \[
      \begin{prooftree}
        \resplist{\Delta}{\reactlist{}}
        \justifies
        \Delta, A \rightarrow_{\reactlist{}}^S B, A
        \models_{\basepluslist{\Delta}{\reactlist{}}} \Delta, B
      \end{prooftree}
    \]

    Then,

    \[
      \begin{prooftree}
        \[\justifies \zsyseq{\Gamma}{A}{[]}{A}\]
        \qquad
        \zsyseq{\Gamma}{\Delta,B}{[]}{\Delta^{\otimes} \otimes B}
        \qquad
        \resplist{\Delta}{\reactlist{}}
        \justifies
        \zsyseq{\Gamma}{\Delta,A \rightarrow_{\reactlist{}} B}{
          \basepluslist{\Delta}{l}
        }{\Delta^{\otimes} \otimes B}
      \end{prooftree}
    \]

  \item $\otimes L$ and $\otimes R$ are straightforward.

  \item Composition

    \[
      \begin{prooftree}
        \Sigma \models_{\reactlist{1}} \Delta
        \qquad
        \Delta \models_{\reactlist{2}} \nabla
        \justifies
        \Sigma \models_{\listplus{\reactlist{1}}{\reactlist{2}}} \nabla
      \end{prooftree}
    \]

    Then

    \[
      \begin{prooftree}
        \zsyseq{\Gamma}{\Sigma}{\reactlist{1}}{\Delta^{\otimes}}
        \qquad
        \[
          \zsyseq{\Gamma}{\Delta}{\reactlist{2}}{\nabla^{\otimes}}
          \justifies
          \zsyseq{\Gamma}{\Delta^{\otimes}}{\reactlist{2}}{\nabla^{\otimes}}
        \]
        \justifies
        \zsyseq{\Gamma}{\Sigma}{\listplus{\reactlist{1}}{\reactlist{2}}}{\nabla^{\otimes}}
        \using{cut}
      \end{prooftree}
    \]

  \end{enumerate}
\end{proof}

\subsection{Remarks}
\label{sec:sequentcalculus-remarks}

In the previous section we defined the logical calculus that we intended to
implement in the theorem prover. The calculus has been defined to allow a
comparison with the original calculus of Zsyntax in \cite{adding-logic} to be as
easy as possible, but such formulation is for many reasons not very suitable for
automatic proof search and machine implementation.

A sequent calculus formulation is instead more convenient to study the
proof-theoretic properties of the logic and the automated deduction techniques
that can be applied. This is the reason why this section gives a sequent
translation of the material of the previous one.

The calculus we just defined will be used as the reference in both the
subsequent proof-theoretic studies and in the implementation. The next step will
be to find an automatic search procedure for such calculus, by adapting some
well-known techniques for linear logic to our case.  Linear logic is known to be
particularly tricky to mechanize in an efficient way, given its multiple sources
of non-determinism. We will therefore draw from \cite{haudhuri-thesis} and
incrementally modify our sequent calculus to allow more efficient proof search
techniques to be employed.

%%% Local Variables:
%%% mode: latex
%%% TeX-master: "../docs"
%%% End:
