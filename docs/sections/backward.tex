\section{Backward sequent calculus}

The objective is to obtain a calculus for Zsyntax that is amenable to efficient proof
search, while being sound and complete with the original formulation. This
basically means that we are going to need some sort of sequent calculus enjoying
properties like the subformula property and cut admissibility.
Given the obvious similarities between a Zsyntax proof and a proof in
intuitionistic linear logic, it seems reasonable to aim for a slightly modified
version of the sequent calculus for intuitionistic linear logic in our search
for such a calculus. The rest of this section does exactly that.

\begin{align*}
  \biocore{A} & \equiv \{A\} \ \text{if } A\ \text{is a bio formula} \\
  \biocore{A \otimes B} & \equiv \biocore{A} \cup \biocore{B} \\
  \biocore{A \deflimp B} & \equiv S
\end{align*}

the biocore of a context is defined in the obvious way:

\[
  \biocore{A_1, A_2, \dots, A_n} \equiv \bigcup_{i = 1}^{n}\biocore{A_i}
\]

\[
  \respects{\Delta}{\ctrlset{}} \equiv \forall A \in \biocore{\Delta}, A \notin \ctrlset{}
\]

From this definition, it follows immediately that

\[
  \forall \ctrlset{}, \respects{\emptyset}{\ctrlset{}}, \qquad
  \forall \Delta, \respects{\Delta}{\emptyset}
\]

where $\emptyset$ represents both the empty list and the empty set by abuse of
notation.

\begin{table}[h]
  \centering
  \bgroup
  \def\arraystretch{2}%  1 is the default, change whatever you need
  \begin{tabular}[h]{| L | L |}
    \hline
    \text{Zsyntax} & \text{Provability relation}\\
    \text{implicit} & \begin{prooftree}\justifies \zsy{\Gamma}{\emptyset}{\Gamma}\end{prooftree}
    \\
    \hline

    \text{implicit} &
                      \begin{prooftree}
                        \zsy{\Gamma}{\ctrlset{1}}{\Delta} \quad
                        \zsy{\Delta}{\ctrlset{2}}{\nabla}
                        \justifies
                        \zsy{\Gamma}{\ctrlset{1} \cup \ctrlset{2}}{\nabla}
                      \end{prooftree} \\

    \hline

    \begin{prooftree}
      \Gamma, \Delta_1 \; : \; \respects{\Gamma}{\ctrlset{i}}
      \justifies
      \Gamma, \Delta_2
      \using{\mathrm{Ax}_i}
    \end{prooftree} &
                      \begin{prooftree}
                        \respects{\Gamma}{\ctrlset{i}}
                        \justifies
                        \zsy{\Gamma, \Delta_1}{\ctrlset{i}}{\Gamma, \Delta_2}
                      \end{prooftree} \\

    \hline

    \begin{prooftree}
      \Gamma, A, B
      \justifies
      \Gamma, A \otimes B
    \end{prooftree} & \zsy{\Gamma, A, B}{\emptyset}{\Gamma, A \otimes B} \\

    \hline

    \begin{prooftree}
      \Gamma, A \otimes B
      \justifies
      \Gamma, A, B
    \end{prooftree} & \zsy{\Gamma, A \otimes B}{\emptyset}{\Gamma, A, B} \\

    \hline

    \begin{prooftree}
      \Gamma, \Delta \qquad
      \[
        \Gamma, A
        \using
        \ctrlset{}
        \leadsto
        B
      \]
      \justifies
      A \limp^{\biocore{\Gamma}}_{\ctrlset{}} B, \Delta
    \end{prooftree}
                   &
                     \begin{prooftree}
                       \zsy{\Gamma, A}{\ctrlset{}}{B}
                       \justifies
                       \zsy{\Gamma, \Delta}{\emptyset}{A \limp^{\biocore{\Gamma}}_{\ctrlset{}} B, \Delta}
                     \end{prooftree} \\

    \hline

    \begin{prooftree}
      \Gamma, A, A \limp^{S}_{\ctrlset{}} B
      \; : \; \respects{\Gamma}{\ctrlset{}}
      \justifies
      \Gamma, B
    \end{prooftree}
                   &
                     \begin{prooftree}
                       \respects{\Gamma}{\ctrlset{}}
                       \justifies
                       \zsy{\Gamma, A, A \limp^{S}_{\ctrlset{}}
                         B}{\ctrlset{}}{\Gamma, B}
                     \end{prooftree} \\
    \hline
  \end{tabular}
  \egroup
  \caption{Caption}
\end{table}

Linear logic inspired calculus... One where if you remove the annotations, you
have a standard linear logic sequent calculus.

So we try to mold our rules into shape by starting from a direct translation of
the zsyntax rules and refining them into more standard form.

In intuitionistic linear logic, the linear context can be thought as the list of
resources, that are modified by the rules. In a sense, this is what the standard
rules of zsyntax do: change a list of resources into another...

\paragraph{Identity}

The trivial reaction is the one that does nothing:

\[
  \begin{prooftree}
    \justifies
    \zsyseq{\Gamma}{A}{\ctrlset{}}{A}
  \end{prooftree}
\]

for any control set $\ctrlset{}$. It can be proved that we can rescrict $A$ to
the atomic case while retaining completeness.

\paragraph{Biological axioms}

The calculus has a built-in way to account for (biological) axioms with the
unrestricted context. To make an axiom $A$ available to a derivation, it
sufficies to add it to the unrestricted context. Formulas in the unrestricted
context can of course be used ad libitum:

\[
  \begin{prooftree}
    \zsyseq{\Gamma, A}{\Delta, A}{\ctrlset{}}{C}
    \justifies
    \zsyseq{\Gamma, A}{\Delta}{\ctrlset{}}{C}
  \end{prooftree}
\]

\paragraph{$\otimes$ elimination}

A direct translation yields a rule that is already in an acceptable form for a
left rule.

\[
  \begin{prooftree}
    \bkwseq{\Gamma}{\Delta, A, B}{C}
    \justifies
    \bkwseq{\Gamma}{\Delta, A \otimes B}{C}
  \end{prooftree}
\]

\paragraph{$\otimes$ introduction}

The $\otimes$ introduction rule of the original calculus is the exact inverse of
the elimination rule, so a direct translation from Zsyntax to sequent calculus
would be the following:

\[
  \begin{prooftree}
    \zsyseq{\Gamma}{\Delta, A \otimes B}{\ctrlset{}}{C}
    \justifies
    \zsyseq{\Gamma}{\Delta, A, B}{\ctrlset{}}{C}
  \end{prooftree}
\]

This, however, is not a good rule as it augments the complexity of the sequent,
and breaks the left-right symmetry. We first notice that the above rule and the
following axiom are equivalent

\[
  \begin{prooftree}
    \justifies
    \zsyseq{\Gamma}{A,B}{\emptyset}{A \otimes B}
  \end{prooftree}
\]

The second comes from the first by identity. The first comes from the second by
cut:

\[
  \begin{prooftree}
    \[ \justifies \zsyseq{\Gamma}{A,B}{\emptyset}{A \otimes B} \]
    \qquad
    \zsyseq{\Gamma}{\Delta, A \otimes B}{\ctrlset{}}{C}
    \justifies
    \zsyseq{\Gamma}{\Delta, A, B}{\ctrlset{}}{C}
  \end{prooftree}
\]

However, we have to do some additional cuts to the axiom above in order to
obtain a rule that allows us to eliminate the cut rule. There are actually two
ways to do it (yielding in essence two different rules), but this is just a
consequence of the fact that Zsyntax is aware of the ordering in which
deductions (i.e., reactions) are performed.

\[
  \begin{prooftree}
    \zsyseq{\Gamma}{\Delta_2}{\ctrlset{2}}{B} \qquad
    \[
      \zsyseq{\Gamma}{\Delta_1}{\ctrlset{1}}{A} \qquad
      \[\justifies \zsyseq{\Gamma}{A,B}{\emptyset}{A \otimes B} \]
      \justifies
      \zsyseq{\Gamma}{\Delta_1, B}{\ctrlset{1}}{A \otimes B}
      \using{\respects{B}{\ctrlset{1}}}
    \]
    \justifies
    \zsyseq{\Gamma}{\Delta_1, \Delta_2}{\ctrlset{1} \cup \ctrlset{2}}{A \otimes B}
    \using{\respects{\Delta_1}{\ctrlset{2}}}
  \end{prooftree}
\]

thus yielding the following rule

\[
  \begin{prooftree}
    \zsyseq{\Gamma}{\Delta_1}{\ctrlset{1}}{A} \qquad
    \zsyseq{\Gamma}{\Delta_2}{\ctrlset{2}}{B}
    \justifies
    \zsyseq{\Gamma}{\Delta_1, \Delta_2}{\ctrlset{1} \cup \ctrlset{2}}{A \otimes B}
    \using{\respects{\Delta_1}{\ctrlset{2}} \wedge \respects{B}{\ctrlset{1}}}
  \end{prooftree}
\]

Or


\[
  \begin{prooftree}
    \zsyseq{\Gamma}{\Delta_1}{\ctrlset{1}}{A} \qquad
    \[
      \zsyseq{\Gamma}{\Delta_2}{\ctrlset{2}}{B} \qquad
      \[\justifies \zsyseq{\Gamma}{A,B}{\emptyset}{A \otimes B} \]
      \justifies
      \zsyseq{\Gamma}{\Delta_2, A}{\ctrlset{2}}{A \otimes B}
      \using{\respects{A}{\ctrlset{2}}}
    \]
    \justifies
    \zsyseq{\Gamma}{\Delta_1, \Delta_2}{\ctrlset{1} \cup \ctrlset{2}}{A \otimes B}
    \using{\respects{\Delta_2}{\ctrlset{1}}}
  \end{prooftree}
\]

thus yielding the following rule

\[
  \begin{prooftree}
    \zsyseq{\Gamma}{\Delta_1}{\ctrlset{1}}{A} \qquad
    \zsyseq{\Gamma}{\Delta_2}{\ctrlset{2}}{B}
    \justifies
    \zsyseq{\Gamma}{\Delta_1, \Delta_2}{\ctrlset{1} \cup \ctrlset{2}}{A \otimes B}
    \using{\respects{\Delta_2}{\ctrlset{1}} \wedge \respects{A}{\ctrlset{2}}}
  \end{prooftree}
\]

\paragraph{$\limp$ introduction}

A direct translation to our single-succedent sequent calculus would be the
following:

\[
  \begin{prooftree}
    \zsyseq{\Gamma}{\Delta_1, A}{\ctrlset{1}}{B} \qquad
    \zsyseq{\Gamma}{\Delta_2, A \limp^{\biocore{\Delta_1}}_{\ctrlset{1}}
      B}{\ctrlset{2}}{C}
    \justifies
    \zsyseq{\Gamma}{\Delta_1, \Delta_2}{\ctrlset{2}}{C}
  \end{prooftree}
\]

However, this rule again has the drawback that it increases the complexity of
the sequent backwards, and breaks the left-right symmetry. A better, more
standard rule can be obtained as a special case of the one above:

\[
  \begin{prooftree}
    \zsyseq{\Gamma}{\Delta_1, A}{\ctrlset{1}}{B}
    \qquad
    \zsyseq{\Gamma}{A \limp^{\biocore{\Delta_1}}_{\ctrlset{1}}
      B}{\emptyset}{A \limp^{\biocore{\Delta_1}}_{\ctrlset{1}}
      B}
    \justifies
    \zsyseq{\Gamma}{\Delta_1}{\emptyset}{A \limp^{\biocore{\Delta_1}}_{\ctrlset{1}}
      B}
  \end{prooftree}
\]

thus yielding the following:

\[
  \begin{prooftree}
    \zsyseq{\Gamma}{\Delta, A}{\ctrlset{}}{B}
    \justifies
    \zsyseq{\Gamma}{\Delta}{\emptyset}{A \limp^{\biocore{\Delta}}_{\ctrlset{}}
      B}
  \end{prooftree}
\]

The equivalence is witnessed by the fact that the first rule can be obtained from
the second by cut:

\[
  \begin{prooftree}
    \[
      \zsyseq{\Gamma}{\Delta_1, A}{\ctrlset{1}}{B}
      \justifies
      \zsyseq{\Gamma}{\Delta}{\emptyset}{A \limp^{\biocore{\Delta_1}}_{\ctrlset{1}}
        B}
    \]
    \qquad
    \zsyseq{\Gamma}{\Delta_2, A \limp^{\biocore{\Delta_1}}_{\ctrlset{1}}
      B}{\ctrlset{2}}{C}
    \justifies
    \zsyseq{\Gamma}{\Delta_1, \Delta_2}{\ctrlset{2}}{C}
  \end{prooftree}
\]

\paragraph{$\limp$ elimination}

The direct translation would be the following:

\[
  \begin{prooftree}
    \zsyseq{\Gamma}{\Delta, B}{\ctrlset{}'}{C}
    \justifies
    \zsyseq{\Gamma}{\Delta, A \limp^{\biocore{S}}_{\ctrlset{}}
      B, A}{\ctrlset{} \cup \ctrlset{}'}{C}
    \using{\respects{\Delta}{\ctrlset{}}}
  \end{prooftree}
\]

We can get to a better left rule that has only the implication as principal
formula with cut:


\[
  \begin{prooftree}
    \zsyseq{\Gamma}{\Delta_1}{\emptyset}{A}\qquad
    \[
      \zsyseq{\Gamma}{\Delta_2, B}{\ctrlset{2}}{C}
      \justifies
      \zsyseq{\Gamma}{\Delta_2, A \limp^{\biocore{S}}_{\ctrlset{1}}
        B, A}{\ctrlset{2}}{C}
      \using{\respects{\Delta_2}{\ctrlset{1}}}
    \]
    \justifies
    \zsyseq{\Gamma}{\Delta_1, \Delta_2, A \limp^{\biocore{S}}_{\ctrlset{1}} B}{\ctrlset{1}\cup\ctrlset{2}}{C}
  \end{prooftree}
\]

thus yielding the following rule:

\[
  \begin{prooftree}
    \zsyseq{\Gamma}{\Delta_1}{\emptyset}{A}\qquad
    \zsyseq{\Gamma}{\Delta_2, B}{\ctrlset{2}}{C}
    \justifies
    \zsyseq{\Gamma}{\Delta_1, \Delta_2, A \limp^{\biocore{S}}_{\ctrlset{1}} B}{\ctrlset{1}\cup\ctrlset{2}}{C}
    \using{\respects{\Delta_2}{\ctrlset{1}}}
  \end{prooftree}
\]

\begin{figure}[ht]
  \begin{mdframed}

    \[
      \begin{prooftree}
        \justifies
        \zsyseq{\Gamma}{A}{\ctrlset{}}{A}
      \end{prooftree}
      \qquad \qquad
      \begin{prooftree}
        \zsyseq{\Gamma, A}{\Delta, A}{\ctrlset{}}{C}
        \justifies
        \zsyseq{\Gamma, A}{\Delta}{\ctrlset{}}{C}
      \end{prooftree}
    \]

    \[
      \begin{prooftree}
        \zsyseq{\Gamma}{\Delta_1}{\ctrlset{1}}{A} \qquad
        \zsyseq{\Gamma}{\Delta_2}{\ctrlset{2}}{B}
        \justifies
        \zsyseq{\Gamma}{\Delta_1, \Delta_2}{\ctrlset{1} \cup \ctrlset{2}}{A \otimes B}
        \using{(\respects{\Delta_1}{\ctrlset{2}} \wedge
          \respects{B}{\ctrlset{1}}) \vee
        (\respects{\Delta_2}{\ctrlset{1}} \wedge \respects{A}{\ctrlset{2}})}
      \end{prooftree}
    \]

    \[
      \begin{prooftree}
        \zsyseq{\Gamma}{\Delta, A \otimes B}{\ctrlset{}}{C}
        \justifies
        \zsyseq{\Gamma}{\Delta, A, B}{\ctrlset{}}{C}
      \end{prooftree}
      \qquad \qquad
      \begin{prooftree}
        \zsyseq{\Gamma}{\Delta, A}{\ctrlset{}}{B}
        \justifies
        \zsyseq{\Gamma}{\Delta}{\emptyset}{A \limp^{\biocore{\Delta}}_{\ctrlset{}}
          B}
      \end{prooftree}
    \]

    \[
      \begin{prooftree}
        \zsyseq{\Gamma}{\Delta_1}{\emptyset}{A}\qquad
        \zsyseq{\Gamma}{\Delta_2, B}{\ctrlset{2}}{C}
        \justifies
        \zsyseq{\Gamma}{\Delta_1, \Delta_2, A \limp^{\biocore{S}}_{\ctrlset{1}} B}{\ctrlset{1}\cup\ctrlset{2}}{C}
        \using{\respects{\Delta_2}{\ctrlset{1}}}
      \end{prooftree}
    \]

  \end{mdframed}
  \caption{\label{bkwseqcalc} Annotated backward sequent calculus.}
\end{figure}

%%% Local Variables:
%%% mode: latex
%%% TeX-master: "../docs"
%%% End:
