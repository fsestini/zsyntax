\documentclass{article}

\usepackage{amsmath}
\usepackage{amssymb}
\usepackage{dsfont}
\usepackage{prooftree}
\usepackage{amsthm}
\newtheorem{theorem}{Theorem}
\newtheorem{definition}{Definition}
\newtheorem{proposition}{Proposition}

\def\limp {\mathbin{{-}\mkern-3.5mu{\circ}}}

\begin{document}

\section{The calculus}

We consider a very simple fragment of propositional intuitionistic
linear logic, which comprises the multiplicative connectives
$\otimes, \limp, \textbf{1}$, and the additive connective $\oplus$.

\[
  \begin{prooftree}
    \justifies
    \Gamma; P \Longrightarrow P
    \using{init}
  \end{prooftree}
  \qquad \qquad
  \begin{prooftree}
    \Gamma, A; \Delta, A \Longrightarrow C
    \justifies
    \Gamma, A; \Delta \Longrightarrow C
    \using{copy}
  \end{prooftree}
\]

\[
  \begin{prooftree}
    \Gamma; \Delta \Longrightarrow A
    \qquad
    \Gamma; \Delta' \Longrightarrow B
    \justifies
    \Gamma; \Delta, \Delta' \Longrightarrow A \otimes B
    \using{\otimes R}
  \end{prooftree}
  \qquad \qquad
  \begin{prooftree}
    \Gamma; \Delta, A, B \Longrightarrow C
    \justifies
    \Gamma; \Delta, A \otimes B \Longrightarrow C
    \using{\otimes L}
  \end{prooftree}
\]

\[
  \begin{prooftree}
    \justifies
    \Gamma; \cdot \Longrightarrow \textbf{1}
    \using{\textbf{1} R}
  \end{prooftree}
  \qquad \qquad
  \begin{prooftree}
    \Gamma; \Delta \Longrightarrow C
    \justifies
    \Gamma; \Delta, \textbf{1} \Longrightarrow C
    \using{\textbf{1} L}
  \end{prooftree}
\]

\[
  \begin{prooftree}
    \Gamma; \Delta \Longrightarrow A
    \qquad
    \Gamma; \Delta', B \Longrightarrow C
    \justifies
    \Gamma; \Delta, \Delta', A \limp B \Longrightarrow C
    \using{\limp L}
  \end{prooftree}
  \qquad \qquad
  \begin{prooftree}
    \Gamma; \Delta, A \Longrightarrow B
    \justifies
    \Gamma; \Delta \Longrightarrow A \limp B
    \using{\limp R}
  \end{prooftree}
\]


\[
  \begin{prooftree}
    \Gamma; \Delta, A \Longrightarrow C
    \qquad
    \Gamma; \Delta, B \Longrightarrow C
    \justifies
    \Gamma; \Delta, A \oplus B \Longrightarrow C
    \using{\oplus L}
  \end{prooftree}
  \qquad
  \begin{prooftree}
    \Gamma; \Delta \Longrightarrow A
    \justifies
    \Gamma; \Delta \Longrightarrow A \oplus B
    \using{\oplus R 1}
  \end{prooftree}
  \qquad
  \begin{prooftree}
    \Gamma; \Delta \Longrightarrow B
    \justifies
    \Gamma; \Delta \Longrightarrow A \oplus B
    \using{\oplus R 2}
  \end{prooftree}
\]

\section{Cut elimination}

We first show that restricting the identity axiom on atomic formulas
only is conservative:

\begin{theorem}
  The identity axiom $\Gamma; A \Longrightarrow A$, where $A$ is of
  arbitrary complexity, is admissible.
\end{theorem}
\begin{proof}
  TODO.
\end{proof}

\begin{theorem}[Linear cut elimination]
  If $\Gamma; \Delta \Longrightarrow A$ and $\Gamma; \Delta', A
  \Longrightarrow C$, then $\Gamma; \Delta, \Delta' \Longrightarrow C$.
\end{theorem}
\begin{proof}
  By nested induction on the height of the derivations of the
  premises, and the complexity of the cut formula. We can distinguish
  some cases:

  \begin{enumerate}
  \item One of the premises is an axiom. Then:

    \[
      \begin{prooftree}
        \[\justifies \Gamma; P \Longrightarrow P\] \qquad
        \[\mathcal{E}
        \leadsto
        \Gamma; P \Longrightarrow C \]
        \justifies
        \Gamma; P \Longrightarrow C
      \end{prooftree}
    \]

    then, just take $\mathcal{E}$. The second case is:

    \[
      \begin{prooftree}
        \[\mathcal{D} \leadsto \Gamma; \Delta \Longrightarrow P\]
        \qquad
        \[\justifies \Gamma; P \Longrightarrow P\]
        \justifies
        \Gamma; \Delta \Longrightarrow P
      \end{prooftree}
    \]

    then, just take $\mathcal{D}$.
    
  \item (Principal cuts) The cut formula is introduced by a right rule
    in the left premise, and eliminated by a left rule in the right
    premise... TODO.
  \item (Left-commutative cases) The cut formula is a side formula in
    the left premise. Then, the cut is just routinely moved to the
    premises of the last rule that has been used to derive the left
    premise, and the inductive hypothesis is applied since we act on
    derivations of strictly smaller height.

    \begin{enumerate}
    \item The last rule is $copy$:
      
      \[
        \begin{prooftree}
          \[
            \Gamma, A ; \Delta, A \Longrightarrow B
            \justifies
            \Gamma, A; \Delta \Longrightarrow B
            \using{copy}
          \] \qquad
          \Gamma, A; \Delta', B \Longrightarrow C
          \justifies
          \Gamma, A ; \Delta, \Delta' \Longrightarrow C
        \end{prooftree}
      \]

      Then,

      \[
        \begin{prooftree}
          \[
            \Gamma, A; \Delta, A \Longrightarrow B
            \qquad
            \Gamma, A; \Delta', B \Longrightarrow C
            \justifies
            \Gamma, A; \Delta, \Delta', A \Longrightarrow C
            \using{cut}
          \]
          \justifies
          \Gamma, A ; \Delta, \Delta' \Longrightarrow C
          \using{copy}
        \end{prooftree}
      \]

      
    \item The last rule was...
    \end{enumerate}
    
  \item (Right-commutative cases) The cut formula is a side formula in
    the right premise. Then, the cut is just routinely moved to the
    premises of the last rule that has been used to derive the right
    premise, and the inductive hypothesis is applied since we act on
    derivations of strictly smaller height.

    \begin{enumerate}
    \item The last rule is $copy$:

      \[
        \begin{prooftree}
          \Gamma, A ; \Delta \Longrightarrow C
          \qquad
          \[
            \Gamma, A; \Delta', C, A \Longrightarrow D
            \justifies
            \Gamma, A; \Delta', C \Longrightarrow D
            \using{copy}
          \]
          \justifies
          \Gamma, A; \Delta, \Delta' \Longrightarrow D
          \using{cut}
        \end{prooftree}
      \]

      then

      \[
        \begin{prooftree}
          \[
            \Gamma, A ; \Delta \Longrightarrow C
            \qquad
            \Gamma, A; \Delta', C, A \Longrightarrow D
            \justifies
            \Gamma, A; \Delta, \Delta', A \Longrightarrow D
            \using{cut}
          \]
          \justifies
          \Gamma, A; \Delta, \Delta' \Longrightarrow D
          \using{copy}
        \end{prooftree}
      \]
    \end{enumerate}
  \end{enumerate}
\end{proof}

\begin{theorem}[Persistent cut elimination]
  If $\Gamma; \cdot \Longrightarrow A$ and $\Gamma, A; \Delta
  \Longrightarrow C$, then $\Gamma; \Delta \Longrightarrow C$.
\end{theorem}
\begin{proof}
  By structural induction on the height of the derivations.
  \begin{enumerate}
  \item One of the premises is an initial sequent. Hence, the right
    premise is, and

    \[
      \begin{prooftree}
        \Gamma; \cdot \Longrightarrow A
        \qquad
        \[ \justifies \Gamma, A; P \Longrightarrow P\]
        \justifies
        \Gamma; P \Longrightarrow P
      \end{prooftree}
    \]
    
    but then also $\Gamma; P \Longrightarrow P$ is an identity axiom,
    and can be derived without cut.
    
  \item All other cases are treated as right-commutative cuts, except
    for the case where the last rule in the right premise is
    $copy$. In this case, the cut is

    \[
      \begin{prooftree}
        \Gamma; \cdot \Longrightarrow A
        \qquad
        \[
          \Gamma, A; \Delta, A \Longrightarrow C
          \justifies
          \Gamma, A ; \Delta \Longrightarrow C
          \using{copy}
        \]
        \justifies
        \Gamma; \Delta \Longrightarrow C
        \using{cut!}
      \end{prooftree}
    \]

    But then, by inductive hypothesis and admissibility of linear cut,
    we have

    \[
      \begin{prooftree}
        \Gamma; \cdot \Longrightarrow A
        \qquad
        \[
          \Gamma; \cdot \Longrightarrow A
          \qquad
          \Gamma, A; \Delta, A \Longrightarrow C
          \justifies
          \Gamma; \Delta, A \Longrightarrow C
          \using{cut!}
        \]
        \justifies
        \Gamma; \Delta \Longrightarrow C
        \using{cut}
      \end{prooftree}
    \]

  \end{enumerate}
\end{proof}

\section{Forward sequent calculus}

Since weakening and contraction are admissible for the unrestricted
zone, we can treat $\Gamma$ as a set.

\[
  \begin{prooftree}
    \justifies
    \cdot ; P \longrightarrow P
    \using{init}
  \end{prooftree}
  \qquad \qquad
  \begin{prooftree}
    \Gamma; \Delta, A \longrightarrow C
    \justifies
    \Gamma \cup \{A\}; \Delta \longrightarrow C
    \using{copy}
  \end{prooftree}
\]

\[
  \begin{prooftree}
    \Gamma; \Delta \longrightarrow A
    \qquad
    \Gamma'; \Delta' \longrightarrow B
    \justifies
    \Gamma\cup \Gamma'; \Delta, \Delta' \longrightarrow A \otimes B
  \end{prooftree}
  \qquad \qquad
  \begin{prooftree}
    \Gamma; \Delta, A, B \longrightarrow C
    \justifies
    \Gamma; \Delta, A \otimes B \longrightarrow C
  \end{prooftree}
\]

\[
  \begin{prooftree}
    \justifies
    \cdot; \cdot \longrightarrow \textbf{1}  
  \end{prooftree}
  \qquad \qquad
  \begin{prooftree}
    \Gamma; \Delta \longrightarrow C
    \justifies
    \Gamma; \Delta, \textbf{1} \longrightarrow C
  \end{prooftree}
\]

\[
  \begin{prooftree}
    \Gamma; \Delta \longrightarrow A
    \qquad
    \Gamma'; \Delta', B \longrightarrow C
    \justifies
    \Gamma \cup \Gamma'; \Delta, \Delta', A \limp B \longrightarrow C
    \using{\limp L}
  \end{prooftree}
  \qquad \qquad
  \begin{prooftree}
    \Gamma; \Delta, A \longrightarrow B
    \justifies
    \Gamma; \Delta \longrightarrow A \limp B
    \using{\limp R}
  \end{prooftree}
\]

\begin{definition}
  \begin{enumerate}
  \item A forward sequent is of the form $\Gamma; \Delta
    \longrightarrow C$, where $\Gamma$ and $\Delta$ are the
    unrestricted and linear resources respectively;
  \item The correspondence relation $\prec$ between forward and
    backward sequents is defined as follows:
    $(\Gamma; \Delta \longrightarrow C) \prec
    (\Gamma';\Delta\Longrightarrow C) \; iff \; \Gamma \subseteq
    \Gamma'$. The forward sequent $s$ is sound if for every backward
    sequent $s'$ such that $s \prec s'$, $s'$ is derivable in the
    backward calculus;
  \item The subsumption relation $\leq$ between forward sequents is
    the smallest relation to satisfy:

    \[
      (Gamma; \Delta \longrightarrow C) \leq (\Gamma'; \Delta
      \longrightarrow C)
    \]

    where $\Gamma \subseteq \Gamma'$.
  \end{enumerate}
\end{definition}

Obviously, if $s_1 \leq s_2$ and $s_2 \prec s$, then $s_1 \prec s$.

\begin{definition}
  A rule with conclusion $s$ and premises $s_1, \dots, s_n$ is said to
  satisfy the irredundancy property if for no $i \in \{1, \dots, n\}$,
  $s_i \leq s$.
\end{definition}

\begin{proposition}
  All forward rules satisfy the irredundancy property.
\end{proposition}

\begin{theorem}[Soundness]
  If $\Gamma; \Delta \longrightarrow C$ is derivable, then it is
  sound.
\end{theorem}

\begin{theorem}[Completeness]
  If $\Gamma; \Delta \Longrightarrow C$ is derivable, then there
  exists a derivable forward sequent
  $\Gamma'; \Delta' \longrightarrow C$ such that
  $(\Gamma'; \Delta' \longrightarrow C) \prec (\Gamma; \Delta
  \Longrightarrow C)$.
\end{theorem}

\section{Focused derivations}



\end{document}

