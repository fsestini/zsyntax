\documentclass{article}

\usepackage{amsmath}
\usepackage{amssymb}
\usepackage{dsfont}
\usepackage{prooftree}
\usepackage{amsthm}
\usepackage{mathtools}

\newtheorem{theorem}{Theorem}
\newtheorem{lemma}{Lemma}
\newtheorem{definition}{Definition}
\newtheorem{proposition}{Proposition}
\newtheorem{corollary}{Corollary}

\def\limp {\mathbin{{-}\mkern-3.5mu{\circ}}}
\newcommand{\neuseqsymb}{
  \mathrel{\Longrightarrow\!\!\!\!\!\!\!\!\Longrightarrow}
  %\mathrlap{\,\Longrightarrow}\Longrightarrow
}
\newcommand{\neuseq}[3]{#1 ; #2 \neuseqsymb #3}
\newcommand{\brfrel}[1]{\textsf{foc}^+_{\Uparrow}(#1)}
\newcommand{\blfrel}[1]{\textsf{foc}^-_{\Uparrow}(#1)}
\newcommand{\bactrel}[1]{\textsf{act}_{\Uparrow}(#1)}
\newcommand{\relj}[3]{#1 [#2] \hookrightarrow #3}
\newcommand{\btriseq}[4]{#1; #2; #3 \Longrightarrow #4}
\newcommand{\rfocseq}[3]{#1; #2 \gg #3}
\newcommand{\lfocseq}[4]{#1; #2; #3 \ll #4}

\newcommand{\rinit}{\textsc{rinit}}
\newcommand{\linit}{\textsc{linit}}
\newcommand{\lact}{\textsc{lact}}
\newcommand{\ract}{\textsc{ract}}
\newcommand{\bact}{\textsc{bact}}
\newcommand{\rfoc}{\textsc{rfoc}}
\newcommand{\lfoc}{\textsc{lfoc}}
\newcommand{\matchrule}{\textsc{match}}
\newcommand{\matchprimerule}{\textsc{match'}}
\newcommand{\rightfocusrule}{\textsc{right-focus}}
\newcommand{\leftfocusrule}{\textsc{left-focus}}
\newcommand{\copyfocusrule}{\textsc{copy-focus}}
\newcommand{\rblur}{\textsc{rblur}}
\newcommand{\lblur}{\textsc{lblur}}
\newcommand{\rblurstar}{\textsc{rblur}^*}
\newcommand{\lblurstar}{\textsc{lblur}^*}
\newcommand{\faplus}{\textsc{FA}^+}
\newcommand{\faplusstar}{\textsc{FA}^{+*}}
\newcommand{\faminusstar}{\textsc{FA}^{-*}}
\newcommand{\copyrule}{\textsc{copy}}


\title{A forward focused calculus for Zsyntax}
\author{Filippo Sestini}

\begin{document}

\maketitle

\section{Backward sequent calculus}

\subsection{From natural deduction to sequent calculus}

...



We consider a very simple fragment of propositional intuitionistic
linear logic, which comprises the multiplicative connectives
$\otimes, \limp, \textbf{1}$, and the additive connective $\oplus$.

\[
  \begin{prooftree}
    \justifies
    \Gamma; P \Longrightarrow P
    \using{init}
  \end{prooftree}
  \qquad \qquad
  \begin{prooftree}
    \Gamma, A; \Delta, A \Longrightarrow C
    \justifies
    \Gamma, A; \Delta \Longrightarrow C
    \using{copy}
  \end{prooftree}
\]

\[
  \begin{prooftree}
    \Gamma; \Delta \Longrightarrow A
    \qquad
    \Gamma; \Delta' \Longrightarrow B
    \justifies
    \Gamma; \Delta, \Delta' \Longrightarrow A \otimes B
    \using{\otimes R}
  \end{prooftree}
  \qquad \qquad
  \begin{prooftree}
    \Gamma; \Delta, A, B \Longrightarrow C
    \justifies
    \Gamma; \Delta, A \otimes B \Longrightarrow C
    \using{\otimes L}
  \end{prooftree}
\]

% \[
%   \begin{prooftree}
%     \justifies
%     \Gamma; \cdot \Longrightarrow \textbf{1}
%     \using{\textbf{1} R}
%   \end{prooftree}
%   \qquad \qquad
%   \begin{prooftree}
%     \Gamma; \Delta \Longrightarrow C
%     \justifies
%     \Gamma; \Delta, \textbf{1} \Longrightarrow C
%     \using{\textbf{1} L}
%   \end{prooftree}
% \]

\[
  \begin{prooftree}
    \Gamma; \Delta \Longrightarrow A
    \qquad
    \Gamma; \Delta', B \Longrightarrow C
    \justifies
    \Gamma; \Delta, \Delta', A \limp B \Longrightarrow C
    \using{\limp L}
  \end{prooftree}
  \qquad \qquad
  \begin{prooftree}
    \Gamma; \Delta, A \Longrightarrow B
    \justifies
    \Gamma; \Delta \Longrightarrow A \limp B
    \using{\limp R}
  \end{prooftree}
\]


% \[
%   \begin{prooftree}
%     \Gamma; \Delta, A \Longrightarrow C
%     \qquad
%     \Gamma; \Delta, B \Longrightarrow C
%     \justifies
%     \Gamma; \Delta, A \oplus B \Longrightarrow C
%     \using{\oplus L}
%   \end{prooftree}
%   \qquad
%   \begin{prooftree}
%     \Gamma; \Delta \Longrightarrow A
%     \justifies
%     \Gamma; \Delta \Longrightarrow A \oplus B
%     \using{\oplus R 1}
%   \end{prooftree}
%   \qquad
%   \begin{prooftree}
%     \Gamma; \Delta \Longrightarrow B
%     \justifies
%     \Gamma; \Delta \Longrightarrow A \oplus B
%     \using{\oplus R 2}
%   \end{prooftree}
% \]

\subsection{Cut elimination}

We first show that restricting the identity axiom on atomic formulas
only is conservative:

\begin{theorem}
  The identity axiom $\Gamma; A \Longrightarrow A$, where $A$ is of
  arbitrary complexity, is admissible.
\end{theorem}
\begin{proof}
  TODO.
\end{proof}

\begin{theorem}[Linear cut elimination]
  If $\Gamma; \Delta \Longrightarrow A$ and $\Gamma; \Delta', A
  \Longrightarrow C$, then $\Gamma; \Delta, \Delta' \Longrightarrow C$.
\end{theorem}
\begin{proof}
  By nested induction on the height of the derivations of the
  premises, and the complexity of the cut formula. We can distinguish
  some cases:

  \begin{enumerate}
  \item One of the premises is an axiom. Then:

    \[
      \begin{prooftree}
        \[\justifies \Gamma; P \Longrightarrow P\] \qquad
        \[\mathcal{E}
        \leadsto
        \Gamma; P \Longrightarrow C \]
        \justifies
        \Gamma; P \Longrightarrow C
      \end{prooftree}
    \]

    then, just take $\mathcal{E}$. The second case is:

    \[
      \begin{prooftree}
        \[\mathcal{D} \leadsto \Gamma; \Delta \Longrightarrow P\]
        \qquad
        \[\justifies \Gamma; P \Longrightarrow P\]
        \justifies
        \Gamma; \Delta \Longrightarrow P
      \end{prooftree}
    \]

    then, just take $\mathcal{D}$.
    
  \item (Principal cuts) The cut formula is introduced by a right rule
    in the left premise, and eliminated by a left rule in the right
    premise... TODO.
  \item (Left-commutative cases) The cut formula is a side formula in
    the left premise. Then, the cut is just routinely moved to the
    premises of the last rule that has been used to derive the left
    premise, and the inductive hypothesis is applied since we act on
    derivations of strictly smaller height.

    \begin{enumerate}
    \item The last rule is $copy$:
      
      \[
        \begin{prooftree}
          \[
            \Gamma, A ; \Delta, A \Longrightarrow B
            \justifies
            \Gamma, A; \Delta \Longrightarrow B
            \using{copy}
          \] \qquad
          \Gamma, A; \Delta', B \Longrightarrow C
          \justifies
          \Gamma, A ; \Delta, \Delta' \Longrightarrow C
        \end{prooftree}
      \]

      Then,

      \[
        \begin{prooftree}
          \[
            \Gamma, A; \Delta, A \Longrightarrow B
            \qquad
            \Gamma, A; \Delta', B \Longrightarrow C
            \justifies
            \Gamma, A; \Delta, \Delta', A \Longrightarrow C
            \using{cut}
          \]
          \justifies
          \Gamma, A ; \Delta, \Delta' \Longrightarrow C
          \using{copy}
        \end{prooftree}
      \]

      
    \item The last rule was...
    \end{enumerate}
    
  \item (Right-commutative cases) The cut formula is a side formula in
    the right premise. Then, the cut is just routinely moved to the
    premises of the last rule that has been used to derive the right
    premise, and the inductive hypothesis is applied since we act on
    derivations of strictly smaller height.

    \begin{enumerate}
    \item The last rule is $copy$:

      \[
        \begin{prooftree}
          \Gamma, A ; \Delta \Longrightarrow C
          \qquad
          \[
            \Gamma, A; \Delta', C, A \Longrightarrow D
            \justifies
            \Gamma, A; \Delta', C \Longrightarrow D
            \using{copy}
          \]
          \justifies
          \Gamma, A; \Delta, \Delta' \Longrightarrow D
          \using{cut}
        \end{prooftree}
      \]

      then

      \[
        \begin{prooftree}
          \[
            \Gamma, A ; \Delta \Longrightarrow C
            \qquad
            \Gamma, A; \Delta', C, A \Longrightarrow D
            \justifies
            \Gamma, A; \Delta, \Delta', A \Longrightarrow D
            \using{cut}
          \]
          \justifies
          \Gamma, A; \Delta, \Delta' \Longrightarrow D
          \using{copy}
        \end{prooftree}
      \]
    \end{enumerate}
  \end{enumerate}
\end{proof}

\begin{theorem}[Persistent cut elimination]
  If $\Gamma; \cdot \Longrightarrow A$ and $\Gamma, A; \Delta
  \Longrightarrow C$, then $\Gamma; \Delta \Longrightarrow C$.
\end{theorem}
\begin{proof}
  By structural induction on the height of the derivations.
  \begin{enumerate}
  \item One of the premises is an initial sequent. Hence, the right
    premise is, and

    \[
      \begin{prooftree}
        \Gamma; \cdot \Longrightarrow A
        \qquad
        \[ \justifies \Gamma, A; P \Longrightarrow P\]
        \justifies
        \Gamma; P \Longrightarrow P
      \end{prooftree}
    \]
    
    but then also $\Gamma; P \Longrightarrow P$ is an identity axiom,
    and can be derived without cut.
    
  \item All other cases are treated as right-commutative cuts, except
    for the case where the last rule in the right premise is
    $copy$. In this case, the cut is

    \[
      \begin{prooftree}
        \Gamma; \cdot \Longrightarrow A
        \qquad
        \[
          \Gamma, A; \Delta, A \Longrightarrow C
          \justifies
          \Gamma, A ; \Delta \Longrightarrow C
          \using{copy}
        \]
        \justifies
        \Gamma; \Delta \Longrightarrow C
        \using{cut!}
      \end{prooftree}
    \]

    But then, by inductive hypothesis and admissibility of linear cut,
    we have

    \[
      \begin{prooftree}
        \Gamma; \cdot \Longrightarrow A
        \qquad
        \[
          \Gamma; \cdot \Longrightarrow A
          \qquad
          \Gamma, A; \Delta, A \Longrightarrow C
          \justifies
          \Gamma; \Delta, A \Longrightarrow C
          \using{cut!}
        \]
        \justifies
        \Gamma; \Delta \Longrightarrow C
        \using{cut}
      \end{prooftree}
    \]

  \end{enumerate}
\end{proof}

\section{Forward sequent calculus}

Since weakening and contraction are admissible for the unrestricted
zone, we can treat $\Gamma$ as a set.

\[
  \begin{prooftree}
    \justifies
    \cdot ; P \longrightarrow P
    \using{init}
  \end{prooftree}
  \qquad \qquad
  \begin{prooftree}
    \Gamma; \Delta, A \longrightarrow C
    \justifies
    \Gamma \cup \{A\}; \Delta \longrightarrow C
    \using{copy}
  \end{prooftree}
\]

\[
  \begin{prooftree}
    \Gamma; \Delta \longrightarrow A
    \qquad
    \Gamma'; \Delta' \longrightarrow B
    \justifies
    \Gamma\cup \Gamma'; \Delta, \Delta' \longrightarrow A \otimes B
  \end{prooftree}
  \qquad \qquad
  \begin{prooftree}
    \Gamma; \Delta, A, B \longrightarrow C
    \justifies
    \Gamma; \Delta, A \otimes B \longrightarrow C
  \end{prooftree}
\]

% \[
%   \begin{prooftree}
%     \justifies
%     \cdot; \cdot \longrightarrow \textbf{1}  
%   \end{prooftree}
%   \qquad \qquad
%   \begin{prooftree}
%     \Gamma; \Delta \longrightarrow C
%     \justifies
%     \Gamma; \Delta, \textbf{1} \longrightarrow C
%   \end{prooftree}
% \]

\[
  \begin{prooftree}
    \Gamma; \Delta \longrightarrow A
    \qquad
    \Gamma'; \Delta', B \longrightarrow C
    \justifies
    \Gamma \cup \Gamma'; \Delta, \Delta', A \limp B \longrightarrow C
    \using{\limp L}
  \end{prooftree}
  \qquad \qquad
  \begin{prooftree}
    \Gamma; \Delta, A \longrightarrow B
    \justifies
    \Gamma; \Delta \longrightarrow A \limp B
    \using{\limp R}
  \end{prooftree}
\]

\begin{definition}
  \begin{enumerate}
  \item A forward sequent is of the form $\Gamma; \Delta
    \longrightarrow C$, where $\Gamma$ and $\Delta$ are the
    unrestricted and linear resources respectively;
  \item The correspondence relation $\prec$ between forward and
    backward sequents is defined as follows:
    $(\Gamma; \Delta \longrightarrow C) \prec
    (\Gamma';\Delta\Longrightarrow C) \; \text{iff} \; \Gamma \subseteq
    \Gamma'$. The forward sequent $s$ is sound if for every backward
    sequent $s'$ such that $s \prec s'$, $s'$ is derivable in the
    backward calculus;
  \item The symbol $\prec$ is overloaded to represent the \emph{subsumption}
    relation between forward sequents, as the smallest relation to satisfy:

    \[
      (\Gamma; \Delta \longrightarrow C) \prec (\Gamma'; \Delta
      \longrightarrow C) \; \text{iff} \; \Gamma \subseteq \Gamma'
    \]
  \end{enumerate}
\end{definition}

\begin{definition}
  A rule with conclusion $s$ and premises $s_1, \dots, s_n$ is said to
  satisfy the irredundancy property if for no $i \in \{1, \dots, n\}$,
  $s_i \leq s$.
\end{definition}

\begin{proposition}
  All forward rules satisfy the irredundancy property.
\end{proposition}

\begin{theorem}[Soundness]
  If $\Gamma; \Delta \longrightarrow C$ is derivable, then it is
  sound.
\end{theorem}

\begin{theorem}[Completeness]
  If $\Gamma; \Delta \Longrightarrow C$ is derivable, then there
  exists a derivable forward sequent
  $\Gamma'; \Delta' \longrightarrow C$ such that
  $(\Gamma'; \Delta' \longrightarrow C) \prec (\Gamma; \Delta
  \Longrightarrow C)$.
\end{theorem}

\section{The Inverse Method}

...

\subsection{Subformula property}

...


\begin{definition}
  A decorated sequent is of the form $\Gamma^-_! ; \Delta^-_. \Longrightarrow
  C^+_.$ .
\end{definition}

\begin{definition}[Decorated subformula relation]
  The decorated subformula relation $\leq$ between decorated propositions is the
  reflexive-transitive closure of the following cases:

  \begin{alignat*}{2}
    & A^{\pm}_{.} \leq (A \otimes B)_a^{\pm} \qquad & B^{\pm}_{.} \leq (A \otimes
    B)_a^{\pm} \\
    & A^{\mp}_{.} \leq (A \limp B)_a^{\pm} & B^{\pm}_{.} \leq (A \limp
    B)_a^{\pm}
  \end{alignat*}
  \[
    A^{\pm}_{.} \leq A^{\pm}_{!}
  \]
\end{definition}

\begin{definition}
  A decorated sequent $s_1 \equiv \Gamma^-_! ; \Delta^-_. \Longrightarrow C^+_.$
  is a subsequent of the decorated sequent
  $\Gamma'^-_! ; \Delta'^-_. \Longrightarrow C'^+_.$, written $s_1 \leq s_2$ if

  \[
    \Gamma'^-_! \cup \Delta'^-_. \cup C'^+_. \leq \Gamma^-_! \cup
    \Delta^-_. \cup C^+_.
  \]
\end{definition}

\begin{theorem}[Subformula property]
  If $\Gamma'; \Delta' \Longrightarrow C'$ appears in a proof of
  $\Gamma; \Delta \Longrightarrow C$, then

  \[
    \Gamma'^-_! \cup \Delta'^-_. \cup C'^+_. \leq \Gamma^-_! \cup
    \Delta^-_. \cup C^+_.
  \]
\end{theorem}
\begin{proof}
  By straightforward inspection of the rules of the backward calculus. It
  sufficies to observe that, if the following is such a rule:

  \[
    \begin{prooftree}
      s_1 \quad s_2 \quad \dots \quad s_n
      \justifies
      s
    \end{prooftree}
  \]

  then $s_i \leq s$ for all $i \in \{1, \dots, n\}$. The thesis follows by
  transitivity of the subformula relation.
\end{proof}

Since sequents in the forward calculus contain a subset of formulas in the
backward calculus, it follows that

\begin{corollary}
  If $\Gamma'; \Delta' \longrightarrow C'$ appears in a proof of
  $\Gamma; \Delta \longrightarrow C$, then

  \[
    \Gamma'^-_! \cup \Delta'^-_. \cup C'^+_. \leq \Gamma^-_! \cup
    \Delta^-_. \cup C^+_.
  \]
\end{corollary}

\subsection{Sequent representation}

\begin{definition}
  A forward sequent is represented as follows:

  \[
    u_1 \# A_1, \dots, u_m \# A_n ; l_1^{k_1} \# B_1, \dots, l_n^{k_n} \# B_k
    \longrightarrow r \# C
  \]
\end{definition}

...

In the saturation-based search that we use in the forward direction, there is a
form of non-determinism in selecting sequents for applying rules. It is
therefore important that the database of sequents available as candidates is as
less redundant as possible. We therefore need to check for sequent subsumption
whenever a new sequent is created (\emph{forward subsumption}. In implementing
subsumption checks, it is important to detect failures as early as possible,
because the vast majority of checks it likely to fail. The usual strategy is to
perform a sequence of hierarchical tests that imply subsumption if they all
succeed. Performing these checks in sequence allows us to stop as soon as one of
them fails.

\begin{definition}[Hierarchical subsumption tests]
  A decorated forward sequent $s_1 \equiv \Gamma; \Delta \longrightarrow C$ does
  not subsume $s_2 \equiv \Gamma'; \Delta' \longrightarrow C'$, if

  \begin{enumerate}
  \item $C \neq C'$, or
  \item $\Delta \not \subseteq \Delta'$, or
  \item $\Gamma \not \subseteq \Gamma'$, or
  \item $s_1 \not \prec s_2$.
  \end{enumerate}

  Where $\Gamma \subseteq \Gamma'$ if for every $l \in \mathrm{dom}(\Gamma)$,
  $\mathrm{mult}(\Gamma, l) \leq \mathrm{mult}(\Gamma', l)$, and similarly for
  $\Delta$.
\end{definition}

Add Theorem 4.12 of the thesis...

\section{Focused derivations}

A backward focused proof has two phases. In the active phase all possible rules
are applied in an arbitrary order to asynchronous propositions. When only
synchronous propositions remain, one proposition is selected and a \emph{focused
  phase} for that proposition begins.

As out backward linear sequent calculus is two-sided, we have left- and right-
synchronous and asynchronous connectives.

\begin{table}[h]
  \centering
  \begin{tabular}{|l|l|}
    \hline
    \textbf{symbol} & \textbf{meaning} \\
    $P$ & left-synchronous ($\limp$) \\
    $Q$ & right-synchronous ($\otimes$) \\
    $L$ & left-asynchronous ($\otimes$) \\
    $R$ & right-asynchronous ($\limp$)
  \end{tabular}
\end{table}

The above table does not include the atomic propositions. For reasons explained
in [thesis], here we are forced to treat them as synchronous. However, we can
differenciate the atoms by means of a \emph{focusing bias}, which indicates
whether the atomic proposition under focus must immediately be derived in an
initial sequent.

The backward focusing calculus consists of the following kinds of sequents:

\begin{table}[h]
  \centering
  \begin{tabular}{ll}
    \hline
    \textbf{symbol} & \textbf{meaning} \\
    $\Gamma; \Delta \gg A$ & right-focal sequent with $A$ under focus \\
    $\Gamma; \Delta; A \ll Q$ & left-focal sequent with $A$ under focus \\
    $\Gamma; \Delta; \Omega \Longrightarrow C; \cdot$ & right-active sequent \\
    $\Gamma; \Delta; \Omega \Longrightarrow \cdot; Q$ & left-active sequent
  \end{tabular}
\end{table}

Here, $\Delta$ contains only left-synchronous propositions, i.e., it is of the
form $P_1, \dots, P_n$. $\Omega$ is an ordered context of propositions which may
be synchronous of asynchronous.

For active sequents the right active propositions are decomposed until they
become right-synchronous, i.e. the sequent is of the form $\Gamma; \Delta;
\Omega \Longrightarrow Q; \cdot$. The right hand side is then changed to $\cdot;
Q$. Similarly, the propositions in $\Omega$ are decomposed except when the
proposition is left-synchronous, in which case it is transferred to $\Delta$.

Eventually, the active sequent is reduced to the form $\Gamma; \Delta; \cdot
\Longrightarrow \cdot \cdot; Q$, which we call \emph{neutral sequents}.
A focusing phase is launched from such a neutral sequent by selecting a
\emph{focusable} proposition and giving it the corresponsing focus.

\begin{definition}[Focusable proposition]
  \begin{enumerate}
  \item A proposition is right-focusable if it is right-synchronous and not a
    right-biased atom;
  \item A proposition is left-focusable if it is left-sunchronous and not a
    left-biased atom.
  \end{enumerate}
\end{definition}

When we are in a neutral sequent, we may copy a proposition out of the
unrestricted context and immediately focus on it, regardless of whether it is
focusable or not. If this proposition is actually left-asynchronous, then we
will immediately remove focus on it and transition to an active phase.

There are two forms of the initial sequent, corresponding to the two focusing
biases. If the focal proposition becomes atomic, we terminate with one of the
two initial forms. If the focal proposition is asynchronous, we blur the focus.
If the focal proposition is atomic and of the wrong bias, then also we blur the
focus, but in this case we transition directly to the neutral sequent instead of
entering the active phase.

Decomposing focal propositions uses non-invertible rules for that proposition,
and focus is maintained to the operands of the top-level connective of the
proposition.


\[
  \begin{prooftree}
    p \; \text{left-biased}
    \justifies
    \Gamma; p \gg p
    \using{rinit}
  \end{prooftree}
  \qquad \qquad
  \begin{prooftree}
    p \; \text{right-biased}
    \justifies
    \Gamma; \cdot; p \ll p
    \using{linit}
  \end{prooftree}
\]

\[
  \begin{prooftree}
    \Gamma; \Delta; \Omega \Longrightarrow \cdot; Q
    \justifies
    \Gamma; \Delta; \Omega \Longrightarrow Q; \cdot
    \using{ract}
  \end{prooftree}
  \qquad \qquad
  \begin{prooftree}
    \Gamma; \Delta, P; \Omega \cdot \Omega' \Longrightarrow \gamma
    \justifies
    \Gamma; \Delta; \Omega \cdot P \cdot \Omega' \Longrightarrow \gamma
    \using{lact}
  \end{prooftree}
\]

\[
  \begin{prooftree}
    \Gamma; \Delta \gg Q \qquad Q\; \text{right-focusable}
    \justifies
    \Gamma; \Delta; \cdot \Longrightarrow \cdot ; Q
    \using{rfoc}
  \end{prooftree}
  \qquad \qquad
  \begin{prooftree}
    \Gamma; \Delta; P \ll Q \qquad P\; \text{left-focusable}
    \justifies
    \Gamma; \Delta, P; \cdot \Longrightarrow \cdot; Q
    \using{lfoc}
  \end{prooftree}
\]

\[
  \begin{prooftree}
    \Gamma; \Delta; \cdot \Longrightarrow R; \cdot
    \justifies
    \Gamma; \Delta \gg R
    \using{rblur}
  \end{prooftree}
  \qquad \qquad
  \begin{prooftree}
    \Gamma; \Delta; L \Longrightarrow \cdot; Q
    \justifies
    \Gamma; \Delta; L \ll Q
    \using{lblur}
  \end{prooftree}
\]

\[
  \begin{prooftree}
    \Gamma; \Delta; \cdot \Longrightarrow \cdot ; p \qquad p \; \text{right-biased}
    \justifies
    \Gamma; \Delta \gg p
    \using{rblur^*}
  \end{prooftree}
  \qquad \qquad
  \begin{prooftree}
    \Gamma; \Delta, p; \cdot \Longrightarrow \cdot; Q \qquad p \; \text{left-biased}
    \justifies
    \Gamma; \Delta; p \ll Q
    \using{lblur^*}
  \end{prooftree}
\]

\[
  \begin{prooftree}
    \Gamma, A; \Delta ; A \ll Q
    \justifies
    \Gamma, A; \Delta; \cdot \Longrightarrow \cdot; Q
    \using{copy}
  \end{prooftree}
\]

\[
  \begin{prooftree}
    \Gamma; \Delta; \Omega \cdot A \cdot B \cdot \Omega' \Longrightarrow \gamma
    \justifies
    \Gamma; \Delta; \Omega \cdot A \otimes B \cdot \Omega' \Longrightarrow
    \gamma
    \using{\otimes L}
  \end{prooftree}
  \qquad \qquad
  \begin{prooftree}
    \Gamma; \Delta_1 \gg A \qquad \Gamma; \Delta_2 \gg B
    \justifies
    \Gamma; \Delta_1, \Delta_2 \gg A \otimes B
    \using{\otimes R}
  \end{prooftree}
\]

\[
  \begin{prooftree}
    \Gamma; \Delta_1; B \ll Q \qquad \Gamma; \Delta_2 \gg A
    \justifies
    \Gamma; \Delta_1, \Delta_2 ; A \limp B \ll Q
    \using{\limp L}
  \end{prooftree}
  \qquad \qquad
  \begin{prooftree}
    \Gamma; \Delta; \Omega \cdot A \Longrightarrow B; \cdot
    \justifies
    \Gamma; \Delta; \Omega \Longrightarrow A \limp B; \cdot
    \using{\limp R}
  \end{prooftree}
\]

\subsection{Backward derived rules}

The primary benefit of focusing is the ability to generate derived ``big step''
inference rules: the intermediate results of a focusing or active phase are not
important, since those steps are ``forced'' in some way. Each derived rule
starts (at the bottom) with a neutral sequent from which a synchronous
proposition is selected for focus, and the focusing steps are followed. Then the
active rules are applied, and eventually we obtain a collection of neutral
sequents as the leaves. These neutral sequents are then treated as the premises
of the derived rule that produces the neutral sequent with which we started.

We first construct the backward derived rules. Then we will move to their
forward version. The general design is that intermediate sequents in the eager
active and focusing phases are not be stored in any sequent database; instead,
all sequents constructed during search are neutral sequents at the phase
boundaries. This is achieved by first precomputing the derived rules that
correspond to the frontier literals of the goal sequent.

For any given proposition, we are interested in constructing a derived inference
for the proposition corresponding to a single pair of focusing and inverse
phases.

The idea is to interpret a proposition itself as the derived rules that it
embodies. Every propostiion is viewed as a relation between the conclusion of
the rule and its premises at the leaves of the bipole. Both the conclusion and
the premises are neutral sequents, which we indicate as
$\neuseq{\Gamma}{\Delta}{Q}$.

There are three classes of relational interpretations:

\begin{enumerate}
\item Right focal relations for the focus formula $A$, written $\brfrel{A}$;
\item Left focal relations for the focus formula $A$, written $\blfrel{A}$;
\item Active relations, written
  $\bactrel{\btriseq{\Gamma}{\Delta}{\Omega}{\xi}}$, where $\xi$ is either
  $\cdot$ or a proposition $C$.
\end{enumerate}

\[
  \begin{prooftree}
    p \; \text{right-biased}
    \justifies
    \relj{\blfrel{p}}{\neuseq{\Gamma}{\cdot}{p}}{\cdot}
    \using{linit}
  \end{prooftree}\qquad\qquad
  \begin{prooftree}
    p \; \text{left-biased}
    \justifies
    \relj{\brfrel{p}}{\neuseq{\Gamma}{p}{\cdot}}{\cdot}
    \using{rinit}
  \end{prooftree}
\]

\[
  \begin{prooftree}
    \relj{\brfrel{A}}{\neuseq{\Gamma}{\Delta_1}{\cdot}}{\Sigma_1}
    \qquad
    \relj{\brfrel{B}}{\neuseq{\Gamma}{\Delta_2}{\cdot}}{\Sigma_2}
    \justifies
    \relj{\brfrel{A \otimes B}}{\neuseq{\Gamma}{\Delta_1,
        \Delta_2}{\cdot}}{\Sigma_1 \cdot \Sigma_2}
    \using{\otimes F}
  \end{prooftree}
\]

\[
  \begin{prooftree}
    \relj{\bactrel{\btriseq{\cdot}{\cdot}{\cdot}{R}}}{s}{\Sigma}
    \justifies
    \relj{\brfrel{R}}{s}{\Sigma}
    \using{FA^+}
  \end{prooftree}
  \qquad \qquad
  \begin{prooftree}
    p \; \text{right-biased}
    \justifies
    \relj{\brfrel{p}}{\neuseq{\Gamma}{\Delta}{\cdot}}{\neuseq{\Gamma}{\Delta}{p}}
    \using{conjecture^+}
  \end{prooftree}
\]

\[
  \begin{prooftree}
    \relj{\blfrel{B}}{\neuseq{\Gamma}{\Delta_1}{Q}}{\Sigma_1} \qquad
    \relj{\brfrel{A}}{\neuseq{\Gamma}{\Delta_2}{\cdot}}{\Sigma_2}
    \justifies
    \relj{\blfrel{A \limp B}}{\neuseq{\Gamma}{\Delta_1,\Delta_2}{Q}}{\Sigma_1
      \cdot \Sigma_2}
    \using{\limp F}
  \end{prooftree}
\]

\[
  \begin{prooftree}
    \relj{\bactrel{\btriseq{\cdot}{\cdot}{L}{\cdot}}}{s}{\Sigma}
    \justifies
    \relj{\blfrel{L}}{s}{\Sigma}
    \using{FA^-}
  \end{prooftree}
  \qquad \qquad
  \begin{prooftree}
    p \; \text{left-biased}
    \justifies
    \relj{\blfrel{p}}{\neuseq{\Gamma}{\Delta}{Q}}{\neuseq{\Gamma}{\Delta, p}{Q}}
    \using{conjecture^-}
  \end{prooftree}
\]

\[
  \begin{prooftree}
    \relj{\bactrel{\btriseq{\Gamma}{\Delta}{\Omega \cdot A \cdot B \cdot
          \Omega'}{\xi}}}{s}{\Sigma}
    \justifies
    \relj{\bactrel{\btriseq{\Gamma}{\Delta}{\Omega \cdot A \otimes B \cdot
          \Omega'}{\xi}}}{s}{\Sigma}
    \using{\otimes A}
  \end{prooftree}\qquad \qquad
  \begin{prooftree}
    \relj{\bactrel{\btriseq{\Gamma}{\Delta}{\Omega \cdot A \cdot
          \Omega'}{B}}}{s}{\Sigma}
    \justifies
    \relj{\bactrel{\btriseq{\Gamma}{\Delta}{\Omega \cdot
          \Omega'}{A \limp B}}}{s}{\Sigma}
    \using{\limp A}
  \end{prooftree}
\]

\[
  \begin{prooftree}
    \relj{
      \bactrel{
        \btriseq{\Gamma}{\Delta,P}{\Omega \cdot \Omega'}{\xi}
      }
    }{s}{\Sigma}
    \justifies
    \relj{\bactrel{\btriseq{\Gamma}{\Delta}{\Omega \cdot P \cdot
          \Omega'}{\xi}}}{s}{\Sigma}
    \using{bact}
  \end{prooftree}
\]

\[
  \begin{prooftree}
    \justifies
    \relj{
      \bactrel{\btriseq{\Gamma}{\Delta}{\cdot}{\cdot}}
    }{
      \neuseq{\Gamma'}{\Delta'}{Q}
    }{
      \neuseq{\Gamma, \Gamma'}{\Delta, \Delta'}{Q}
    }
    \using{match}
  \end{prooftree}
\]

\[
  \begin{prooftree}
    \justifies
    \relj{
      \bactrel{\btriseq{\Gamma}{\Delta}{\cdot}{Q}}
    }{
      \neuseq{\Gamma'}{\Delta'}{\cdot}
    }{
      \neuseq{\Gamma, \Gamma'}{\Delta, \Delta'}{Q}
    }
    \using{match'}
  \end{prooftree}
\]

\[
  \begin{prooftree}
    (\relj{\brfrel{Q}}{\neuseq{\Gamma}{\Delta}{\cdot}}{s_1 \cdot s_2 \dots s_n})
    \quad s_1 \quad s_2 \quad \dots \quad s_n
    \justifies
    \neuseq{\Gamma}{\Delta}{Q}
    \using{right-focus}
  \end{prooftree}
\]

\[
  \begin{prooftree}
    (\relj{\blfrel{Q}}{\neuseq{\Gamma}{\Delta}{Q}}{s_1 \cdot s_2 \dots s_n})
    \quad s_1 \quad s_2 \quad \dots \quad s_n
    \justifies
    \neuseq{\Gamma}{\Delta, P}{Q}
    \using{left-focus}
  \end{prooftree}
\]

\[
  \begin{prooftree}
    (\relj{\blfrel{A}}{\neuseq{\Gamma, A}{\Delta}{Q}}{s_1 \cdot s_2 \dots s_n})
    \quad s_1 \quad s_2 \quad \dots \quad s_n
    \justifies
    \neuseq{\Gamma, A}{\Delta}{Q}
    \using{copy-focus}
  \end{prooftree}
\]

\begin{lemma}\label{completeness-lemma}
  \begin{enumerate}
  \item If $\rfocseq{\Gamma}{\Delta}{A}$, then for some $\Sigma$
    \begin{enumerate}
    \item $\relj{\brfrel{A}}{\neuseq{\Gamma}{\Delta}{\cdot}}{\Sigma}$, and
    \item $\Sigma$ are all derivable
    \end{enumerate}
  \item ...
  \item ...
  \end{enumerate}
\end{lemma}
\begin{proof}
  By simultaneous induction on the height of the derivation.
  \begin{enumerate}
  \item Case $\rinit$:

    \[
      \begin{prooftree}
        p \; \text{left-biased}
        \justifies
        \rfocseq{\Gamma}{p}{p}
        \using{\rinit}
      \end{prooftree}
    \]

    then, just apply the $\rinit$ relation.
  \item Case $\linit$ is similar to the case above.
  \item Case $\ract$ is trivial, since the relations corresponding to the
    premise and conclusion sequents are identical.
  \item Case $\lact$. Straightforward use of the $\bact$ rule for derived
    relations.
  \item Case $\rfoc$.
    
    \[
      \begin{prooftree}
        \Gamma; \Delta \gg Q \qquad Q\; \text{right-focusable}
        \justifies
        \Gamma; \Delta; \cdot \Longrightarrow \cdot ; Q
        \using{rfoc}
      \end{prooftree}
    \]

    Suppose $\gamma_1, \gamma_2$ are such that $\gamma_1 \uplus \gamma_2 =
    Q$. Then, the following:

    \[
      \relj{
        \bactrel{\btriseq{\Gamma}{\Delta}{\cdot}{\gamma_1}}
      }{
        \neuseq{\cdot}{\cdot}{\gamma_2}
      }{
        \neuseq{\Gamma}{\Delta}{Q}
      }
    \]

    is derivable with either $\matchrule$ or $\matchprimerule$. We need to show
    that $\neuseq{\Gamma}{\Delta}{Q}$ is derivable. By inductive hypothesis, we
    have $\relj{\brfrel{Q}}{\neuseq{\Gamma}{\Delta}{\cdot}}{\Sigma}$ for some
    $\Sigma$ all derivable. But then,

    \[
      \begin{prooftree}
        \relj{\brfrel{Q}}{\neuseq{\Gamma}{\Delta}{\cdot}}{\Sigma}
        \qquad \Sigma
        \justifies
        \neuseq{\Gamma}{\Delta}{Q}
        \using{\rightfocusrule}
      \end{prooftree}
    \]

  \item Case $\lfoc$ is just analogous to $\rfoc$, with an application of
    $\leftfocusrule$.
  \item Case $\rblur$.

    \[
      \begin{prooftree}
        \Gamma; \Delta; \cdot \Longrightarrow R; \cdot
        \justifies
        \Gamma; \Delta \gg R
        \using{\rblur}
      \end{prooftree}
    \]

    By inductive hypothesis, we have
    $\relj{\bactrel{\btriseq{\cdot}{\cdot}{\cdot}{R}}}{\neuseq{\Gamma}{\Delta}{\cdot}}{\Sigma}$,
    where all $\Sigma$ are derivable. But
    then, we can apply the rule $\faplus$ to get the thesis

    \[
      \begin{prooftree}
        \relj{\bactrel{\btriseq{\cdot}{\cdot}{\cdot}{R}}}{\neuseq{\Gamma}{\Delta}{\cdot}}{\Sigma}
        \justifies
        \relj{\brfrel{R}}{\neuseq{\Gamma}{\Delta}{\cdot}}{\Sigma}
        \using{FA^+}
      \end{prooftree}
    \]
    
  \item Case $\lblur$ is dual to $\rblur$.
  \item Case $\rblurstar$.

    \[
      \begin{prooftree}
        \Gamma; \Delta; \cdot \Longrightarrow \cdot ; p \qquad p \; \text{right-biased}
        \justifies
        \Gamma; \Delta \gg p
        \using{rblur^*}
      \end{prooftree}
    \]

    By inductive hypothesis,
    $\relj{\bactrel{\btriseq{\Gamma}{\Delta}{\cdot}{p}}}{\neuseq{\cdot}{\cdot}{\cdot}}{\Sigma}$,
    that must have been derived by $\matchprimerule$. Therefore, $\Sigma \equiv
    \neuseq{\Gamma}{\Delta}{p}$, which is by hypothesis derivable.
    Then, we can apply $\faplusstar$ to get
    $\relj{\brfrel{p}}{\neuseq{\Gamma}{\Delta}{\cdot}}{\neuseq{\Gamma}{\Delta}{p}}$.

  \item Case $\lblurstar$ is dual to $\rblurstar$.
  \item Case $\copyrule$.

    \[
      \begin{prooftree}
        \Gamma, A; \Delta ; A \ll Q
        \justifies
        \Gamma, A; \Delta; \cdot \Longrightarrow \cdot; Q
        \using{\copyrule}
      \end{prooftree}
    \]

    We can apply $\matchrule$ to immediately derive

    \[
      \begin{prooftree}
        \justifies
        \relj{\bactrel{\btriseq{\Gamma,
              A}{\Delta}{\cdot}{\cdot}}}{\neuseq{\cdot}{\cdot}{Q}}{\neuseq{\Gamma,
            A}{\Delta}{Q}}
        \using{\matchrule}
      \end{prooftree}
    \]

    and see that $\neuseq{\Gamma, A}{\Delta}{Q}$ is derivable by
    the inductive hypothesis
    $\relj{\blfrel{A}}{\neuseq{\Gamma,A}{\Delta}{Q}}{\Sigma}$ and an application
    of $\copyfocusrule$.
  \item The remaining rules of the connectives are straightforward application
    of the inductive hypothesis.
  \end{enumerate}
\end{proof}

\begin{theorem}[Completeness]
  If $\btriseq{\Gamma}{\Delta}{\cdot}{\cdot ; Q}$, then
  $\neuseq{\Gamma}{\Delta}{Q}$.
\end{theorem}
\begin{proof}
  Straightforward application of Lemma~\ref{completeness-lemma}. The last rule
  used to derive $\btriseq{\Gamma}{\Delta}{\cdot}{\cdot ; Q}$ is one of
  $\lfoc, \rfoc$ or $\copyrule$; correspondingly, by the Lemma, we have the
  derived rules to use as premises of $\rightfocusrule, \leftfocusrule$ and
  $\copyfocusrule$ which derive $\neuseq{\Gamma}{\Delta}{Q}$.
\end{proof}

\subsection{Forward focusing}

\end{document}

