\documentclass{article}

\usepackage{amsmath}
\usepackage{amssymb}
\usepackage{dsfont}
\usepackage{prooftree}
\usepackage{amsthm}
\usepackage{mathtools}
\usepackage{mdframed}

\newtheorem{theorem}{Theorem}
\newtheorem{lemma}{Lemma}
\newtheorem{definition}{Definition}
\newtheorem{proposition}{Proposition}
\newtheorem{corollary}{Corollary}
\newtheorem{example}{Example}
\newtheorem{fact}{Fact}

\def\limp {\mathbin{{-}\mkern-3.5mu{\circ}}}

% Standard backward sequents
\newcommand{\bkwseq}[3]{#1 ; #2 \Longrightarrow #3}

% backward and forward neutral sequents symbols and macros
\newcommand{\bneuseqsymb}{
  \mathrel{\Longrightarrow\!\!\!\!\!\!\!\!\Longrightarrow}}
% \newcommand{\fneuseqsymb}{\twoheadrightarrow}
\newcommand{\fneuseqsymb}{
  \mathrel{\longrightarrow\!\!\!\!\!\!\!\!\longrightarrow}}
\newcommand{\bneuseq}[3]{#1 ; #2 \bneuseqsymb #3}
\newcommand{\fneuseq}[3]{#1 ; #2 \fneuseqsymb #3}

% backward derived-rule relations
\newcommand{\brfrel}[1]{\textsf{foc}^+_{\Uparrow}(#1)}
\newcommand{\blfrel}[1]{\textsf{foc}^-_{\Uparrow}(#1)}
\newcommand{\bactrel}[1]{\textsf{act}_{\Uparrow}(#1)}

% forward derived-rule relations
\newcommand{\frfrel}[1]{\textsf{foc}^+_{\Downarrow}(#1)}
\newcommand{\flfrel}[1]{\textsf{foc}^-_{\Downarrow}(#1)}
\newcommand{\factrel}[1]{\textsf{act}_{\Downarrow}(#1)}

% Relation judgement   rel[...] ~--> res
\newcommand{\relj}[3]{#1 [#2] \hookrightarrow #3}

% Three-place bold sequents     G, D, O ===> C
\newcommand{\btriseq}[4]{#1; #2; #3 \Longrightarrow #4}

% Focusing sequents
\newcommand{\rfocseq}[3]{#1; #2 \gg #3}
\newcommand{\lfocseq}[4]{#1; #2; #3 \ll #4}

\newcommand{\init}{\textsc{init}}
\newcommand{\rinit}{\textsc{rinit}}
\newcommand{\linit}{\textsc{linit}}
\newcommand{\lact}{\textsc{lact}}
\newcommand{\ract}{\textsc{ract}}
\newcommand{\bact}{\textsc{bact}}
\newcommand{\rfoc}{\textsc{rfoc}}
\newcommand{\lfoc}{\textsc{lfoc}}
\newcommand{\matchrule}{\textsc{match}}
\newcommand{\matchprimerule}{\textsc{match'}}
\newcommand{\rightfocusrule}{\textsc{right-focus}}
\newcommand{\leftfocusrule}{\textsc{left-focus}}
\newcommand{\copyfocusrule}{\textsc{copy-focus}}
\newcommand{\rblur}{\textsc{rblur}}
\newcommand{\lblur}{\textsc{lblur}}
\newcommand{\rblurstar}{\textsc{rblur}^*}
\newcommand{\lblurstar}{\textsc{lblur}^*}
\newcommand{\faplus}{\textsc{FA}^+}
\newcommand{\faminus}{\textsc{FA}^-}
\newcommand{\faplusstar}{\textsc{FA}^{+*}}
\newcommand{\faminusstar}{\textsc{FA}^{-*}}
\newcommand{\copyrule}{\textsc{copy}}
\newcommand{\ctrlset}[1]{\mathfrak{S}_{#1}}

\newcommand{\respects}[2]{#1 \, || \, #2}

\newcommand{\labels}[2]{#1 \, \# \, #2}

\newcommand{\strong}[1]{\textbf{#1}}

\newcommand{\subsequent}[2]{#1 \leq #2}
\newcommand{\frwdseq}[3]{#1 ; #2 \longrightarrow #3}

% Formula decoration
\newcommand{\decplus}[1]{#1^+}
\newcommand{\decminus}[1]{#1^-}
\newcommand{\declinear}[1]{#1_.}
\newcommand{\decunrestr}[1]{#1_!}
\newcommand{\decpluslinear}[1]{\decplus{\declinear{#1}}}
\newcommand{\decminuslinear}[1]{\decminus{\declinear{#1}}}
\newcommand{\decplusminus}[1]{#1^{\pm}}
\newcommand{\decminusplus}[1]{#1^{\mp}}

\title{A forward focused calculus for Zsyntax}
\author{Filippo Sestini}

\begin{document}

\maketitle
\tableofcontents

\section{Backward sequent calculus}

\subsection{Control sets}

\subsubsection{Annotation}

\begin{definition}
  An annotated sequent is syntactically represented as $\Gamma; \Delta
  \Longrightarrow_{\ctrlset{}} C$, where $\Gamma, \Delta$ are the ordinary
  unrestricted and linear contexts, $C$ is the succedent formula and
  $\ctrlset{}$ is a control set.
\end{definition}

\begin{figure}[ht]
  \begin{mdframed}
    \[
      \begin{prooftree}
        \justifies
        \Gamma; P \Longrightarrow_{\emptyset} P
        \using{init}
      \end{prooftree}
      \qquad \qquad
      \begin{prooftree}
        \Gamma, A; \Delta, A \Longrightarrow_{\ctrlset{}} C
        \justifies
        \Gamma, A; \Delta \Longrightarrow_{\ctrlset{}} C
        \using{copy}
      \end{prooftree}
    \]

    \[
      \begin{prooftree}
        \Gamma; \Delta \Longrightarrow_{\ctrlset{1}} A
        \qquad
        \Gamma; \Delta' \Longrightarrow_{\ctrlset{2}} B
        \justifies
        \Gamma; \Delta, \Delta' \Longrightarrow_{\ctrlset{1}\cup \ctrlset{2}} A \otimes B
        \using{\otimes R}
      \end{prooftree}
      \qquad \qquad
      \begin{prooftree}
        \Gamma; \Delta, A, B \Longrightarrow_{\ctrlset{}} C
        \justifies
        \Gamma; \Delta, A \otimes B \Longrightarrow_{\ctrlset{}} C
        \using{\otimes L}
      \end{prooftree}
    \]

    \[
      \begin{prooftree}
        \Gamma; \Delta_1 \Longrightarrow_{\ctrlset{1}} A
        \qquad
        \Gamma; \Delta_2, B \Longrightarrow_{\ctrlset{2}} C
        \justifies
        \Gamma; \Delta_1, \Delta_2, A \limp B
        \Longrightarrow_{\ctrlset{1} \cup \ctrlset{2} \cup \ctrlset{A \rightarrow B}} C
      \end{prooftree}
      \qquad \qquad
      \begin{prooftree}
        \Gamma; \Delta, A \Longrightarrow_{\ctrlset{}} B
        \justifies
        \Gamma; \Delta \Longrightarrow_{\ctrlset{}} A \limp B
        \using{\limp R}
      \end{prooftree}
    \]
  \end{mdframed}
  \caption{\label{myfig} Annotated backward sequent calculus.}
\end{figure}



\subsubsection{Checking}

Consider the usual example of .... More precisely, we have
an axiom $E\otimes S \rightarrow E \odot S$,
with $\ctrlset{E\otimes S \rightarrow E \odot S} = \{I\}$, thus we expect
$E \otimes S\otimes I \limp E \odot S \otimes I$ to \emph{not} be valid in our
checked, annotated sequent calculus. The following is the annotated derivation,
where $\Gamma \equiv E\otimes S \rightarrow E \odot S$:

\[
  \begin{prooftree}
    \[
      \[
        \[ \justifies \Gamma; I \Longrightarrow_{\emptyset} I \using{\init} \]
        \qquad
        \[
          \[
            \[
              \[
                \justifies
                \Gamma; E \Longrightarrow_{\emptyset} E
                \using{\init}
              \]\qquad
              \[
                \justifies
                \Gamma; S \Longrightarrow_{\emptyset} S
                \using{\init}
              \]
              \justifies
              \Gamma; E , S \Longrightarrow_{\emptyset} E \otimes S
              \using{\otimes R}
            \]
            \qquad
            \[
              \justifies \Gamma; E\odot S \Longrightarrow_{\emptyset} E \odot S
              \using{\init}
            \]
            \justifies
            \Gamma; E\otimes S \limp E\odot S, E , S \Longrightarrow_{\{I\}} E \odot S
            \using{\limp L}
          \]
          \justifies
          \Gamma; E , S \Longrightarrow_{\{I\}} E \odot S
          \using{\copyrule}
        \]
        \justifies
        \Gamma; E , S , I \Longrightarrow_{\{I\}} E \odot S \otimes I
        \using{\otimes R}
      \]
      \justifies
      \Gamma; E \otimes S \otimes I \Longrightarrow_{\{I\}}  E \odot S \otimes I
      \using{\otimes L \times 3}
    \]
    \justifies
    \Gamma; \cdot \Longrightarrow_{\{I\}} E \otimes S \otimes I \limp E \odot S \otimes
   I
   \using{\limp R}
  \end{prooftree}
\]

Therefore, if we ignore the constraints of the control sets, the formula is
actually derivable (since it is a theorem in Linear Logic). A first attempt to
block this may be to impose a side condition to the applicability of the rule
$\otimes R$, restricting it to conclusion sequents in which the context respects
the control sets of both premises:


\[
  \begin{prooftree}
    \Gamma; \Delta_1 \Longrightarrow_{\ctrlset{1}} A
    \qquad
    \Gamma; \Delta_2 \Longrightarrow_{\ctrlset{2}} B
    \justifies
    \Gamma; \Delta_1, \Delta_2 \Longrightarrow_{\ctrlset{1} \cup \ctrlset{2}} A
    \otimes B
  \end{prooftree}
  \quad
  \respects{(\Gamma, \Delta_1, \Delta_2)}{(\ctrlset{1} \cup \ctrlset{2})}
\]

a similar addition can be done to $\limp L$, for the same reason:

\[
  \begin{prooftree}
    \Gamma; \Delta_1 \Longrightarrow_{\ctrlset{1}} A
    \qquad
    \Gamma; \Delta_2, B \Longrightarrow_{\ctrlset{2}} C
    \justifies
    \Gamma; \Delta_1, \Delta_2, A \limp B
    \Longrightarrow_{\ctrlset{1} \cup \ctrlset{2} \cup \ctrlset{A \rightarrow B}} C
  \end{prooftree}
  \quad
  \respects{(\Gamma, \Delta_1, \Delta_2)}{(\ctrlset{1} \cup \ctrlset{2})}
\]

It is immediate to notice that, with these new side conditions, the formula
above is no longer derivable. This modification, however, turns out to be too
restrictive. Suppose we have $\ctrlset{A \rightarrow B} = \{I\}$. The following
is valid in Zsyntax:

\[
  I \rightarrow A, A \rightarrow B, I, A \models A \otimes B
\]

since we have the following derivation:

\[
  \begin{prooftree}
    \[
      \[
        I \rightarrow A, A \rightarrow B, I, A
        \justifies
        A, A, A \rightarrow B
        \using{\rightarrow \mathcal{E}}
      \]
      \justifies
      A, B
      \using{\rightarrow \mathcal{E}}
    \]
    \justifies
    A \otimes B
    \using{\otimes \mathcal{I}}
  \end{prooftree}
\]

Notice that here we are indeed allowed to eliminate $A \rightarrow B$, because
the inhibitor $I$ which may have blocked the reaction has already been
eliminated in a previous stage of the deduction/reaction. An easy check can
verify that, with the side conditions above, the sequent
$\Gamma; \cdot \Longrightarrow I \otimes A \limp A \otimes B$, where $\Gamma$
contains the axioms as usual, cannot be derived.

The reason is that the side conditions, formulated as above, cannot capture the
temporal differences in the reactions. In this example, the reaction starts with
$I$, which is indeed an inhibitor for $A \rightarrow B$, but since the reaction
between $A \rightarrow B$ and $A$ happens at a later stage, when $I$ has
disappeared from the \emph{current state}, its presence in the control set of $A
\rightarrow B$ causes no problem.

In order to generalize this, consider the premises of $\limp L$, that is,
$\Gamma; \Delta_1 \Longrightarrow_{\ctrlset{1}} A$ and
$\Gamma; \Delta_2, B \Longrightarrow_{\ctrlset{2}} C$. These may be interpreted
as having two deductions

\[
  \begin{prooftree}
    A_1, \dots A_n, \Delta_1
    \leadsto
    A
  \end{prooftree}
  \qquad
  \begin{prooftree}
    A_{n+1}, \dots A_{n+m}, \Delta_2, B
    \leadsto
    C
  \end{prooftree}
\]

where $\forall i \in \{1,\dots,n+m\}, A_i \in \Gamma$, and that result from
rules that can be applied under a control set respectively $\ctrlset{1}$ and
$\ctrlset{2}$.  Suppose now that we start from a Z-state
$A_1, \dots A_n, \Delta_1, A_{n+1}, \dots A_{n+m}, \Delta_2, A\rightarrow B$. It
is understood that we can reproduce the first deduction to get

\[
  \begin{prooftree}
    A_1, \dots A_n, \Delta_1, A_{n+1}, \dots A_{n+m}, \Delta_2, A \rightarrow B
    \leadsto
    A, A_{n+1}, \dots A_{n+m}, \Delta_2, B, A \rightarrow B
  \end{prooftree}
\]

as long as all formulas in $A_{n+1}, \dots A_{n+m}, \Delta_2, A \rightarrow B$
respect the constraints imposed by the control set $\ctrlset{1}$. Then, as long
as $A_{n+1}, \dots A_{n+m}, \Delta_2$ respect $\ctrlset{A\rightarrow B}$, we
can eliminate the implication

\[
  \begin{prooftree}
    \[
      A_1, \dots A_n, \Delta_1, A_{n+1}, \dots A_{n+m}, \Delta_2, A \rightarrow B
      \leadsto
      A, A_{n+1}, \dots A_{n+m}, \Delta_2, A \rightarrow B
    \]
    \justifies
    A_{n+1}, \dots A_{n+m}, \Delta_2, B
    \using{\rightarrow \mathcal{E}}
  \end{prooftree}
\]

now we can just reproduce the second derivation, to get


\[
  \begin{prooftree}
    \[
      \[
        A_1, \dots A_n, \Delta_1, A_{n+1}, \dots A_{n+m}, \Delta_2, A \rightarrow B
        \leadsto
        A, A_{n+1}, \dots A_{n+m}, \Delta_2, A \rightarrow B
      \]
      \justifies
      A_{n+1}, \dots A_{n+m}, \Delta_2, B
      \using{\rightarrow \mathcal{E}}
    \]
    \leadsto
    C
  \end{prooftree}
\]

We can relax the side conditions to allow for these kinds of deduction in the
sequent calculus, as follows:

\[
  \begin{prooftree}
    \Gamma; \Delta_1 \Longrightarrow_{\ctrlset{1}} A
    \qquad
    \Gamma; \Delta_2, B \Longrightarrow_{\ctrlset{2}} C
    \justifies
    \Gamma; \Delta_1, \Delta_2, A \limp B
    \Longrightarrow_{\ctrlset{1} \cup \ctrlset{2} \cup \ctrlset{A \rightarrow B}} C
  \end{prooftree}
  \quad \text{provided} \quad
  \begin{cases}
    \respects{(\Gamma, \Delta_2, A \rightarrow B)}{\ctrlset{1}} \\
    \respects{(\Gamma, \Delta_2)}{\ctrlset{A \rightarrow B}}
  \end{cases}
\]

A similar situation arises with the tensor product. Consider the premises of the
rule $\otimes R$, i.e.,
$\Gamma; \Delta_1 \Longrightarrow_{\ctrlset{1}} A$ and
$\Gamma; \Delta_2 \Longrightarrow_{\ctrlset{2}} B$. This means that we have

\[
  \begin{prooftree}
    A_1, \dots A_n, \Delta_1
    \leadsto
    A
  \end{prooftree}
  \qquad
  \begin{prooftree}
    A_{n+1}, \dots A_{n+m}, \Delta_2
    \leadsto
    B
  \end{prooftree}
\]

where $\forall i \in \{1,\dots,n+m\}, A_i \in \Gamma$, and that result from
rules that can be applied under a control set respectively $\ctrlset{1}$ and
$\ctrlset{2}$.  Suppose now that we start from a Z-state
$A_1, \dots A_n, \Delta_1, A_{n+1}, \dots A_{n+m}, \Delta_2$.  Then, if we view
the two deductions above as monolithic, we have essentially three different
ways to derive $A \otimes B$.

\begin{enumerate}
\item We may start by reproducing the first derivation, followed by the second:

  \[
    \begin{prooftree}
      \[
        \[
          A_1, \dots A_n, \Delta_1, A_{n+1}, \dots A_{n+m}, \Delta_2
          \leadsto
          A, A_{n+1}, \dots A_{n+m}, \Delta_2
        \]
        \leadsto
        A, B
      \]
      \justifies
      A \otimes B
    \end{prooftree}
  \]

  provided $A_{n+1}, \dots A_{n+m}, \Delta_2$ respects $\ctrlset{1}$ and
  $A$ respects $\ctrlset{2}$.
  
\item We may start by reproducing the second derivation, followed by the first:

  \[
    \begin{prooftree}
      \[
        \[
          A_1, \dots A_n, \Delta_1, A_{n+1}, \dots A_{n+m}, \Delta_2
          \leadsto
          A_1, \dots A_n, \Delta_1, B
        \]
        \leadsto
        A, B
      \]
      \justifies
      A \otimes B
    \end{prooftree}
  \]

  provided $A_1, \dots A_n, \Delta_1$ respects $\ctrlset{2}$ and
  $B$ respects $\ctrlset{1}$.

\item We may reproduce the two derivations in parallel, as allowed by Zsyntax:

  \[
    \begin{prooftree}
      \[
        A_1, \dots A_n, \Delta_1, A_{n+1}, \dots A_{n+m}, \Delta_2
        \leadsto
        A, B
      \]
      \justifies
      A \otimes B
    \end{prooftree}
  \]

  provided both $\Delta_1$ and $\Delta_2$ respect $\ctrlset{1} \cup \ctrlset{2}$.
\end{enumerate}

We can relax the side conditions to allow for these kinds of deduction in the
sequent calculus, as follows:

\[
  \begin{prooftree}
    \Gamma; \Delta_1 \Longrightarrow_{\ctrlset{1}} A
    \qquad
    \Gamma; \Delta_2 \Longrightarrow_{\ctrlset{2}} B
    \justifies
    \Gamma; \Delta_1, \Delta_2 \Longrightarrow_{\ctrlset{1} \cup \ctrlset{2}} A
    \otimes B
  \end{prooftree}\quad \text{provided} \quad
  \begin{cases}
    ((\Gamma, \Delta_2) || \ctrlset{1} \wedge (\Gamma, A) || \ctrlset{2}) \; \vee \\
    ((\Gamma, \Delta_1) || \ctrlset{2} \wedge (\Gamma, B) || \ctrlset{1}) \; \vee \\
    ((\Gamma, \Delta_1, \Delta_2) || \ctrlset{1} \cup \ctrlset{2})
  \end{cases}
\]

Notice that, with these modifications, it does not hold that if
$\Gamma; \Delta \Longrightarrow_{\ctrlset{}} C$ then
$\respects{(\Gamma, \Delta)}{\ctrlset{}}$.  Nevertheless, this has no influence
on the fact that we must do the union of the control sets of the premises in the
conclusion sequent. In fact, the control set annotation in a sequent tells us
what must be true externally, i.e., in a hypothetical additional context in
which we may want to make the reaction happen, whereas the check condition tells
us what must hold internally, i.e. in the starting context/state
$\Gamma, \Delta$ itself, in order for the sequent to represent a biologically
correct reaction. It follows that it makes sense for the check conditions to be
less restrictive that the ones we may think of (and that we described as our
first attempt), since they may take into account the order in which internal
reactions, represented by the premise sequents, take place. Instead, the control
set that annotates the conclusion sequent must account for all these conditions
regardless of the order in which they occur
internally. Figure~\ref{annotated-checked} shows the complete calculus with the
updated side conditions.

\begin{figure}[ht]
  \begin{mdframed}
    \[
      \begin{prooftree}
        \justifies
        \Gamma; P \Longrightarrow^c_{\emptyset} P
        \using{\init}
      \end{prooftree}
      \qquad \qquad
      \begin{prooftree}
        \Gamma, A; \Delta, A \Longrightarrow^c_{\ctrlset{}} C
        \justifies
        \Gamma, A; \Delta \Longrightarrow^c_{\ctrlset{}} C
        \using{\copyrule}
      \end{prooftree}
    \]

    \[
      \begin{prooftree}
        \Gamma; \Delta, A, B \Longrightarrow^c_{\ctrlset{}} C
        \justifies
        \Gamma; \Delta, A \otimes B \Longrightarrow^c_{\ctrlset{}} C
        \using{\otimes L}
      \end{prooftree}
      \qquad \qquad
      \begin{prooftree}
        \Gamma; \Delta, A \Longrightarrow^c_{\ctrlset{}} B
        \justifies
        \Gamma; \Delta \Longrightarrow^c_{\ctrlset{}} A \limp B
        \using{\limp R}
      \end{prooftree}
    \]

    \[
      \begin{prooftree}
        \Gamma; \Delta_1 \Longrightarrow_{\ctrlset{1}} A
        \qquad
        \Gamma; \Delta_2 \Longrightarrow_{\ctrlset{2}} B
        \justifies
        \Gamma; \Delta_1, \Delta_2 \Longrightarrow_{\ctrlset{1} \cup \ctrlset{2}} A
        \otimes B
        \using{\otimes R}
      \end{prooftree}\quad
      \begin{cases}
        ((\Gamma, \Delta_2) || \ctrlset{1} \wedge (\Gamma, A) || \ctrlset{2}) \; \vee \\
        ((\Gamma, \Delta_1) || \ctrlset{2} \wedge (\Gamma, B) || \ctrlset{1}) \; \vee \\
        ((\Gamma, \Delta_1, \Delta_2) || \ctrlset{1} \cup \ctrlset{2})
      \end{cases}
    \]
    
    \[
      \begin{prooftree}
        \Gamma; \Delta_1 \Longrightarrow_{\ctrlset{1}} A
        \qquad
        \Gamma; \Delta_2, B \Longrightarrow_{\ctrlset{2}} C
        \justifies
        \Gamma; \Delta_1, \Delta_2, A \limp B
        \Longrightarrow_{\ctrlset{1} \cup \ctrlset{2} \cup \ctrlset{A
            \rightarrow B}} C
        \using{\limp L}
      \end{prooftree}
      \quad
      \begin{cases}
        \respects{(\Gamma, \Delta_2, A \rightarrow B)}{\ctrlset{1}} \\
        \respects{(\Gamma, \Delta_2)}{\ctrlset{A \rightarrow B}}
      \end{cases}
    \]
    
  \end{mdframed}
  \caption{\label{annotated-checked} Annotated checked backward sequent calculus.}
\end{figure}

\begin{definition}
  We augment the notation $\Gamma \vdash \Delta$ w.r.t. theorems of Zsyntax with
  the notation $\Gamma \vdash_{\ctrlset{}} \Delta$ for a control set
  $\ctrlset{}$, to mean that it is possible to derive

  \[
    \begin{prooftree}
      \nabla, \Gamma
      \leadsto
      \nabla, \Delta
    \end{prooftree}
  \]

  for every $\nabla$ such that $\respects{\nabla}{\ctrlset{}}$.
\end{definition}

\begin{theorem}[Soundness]
  If $\Gamma; \Delta \Longrightarrow^c_{\ctrlset{}} C$, then
  $A_1, \dots, A_n, \Delta \vdash_{\ctrlset{}} C$ where $A_i \in \Gamma$ for
  all $i \in \{ 1, \dots, n\}$.
\end{theorem}
\begin{proof}
  By induction on the height of the derivation.

  \begin{enumerate}
  \item The last rule is $\init$. Then, the sequent is $\Gamma; P
    \Longrightarrow_{\emptyset} P$, and $P \models_{\emptyset} P$ by reflexivity
    of the $\models$ relation.

  \item The last rule is $\copyrule$. Then, by inductive hypothesis, there is a
    derivation $\mathcal{D}$ such that

    \[
      \begin{prooftree}
        A_1, \dots, A_n, \Delta, A
        \leadsto
        C
      \end{prooftree}
    \]

    where $A \in \Gamma$ and $A_i \in \Gamma$ for all $i \in \{ 1, \dots, n\}$.
    But then $\mathcal{D}$ satisfies the thesis.

  \item The last rule is $\otimes L$. Then, by inductive hypothesis, we have a
    derivation
    
    \[
      \begin{prooftree}
        A_1, \dots, A_n, \Delta, A, B
        \leadsto
        C
      \end{prooftree}
    \]

    where $A_i \in \Gamma$ for all $i \in \{ 1, \dots, n\}$.
    Then, by $\otimes \mathcal{E}$, we get

    \[
      \begin{prooftree}
        \[
          A_1, \dots, A_n, \Delta, A \otimes B
          \justifies
          A_1, \dots, A_n, \Delta, A, B
          \using{\otimes \mathcal{E}}
        \]
        \leadsto
        C
      \end{prooftree}
    \]
    
    since $\otimes \mathcal{E}$ does not introduce any additional constraints,
    we can conclude the thesis, i.e., $A_1, \dots, A_n, \Delta, A \otimes B
    \models_{\ctrlset{}} C$.

  \item The last rule is $\limp R$. Then, by inductive hypothesis, we have a
    derivation

    \[
      \begin{prooftree}
        A_1, \dots, A_n, \Delta, A
        \leadsto
        B
      \end{prooftree}
    \]

    where $A_i \in \Gamma$ for all $i \in \{ 1, \dots, n\}$.
    Then, by $\rightarrow \mathcal{I}$, we may deduce $A \rightarrow B$ from
    $A_1, \dots, A_n, \Delta$, without adding additional constraints to those of
    the above derivation. Hence, $A_1, \dots, A_n, \Delta \models_{\ctrlset{}} A
    \rightarrow B$.

  \item The last rule is $\otimes R$. Then, by inductive hypothesis

    
    \[
      \begin{prooftree}
        A_1, \dots, A_n, \Delta_1
        \leadsto
        A
      \end{prooftree}
      \qquad \qquad
      \begin{prooftree}
        A_{n+1}, \dots, A_{n+m}, \Delta_2
        \leadsto
        B
      \end{prooftree}
    \]

    where $A_i \in \Gamma$ for all $i \in \{ 1, \dots, n+m\}$.
    We now distinguish the three cases in which the check condition can be
    satisfied:

    \begin{enumerate}
    \item Case
      $\respects{\Gamma, \Delta_2}{\ctrlset{1}} \wedge \respects{\Gamma,
        A}{\ctrlset{2}}$. Then, we can derive

      \[
        \begin{prooftree}
          \[
            \[
              A_1, \dots, A_n, \Delta_1, A_{n+1}, \dots, A_{n+m}, \Delta_2
              \leadsto
              A, A_{n+1}, \dots, A_{n+m}, \Delta_2
            \]
            \leadsto
            A, B
          \]
          \justifies
          A \otimes B
        \end{prooftree}
      \]

      which holds for any additional context as long as it respects the control
      set involved in the two intermediate derivations, namely $\ctrlset{1},
      \ctrlset{2}$. Hence,
      $A_1, \dots, A_n, \Delta_1, A_{n+1}, \dots, A_{n+m},
      \Delta_2 \models_{\ctrlset{1} \cup \ctrlset{2}} A \otimes B$.
      
    \item Case
      $\respects{\Gamma, \Delta_1}{\ctrlset{2}} \wedge \respects{\Gamma,
        B}{\ctrlset{1}}$. Then, we can derive

      \[
        \begin{prooftree}
          \[
            \[
              A_1, \dots, A_n, \Delta_1, A_{n+1}, \dots, A_{n+m}, \Delta_2
              \leadsto
              A_1, \dots, A_n, \Delta_1, B
            \]
            \leadsto
            A, B
          \]
          \justifies
          A \otimes B
        \end{prooftree}
      \]

      As above, from this we may conclude

      \[
        A_1, \dots, A_n, \Delta_1, A_{n+1}, \dots, A_{n+m},
        \Delta_2 \vdash_{\ctrlset{1} \cup \ctrlset{2}} A \otimes B
      \]

    \item Case
      $\respects{(\Gamma, \Delta_1,
        \Delta_2)}{(\ctrlset{1}\cup\ctrlset{2})}$. Since Zsyntax allows for
      simultaneous application of inference rules, given the check condition we
      can derive

      \[
        \begin{prooftree}
          \[
            A_1, \dots, A_n, \Delta_1, A_{n+1}, \dots, A_{n+m}, \Delta_2
            \leadsto
            A, B
          \]
          \justifies
          A \otimes B
        \end{prooftree}
      \]
    \end{enumerate}
    
  \item The last rule is $\limp L$. Then, by inductive hypothesis, we have

        
    \[
      \begin{prooftree}
        A_1, \dots, A_n, \Delta_1
        \leadsto
        A
      \end{prooftree}
      \qquad \qquad
      \begin{prooftree}
        A_{n+1}, \dots, A_{n+m}, \Delta_2, B
        \leadsto
        C
      \end{prooftree}
    \]

    where $A_i \in \Gamma$ for all $i \in \{ 1, \dots, n+m\}$.
    By hypothesis the check condition is satisfied, hence
    $\respects{(\Gamma, \Delta_2, A \limp B)}{\ctrlset{1}}$ and
    $\respects{(\Gamma, \Delta_2)}{\ctrlset{A \rightarrow B}}$.
    Then, we may derive

    \[
      \begin{prooftree}
        \[
          \[
            A_1, \dots, A_n, \Delta_1, A_{n+1}, \dots, A_{n+m}, \Delta_2, A
            \rightarrow B
            \leadsto
            A_{n+1}, \dots, A_{n+m}, \Delta_2, A, A \rightarrow B
          \]
          \justifies
          A_{n+1}, \dots, A_{n+m}, \Delta_2, B
          \using{\rightarrow \mathcal{E}}
        \]
        \leadsto
        C
      \end{prooftree}
    \]
    
    which justifies the thesis

    \[
      A_1, \dots, A_n, \Delta_1, A_{n+1}, \dots, A_{n+m}, \Delta_2, A \rightarrow
      B \vdash_{\ctrlset{1}\cup\ctrlset{2}\cup\ctrlset{A \rightarrow B}} C
    \]
    
  \end{enumerate}
\end{proof}

\subsection{From natural deduction to sequent calculus}

...



We consider a very simple fragment of propositional intuitionistic
linear logic, which comprises the multiplicative connectives
$\otimes, \limp, \textbf{1}$, and the additive connective $\oplus$.

\[
  \begin{prooftree}
    \justifies
    \Gamma; P \Longrightarrow P
    \using{init}
  \end{prooftree}
  \qquad \qquad
  \begin{prooftree}
    \Gamma, A; \Delta, A \Longrightarrow C
    \justifies
    \Gamma, A; \Delta \Longrightarrow C
    \using{copy}
  \end{prooftree}
\]

\[
  \begin{prooftree}
    \Gamma; \Delta \Longrightarrow A
    \qquad
    \Gamma; \Delta' \Longrightarrow B
    \justifies
    \Gamma; \Delta, \Delta' \Longrightarrow A \otimes B
    \using{\otimes R}
  \end{prooftree}
  \qquad \qquad
  \begin{prooftree}
    \Gamma; \Delta, A, B \Longrightarrow C
    \justifies
    \Gamma; \Delta, A \otimes B \Longrightarrow C
    \using{\otimes L}
  \end{prooftree}
\]

% \[
%   \begin{prooftree}
%     \justifies
%     \Gamma; \cdot \Longrightarrow \textbf{1}
%     \using{\textbf{1} R}
%   \end{prooftree}
%   \qquad \qquad
%   \begin{prooftree}
%     \Gamma; \Delta \Longrightarrow C
%     \justifies
%     \Gamma; \Delta, \textbf{1} \Longrightarrow C
%     \using{\textbf{1} L}
%   \end{prooftree}
% \]

\[
  \begin{prooftree}
    \Gamma; \Delta \Longrightarrow A
    \qquad
    \Gamma; \Delta', B \Longrightarrow C
    \justifies
    \Gamma; \Delta, \Delta', A \limp B \Longrightarrow C
    \using{\limp L}
  \end{prooftree}
  \qquad \qquad
  \begin{prooftree}
    \Gamma; \Delta, A \Longrightarrow B
    \justifies
    \Gamma; \Delta \Longrightarrow A \limp B
    \using{\limp R}
  \end{prooftree}
\]


% \[
%   \begin{prooftree}
%     \Gamma; \Delta, A \Longrightarrow C
%     \qquad
%     \Gamma; \Delta, B \Longrightarrow C
%     \justifies
%     \Gamma; \Delta, A \oplus B \Longrightarrow C
%     \using{\oplus L}
%   \end{prooftree}
%   \qquad
%   \begin{prooftree}
%     \Gamma; \Delta \Longrightarrow A
%     \justifies
%     \Gamma; \Delta \Longrightarrow A \oplus B
%     \using{\oplus R 1}
%   \end{prooftree}
%   \qquad
%   \begin{prooftree}
%     \Gamma; \Delta \Longrightarrow B
%     \justifies
%     \Gamma; \Delta \Longrightarrow A \oplus B
%     \using{\oplus R 2}
%   \end{prooftree}
% \]

\subsection{Cut elimination}

We first show that restricting the identity axiom on atomic formulas
only is conservative:

\begin{theorem}
  The identity axiom $\Gamma; A \Longrightarrow A$, where $A$ is of
  arbitrary complexity, is admissible.
\end{theorem}
\begin{proof}
  TODO.
\end{proof}

\begin{theorem}[Linear cut elimination]
  If $\Gamma; \Delta \Longrightarrow A$ and $\Gamma; \Delta', A
  \Longrightarrow C$, then $\Gamma; \Delta, \Delta' \Longrightarrow C$.
\end{theorem}
\begin{proof}
  By nested induction on the height of the derivations of the
  premises, and the complexity of the cut formula. We can distinguish
  some cases:

  \begin{enumerate}
  \item One of the premises is an axiom. Then:

    \[
      \begin{prooftree}
        \[\justifies \Gamma; P \Longrightarrow P\] \qquad
        \[\mathcal{E}
        \leadsto
        \Gamma; P \Longrightarrow C \]
        \justifies
        \Gamma; P \Longrightarrow C
      \end{prooftree}
    \]

    then, just take $\mathcal{E}$. The second case is:

    \[
      \begin{prooftree}
        \[\mathcal{D} \leadsto \Gamma; \Delta \Longrightarrow P\]
        \qquad
        \[\justifies \Gamma; P \Longrightarrow P\]
        \justifies
        \Gamma; \Delta \Longrightarrow P
      \end{prooftree}
    \]

    then, just take $\mathcal{D}$.
    
  \item (Principal cuts) The cut formula is introduced by a right rule
    in the left premise, and eliminated by a left rule in the right
    premise... TODO.
  \item (Left-commutative cases) The cut formula is a side formula in
    the left premise. Then, the cut is just routinely moved to the
    premises of the last rule that has been used to derive the left
    premise, and the inductive hypothesis is applied since we act on
    derivations of strictly smaller height.

    \begin{enumerate}
    \item The last rule is $copy$:
      
      \[
        \begin{prooftree}
          \[
            \Gamma, A ; \Delta, A \Longrightarrow B
            \justifies
            \Gamma, A; \Delta \Longrightarrow B
            \using{copy}
          \] \qquad
          \Gamma, A; \Delta', B \Longrightarrow C
          \justifies
          \Gamma, A ; \Delta, \Delta' \Longrightarrow C
        \end{prooftree}
      \]

      Then,

      \[
        \begin{prooftree}
          \[
            \Gamma, A; \Delta, A \Longrightarrow B
            \qquad
            \Gamma, A; \Delta', B \Longrightarrow C
            \justifies
            \Gamma, A; \Delta, \Delta', A \Longrightarrow C
            \using{cut}
          \]
          \justifies
          \Gamma, A ; \Delta, \Delta' \Longrightarrow C
          \using{copy}
        \end{prooftree}
      \]

      
    \item The last rule was...
    \end{enumerate}
    
  \item (Right-commutative cases) The cut formula is a side formula in
    the right premise. Then, the cut is just routinely moved to the
    premises of the last rule that has been used to derive the right
    premise, and the inductive hypothesis is applied since we act on
    derivations of strictly smaller height.

    \begin{enumerate}
    \item The last rule is $copy$:

      \[
        \begin{prooftree}
          \Gamma, A ; \Delta \Longrightarrow C
          \qquad
          \[
            \Gamma, A; \Delta', C, A \Longrightarrow D
            \justifies
            \Gamma, A; \Delta', C \Longrightarrow D
            \using{copy}
          \]
          \justifies
          \Gamma, A; \Delta, \Delta' \Longrightarrow D
          \using{cut}
        \end{prooftree}
      \]

      then

      \[
        \begin{prooftree}
          \[
            \Gamma, A ; \Delta \Longrightarrow C
            \qquad
            \Gamma, A; \Delta', C, A \Longrightarrow D
            \justifies
            \Gamma, A; \Delta, \Delta', A \Longrightarrow D
            \using{cut}
          \]
          \justifies
          \Gamma, A; \Delta, \Delta' \Longrightarrow D
          \using{copy}
        \end{prooftree}
      \]
    \end{enumerate}
  \end{enumerate}
\end{proof}

\begin{theorem}[Persistent cut elimination]
  If $\Gamma; \cdot \Longrightarrow A$ and $\Gamma, A; \Delta
  \Longrightarrow C$, then $\Gamma; \Delta \Longrightarrow C$.
\end{theorem}
\begin{proof}
  By structural induction on the height of the derivations.
  \begin{enumerate}
  \item One of the premises is an initial sequent. Hence, the right
    premise is, and

    \[
      \begin{prooftree}
        \Gamma; \cdot \Longrightarrow A
        \qquad
        \[ \justifies \Gamma, A; P \Longrightarrow P\]
        \justifies
        \Gamma; P \Longrightarrow P
      \end{prooftree}
    \]
    
    but then also $\Gamma; P \Longrightarrow P$ is an identity axiom,
    and can be derived without cut.
    
  \item All other cases are treated as right-commutative cuts, except
    for the case where the last rule in the right premise is
    $copy$. In this case, the cut is

    \[
      \begin{prooftree}
        \Gamma; \cdot \Longrightarrow A
        \qquad
        \[
          \Gamma, A; \Delta, A \Longrightarrow C
          \justifies
          \Gamma, A ; \Delta \Longrightarrow C
          \using{copy}
        \]
        \justifies
        \Gamma; \Delta \Longrightarrow C
        \using{cut!}
      \end{prooftree}
    \]

    But then, by inductive hypothesis and admissibility of linear cut,
    we have

    \[
      \begin{prooftree}
        \Gamma; \cdot \Longrightarrow A
        \qquad
        \[
          \Gamma; \cdot \Longrightarrow A
          \qquad
          \Gamma, A; \Delta, A \Longrightarrow C
          \justifies
          \Gamma; \Delta, A \Longrightarrow C
          \using{cut!}
        \]
        \justifies
        \Gamma; \Delta \Longrightarrow C
        \using{cut}
      \end{prooftree}
    \]

  \end{enumerate}
\end{proof}

\section{Forward sequent calculus}

Since weakening and contraction are admissible for the unrestricted
zone, we can treat $\Gamma$ as a set.

\[
  \begin{prooftree}
    \justifies
    \cdot ; P \longrightarrow P
    \using{init}
  \end{prooftree}
  \qquad \qquad
  \begin{prooftree}
    \Gamma; \Delta, A \longrightarrow C
    \justifies
    \Gamma \cup \{A\}; \Delta \longrightarrow C
    \using{copy}
  \end{prooftree}
\]

\[
  \begin{prooftree}
    \Gamma; \Delta \longrightarrow A
    \qquad
    \Gamma'; \Delta' \longrightarrow B
    \justifies
    \Gamma\cup \Gamma'; \Delta, \Delta' \longrightarrow A \otimes B
  \end{prooftree}
  \qquad \qquad
  \begin{prooftree}
    \Gamma; \Delta, A, B \longrightarrow C
    \justifies
    \Gamma; \Delta, A \otimes B \longrightarrow C
  \end{prooftree}
\]

% \[
%   \begin{prooftree}
%     \justifies
%     \cdot; \cdot \longrightarrow \textbf{1}  
%   \end{prooftree}
%   \qquad \qquad
%   \begin{prooftree}
%     \Gamma; \Delta \longrightarrow C
%     \justifies
%     \Gamma; \Delta, \textbf{1} \longrightarrow C
%   \end{prooftree}
% \]

\[
  \begin{prooftree}
    \Gamma; \Delta \longrightarrow A
    \qquad
    \Gamma'; \Delta', B \longrightarrow C
    \justifies
    \Gamma \cup \Gamma'; \Delta, \Delta', A \limp B \longrightarrow C
    \using{\limp L}
  \end{prooftree}
  \qquad \qquad
  \begin{prooftree}
    \Gamma; \Delta, A \longrightarrow B
    \justifies
    \Gamma; \Delta \longrightarrow A \limp B
    \using{\limp R}
  \end{prooftree}
\]

\begin{theorem}[Soundness]
  If $\Gamma ; \Delta \longrightarrow C$, then
  $\Gamma ; \Delta \Longrightarrow C$.
\end{theorem}
\begin{proof}
  Just notice that every forward derivation is also a backward derivation, with
  the difference that in the latter unrestricted contexts are copied in all
  the premises. The details are a straightforward induction on the height of a
  forward derivation, plus weakening for the backward calculus.
\end{proof}

\begin{theorem}[Completeness]
  If $\Gamma ; \Delta \Longrightarrow C$, then
  $\Gamma' ; \Delta \longrightarrow C$ for some $\Gamma' \subseteq \Gamma$.
\end{theorem}
\begin{proof}
  Straightforward induction on the backward derivation.
\end{proof}

\begin{example}
  The following is an example derivation of

  $$\frwdseq{R_1, R_3}{\cdot}{q \otimes n \limp d \otimes d \otimes d}$$

  where $R_1 \equiv n \otimes n \limp d$ and
  $R_3 \equiv q \limp d \otimes d \otimes n$.

  \[
    \begin{prooftree}
      \[
        \[
          \[
            \frwdseq{\cdot}{q}{q}
            \[
              \[
                \[
                  \frwdseq{}{d}{d}
                  \quad
                  \frwdseq{}{d}{d}
                  \justifies
                  \frwdseq{}{d, d}{d \otimes d}
                \]
                \quad
                \[
                  \[
                    \frwdseq{}{d}{d}
                    \quad
                    \[
                      \frwdseq{}{n}{n}
                      \quad
                      \frwdseq{}{n}{n}
                      \justifies
                      \frwdseq{}{n, n}{n \otimes n}
                    \]
                    \justifies
                    \frwdseq{}{n, n, n \otimes n \limp d}{d}
                  \]
                  \justifies
                  \frwdseq{R_1}{n, n}{d}
                \]
                \justifies
                \frwdseq{R_1}{n, d, d, n}{d \otimes d \otimes d}
              \]
              \justifies
              \frwdseq{R_1}{n, d \otimes d \otimes n}{d \otimes d \otimes d}
            \]
            \justifies
            \frwdseq{R_1}{q, n, q \limp d \otimes d \otimes n}{d \otimes d \otimes d}
          \]
          \justifies
          \frwdseq{R_1, R_3}{q, n}{d \otimes d \otimes d}
        \]
        \justifies
        \frwdseq{R_1, R_3}{q \otimes n}{d \otimes d \otimes d}
      \]
      \justifies
      \frwdseq{R_1, R_3}{\cdot}{q \otimes n \limp d \otimes d \otimes d}
    \end{prooftree}
  \]
\end{example}

% \begin{definition}
%   \begin{enumerate}
%   \item A forward sequent is of the form $\Gamma; \Delta
%     \longrightarrow C$, where $\Gamma$ and $\Delta$ are the
%     unrestricted and linear resources respectively;
%   \item The correspondence relation $\prec$ between forward and
%     backward sequents is defined as follows:
%     $(\Gamma; \Delta \longrightarrow C) \prec
%     (\Gamma';\Delta\Longrightarrow C) \; \text{iff} \; \Gamma \subseteq
%     \Gamma'$. The forward sequent $s$ is sound if for every backward
%     sequent $s'$ such that $s \prec s'$, $s'$ is derivable in the
%     backward calculus;
%   \item The symbol $\prec$ is overloaded to represent the \emph{subsumption}
%     relation between forward sequents, as the smallest relation to satisfy:

%     \[
%       (\Gamma; \Delta \longrightarrow C) \prec (\Gamma'; \Delta
%       \longrightarrow C) \; \text{iff} \; \Gamma \subseteq \Gamma'
%     \]
%   \end{enumerate}
% \end{definition}

% \begin{definition}
%   A rule with conclusion $s$ and premises $s_1, \dots, s_n$ is said to
%   satisfy the irredundancy property if for no $i \in \{1, \dots, n\}$,
%   $s_i \leq s$.
% \end{definition}

% \begin{proposition}
%   All forward rules satisfy the irredundancy property.
% \end{proposition}

% \begin{theorem}[Soundness]
%   If $\Gamma; \Delta \longrightarrow C$ is derivable, then it is
%   sound.
% \end{theorem}

% \begin{theorem}[Completeness]
%   If $\Gamma; \Delta \Longrightarrow C$ is derivable, then there
%   exists a derivable forward sequent
%   $\Gamma'; \Delta' \longrightarrow C$ such that
%   $(\Gamma'; \Delta' \longrightarrow C) \prec (\Gamma; \Delta
%   \Longrightarrow C)$.
% \end{theorem}

\section{The Inverse Method}

...

\subsection{Subformula property}

The key technical property that makes the inverse method possible is the
\emph{subformula property}. This property means, pragmatically, that we only
need to consider sequents composed of subformulas of the goal sequent when
searching for a cut-free proof.

We formalize this idea in terms of a \emph{subformula relation} for
propositions. To do so, we decorate subformulas with certain marks:

\begin{itemize}
\item \emph{Polarity}, written as a superscript + or - symbol;
\item \emph{Availability}, written as a superscript ! (for ``unrestricted'') or
  . (``linear''). Subformulas of an unrestricted formula \emph{do not} inherit
  the decoration.
\end{itemize}

The availability signs determine whether the formula is allowed to occur in the
unrestricted context, and thus serve as a guide for the copy rule.

\begin{definition}
  A decorated sequent is of the form $\Gamma^-_! ; \Delta^-_. \Longrightarrow
  C^+_.$ .
\end{definition}

\begin{definition}[Decorated subformula relation]
  The decorated subformula relation $\leq$ between decorated propositions is the
  reflexive-transitive closure of the following cases:

  \begin{alignat*}{2}
    & A^{\pm}_{.} \leq (A \otimes B)_a^{\pm} \qquad & B^{\pm}_{.} \leq (A \otimes
    B)_a^{\pm} \\
    & A^{\mp}_{.} \leq (A \limp B)_a^{\pm} & B^{\pm}_{.} \leq (A \limp
    B)_a^{\pm}
  \end{alignat*}
  \[
    A^{\pm}_{.} \leq A^{\pm}_{!}
  \]
\end{definition}

We assume the standard pointwise extension of this relation to sets of decorated
propositions:

\[
  S \leq T \equiv \forall A \in S, \exists B \in T : A \leq B
\]

\begin{definition}
  A decorated sequent $s_1 \equiv \Gamma^-_! ; \Delta^-_. \Longrightarrow C^+_.$
  is a subsequent of the decorated sequent
  $\Gamma'^-_! ; \Delta'^-_. \Longrightarrow C'^+_.$, written $s_1 \leq s_2$ if

  \[
    \subsequent{\Gamma'^-_! \cup \Delta'^-_. \cup C'^+_.}
    {\Gamma^-_! \cup \Delta^-_. \cup C^+_.}
  \]
\end{definition}

\begin{theorem}[Subformula property]
  If $\Gamma'; \Delta' \Longrightarrow C'$ appears in a proof of
  $\Gamma; \Delta \Longrightarrow C$, then

  \[
    \subsequent{\Gamma'; \Delta' \Longrightarrow C'}
    {\Gamma; \Delta \Longrightarrow C}
  \]
\end{theorem}
\begin{proof}
  By straightforward inspection of the rules of the backward calculus. It
  sufficies to observe that, if the following is such a rule:

  \[
    \begin{prooftree}
      s_1 \quad s_2 \quad \dots \quad s_n
      \justifies
      s
    \end{prooftree}
  \]

  then $s_i \leq s$ for all $i \in \{1, \dots, n\}$. The thesis follows by
  transitivity of the subformula relation.
\end{proof}

The same reasoning holds for the forward calculus, where the subsequent relation
is extended to forward sequents in the obvious way.

\begin{corollary}
  If $\Gamma'; \Delta' \longrightarrow C'$ appears in a proof of
  $\Gamma; \Delta \longrightarrow C$, then

  \[
    \subsequent{
      \Gamma'; \Delta' \longrightarrow C'
    }{
      \Gamma; \Delta \longrightarrow C
    }
  \]
\end{corollary}
\begin{proof}
  Immediate.
\end{proof}

\subsection{Sequent representation}

\begin{definition}
  A forward sequent is represented as follows:

  \[
    u_1 \# A_1, \dots, u_m \# A_n ; l_1^{k_1} \# B_1, \dots, l_n^{k_n} \# B_k
    \longrightarrow r \# C
  \]
\end{definition}

...

\subsection{Subsumption}

In the saturation-based search that we use in the forward direction, there is a
form of non-determinism in selecting sequents for applying rules. It is
therefore important that the database of sequents available as candidates is as
less redundant as possible. We therefore need to check for \emph{sequent
  subsumption} whenever a new sequent is created (\emph{forward
  subsumption}). In implementing subsumption checks, it is important to detect
failures as early as possible, because the vast majority of checks it likely to
fail. The usual strategy is to perform a sequence of hierarchical tests that
imply subsumption if they all succeed. Performing these checks in sequence
allows us to stop as soon as one of them fails.

\begin{definition}[Hierarchical subsumption tests]
  A forward sequent $s_1 \equiv \Gamma; \Delta \longrightarrow C$ does not
  subsume $s_2 \equiv \Gamma'; \Delta' \longrightarrow C'$, if

  \begin{enumerate}
  \item $C \neq C'$, or
  \item $\Delta \not \subseteq \Delta'$, or
  \item $\Gamma \not \subseteq \Gamma'$, or
  \item $s_1 \not \prec s_2$.
  \end{enumerate}

  Where $C = C'$ compares both formulas and labels, $\Gamma \subseteq \Gamma'$
  if for every $l \in \mathrm{dom}(\Gamma)$,
  $\mathrm{mult}(\Gamma, l) \leq \mathrm{mult}(\Gamma', l)$, and similarly for
  $\Delta$.
\end{definition}

It is trivial to verify that a sequent subsumes another sequent if and only if
the subsumption test above fails.

\section{Focused derivations}

A backward focused proof has two phases. In the active phase all possible rules
are applied in an arbitrary order to asynchronous propositions. When only
synchronous propositions remain, one proposition is selected and a \emph{focused
  phase} for that proposition begins.

As out backward linear sequent calculus is two-sided, we have left- and right-
synchronous and asynchronous connectives.

\begin{table}[h]
  \centering
  \begin{tabular}{|l|l|}
    \hline
    \textbf{symbol} & \textbf{meaning} \\
    \hline
    $P$ & left-synchronous ($\limp$) \\
    $Q$ & right-synchronous ($\otimes$) \\
    $L$ & left-asynchronous ($\otimes$) \\
    $R$ & right-asynchronous ($\limp$) \\
    \hline
  \end{tabular}
\end{table}

The above table does not include the atomic propositions. For reasons explained
in [thesis], here we are forced to treat them as synchronous. However, we can
differenciate the atoms by means of a \emph{focusing bias}, which indicates
whether the atomic proposition under focus must immediately be derived in an
initial sequent.

The backward focusing calculus consists of the following kinds of sequents:

\begin{table}[h]
  \centering
  \begin{tabular}{|l|l|}
    \hline
    \textbf{symbol} & \textbf{meaning} \\
    \hline
    $\Gamma; \Delta \gg A$ & right-focal sequent with $A$ under focus \\
    $\Gamma; \Delta; A \ll Q$ & left-focal sequent with $A$ under focus \\
    $\Gamma; \Delta; \Omega \Longrightarrow C; \cdot$ & right-active sequent \\
    $\Gamma; \Delta; \Omega \Longrightarrow \cdot; Q$ & left-active sequent \\
    \hline
  \end{tabular}
\end{table}

Here, $\Delta$ contains only left-synchronous propositions, i.e., it is of the
form $P_1, \dots, P_n$. $\Omega$ is an ordered context of propositions which may
be synchronous of asynchronous.

For active sequents the right active propositions are decomposed until they
become right-synchronous, i.e. the sequent is of the form $\Gamma; \Delta;
\Omega \Longrightarrow Q; \cdot$. The right hand side is then changed to $\cdot;
Q$. Similarly, the propositions in $\Omega$ are decomposed except when the
proposition is left-synchronous, in which case it is transferred to $\Delta$.

Eventually, the active sequent is reduced to the form $\Gamma; \Delta; \cdot
\Longrightarrow \cdot \cdot; Q$, which we call \emph{neutral sequents}.
A focusing phase is launched from such a neutral sequent by selecting a
\emph{focusable} proposition and giving it the corresponsing focus.

\begin{definition}[Focusable proposition]
  \begin{enumerate}
  \item A proposition is right-focusable if it is right-synchronous and not a
    right-biased atom;
  \item A proposition is left-focusable if it is left-sunchronous and not a
    left-biased atom.
  \end{enumerate}
\end{definition}

When we are in a neutral sequent, we may copy a proposition out of the
unrestricted context and immediately focus on it, regardless of whether it is
focusable or not. If this proposition is actually left-asynchronous, then we
will immediately remove focus on it and transition to an active phase.

There are two forms of the initial sequent, corresponding to the two focusing
biases. If the focal proposition becomes atomic, we terminate with one of the
two initial forms. If the focal proposition is asynchronous, we blur the focus.
If the focal proposition is atomic and of the wrong bias, then also we blur the
focus, but in this case we transition directly to the neutral sequent instead of
entering the active phase.

Decomposing focal propositions uses non-invertible rules for that proposition,
and focus is maintained to the operands of the top-level connective of the
proposition.


\[
  \begin{prooftree}
    p \; \text{left-biased}
    \justifies
    \Gamma; p \gg p
    \using{rinit}
  \end{prooftree}
  \qquad \qquad
  \begin{prooftree}
    p \; \text{right-biased}
    \justifies
    \Gamma; \cdot; p \ll p
    \using{linit}
  \end{prooftree}
\]

\[
  \begin{prooftree}
    \Gamma; \Delta; \Omega \Longrightarrow \cdot; Q
    \justifies
    \Gamma; \Delta; \Omega \Longrightarrow Q; \cdot
    \using{ract}
  \end{prooftree}
  \qquad \qquad
  \begin{prooftree}
    \Gamma; \Delta, P; \Omega \cdot \Omega' \Longrightarrow \gamma
    \justifies
    \Gamma; \Delta; \Omega \cdot P \cdot \Omega' \Longrightarrow \gamma
    \using{lact}
  \end{prooftree}
\]

\[
  \begin{prooftree}
    \Gamma; \Delta \gg Q \qquad Q\; \text{right-focusable}
    \justifies
    \Gamma; \Delta; \cdot \Longrightarrow \cdot ; Q
    \using{rfoc}
  \end{prooftree}
  \qquad \qquad
  \begin{prooftree}
    \Gamma; \Delta; P \ll Q \qquad P\; \text{left-focusable}
    \justifies
    \Gamma; \Delta, P; \cdot \Longrightarrow \cdot; Q
    \using{lfoc}
  \end{prooftree}
\]

\[
  \begin{prooftree}
    \Gamma; \Delta; \cdot \Longrightarrow R; \cdot
    \justifies
    \Gamma; \Delta \gg R
    \using{rblur}
  \end{prooftree}
  \qquad \qquad
  \begin{prooftree}
    \Gamma; \Delta; L \Longrightarrow \cdot; Q
    \justifies
    \Gamma; \Delta; L \ll Q
    \using{lblur}
  \end{prooftree}
\]

\[
  \begin{prooftree}
    \Gamma; \Delta; \cdot \Longrightarrow \cdot ; p \qquad p \; \text{right-biased}
    \justifies
    \Gamma; \Delta \gg p
    \using{rblur^*}
  \end{prooftree}
  \qquad \qquad
  \begin{prooftree}
    \Gamma; \Delta, p; \cdot \Longrightarrow \cdot; Q \qquad p \; \text{left-biased}
    \justifies
    \Gamma; \Delta; p \ll Q
    \using{lblur^*}
  \end{prooftree}
\]

\[
  \begin{prooftree}
    \Gamma, A; \Delta ; A \ll Q
    \justifies
    \Gamma, A; \Delta; \cdot \Longrightarrow \cdot; Q
    \using{copy}
  \end{prooftree}
\]

\[
  \begin{prooftree}
    \Gamma; \Delta; \Omega \cdot A \cdot B \cdot \Omega' \Longrightarrow \gamma
    \justifies
    \Gamma; \Delta; \Omega \cdot A \otimes B \cdot \Omega' \Longrightarrow
    \gamma
    \using{\otimes L}
  \end{prooftree}
  \qquad \qquad
  \begin{prooftree}
    \Gamma; \Delta_1 \gg A \qquad \Gamma; \Delta_2 \gg B
    \justifies
    \Gamma; \Delta_1, \Delta_2 \gg A \otimes B
    \using{\otimes R}
  \end{prooftree}
\]

\[
  \begin{prooftree}
    \Gamma; \Delta_1; B \ll Q \qquad \Gamma; \Delta_2 \gg A
    \justifies
    \Gamma; \Delta_1, \Delta_2 ; A \limp B \ll Q
    \using{\limp L}
  \end{prooftree}
  \qquad \qquad
  \begin{prooftree}
    \Gamma; \Delta; \Omega \cdot A \Longrightarrow B; \cdot
    \justifies
    \Gamma; \Delta; \Omega \Longrightarrow A \limp B; \cdot
    \using{\limp R}
  \end{prooftree}
\]

\begin{theorem}[Soundness]
  \begin{enumerate}
  \item If $\rfocseq{\Gamma}{\Delta}{A}$, then $\Gamma; \Delta \Longrightarrow
    A$;
  \item If $\lfocseq{\Gamma}{\Delta}{A}{Q}$, then $\Gamma ; \Delta, A
    \Longrightarrow Q$;
  \item If $\btriseq{\Gamma}{\Delta}{\Omega}{C ; \cdot}$, then $\Gamma; \Delta,
    \Omega \Longrightarrow C$;
  \item If $\btriseq{\Gamma}{\Delta}{\Omega}{\cdot ; C}$, then $\Gamma; \Delta,
    \Omega \Longrightarrow C$.
  \end{enumerate}
\end{theorem}
\begin{proof}
  The four assertions are proved simultaneously by straightforward induction on
  the height of the derivation.
\end{proof}

\begin{theorem}[Completeness]
  If $\Gamma ; \Delta \Longrightarrow C$, then
  $\Gamma; \cdot ; \Omega \Longrightarrow C ; \cdot$, where $\Omega$ is some
  listing of $\Delta$.
\end{theorem}
\begin{proof}
  This proof is much more involved than the one for soundness, and can be found
  in [thesis] for the full intuitionistic linear logic. It can be easily adapted
  to our small fragment since it does not rely on the presence of all
  connectives.
\end{proof}

\subsection{Backward derived rules}

The primary benefit of focusing is the ability to generate derived ``big step''
inference rules: the intermediate results of a focusing or active phase are not
important, since those steps are ``forced'' in some way. Each derived rule
starts (at the bottom) with a neutral sequent from which a synchronous
proposition is selected for focus, and the focusing steps are followed. Then the
active rules are applied, and eventually we obtain a collection of neutral
sequents as the leaves. These neutral sequents are then treated as the premises
of the derived rule that produces the neutral sequent with which we started.

We first construct the backward derived rules. Then we will move to their
forward version. The general design is that intermediate sequents in the eager
active and focusing phases are not be stored in any sequent database; instead,
all sequents constructed during search are neutral sequents at the phase
boundaries. This is achieved by first precomputing the derived rules that
correspond to the frontier literals of the goal sequent.

For any given proposition, we are interested in constructing a derived inference
for the proposition corresponding to a single pair of focusing and inverse
phases.

The idea is to interpret a proposition itself as the derived rules that it
embodies. Every propostiion is viewed as a relation between the conclusion of
the rule and its premises at the leaves of the bipole. Both the conclusion and
the premises are neutral sequents, which we indicate as
$\bneuseq{\Gamma}{\Delta}{Q}$.

There are three classes of relational interpretations:

\begin{enumerate}
\item Right focal relations for the focus formula $A$, written $\brfrel{A}$;
\item Left focal relations for the focus formula $A$, written $\blfrel{A}$;
\item Active relations, written
  $\bactrel{\btriseq{\Gamma}{\Delta}{\Omega}{\xi}}$, where $\xi$ is either
  $\cdot$ or a proposition $C$.
\end{enumerate}

\[
  \begin{prooftree}
    p \; \text{right-biased}
    \justifies
    \relj{\blfrel{p}}{\bneuseq{\Gamma}{\cdot}{p}}{\cdot}
    \using{linit}
  \end{prooftree}\qquad\qquad
  \begin{prooftree}
    p \; \text{left-biased}
    \justifies
    \relj{\brfrel{p}}{\bneuseq{\Gamma}{p}{\cdot}}{\cdot}
    \using{rinit}
  \end{prooftree}
\]

\[
  \begin{prooftree}
    \relj{\brfrel{A}}{\bneuseq{\Gamma}{\Delta_1}{\cdot}}{\Sigma_1}
    \qquad
    \relj{\brfrel{B}}{\bneuseq{\Gamma}{\Delta_2}{\cdot}}{\Sigma_2}
    \justifies
    \relj{\brfrel{A \otimes B}}{\bneuseq{\Gamma}{\Delta_1,
        \Delta_2}{\cdot}}{\Sigma_1 \cdot \Sigma_2}
    \using{\otimes F}
  \end{prooftree}
\]

\[
  \begin{prooftree}
    \relj{\bactrel{\btriseq{\cdot}{\cdot}{\cdot}{R}}}{s}{\Sigma}
    \justifies
    \relj{\brfrel{R}}{s}{\Sigma}
    \using{FA^+}
  \end{prooftree}
  \qquad \qquad
  \begin{prooftree}
    p \; \text{right-biased}
    \justifies
    \relj{\brfrel{p}}{\bneuseq{\Gamma}{\Delta}{\cdot}}{\bneuseq{\Gamma}{\Delta}{p}}
    \using{conjecture^+}
  \end{prooftree}
\]

\[
  \begin{prooftree}
    \relj{\blfrel{B}}{\bneuseq{\Gamma}{\Delta_1}{Q}}{\Sigma_1} \qquad
    \relj{\brfrel{A}}{\bneuseq{\Gamma}{\Delta_2}{\cdot}}{\Sigma_2}
    \justifies
    \relj{\blfrel{A \limp B}}{\bneuseq{\Gamma}{\Delta_1,\Delta_2}{Q}}{\Sigma_1
      \cdot \Sigma_2}
    \using{\limp F}
  \end{prooftree}
\]

\[
  \begin{prooftree}
    \relj{\bactrel{\btriseq{\cdot}{\cdot}{L}{\cdot}}}{s}{\Sigma}
    \justifies
    \relj{\blfrel{L}}{s}{\Sigma}
    \using{FA^-}
  \end{prooftree}
  \qquad \qquad
  \begin{prooftree}
    p \; \text{left-biased}
    \justifies
    \relj{\blfrel{p}}{\bneuseq{\Gamma}{\Delta}{Q}}{\bneuseq{\Gamma}{\Delta, p}{Q}}
    \using{conjecture^-}
  \end{prooftree}
\]

\[
  \begin{prooftree}
    \relj{\bactrel{\btriseq{\Gamma}{\Delta}{\Omega \cdot A \cdot B \cdot
          \Omega'}{\xi}}}{s}{\Sigma}
    \justifies
    \relj{\bactrel{\btriseq{\Gamma}{\Delta}{\Omega \cdot A \otimes B \cdot
          \Omega'}{\xi}}}{s}{\Sigma}
    \using{\otimes A}
  \end{prooftree}\qquad \qquad
  \begin{prooftree}
    \relj{\bactrel{\btriseq{\Gamma}{\Delta}{\Omega \cdot A \cdot
          \Omega'}{B}}}{s}{\Sigma}
    \justifies
    \relj{\bactrel{\btriseq{\Gamma}{\Delta}{\Omega \cdot
          \Omega'}{A \limp B}}}{s}{\Sigma}
    \using{\limp A}
  \end{prooftree}
\]

\[
  \begin{prooftree}
    \relj{
      \bactrel{
        \btriseq{\Gamma}{\Delta,P}{\Omega \cdot \Omega'}{\xi}
      }
    }{s}{\Sigma}
    \justifies
    \relj{\bactrel{\btriseq{\Gamma}{\Delta}{\Omega \cdot P \cdot
          \Omega'}{\xi}}}{s}{\Sigma}
    \using{bact}
  \end{prooftree}
\]

\[
  \begin{prooftree}
    \justifies
    \relj{
      \bactrel{\btriseq{\Gamma}{\Delta}{\cdot}{\cdot}}
    }{
      \bneuseq{\Gamma'}{\Delta'}{Q}
    }{
      \bneuseq{\Gamma, \Gamma'}{\Delta, \Delta'}{Q}
    }
    \using{match}
  \end{prooftree}
\]

\[
  \begin{prooftree}
    \justifies
    \relj{
      \bactrel{\btriseq{\Gamma}{\Delta}{\cdot}{Q}}
    }{
      \bneuseq{\Gamma'}{\Delta'}{\cdot}
    }{
      \bneuseq{\Gamma, \Gamma'}{\Delta, \Delta'}{Q}
    }
    \using{match'}
  \end{prooftree}
\]

\[
  \begin{prooftree}
    (\relj{\brfrel{Q}}{\bneuseq{\Gamma}{\Delta}{\cdot}}{s_1 \cdot s_2 \dots s_n})
    \quad s_1 \quad s_2 \quad \dots \quad s_n
    \justifies
    \bneuseq{\Gamma}{\Delta}{Q}
    \using{right-focus}
  \end{prooftree}
\]

\[
  \begin{prooftree}
    (\relj{\blfrel{Q}}{\bneuseq{\Gamma}{\Delta}{Q}}{s_1 \cdot s_2 \dots s_n})
    \quad s_1 \quad s_2 \quad \dots \quad s_n
    \justifies
    \bneuseq{\Gamma}{\Delta, P}{Q}
    \using{left-focus}
  \end{prooftree}
\]

\[
  \begin{prooftree}
    (\relj{\blfrel{A}}{\bneuseq{\Gamma, A}{\Delta}{Q}}{s_1 \cdot s_2 \dots s_n})
    \quad s_1 \quad s_2 \quad \dots \quad s_n
    \justifies
    \bneuseq{\Gamma, A}{\Delta}{Q}
    \using{copy-focus}
  \end{prooftree}
\]

\begin{lemma}\label{completeness-lemma}
  \begin{enumerate}
  \item If $\rfocseq{\Gamma}{\Delta}{A}$, then for some $\Sigma$
    \begin{enumerate}
    \item $\relj{\brfrel{A}}{\bneuseq{\Gamma}{\Delta}{\cdot}}{\Sigma}$, and
    \item $\Sigma$ are all derivable
    \end{enumerate}
  \item ...
  \item ...
  \end{enumerate}
\end{lemma}
\begin{proof}
  By simultaneous induction on the height of the derivation.
  \begin{enumerate}
  \item Case $\rinit$:

    \[
      \begin{prooftree}
        p \; \text{left-biased}
        \justifies
        \rfocseq{\Gamma}{p}{p}
        \using{\rinit}
      \end{prooftree}
    \]

    then, just apply the $\rinit$ relation.
  \item Case $\linit$ is similar to the case above.
  \item Case $\ract$ is trivial, since the relations corresponding to the
    premise and conclusion sequents are identical.
  \item Case $\lact$. Straightforward use of the $\bact$ rule for derived
    relations.
  \item Case $\rfoc$.
    
    \[
      \begin{prooftree}
        \Gamma; \Delta \gg Q \qquad Q\; \text{right-focusable}
        \justifies
        \Gamma; \Delta; \cdot \Longrightarrow \cdot ; Q
        \using{rfoc}
      \end{prooftree}
    \]

    Suppose $\gamma_1, \gamma_2$ are such that $\gamma_1 \uplus \gamma_2 =
    Q$. Then, the following:

    \[
      \relj{
        \bactrel{\btriseq{\Gamma}{\Delta}{\cdot}{\gamma_1}}
      }{
        \bneuseq{\cdot}{\cdot}{\gamma_2}
      }{
        \bneuseq{\Gamma}{\Delta}{Q}
      }
    \]

    is derivable with either $\matchrule$ or $\matchprimerule$. We need to show
    that $\bneuseq{\Gamma}{\Delta}{Q}$ is derivable. By inductive hypothesis, we
    have $\relj{\brfrel{Q}}{\bneuseq{\Gamma}{\Delta}{\cdot}}{\Sigma}$ for some
    $\Sigma$ all derivable. But then,

    \[
      \begin{prooftree}
        \relj{\brfrel{Q}}{\bneuseq{\Gamma}{\Delta}{\cdot}}{\Sigma}
        \qquad \Sigma
        \justifies
        \bneuseq{\Gamma}{\Delta}{Q}
        \using{\rightfocusrule}
      \end{prooftree}
    \]

  \item Case $\lfoc$ is just analogous to $\rfoc$, with an application of
    $\leftfocusrule$.
  \item Case $\rblur$.

    \[
      \begin{prooftree}
        \Gamma; \Delta; \cdot \Longrightarrow R; \cdot
        \justifies
        \Gamma; \Delta \gg R
        \using{\rblur}
      \end{prooftree}
    \]

    By inductive hypothesis, we have
    $\relj{\bactrel{\btriseq{\cdot}{\cdot}{\cdot}{R}}}{\bneuseq{\Gamma}{\Delta}{\cdot}}{\Sigma}$,
    where all $\Sigma$ are derivable. But
    then, we can apply the rule $\faplus$ to get the thesis

    \[
      \begin{prooftree}
        \relj{\bactrel{\btriseq{\cdot}{\cdot}{\cdot}{R}}}{\bneuseq{\Gamma}{\Delta}{\cdot}}{\Sigma}
        \justifies
        \relj{\brfrel{R}}{\bneuseq{\Gamma}{\Delta}{\cdot}}{\Sigma}
        \using{FA^+}
      \end{prooftree}
    \]
    
  \item Case $\lblur$ is dual to $\rblur$.
  \item Case $\rblurstar$.

    \[
      \begin{prooftree}
        \Gamma; \Delta; \cdot \Longrightarrow \cdot ; p \qquad p \; \text{right-biased}
        \justifies
        \Gamma; \Delta \gg p
        \using{rblur^*}
      \end{prooftree}
    \]

    By inductive hypothesis,
    $\relj{\bactrel{\btriseq{\Gamma}{\Delta}{\cdot}{p}}}{\bneuseq{\cdot}{\cdot}{\cdot}}{\Sigma}$,
    that must have been derived by $\matchprimerule$. Therefore, $\Sigma \equiv
    \bneuseq{\Gamma}{\Delta}{p}$, which is by hypothesis derivable.
    Then, we can apply $\faplusstar$ to get
    $\relj{\brfrel{p}}{\bneuseq{\Gamma}{\Delta}{\cdot}}{\bneuseq{\Gamma}{\Delta}{p}}$.

  \item Case $\lblurstar$ is dual to $\rblurstar$.
  \item Case $\copyrule$.

    \[
      \begin{prooftree}
        \Gamma, A; \Delta ; A \ll Q
        \justifies
        \Gamma, A; \Delta; \cdot \Longrightarrow \cdot; Q
        \using{\copyrule}
      \end{prooftree}
    \]

    We can apply $\matchrule$ to immediately derive

    \[
      \begin{prooftree}
        \justifies
        \relj{\bactrel{\btriseq{\Gamma,
              A}{\Delta}{\cdot}{\cdot}}}{\bneuseq{\cdot}{\cdot}{Q}}{\bneuseq{\Gamma,
            A}{\Delta}{Q}}
        \using{\matchrule}
      \end{prooftree}
    \]

    and see that $\bneuseq{\Gamma, A}{\Delta}{Q}$ is derivable by
    the inductive hypothesis
    $\relj{\blfrel{A}}{\bneuseq{\Gamma,A}{\Delta}{Q}}{\Sigma}$ and an application
    of $\copyfocusrule$.
  \item The remaining rules of the connectives are straightforward application
    of the inductive hypothesis.
  \end{enumerate}
\end{proof}

\begin{theorem}[Completeness]
  If $\btriseq{\Gamma}{\Delta}{\cdot}{\cdot ; Q}$, then
  $\bneuseq{\Gamma}{\Delta}{Q}$.
\end{theorem}
\begin{proof}
  Straightforward application of Lemma~\ref{completeness-lemma}. The last rule
  used to derive $\btriseq{\Gamma}{\Delta}{\cdot}{\cdot ; Q}$ is one of
  $\lfoc, \rfoc$ or $\copyrule$; correspondingly, by the Lemma, we have the
  derived rules to use as premises of $\rightfocusrule, \leftfocusrule$ and
  $\copyfocusrule$ which derive $\bneuseq{\Gamma}{\Delta}{Q}$.
\end{proof}

\subsection{Forward focusing}

In the following rules, $\gamma \setminus \xi$ is defined as $\gamma$ if $\xi = \cdot$,
and as $\cdot$ if $\gamma = \xi$.

\[
  \begin{prooftree}
    p \; \text{left-biased}
    \justifies
    \relj{\frfrel{p}}{\cdot}{\fneuseq{\cdot}{p}{\cdot}}
    \using{\linit}
  \end{prooftree}
  \qquad\qquad
  \begin{prooftree}
    p \; \text{right-based}
    \justifies
    \relj{\flfrel{p}}{\cdot}{\fneuseq{\cdot}{\cdot}{p}}
    \using{\rinit}
  \end{prooftree}
\]


\[
  \begin{prooftree}
    \relj{
      \factrel{\btriseq{\cdot}{\cdot}{\cdot}{R}}
    }{
      \Sigma
    }{
      s
    }
    \justifies
    \relj{\frfrel{R}}{\Sigma}{s}
    \using{\faplus}
  \end{prooftree}
  \qquad \qquad
    \begin{prooftree}
    \relj{\factrel{\btriseq{\cdot}{\cdot}{L}{\cdot}}}{\Sigma}{s}
    \justifies
    \relj{\flfrel{L}}{\Sigma}{s}
    \using{\faminus}
  \end{prooftree}
\]

\[
  \begin{prooftree}
    \relj{\flfrel{B}}{\Sigma_1}{\fneuseq{\Gamma_1}{\Delta_1}{\gamma}}
    \qquad
    \relj{\frfrel{A}}{\Sigma_2}{\fneuseq{\Gamma_2}{\Delta_2}{\cdot}}
    \justifies
    \relj{\flfrel{A \limp B}}{\Sigma_1 \cdot \Sigma_2}{\fneuseq{\Gamma_1,
        \Gamma_2}{\Delta_1, \Delta_2}{\gamma}}
    \using{\limp F}
  \end{prooftree}
\]

\[
  \begin{prooftree}
    \relj{\frfrel{A}}{\Sigma_1}{\fneuseq{\Gamma_1}{\Delta_1}{\cdot}}
    \qquad
    \relj{\frfrel{B}}{\Sigma_2}{\fneuseq{\Gamma_2}{\Delta_2}{\cdot}}
    \justifies
    \relj{\frfrel{A \otimes B}}{\Sigma_1 \cdot \Sigma_2}{\fneuseq{\Gamma_1,
        \Gamma_2}{\Delta_1, \Delta_2}{\cdot}}
    \using{\otimes F}
  \end{prooftree}
\]

\[
  \begin{prooftree}
    \relj{\factrel{\btriseq{\Gamma}{\Delta}{\Omega \cdot A}{B}}}{\Sigma}{s}
    \justifies
    \relj{\factrel{\btriseq{\Gamma}{\Delta}{\Omega}{A \limp B}}}{\Sigma}{s}
    \using{\limp A}
  \end{prooftree}
\]

\[
  \begin{prooftree}
    \relj{\factrel{
        \btriseq{\Gamma}{\Delta}{\Omega \cdot A \cdot B \cdot \Omega'}{\xi}}
    }{\Sigma}{s}
    \justifies
    \relj{\factrel{
        \btriseq{\Gamma}{\Delta}{\Omega \cdot A \otimes B \cdot \Omega'}{\xi}}
    }{\Sigma}{s}
    \using{\otimes A}
  \end{prooftree}
\]

\[
  \begin{prooftree}
    \relj{\factrel{\btriseq{\Gamma}{\Delta, P}{\Omega \cdot \Omega'}{\xi}}}{\Sigma}{s}
    \justifies
    \relj{\factrel{\btriseq{\Gamma}{\Delta}{\Omega \cdot P \cdot \Omega'}{\xi}}}{\Sigma}{s}
    \using{act}
  \end{prooftree}
\]

\[
  \begin{prooftree}
    \xi \subseteq \gamma
    \justifies
    \relj{
      \factrel{\btriseq{\Gamma}{\Delta}{\cdot}{\xi}}
    }{
      \fneuseq{\Gamma, \Gamma'}{\Delta, \Delta'}{\gamma}
    }{
      \fneuseq{\Gamma'}{\Delta'}{\gamma \setminus \xi}
    }
    \using{match}
  \end{prooftree}
\]

The derived rule for positive subformulas is:

\[
  \begin{prooftree}
    s_1 \quad \dots \quad s_n \quad
    (\relj{\frfrel{Q}}{s_1 \dots s_n}{\fneuseq{\Gamma}{\Delta}{\cdot}})
    \justifies
    \fneuseq{\Gamma}{\Delta}{Q}
  \end{prooftree}
\]

Similarly, for negative propositions, we have two rules:

\[
  \begin{prooftree}
    s_1 \quad \dots \quad s_n \quad
    (\relj{\flfrel{P}}{s_1 \dots s_n}{\fneuseq{\Gamma}{\Delta}{Q}})
    \justifies
    \fneuseq{\Gamma}{\Delta, P}{Q}
  \end{prooftree}
\]

\[
  \begin{prooftree}
    s_1 \quad \dots \quad s_n \quad
    (\relj{\flfrel{A}}{s_1 \dots s_n}{\fneuseq{\Gamma}{\Delta}{Q}})
    \justifies
    \fneuseq{\Gamma, A}{\Delta}{Q}
  \end{prooftree}
\]

\begin{fact}
  \begin{enumerate}
  \item If $\relj{\factrel{\btriseq{\Gamma}{\Delta}{\Omega}{\xi}}}{\Sigma}
    {\fneuseq{\Gamma'}{\Delta'}{\gamma}}$, then either
    \begin{enumerate}
    \item $\xi = \cdot$ and $\gamma = Q$ for some right-synchronous $Q$;
    \item $\xi = C$ and $\gamma = \cdot$ for some $C$.
    \end{enumerate}
    
  \item If $\relj{\flfrel{A}}{\Sigma}{\fneuseq{\Gamma}{\Delta}{\gamma}}$, then
    $\gamma = Q$ for some right-synchronous $Q$.
  \end{enumerate}
\end{fact}

\begin{definition}
  The sequent $\fneuseq{\Gamma}{\Delta}{Q}$ is sound if
  $\btriseq{\Gamma}{\Delta}{\cdot}{\cdot ; Q}$.
\end{definition}

\begin{lemma}
  If $\Sigma$ are sound, then

  \begin{enumerate}
  \item If $\relj{\frfrel{A}}{\Sigma}{\fneuseq{\Gamma}{\Delta}{\cdot}}$,
    then, $\btriseq{\Gamma}{\Delta}{\cdot}{A ; \cdot}$;
  \item If $\relj{\flfrel{A}}{\Sigma}{\fneuseq{\Gamma}{\Delta}{Q}}$,
    then $\btriseq{\Gamma}{\Delta}{\cdot}{\cdot ; Q}$;
  \item If
    $\relj{\factrel{\btriseq{\Gamma}{\Delta}{\Omega}{\cdot}}}{\Sigma}
    {\fneuseq{\Gamma'}{\Delta'}{Q}}$, then
    $\btriseq{\Gamma, \Gamma'}{\Delta, \Delta'}{\Omega}{\cdot ; Q}$;
  \item If
    $\relj{\factrel{\btriseq{\Gamma}{\Delta}{\Omega}{C}}}{\Sigma}
    {\fneuseq{\Gamma'}{\Delta'}{\cdot}}$, then
    $\btriseq{\Gamma, \Gamma'}{\Delta, \Delta'}{\Omega}{C ; \cdot}$.
  % \item If
  %   $\relj{\factrel{\btriseq{\Gamma}{\Delta}{\Omega}{\gamma}}}{\Sigma}
  %   {\fneuseq{\Gamma'}{\Delta'}{C}}$, then $\gamma = C$ and
  %   $\btriseq{\Gamma, \Gamma'}{\Delta, \Delta'}{\Omega}{C ; \cdot}$.
  \end{enumerate}
\end{lemma}
\begin{proof}
  All points proved by simultaneous induction on the height of the
  derivation. We distinguish cases on the last rule used in the derivation:

  \begin{itemize}
  \item Case $\linit$. Hence,
    $\relj{\frfrel{p}}{\cdot}{\fneuseq{\cdot}{p}{\cdot}}$ and $p$ is
    left-biased. Then,

    \[
      \begin{prooftree}
        \[
          \[
            p \; \text{left-biased}
            \justifies
            \rfocseq{\cdot}{p}{p}
          \]
          \justifies
          \btriseq{\cdot}{p}{\cdot}{\cdot ; p}
        \]
        \justifies
        \btriseq{\cdot}{p}{\cdot}{p ; \cdot}
      \end{prooftree}
    \]
    
  \item Case $\rinit$ is similar;
  \item Case $\faplus{}$. 
  \end{itemize}
\end{proof}

\begin{theorem}[Soundness]
  If $\fneuseq{\Gamma}{\Delta}{Q}$, then $\btriseq{\Gamma}{\Delta}{\cdot}{\cdot
  ; Q}$.
\end{theorem}
\begin{proof}
  TODO...
\end{proof}

\section{Focused inverse method}

The primary issue in the forward direction is to enumerate the propositions for
which we need to derive inference rules. As the calculus of derived rules has
only neutral sequents as premises and conclusions, we need only generate rules
for propositions that occur in neutral sequents; we call them \emph{frontier
  propositions}. To find the frontier propositions in a goal sequent, we
abstractly replay the focusing and active phases to identify the phase
transitions. Each transition from an active to a focal phase produces a frontier
proposition. We do not consider negative left-biased focused atoms (or positive
right-biased focused atoms) in the frontier as these will be derived using an
init rule for which there are no premises.

\[
  \decminus{f(p)} = \emptyset \qquad
  \decplus{f(p)} = \decplusminus{a(p)} = \{
  \decplusminus{\declinear{p}} \} \qquad \text{if $p$ right-based}
\]
\[
  \decplus{f(p)} = \emptyset \qquad
  \decminus{f(p)} = \decminusplus{a(p)} = \{
  \decminusplus{\declinear{p}} \} \qquad \text{if $p$ left-biased}
\]
\[
  \decminus{f(A \otimes B)} = \decminus{a(A \otimes B)} \qquad
  \decplus{f(A \otimes B)} = \decplus{f(A)}, \decplus{f(B)}
\]
\[
  \decminus{a(A\otimes B)} = \decminus{a(A)}, \decminus{a(B)} \qquad
  \decplus{a(A\otimes B)} = \decplus{f(A \otimes B)}, \decminuslinear{(A \otimes B)}
\]
\[
  \decminus{f(A \limp B)} = \decplus{f(A)}, \decminus{f(B)} \qquad
  \decplus{f(A \limp B)} = \decplus{a(A \limp B)}
\]
\[
  \decminus{a(A \limp B)} = \decminus{f(A \limp B)}, \decminuslinear{(A \limp
    B)}
  \qquad
  \decplus{a(A\limp B)} = \decminus{a(A)}, \decplus{a(B)}
\]

\begin{example}
  Put example here...
\end{example}

\begin{definition}[Frontier]
  Given a goal $\Gamma;\Delta \Longrightarrow Q$, its frontier contains

  \begin{enumerate}
  \item All top-level propositions in $\decunrestr{\decminus{\Gamma}},
    \decminuslinear{\Delta}$, and $\decpluslinear{Q}$;
  \item For any $A \in \Gamma, \Delta$, the collection $\decminus{f(A)}$;
  \item the collection $\decplus{f(Q)}$.
  \end{enumerate}
\end{definition}

In the preparatory phase for the inverse method, we calculate the frontier
propositions of the goal sequent. There is no need to generate initial sequents
separately, as the executions of negative atoms in the frontier directly give us
the necessary initial sequents.

\section{Search strategy}

\subsection{Rules and rule application}

% From 4.1.5
Rules are precomputed to the specific labels of the subformulas of the goal
sequent. If the positive proposition $\labels{r}{A \otimes B}$ occurs in the
goal sequent, and $\labels{r_A}{A}, \labels{r_B}{B}$ are the labels of the
operands, then the following rule will be generated for this label:

\[
  \begin{prooftree}
    \Gamma_1; \Delta_1 \longrightarrow r_A
    \qquad
    \Gamma_2; \Delta_2 \longrightarrow r_B
    \justifies
    \Gamma_1, \Gamma_2; \Delta_1, \Delta_2 \rightarrow r
    \using{\otimes R}
  \end{prooftree}
\]

To apply a rule to a given input sequent, we have to first \emph{match} a
premiss of the rule to the input sequent.

\subsection{Search procedure}

% From 4.1.6

The search procedure is summarized as follows:

\begin{enumerate}
\item Label all subformulas of the goal sequent, and decorate using signs and
  availabilities;
\item Determine all initial sequents for atomic formulas with both signs;
\item Specialize all left rules for negative subformulas, all right rules for
  positive subformulas, and instances of the ``copy'' rule for unrestriced
  subformulas;
\item Starting from the initial sequents, apply the inference ruels in any order
  that is guaranteed to saturate the search space. Add new facts to a database
  used for forward subsumption;
\item Stop when the goal sequent is matched, or if no rules apply.
\end{enumerate}

The procedure maintains two continually updated sequent databases:

\begin{enumerate}
\item The \strong{kept sequents} database contains new sequents that have not
  been subsumed, but are not yet being considered for rule applications;
\item The \strong{active sequents} database that contains all sequents that
  should be considered for rule applications.
\end{enumerate}

At each loop iteration, a sequent $s$ is selected and removed from the kept
sequents database and inserted into the active sequents database
(``activation''). Then, all specialized rules are matched against $s$ as the
first premise.... [continues, but does not say anything]

Completeness of the loop is a well known property, requiring only the following
properties of the implementation:

\begin{enumerate}
\item \strong{Fair selection}: every sequent in the kept sequents database is
  eventually selected for insertion into the active sequents database;
\item \strong{Fair application}: if a rule can be fully satisfied by sequents in
  the active sequents database (in any order), then the conclusion of this rule
  is eventually considered for insertion into the kept sequents database.
\end{enumerate}

\subsection{Search procedure 2}

% Taken from 5.6

We extend the approach by treating all rules as essentially unary rules that
produce either a conclusion sequent or a partially instantiated rule. Hence we
have a dynamically growing collection of (partially applied) rules in the rule
database.

The inner (?) loop of the search procedure performs the following \emph{lazy
  activation} step until either the goal sequent is subsumed, or no further
rules are applicable to the active sequents. Activation contains a closely
related procedure called \emph{percolation}.

\begin{definition}[Lazy activation]
  In the activation of a sequent $s$, that is, the transferring of $s$ from the
  kept sequents database to the active sequents database, the following steps
  are performed:

  \begin{enumerate}
  \item The sequent is renamed (???) and inserted into the active sequents
    database;
  \item All available rules are applied to $s$. If these applications produced
    new rules $R$, then percolation is performed on $R$ to obtain the full
    collection of new partially instantiated rules;
  \item The new rules are added to the rule database;
  \item All sequents generated during the above rule applications and
    percolation are tested for subsumption in the global index (forward
    subsumption). All unsubsumed sequents are added to the kept sequents
    database.
  \end{enumerate}
\end{definition}

\begin{definition}[Percolation]
  To percolate a collection of rules $R$, the following two steps are performed
  until there are no new additions to $R$:

  \begin{enumerate}
  \item For every sequent in the active set, every rule in $R$ is applied to it;
  \item Any new rules generated are added to $R$.
  \end{enumerate}
\end{definition}

A sequent is added to the kept sequents database if it is not globally subsumed
by some sequent derived earlier.

\subsection{The focused inverse method (to be redistributed)}

% Taken from 6.3

It is possible for a new sequent to be stronger than some sequents already in
the database. In this case, the old weaker sequents are no longer considered for
new derivations (backward subsumption).

\subsubsection{Implementation details}

% Taken from 6.3.1

Given a proposition in the frontier, the focusing phase of the derived rule for
it is completely predictable, sincere there are no disjunctive choices to be
made. The active phase is also predictable, since we do not have additive
connectives.

My addition: Hence, every subformula in the frontier deterministically gives
rise to a unique derived rule schema (which can of course be matched by actual
premise sequents in infinite ways).

Doubt: perche' parla di proposizioni e non di (sotto-)formule? Non avevamo detto
che ogni non-atomic subformula e` distinta da altre subformule relative alla
stessa proposizione (quindi sintatticamente uguali nel caso proposizionale).

\subsection{Search loop}

% taken from "A focusing inverse method theorem prover for first-order
% linear logic"

We use a lazy variant of the OTTER loop as the main loop of the search
procedure. We maintain tow global sets of derived sequents:

\begin{itemize}
\item The \emph{active set} containing sequents to be considered as premises of
  rules;
\item The \emph{inactive set} that contains all facts that have not yet been
  transferred to the active set.
\end{itemize}

The inner loop of the search procedure repeats the following lazy activation
step until either the goal sequent is subsumed (in which case the search is
successful), or no further rules are applicable to the sequents in the acitve
set and the inactive set is exhausted (in which case the search saturates).

\begin{definition}[Lazy activation]
  To activate the sequent $s$, i.e., to transfer it from the inactive to the
  active set, the following steps are performed:

  \begin{enumerate}
  \item After renaming (???), $s$ is inserted into the active set;
  \item All available rules are applied to $s$. If these applications produce
    new rules, say a set $R$, then the following two steps are performed in a
    loop until there are no additions to $R$:

    \begin{enumerate}
    \item For every sequent $s'$ in the active set, every rule in $R$ is applied
      to $s'$;
    \item any new rules generated are added to $R$.
    \end{enumerate}

    This two-steps inner loop is called the \emph{percolation} phase.
  \item The collection of rules $R$ is added to the set of rules;
  \item All sequents generated during the above applications are tested for
    subsumption, and the un-subsumed sequents are added to the inactive set.
  \end{enumerate}
\end{definition}

\begin{theorem}[Completeness of the search loop]
  Let $S_a, S_i$ be the active and inactive sets and $R$ the collection of rules
  of the search procedure before initiating iteration $i$. Suppose $r$ is a rule
  in $R$ and $s_1, s_2, \dots, s_n$ is an ordered list of sequents from
  $S_a \cup S_i$ that would, if matched agains $r$ in this order, produce a
  successful result. Then this match actually occurs, and the result of it added
  to the state of the search, during some iteration $j \geq i$.
\end{theorem}
\begin{proof}
  By induction on $n$. If $n = 0$, then the match occurs vacuously during any
  iteration from $i$ on. If the list is instead $n+1$ elements long, then
  clearly the first $n$ sequents in the list potentially match $r$ successfully,
  and by inductive hypothesis they are eventually matched agains $r$ producing a
  partially applied rule $r'$ during some iteration $j \geq i$. We split cases
  depending on whether $s_{n+1} \in S_a$ or $s_{n+1} \in S_i$ during such
  iteration $j$.

  \begin{itemize}
  \item If $s_{n+1} \in S_a$ during iteration $j$ (whether because it is added
    at the beginning of it or because it is in $S_a$ already), then the
    percolation phase ensures that $r'$ is also matched against
    $s_{n+1}$. Therefore, $r$ is fully applied to $s_1, \dots, s_{n+1}$ during
    iteration $j$, where $j \geq i$ by hypothesis;
    
  \item If $s_{n+1} \in S_i$ during iteration $j$, then $s_{n+1}$ is eventually
    selected for activation during some iteration $k > j$ and matched against
    all rules in the collection of rules at that moment. From the inductive
    hypothesis it follows that $r'$ is added in the collection of rules during
    iteration $j$, so it certainly is in this collection during iteration
    $k$. Therefore, $r'$ is matched agains $s_{n+1}$ during some iteration
    $k \geq i$, hence producing the result of matching $r$ against
    $s_1, \dots, s_{n+1}$ during such iteration.
  \end{itemize}
\end{proof}

\end{document}

